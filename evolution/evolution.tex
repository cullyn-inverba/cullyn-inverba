\documentclass[12pt,a4paper]{article}
\usepackage{inverba}

\newcommand{\userName}{Cullyn Newman} 
\newcommand{\class}{BI 358} 
\newcommand{\institution}{Portland State Univsrsity} 
\newcommand{\theTitle}{\color{G-Leaf}Evolution}

\begin{document}
%%%%%%%%%%%%%%%%%%%%%%%%%%%%%%%%%%%%%%%%%%%%%%%%%%%%%%%%%%%%%%%%%%%%%
\tableofcontents
\cleardoublepage
\fancyhead{}
\fancyhead[R]{\hyperlink{home}{\nouppercase\leftmark}}
%%%%%%%%%%%%%%%%%%%%%%%%%%%%%%%%%%%%%%%%%%%%%%%%%%%%%%%%%%%%%%%%%%%%%

\clearpage
\fancyhead[L]{Introduction}
%\begingroup
%%%%%%%%%%%%%%%%%%%%%%%%%%%%% Chapter 1 %%%%%%%%%%%%%%%%%%%%%%%%%%%%%
%\begingroup
\clearpage
\section{Evolutionary Thinking}\phantomsection
\subsection{Introduction}
\begin{itemize}
    \item Essential questions of evolutionary biology: 
        \begin{itemize}
            \item Why do organisms look so different?
            \item Why develop elaborate sexual traits?
            \item Why do organisms senesce?
        \end{itemize}
    \item Evolution is mainly an historical science and thus must relay on other methods of reconstructing the past or making inferences about evolutionary forces. 
    \item \textbf{Proximate}: a question about a mechanistic cause; provides an immediate explanation about {\color{o-Sun}how} a mechanistic cause functions.
    \item \textbf{Ultimate}: {\color{o-Sun}why}, or the reason, a trait or organism is the way it is; an evolutionary explanation.
    \item Example of proximate vs ultimate in Galapagos finches: 
        \begin{itemize}
            \item Proxmate: developmental growth factor is increased/decreased in some birds.
            \item Ultimate: different habits are selected on breaks that maxmize food gathering ability.
        \end{itemize}
    \item Evolutionary biology's approach to answering questions:
        \begin{itemize}
            \item \textbf{Empirical data}: {\color{o-Sun}observation} studies, experiments; the \textit{comparative method}.
            \item \textbf{Theory}: {\color{o-Sun}predictions} that use models and mathematical reasoning which can be be {\color{o-Sun}tested} with empirical data.
        \end{itemize}
    \item Overview of the components of evolution by natural selection:
        \begin{itemize}
            \item Genetic variation exists, via mutations.
            \item Mutations are heritable.
            \item The is an advantage to survival and/or reproduction from the mutation.
            \item Individuals with the advantage in survival/reproduction are selected for.
        \end{itemize}
\end{itemize}
%\endgroup
%%%%%%%%%%%%%%%%%%%%%%%%%%%%% Chapter 1 %%%%%%%%%%%%%%%%%%%%%%%%%%%%%

%%%%%%%%%%%%%%%%%%%%%%%%%%%%% Chapter 10 %%%%%%%%%%%%%%%%%%%%%%%%%%%%%
%\begingroup
\clearpage
\setcounter{section}{9}
\section{Form and Function}\phantomsection
\subsection{}
\begin{itemize}
    \item 
\end{itemize}
%\endgroup
%%%%%%%%%%%%%%%%%%%%%%%%%%%%% Chapter 10 %%%%%%%%%%%%%%%%%%%%%%%%%%%%%

%%%%%%%%%%%%%%%%%%%%%%%%%%%%% Chapter 4 %%%%%%%%%%%%%%%%%%%%%%%%%%%%%
%\begingroup
\clearpage
\setcounter{section}{3}
\section{Evolutionary Trees}\phantomsection
\subsection{}
\begin{itemize}
    \item 
\end{itemize}
%\endgroup
%%%%%%%%%%%%%%%%%%%%%%%%%%%%% Chapter 4 %%%%%%%%%%%%%%%%%%%%%%%%%%%%%

%%%%%%%%%%%%%%%%%%%%%%%%%%%%% Chapter 3 %%%%%%%%%%%%%%%%%%%%%%%%%%%%%
%\begingroup
\clearpage
\setcounter{section}{2}
\section{Natural Selection}\phantomsection
\subsection{}
\begin{itemize}
    \item 
\end{itemize}
%\endgroup
%%%%%%%%%%%%%%%%%%%%%%%%%%%%% Chapter 3 %%%%%%%%%%%%%%%%%%%%%%%%%%%%%
%\endgroup

\clearpage
\fancyhead[L]{Mechanisms of Evolution}
%\begingroup
%%%%%%%%%%%%%%%%%%%%%%%%%%%%% Chapter 6 %%%%%%%%%%%%%%%%%%%%%%%%%%%%%
%\begingroup
\clearpage
\setcounter{section}{5}
\section{Mendelian Genetics}\phantomsection
\subsection{}
\begin{itemize}
    \item 
\end{itemize}
%\endgroup
%%%%%%%%%%%%%%%%%%%%%%%%%%%%% Chapter 6 %%%%%%%%%%%%%%%%%%%%%%%%%%%%%
%\endgroup
\end{document}