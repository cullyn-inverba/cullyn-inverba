\chapter{Data}

\section{Data Fundamentals}
\begin{itemize}
  \item \ddd{Data}: units qualitative or quantitative information about persons or objects collected via observation.
    \begin{itemize}
      \item Note: data is different from information---information resolves uncertainty, while data has the potential to be transformed into information post-analysis.
      \item Data as a general concept refers to the fact that some existing information or knowledge can be represented in a form suitable for processing.
    \end{itemize}

    \subsection{Data Types}
    \begin{itemize}
      \item Data types have two different general meanings:
        \begin{itemize}
          \item \ddd{Data type (computer science)}: involves the format of data storage and has implications on operations and storage space.
          \item \ddd{Data type (statistics)}: involves the category of data and has implications on the methods used for analysis.
        \end{itemize}
      \item There are many data types, with more specific definitions than the following definitions, but for now these are frequently used and adequate for topics covered.
      \begin{table}[h]
        \centering
        \caption{Relevant Statistical Data Types\strut}
        \begin{tabular}{clll}
          \toprule
          Category & Type & Description & Example  \\
          \midrule
          \multirow{3}{*}{Numerical} & Interval & Degree of difference & Temperature \\
          & Ratio & Interval + meaningful zero & Height \\
          & Discrete & Count (integers) & Population \\
          \midrule
          \multirow{2}{*}{Categorical}& Ordinal & Sortable, discrete & Educational level \\
          & Nominal & Non-sortable, discrete & Movie genre \\
          \bottomrule
        \end{tabular}
      \end{table}
    \end{itemize}

    \subsection{Population vs. Sample Data}
    \begin{itemize}
      \item \ddd{Population data \(\mu\)}: data from \emph{all} members of a group.
      \item \ddd{Sample data \(\hat{\mu}\)}: data from a \emph{subset} of members of a group (hopefully random).
      \item Statistical procedures generally are designed for sample or population data; wrong conclusions can be drawn if the distinction is not clear. 
        \begin{itemize}
          \item Note: most data are sample data in practice, as generalization of populations using sample data is usually the goal of statistics.
        \end{itemize}
      \ddd{Anecdotes}: a case study of a rare occurrence, or a sample size of only one; insights may be possible, but poor confidence in generalizations should be noted.
    \end{itemize}
  
\end{itemize}

\section{Visualizing Data}
\begin{itemize}
  \item 
\end{itemize}

