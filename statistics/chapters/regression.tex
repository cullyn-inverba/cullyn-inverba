\chapter{Regression}

\section{Regression Fundamentals}
\begin{itemize}
  \item[]
  
  \subsection{Model-Fitting}
  \begin{itemize}
    \item \ddd{Model fitting}: the combination of fixed features and free parameters in such a way that fits experimental data to a mathematical models based on adjustments to the free parameters that attempts to explain the \yyy{dependent variable}.
      \begin{itemize}
        \item \ddd{Fixed features \(x_n\)}: \xxx{independent variables} \emph{imposed on the model} based on previous knowledge, understanding, theories, hypotheses, or other evidence. 
        \item \ddd{Free parameters \(\beta_n\)}: variables that cannot be predicted precisely or constrained by the model; they must be \emph{adjusted or estimated}.
        \item \ddd{Intercept \(\beta_\nil\)}: the average when all other parameters are 0.
        \item \ddd{Residual (error) \(\fff{\varepsilon}\)}: the \fff{residual} (innovation, error) \fff{data} which are not directly observed or fit by the model.
      \end{itemize}
    \item \ddd{Model interpolation}: the prediction of other experimental results given prior experimental data and a fitted model based on those data.
    \item \ddd{The general outline of model-fitting}:
    \begin{itemize}
      \item \emph{Define the equation(s) underlying the model}; dependent on data availability. % chktex 36
        \begin{itemize}
          \item If \emph{all} the \xxx{fixed features} are \hyperref[Subsection: Data Types]{\ulink{discrete}}, then use \hyperref[Chapter: Analysis of Variance]{\ulink{ANOVA}}.
          \item If at least \emph{some} \xxx{fixed features} are \emph{continuous}, then use regression.
          \item E.g., height \(\yyy{h}\) is governed by numerous complex interactions, but a simplistic model can be made to estimate the importance of particular fixed features;
          \[%%%%%%%%%%%%%%%%%%%%%%%%%%%%%%%%%%%%%%%%%%%%%%%
          \yyy{h} = \beta_1 \xxx{x_n} + \beta_2 \xxx{x_n} + \beta_3  \xxx{x_n} \quad \xxx{x_1}: \text{sex},\xxx{x_2}: \text{parents' height}, \xxx{x_3}: \text{nutrition}
          \]%%%%%%%%%%%%%%%%%%%%%%%%%%%%%%%%%%%%%%%%%%%%%%%
          
        \end{itemize}
      \item \emph{Map the data to the model equations}, i.e.,  take the real, or simulated, data and map them to the \xxx{fixed features},  yielding a system of equations with a series of unknown parameters. 
      \item \emph{Convert the equations into a matrix-vector equation}, i.e., \(\xxx{\bm{X}}\bm{\beta} = \bm{\yyy{y}}\) (\(\bm{Ax}=\bm{b}\))
      \item \emph{Convert the equations into a matrix-vector equation}, i.e., \hyperref[Subsection: Multiple Regression]{\dlink{the general linear model}}; sometimes simplified to \(\xxx{\bm{X}}\bm{\beta} = \bm{\yyy{y}}\) (linear algebra nomenclature: \(\bm{Ax}=\bm{b}\)).
      \item \emph{Statistical evaluation of the model}, i.e., the application of inferential statistics, e.g., \hyperref[Subsection: Model Significance]{\dlink{model significance}} and \hyperref[Subsection: Standardized Coefficients]{\dlink{coefficients significance}}.
    \end{itemize}
  \end{itemize}

  \subsection{Least-Squares}
  \begin{itemize}
    \item 
  \end{itemize}

  \subsection{Model Significance}
  \begin{itemize}
    \item 
  \end{itemize}

  \subsection{Standardized Coefficients}
  \begin{itemize}
    \item 
  \end{itemize}
  
\end{itemize}

\section{Regression Models}
\begin{itemize}
  \item [] 
  
  \subsection{Simple Regression}
  \begin{itemize}
    \item 
  \end{itemize}
  
  \subsection{Multiple Regression}
  \begin{itemize}
    \item \ddd{General linear model}: a compact way of writing several multiple linear regression models using matrix algebra, i.e.,
    \[%%%%%%%%%%%%%%%%%%%%%%%%%%%%%%%%%%%%%%%%%%%%%%%
    \bm{\yyy{Y}} = \bm{\xxx{X}\beta} + \fff{\varepsilon}
    \]%%%%%%%%%%%%%%%%%%%%%%%%%%%%%%%%%%%%%%%%%%%%%%%
    \begin{itemize}
      \item \(\yyy{\bm{Y}}\): matrix with a series of multivariate measurements, where each column is a set of measurements of one of the \yyy{dependent variables}.
      \item \(\xxx{\bm{X}}\): the \xxx{design matrix}, where each column is a set of observations on \xxx{independent variables}.
      \item \(\bm{\beta}\): the matrix of \(\beta \) coefficients to be estimated.
      \item \(\displaystyle\fff{\varepsilon}\): the matrix of \fff{residuals} associated with the model.
    \end{itemize}
  \end{itemize}
  
  \subsection{Polynomial Regression}
  \begin{itemize}
    \item 
  \end{itemize}
  
  \subsection{Logistic Regression}
  \begin{itemize}
    \item 
  \end{itemize}

  \subsection{Nested Models}
  \begin{itemize}
    \item 
  \end{itemize}
  
\end{itemize}

\section{Statistical Power and Sample Size}
\begin{itemize}
  \item []
  
  \subsection{Statistical Power}
  \begin{itemize}
    \item 
  \end{itemize}

  \subsection{Sample Size}
  \begin{itemize}
    \item 
  \end{itemize}
  
\end{itemize}
