\chapter{Clustering and Dimensionality Reduction}

\section{Clustering}
\begin{itemize}
  \item \ddd{Cluster analysis}: the task or grouping a set of objects in such a way that the objects in the same group (cluster) are more similar to each other than those in other groups.
  \item There is no one specific clustering algorithm, instead there are various algorithms that differ by what constitutes a cluster and how to efficiently find them.
  \item Distance (\hyperref[Subsection: K-Means Clustering]{\dlink{to center}}, or \hyperref[Subsection: K-Nearest Neighbor]{\dlink{to neighbor}}) and \hyperref[Subsection: DBSCAN]{\dlink{ density}} will be the main metrics investigated here, but there are certainly more.
  
  \subsection{K-Means Clustering}
  \begin{itemize}
    \item \ddd{\(k\)-means clustering}: a method of vector quantization that aims to partition \emph{\(n\)~observations} (multidimensional data) into \emph{\(k\) clusters}, wherein each observation belongs to the cluster with the \emph{nearest centroid} (mean).
      \begin{itemize}
        \item Cluster membership is defined using distances to the nearest center, where the aim is to \bbb{minimize within-group} and \rrr{maximizes between-group} distances.
      \end{itemize}
    \item Specifics (or the many alterations) of the algorithm will not be covered, instead, a general outline of the method will be described:
      \begin{enumerate}
        \item Select \(k\) (can be very difficult); many potential methods.
        \item Create \(k\) centroids at random (possibly slightly informed) locations in the dataset.
        \item Compute the sum of squared distances from all data points in the dataset.
        \item Assign each data point to its closest centroid.
        \item Create a new centroid at the average of all data points.
        \item Repeat 3--5 until some measure convergence (how much mean moves each iteration, typically).
      \end{enumerate}
    \item Difficulties with \(k\)-means:
      \begin{itemize}
        \item Evaluation of proper \(k\) is often difficult.
        \item Multidimensional clustering is often hard to visualize, making evaluation even harder sometimes.
        \item Computation is complex (NP-hard); therefore, each result is often different.
        \item Suboptimal when cluster sizes differ greatly.
        \item What makes clusters similar may not be Euclidean based.
      \end{itemize}
  \end{itemize}

  \subsection{DBSCAN}
  \begin{itemize}
    \item \ddd{Desnity-based spatial cluster of applications with nose DBSCAN}: a \emph{density} based \hyperref[Subsection: Parametric vs. Nonparametric]{\ulink{nonparametric}} algorithm that clusters points into groups that are relatively close together; used as means to label and find clusters in data.
    \item \ddd{Step size \(\epsilon\)}: the radius of a point that in combination with specified minimum nearby points \(m\) that determine the density of neighborhood of points with respect to other points.
    \item \rrr{\textbf{Core points \(A\)}}: a point where at least \(m\) points are within distance \(\epsilon \) (+ itself).
    \item \yyy{\textbf{Reachable points \(B\)}}: when the point is within \(\epsilon \) from a core point, but \(< m\) required to be a core point.
    \item \bbb{\textbf{Outliers \(N\)}}: low-density regions, often considered as noise points.
      \begin{center}
        \Image{0.6\columnwidth}{chapters/images/DBSCAN.png}
      \end{center}
    \medskip
      \begin{itemize}
        \item \bbb{Decreasing \(\epsilon \)} \to~\rrr{increases number of clusters}, breaks up clusters.
        \item \rrr{Increasing \(\epsilon \)} \to~\bbb{decreases number of clusters}, combines clusters.
        \item \bbb{Decreasing \(m \)} \to~\bbb{decreases number of clusters}, true clusters may be split up.
        \item \rrr{Increasing \(m \)} \to~\rrr{increases number of clusters}, true clusters may be ignored.
      \end{itemize}
  \end{itemize}

  \subsection{K-Nearest Neighbor}
  \begin{itemize}
    \item \ddd{\(k\)-nearest neighbor \(k\)-NN}: a nonparametric \emph{classification} method applied to points based on the \emph{distance to}, potentially and \emph{plurality of}, its neighboring points.
      \begin{itemize}
        \item Unlike \(k\)-means, data must be labeled; \(k\)-NN is used to classify new data points. 
        \item If \(k=1\), then it's simply assigned to class of nearest neighbor.
      \end{itemize}
    \item Normalizing the training data can greatly improve accuracy of \(k\)-NN, as it relies on distance, which may come in vastly different scales.
    \item \(k\)-NN \emph{regression} can be used in a similar sense, where out output is the property value of the object based on averages of values of \(k\) nearest neighbors.
  \end{itemize}
  
\end{itemize}

\section{Dimensionality Reduction}
\begin{itemize}
  \item \ddd{Dimensionality (dimension) reduction}: the transformation of data from high-dimensional space into a lower-dimensional space, so that the low-dimensional representation retains meaningful properties of the original data.
    \begin{itemize}
      \item Raw data in higher dimensions are often sparse, due to the ``curse of dimensionality.''
      \item Analysis in higher dimensions is often computationally intractable, requiring dimension reduction of some kind in order to be feasible.
    \end{itemize}
  \item Dimension reduction is common in fields that deal with large number of observations and/or variables, e.g., signal processing, speech recognition, neuroinformatics, bioinformatics, etc.
  \item There are several methods, which are commonly divided into linear and non-linear approaches; for now, only primers with simplified math on principal and independent component analysis will be covered.

  \subsection{Primer: Principal Components Analysis}
  \begin{itemize}
    \item \ddd{Principal component analysis (PCA)}: the main linear technique for dimension reduction, wherein linear mapping of the data to lower-dimensional space minimizes variance of data in the new representation.
      \begin{itemize}
        \item The largest eigenvalues (principal components) and corresponding eigenvectors of the data's covariance matrix is used reconstruct (change of basis) a large fraction of the variance of the original data.
        \item Often only the first few principal components are used, as they contribute the vast majority to the systems' energy and the rest tend to be noise.
          \begin{itemize}
            \item These largest values together represent the lower-dimension space that the data will be compressed to.
          \end{itemize}
      \end{itemize}
    \item Limitations of PCA\@:
      \begin{itemize}
        \item Principal components are eigenvectors, thus they are forced to be orthogonal; the dimension of key features often aren't orthogonal.
        \item PCA maximizes variance to find distinctions between dimensions, however, sometimes this might maximize unwanted noise and miss important features within the data.
      \end{itemize}
    \item In conclusion: PCA is great for dimension reduction and visual exploration, especially if the data conform to assumptions, but can be suboptimal when interpreting PCs as factors (unique sources of variance).
  \end{itemize}
    
  \subsection{Primer: Independent Components Analysis}
  \begin{itemize}
    \item \ddd{Independent component analysis (ICA)}: a method for separating multivariate signal into additive subcomponents.
      \begin{itemize}
        \item Performed by assuming that the subcomponents are \emph{non-Gaussian} signals and are statistically \emph{independent} of each other.
      \end{itemize}
    \item Basic outline of ICA\@:
    \begin{itemize}
      \item ``Whiten'' the data, or remove co/variances across different variables (using PCA)
        \begin{itemize}
          \item Linear dependencies are removed, but shared information is preserved.
        \end{itemize}
      \item Rotate axes (oblique) to minimize share information.
        \begin{itemize}
          \item Unlike PCA, ICA is not limited to orthogonal axes, allowing for independent non-orthogonal axes to be found.
        \end{itemize}
     \end{itemize}
  \end{itemize}
    
\end{itemize}

