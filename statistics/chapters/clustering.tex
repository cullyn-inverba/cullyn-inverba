\chapter{Clustering and Dimensionality Reduction}

\section{Clustering}
\begin{itemize}
  \item \ddd{Cluster analysis}: the task or grouping a set of objects in such a way that the objects in the same group (cluster) are more similar to each other than those in other groups.
  \item There is no one specific clustering algorithm, instead there are various algorithms that differ by what constitutes a cluster and how to efficiently find them.
  \item Distance (\hyperref[Subsection: K-Means Clustering]{\dlink{to center}}, or \hyperref[Subsection: K-Nearest Neighbor]{\dlink{to neighbor}}) and \hyperref[Subsection: DBSCAN]{\dlink{ density}} will be the main metrics investigated here, but there are certainly more.
  
  \subsection{K-Means Clustering}
  \begin{itemize}
    \item \ddd{\(k\)-means clustering}: a method of vector quantization that aims to partition \emph{\(n\)~observations} (multidimensional data) into \emph{\(k\) clusters}, wherein each observation belongs to the cluster with the \emph{nearest centroid} (mean).
      \begin{itemize}
        \item Cluster membership is defined using distances to the nearest center, where the aim is to \bbb{minimize within-group} and \rrr{maximizes between-group} distances.
      \end{itemize}
    \item Specifics (or the many alterations) of the algorithm will not be covered, instead, a general outline of the method will be described:
      \begin{enumerate}
        \item Select \(k\) (can be very difficult); many potential methods.
        \item Create \(k\) centroids at random (possibly slightly informed) locations in the dataset.
        \item Compute the sum of squared distances from all data points in the dataset.
        \item Assign each data point to its closest centroid.
        \item Create a new centroid at the average of all data points.
        \item Repeat 3--5 until some measure convergence (how much mean moves each iteration, typically).
      \end{enumerate}
    \item Difficulties with \(k\)-means:
      \begin{itemize}
        \item Evaluation of proper \(k\) is often difficult.
        \item Multidimensional clustering is often hard to visualize, making evaluation even harder sometimes.
        \item Computation is complex (NP-hard); therefore, each result is often different.
        \item Suboptimal when cluster sizes differ greatly.
        \item What makes clusters similar may not be Euclidean based.
      \end{itemize}
  \end{itemize}

  \subsection{DBSCAN}
  \begin{itemize}
    \item \ddd{Desnity-based spatial cluster of applications with nose DBSCAN}: a \emph{density} based \hyperref[Subsection: Parametric vs. Nonparametric]{\ulink{nonparametric}} algorithm that clusters points into groups that are relatively close together.
      \begin{itemize}
        \item BCSCAN is one of the most common clustering algorithms used.
      \end{itemize}
    \item \ddd{Step size \(\epsilon\)}: the radius of a point that in combination with specified minimum nearby points \(m\) that determine the density of neighborhood of points with respect to other points.
    \item \rrr{\textbf{Core points \(A\)}}: a point where at least \(m\) points are within distance \(\epsilon \) (+ itself).
    \item \yyy{\textbf{Reachable points \(B\)}}: when the point is within \(\epsilon \) from a core point, but \(< m\) required to be a core point.
    \item \bbb{\textbf{Outliers \(N\)}}: low-density regions, often considered as noise points.
      \begin{center}
        \Image{0.6\columnwidth}{chapters/images/DBSCAN.png}
      \end{center}
    \medskip
      \begin{itemize}
        \item \bbb{Decreasing \(\epsilon \)} \to~\rrr{increases number of clusters}, breaks up clusters.
        \item \rrr{Increasing \(\epsilon \)} \to~\bbb{decreases number of clusters}, combines clusters.
        \item \bbb{Decreasing \(m \)} \to~\bbb{decreases number of clusters}, true clusters may be split up.
        \item \rrr{Increasing \(m \)} \to~\rrr{increases number of clusters}, true clusters may be ignored.
      \end{itemize}
  \end{itemize}

  \subsection{K-Nearest Neighbor}
  \begin{itemize}
    \item 
  \end{itemize}
  
\end{itemize}

\section{Dimensionality Reduction}
\begin{itemize}
  \item []
  
  \subsection{Primer: Principal Components Analysis}
  \begin{itemize}
    \item []
  \end{itemize}
    
  \subsection{Primer: Independent Components Analysis}
  \begin{itemize}
    \item 
  \end{itemize}
    
\end{itemize}

