\documentclass[plain,basic]{inVerba-notes}

\newcommand{\userName}{Cullyn Newman}
\newcommand{\class}{BI:\@ 428}
\newcommand{\theTitle}{Pedigree Assingment}
\newcommand{\institution}{Portland State University}

\begin{document}
\section*{Huntington Disease, HD} 

\begin{itemize}
    \item \textbf{Description}: A progressive disorder of motor, cognitive, and psychiatric disturbances. Key signs of effected individuals is progressive motor disability featuring chorea, which is the involuntary, irregular, and unpredictable muscle movement. Changes in personality, cognitive decline, and depression are also very common. Symptoms are a result of progressive, selective neural cell loss and atrophy in the caudate and putamen. 
    \item \textbf{Mode of inheritance}: autosomal dominant.
    \item \textbf{Gene responsible and location}: Huntingtin (HTT;\@ 613004) on chromosome 4p16. Caused by an expanded trinucleotide repeat (CAG, encoding glutamine). 
    \item \textbf{Penetrance/prevalence}:
        \begin{itemize}
            \item Normal: 26 or fewer CAG repeats. Not at risk.
            \item Intermediate: 27 to 35. Not at risk, but has risk of having a child with a higher range due to mutations during germline reproduction.
            \item Pathogenic HD-causing alleles: 36 or more repeats. 
                \begin{itemize}
                    \item \textbf{Reduced-penetrance}: 36--39 repeats. At risk, but may not develop symptoms. Elderly asymptomatic individuals are common in this range.
                    \item \textbf{Full-penetrance}: 40 or more. Chances of developing HD increases with both age and number of repeats. Mean age of onset is approximately 45 years.
                \end{itemize}
        \end{itemize}
    \item \textbf{Molecular genetics}: The fact that the phenotype of HD is completely dominant suggested that the disorder results from a gain-of-function mutation in which either the mRNA product or the protein product of the disease allele has some new property or is expressed inappropriately.~\shortcite{myers1982factors}
    \item \textbf{Pathogenesis}: The mutant huntingtin protein in HD results from an expanded CAG repeat leading to an expanded polyglutamine strand at the N terminus and a putative toxic gain of function. Neuropathologic studies show neuronal inclusions containing aggregates of polyglutamines.~\cite{walker2007huntington}
        \begin{itemize}
            \item In addition to Huntington disease, there are at least 8 other diseases of the central nervous system, each of which is known to be associated with a different protein containing an expanded polyglutamine sequence. 
        \end{itemize}
\end{itemize}

\nocite{*}
\bibliographystyle{apacite}
\bibliography{pedigree.bib}
\end{document}