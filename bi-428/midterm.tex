\documentclass[basic]{inVerba-notes}

\newcommand{\userName}{Cullyn Newman}
\newcommand{\class}{BI:\@ 428}
\newcommand{\theTitle}{Midterm}
\newcommand{\institution}{Portland State University}

\begin{document}
  \section{Week 2: Chromosomes}
  \begin{enumerate}
    \item \textbf{What is the ``MAE''?}
    
    MAE, or maternal age effect (or advanced maternal age), is a risk factor that is known to predispose older mothers (ages 35 or more, typically) to various autosomal trisomies (13, 18, and 21). There isn't exactly a threshold age however, it's actually occurs on a continuum, with the first notable increase occurring around age of 30--32, but getting exponentially worse as the age increases.

    More specifically, MAE increases the chances of aneuploidy (presence of abnormal number of chromosomes), which is the result of nondisjunction during cell division, resulting in improper separation chromosomes. The effect on nondisjunction can have various effects depending on when it occurs. 
      \begin{itemize}
        \item Meiotic nondisjunction results in both homologous chromosomes going to the same gamete.
        \item Second meiotic division results in two identical chromatids going to the same gamete.
        \item Somatic cell division results in mosaicism (two or more genetically different sets of cells), assuming it results after first cleavage.
      \end{itemize}

    The coexistence of a normal cell with an aneuploid line can allow the for the development of trisomy, which would otherwise would not survive early on. Exact causes of nondisjunction is not known, but it does appear to be most common in the first meiotic division in females. This is why MAE is a suspected factor, since the chances of nondisjunction probably increased due to errors accumulated with age. This is also why most cases of down syndrome (specifically trisomy 21, the most common trisomy) have two maternal and one paternal chromosomes, giving it three copies of chromosome 21 (karyotype 47,XX,+21/47,XY,+21).

    \item \textbf{What is an acrocentric chromosome?}
    
    Acrocentric refers to the placement, or location, of the centromere on the chromosome. Acrocentric is when the centromere is near one end of the chromosome, while metacentric is when its in the center. Submetacentric is then of course slightly off center.

    The ``p'' are is the shorter arm of the chromosome, and in the case of acrocentric, then it's usually quite short. There short arms consist of DNA that encodes for ribosomal RNA and often contain remain condensed with small knobs referred to as satellites.

    MAE contributes to a large portion of Down syndrome cases, but there are other causes, such as errors during translocations (chromosomal exchange). A Robertsonian translocation is one of these errors (2--5\% of cases), in which the long arm (``q'') of chromosome 21 (an acrocentric chromosome) fuses to the centromere of another chromosome (typically chromosome 14). If both the Robertsonian translocated chromosome and a normal 21 go to the same germ cell during meiosis, then the result will be trisomy 21. Robertsonian translocation is not affected my MAE\@.

    \item \textbf{What is the function of telomerase?}
    
    Telomerase is an enzyme that adds a telomere repeat sequence to the 3' end of the lagging  strand in telomeres through the use of a specific RNA template (GGGTTA in humans). This is important, because the lagging strand must remove the terminal RNA primer during replication. This isn't a problem until you reach the end of a DNA sequence, where there are no new upstream primers for DNA polymerase to do its work.

    Telomerase solves this problem by creating a sequence of DNA that is added on to the end of a chromosome, creating telomeres. Sometimes somatic replicate without telomerase, or other factors inhibit its function, which results chromosomes gradually shortening until vital DNA sequences start to become damaged, leading to senescence. It thought some purposefully lose the function of telomerase to control cell divisions, reducing probability of cancerous growth.  

    \item \textbf{What is constitutive heterochromatin?}

    Constitutive heterochromatin are regions of DNA found throughout the chromosome of eukaryotes. They are found in telomeres and scattered throughout the chromosome, but mostly in the pericentromeric (near the centromere) regions of chromosomes. The is much more in chromosomes 1,9,16, 19, and Y. 

    Visualization is done by using the C-banding technique, which is stains regions of DNA\@. Since constitutive heterochromatin are so condensed, then it is usually very easy to identify them due to darker staining.

    Constitutive heterochromatin mostly consist of various repeats, accounting for about 6.5\% of the human genome. Typically, they do not have many genes, but some have been found. It is thought that histone modifications are one of the main ways of condensing constitutive heterochromatin, which suggests they play a role in some heritable characteristics due to epigenetics. 

    \item \textbf{Describe the process of comparative genomic hybridization (CGH)}
    
    CGH is a molecular cytogenetic (branch of genetics relating chromosomes to behavior) method used to quickly and efficient compare two genomic DNA samples (control and reference) in order to detect deletions or duplications. CGH is only able to detect unbalanced chromosomal abnormalities, since it compares copy number variants to relative ploidy levels. However, it still can lead to identification of candidate genes.

    There are multiple methods, such as FISH (fluorescence in situ hybridization), blocking, DNA labelling, CGH arrays, and others, all of which are applied in a wide variety of techniques with different advantages.

    For example, conventional CGH are mainly used to detect genomic diagnosis and prognosis of cancer, while CGH arrays can be extended and used to find aberrations that cause a variety human genetic disorders. 
    
    \item \textbf{What is the difference between aneuploidy and polyploidy?}
    
    Polyploidy refers to the change of a whole set of chromosomes, and generally defined as the condition in which cells of an organism have more than homologous sets of chromosomes. Rangers from haploid, diploid, triploid\dots\ to decaploid (10) and dodecaploid (12).

    Aneuploidy specifically refers to a particular chromosome that is either over or underrepresented. 

    Polyploid = whole set; all. Aneuploid = part; one.

    \item \textbf{A balanced carrier has as reciprocal translocation between chromosome 4 and 11. What does this mean?}
    
    A reciprocal translocation is the exchange of segments between two or more chromosomes. A balanced reciprocal translocation would generally mean there is no gain or loss of genetic material, but errors can still arise. 

    It is not guaranteed that there will be no phenotypic or chromosomal defect, as a gene could still be disrupted, or an alteration due a gene due to the position effect could still be at play. 

    The best option is to perform chromosomal analysis on both parents. If there is rearrangement already present in one of the parents, then there is probably little chance that any abnormalities are due to the translocation. However, if there is no rearrangement present, then it could be significant, and further testing using uniparental disomy (2 copies from same parent, rather than 1 from both) could help illuminate the cause of any abnormalities if they are present.
  \end{enumerate}

  \section{Week 3: Sexual Development}

  \begin{enumerate}
    \item \textbf{What are the PARs?}
    
    Pseudoautosomal regions (PARs) are homologous sequences of nucleotides on the ends of X and Y chromosomes. They are found in two locations, PAR1 and PAR2, that are on opposite sides and are assumed to evolve independently. I

    There small number of genes (29 found so far) on these sex chromosomes are inherited just like any other autosomal gene. This means that females can inherit an allele originally present on the Y chromosome. 

    The main function of the PARs is to allow the X and Y chromosomes to pair and properly segregate during meiosis in males and allow for ends of the X chromosome to escape X inactivation. 

    \item \textbf{What does the Barr Body represent?}

    The Barr Body represents the inactive X chromosomes; they can be visualized as dense bodies if stained during interphase.

    X inactivation is controlled by the X inactivation center (Xic), usually found near the centromere. Two genes (Tsix, Xist) in Xic appear to play antagonistic role in this process. The loss of Tsix results in increased levels of Xist around the Xic. No loss of Tsix keeps Xist levels low. 

    High levels of Xist will spread out, coating the future inactive chromosome (except for PAR regions as noted in question 1), leading to eventual condensation. The process happens in early embryonic development at random in mammals. 

    Humans that have more than one X chromosome always have one fewer than the total number of X chromosome. Penta-X Klinefelter would have four, penta-X female would have 4, and normal females have 1.

    \item \textbf{How many genes are on the Y chromosome?}
    
    According to CCDS\@: 63. Other sources have as little as 35, up to a max of 73 protein coding genes. The number of non-coding RNA genes rangers from 55 to 109, and pseudogenes range from 381 to 400.

    The Y chromosomes is known to be one of the largest gene deserts in the human genome. Notable genes include: 

    \begin{itemize}
      \item SRY (sex-determing region)
      \item ZFY (zinc finger protein)
      \item TSPY (testis-specific protein)
    \end{itemize}

    There are others, but I don't recall discussing them, or the exact function is unknown.

    \minimal{``Homo sapiens Y chromosome genes''. CCDS Release 20 for Homo sapiens. 2016-09-08. Retrieved 2017-05-28.} % chktex 8

    \item \textbf{What is a pseudogene? What is gynecomastia?}
    
    Pseudogenes are inactive versions of genes, generally existing as an extra copy. Most arise by DNA duplication directly, or indirectly through reverse transcription.

    Gynecomastia is the abnormal non-cancerous enlargement of breasts in men due to hormone imbalances. 

    \item \textbf{Describe in detail the development of the male sexual system, from fertilization to puberty.}
    
    Embryos sexual identity is first determined at fertilization, with the SRY gene on the Y chromosome being the main factor. The SRY gene recruits other genes, such as TDF, which all together begin a cascade of factors that guide male development and suppresses female development.

    Testis begin to develop from the gonadal ridge and the gonads differentiate into spermatogonia instead of the eventual ovary. The formation of the testis trigger the Leydig cells to secrete testosterone, which influences a variety of developmental pathways.
    
    Not all cell tissues are can differentiate. Most primitive cells form two rudimentary duct systems in the embryo which then must develop from. The Müllerian duct is triggered for degradation in males, while testosterone stimulate growth of the male tract, or Wolffian duct. Other sensitive tissues are triggered to develop into the penis in the presence of testosterone.

    Testosterone also sets up hypothalamic pathways, triggering production of gonadotropin-releasing hormones (GnRH), which later influence various aspects of sexual physiology in later development. These hormonal pathways trigger various positive and negative feedback loops that leads to various secondary sex characteristics. In males, facial hair, elongation of Larynx, broader shoulders, increases to body hair in other areas, and increases to total musculature. 

    Adult gender identity is formed due to many of these secondary sex characteristics, hormonal changes in the brain, and society expectations. 

    \item \textbf{Research and discuss in some detail one cause of both male and female pseudohermaphroditism.}
    
    Pseudohermaphroditism describes an organism that is born with primary sex characteristics of one sex, but develops the secondary sex characteristics of another, based on gonadal tissue. True hermaphroditism shows both tissue are present in one individual.

    One potential cause of male pseudohermaphroditism could be mutations affecting the androgen receptor gene, which could cause partial androgen insensitivity, or complete. This means that initial sex development could occur fine, but lack of receptors would not allow for the suppression of secondary female characteristics and development of male secondary sex characteristics triggered by the typically more potent androgen created from testosterone later in development. 

    Female pseudohermaphroditism could be caused by a female fetus is exposed to excess androgenic environments, either from drug intake of ovarian tumors. This would produce a female with typical gonads, but various secondary sex characteristic pathways' feedback loops to be triggered due to accidental exposure.
  \end{enumerate}

  \section{Week 4: Development, Imprinting, and Epigenetics}

  \begin{enumerate}
    \item 1. \textbf{As the fertilized egg goes from a single cell to the multicellular fetus, the cells change from being “totipotent” to “pluripotent” to “multipotent”. What do these terms mean and what development stages do they represent?}
    
    These terms describe cell potency, or the cell's ability to differentiate. The more cells that one cell can turn into, then the greater the potency. Potency exists on a continuum; in regard to the terms given: totipotent > pluripotent > multipotent.

    More specifically, totipotent represents the greatest cell potency; they can differentiate into any embryonic cell, as well as any extraembryonic cell. Spores and zygote are examples of totipotent cells; they are the first stage of development.

    Pluripotent cells are cells that can differentiate into any of the three germ layers (ectoderm, mesoderm, or endoderm). Some pluripotent stems calls can properly represent any embryonic cell (e.g., embryonic stem cells) or impartially repseent one of the three germ layers, but lack ability adopt some characteristics.  

    Multipotent refers to progenitor cells, which like stem cells, but already pushed towards a target cell. Blood cells are a good example as they can still differentiate into different cell types, but are partially developed. 
    
    \item \textbf{Look up one of the development gene defects listed on Table 3.8 in your lecture notes. What gene is involved, what is the resultant phenotype?} 
    
    Werner syndrome ATP-dependent helicase (WRN) is a member of the RecQ  helicase family. RecQ Helicase has been shown to be important in gnome maintenance. The main function of WRN is thought to be that of repairing double strand breaks caused by homologous recombination, or non-homologous end joining, it is effective in replication arrest recovery, aids in repair of single nucleotide damages due to base excision repair, and maybe even telomere maintenance and replication. 

    \minimal{Ding SL, Shen CY (2008). ``Model of human aging: recent findings on Werner's and Hutchinson–Gilford progeria syndromes''. Clin Interv Aging. 3 (3): 431–44.}

    Werner syndrome is caused by mutations to WRN, with more than 20 mutations known to cause the syndrome; usually resulting in a shortened Werner protein. Recent evidence suggests that either the protein is not transported into the nucleus, or it breaks down too quickly, leading to inability to interact with DNA\@. Methylation of WRN can also cause the gene to turn off completely. Overall, Werner syndrome's grand phenotype effect is that of premature aging.

    \qquad

    \item \textbf{What is the difference between a teratogen and a mutagen? What is Dilantin used for medically and what birth defects does it cause?}
    
    A mutagen is an agent that chemically or physically damages genetic material. Most commonly, mutagens cause in increase in mutations. Mutagens are likely to also be carcinogens, which are mutations responsible for causing cancer. Mutagens can be passed down if the germ line is effected. 

    Teratogens are specifically substances that cause birth defects due to changes in the embryonic environment or to the fetus. The DNA can remain perfectly intact, but change to the environment can lead to birth defects due to unwelcomed interactions. 

    Dalatin is anti-seizure medication. It can also be used heart arrhythmias or neuropathic pain. There is evidence that it can act as a teratogen, causing birth defects, though appears to be safe while breastfeeding. It can cause mild to moderate mental deficiency, poor growth, cleft lip, heart defects, short forward positioned nose, and more.

    \item \textbf{Explain the phenomenon of uniparental disomy and its relationship to Angelman and Prader-Willi Syndromes. Why are the effects different given that the same area of chromosome 15 is involved?}
    
    Angelman and Prader-Willi syndromes that are both associated with abnormalities involving imprinted genes in the region of chromosome 15. Around 70\% of cases are due to microdeletions. There determining factor between the two syndromes depends on which parental chromosome it comes from. Paternal 15 results in Prader-Willi, while maternal results in Angelman.

    However, the remaining 30\% cases of Prader-Willi and 1--2\% of Angelman are due to uniparental disomy (UPD). Again, uniparental disomy is when an individual receives two copies, or at least the relative part, from just one parent and not the other. UPD can disrupt parent-specific imprinting and may lead to uncovering of recessive genes.

    Trisomy 15 is more common in maternal nondisjunction, which leads to a more frequent loss of paternal 15 thanks to trisomy rescue mechanisms. This is why Prader-Willi syndrome is more commonly seen in UPD cases. 
    
    \item \textbf{Read, summarize and contrast the application of the term “epigenetics” as reflected in these two articles:}
    
    \minimal{Reik et al. 2001. Epigenetic reprogramming in mammalian development. Science 293:1089--1093.
    
    Nilsson et al. 2018. Environmentally induced epigenetic transgenerational inheritance of disease susceptibility. Environmental Epigenetics 4(2):12--17.}

    Reik et al.\ appears to be using epigenetics as more of means of cell reprogramming, and claims that the reprogramming is what is responsible for genetic imprinting. They claim that sequences that escape the reprogramming may be involved in epigenetic inheritance. They also connect reprogramming is also involved in stem cell differentiation. Essentially, this paper focuses in DNA methylation and how it may be used to suggest a mechanism for such reprogramming. 

    Nilsson et al.\ reviews environmentally induced epigenetic changes and offers a much more broad definition of epigenetics. These authors operate under a more recent definition of, ``the molecular factors and processes around DNA that regulate genome activity independent of DNA sequence, and that are mitotically stable.'' In this paper the authors are trying to understand the molecular mechanisms that make up the epigenome.

    DNA methylation was discussed in Nilsson et al.\ paper of course, and particularly reprogramming was described there as a transgenerational, genome wide reset. Though, they differentiate between true imprinting and imprinted like, which I do not see Reik et al.\ making any kind of distinction. Reik's focus appeared to be more directly involved in mechanism of returning to totipotent of cells, rather than understanding genomic changes over time like Nillson did. 




  \end{enumerate}
\end{document}