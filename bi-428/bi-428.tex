\documentclass{inVerba-notes}

\newcommand{\theTitle}{\href{https://github.com/cullyn-inverba/notes/tree/master/bi-428} {Human Genetics}}

\begin{document}
\tableofcontents

%%%%%%%%%%% DNA Structure and Function %%%%%%%%%%%

%\begingroup
\chapter{DNA Structure and Function}\label{DNA Basics}
\begin{adjustwidth}{1cm}{1cm}
  \dots
\end{adjustwidth}

\section{Deoxyribonucleic Acid}\label{DNA}
\begin{itemize}
    \item \ddd{Deoxyribonucleic Acid (DNA)}: a double helix containing two polynucleotide chains that carries the genetic instructions for all known organisms and many viruses.
        \begin{itemize}
            \item The bases are made of four bases:
                \begin{itemize}
                    \item The \yyy{purine} derivatives: \yyy{adenine (A) and guanine (G)}.
                    \item The \xxx{pyrimidine} derivatives: \xxx{thymine (T) and cytosine (C)}.
                \end{itemize} 
            \item The backbone is made of \emph{alternating deoxyribose} molecules (a ribose missing its 2' oxygen) connected to phosphodiester bonds from \emph{5' \to\ 3'} positions---forming two \emph{antiparallel} strands.
            
                \centering
                \schemestart{}
                \chemname{\chemfig[cram width=2pt]{HO-[2,0.5,2]?<[7,0.7](-[6,0.5]OH)-[,,,,line width=2pt](-[6,0.5]\emph{OH})>[1,0.7](-[6,0.5]OH)-[3,0.7]O-[4]?}}{Ribose}
                \arrow{}
                \chemname{\chemfig[cram width=2pt]{HO-[2,0.5,2]?<[7,0.7](-[6,0.5]OH)-[,,,,line width=2pt](-[6,0.5]\emph{H})>[1,0.7](-[6,0.5]OH)-[3,0.7]O-[4]?}}{Deoxyribose}
                \schemestop{}
        \end{itemize}
    \item Number of \yyy{adenines} = \xxx{thymines}. (A-T)
    \item Number of \yyy{guanines} = \xxx{cytosines}. (C-G)
        \begin{itemize}
            \item Bonds between bases are \emph{noncovalent} (no electron sharing, weak).
            \item C---G pairs form three hydrogen bonds, while A---T forms two; making G---C slightly more stable.
            
            \scriptsize
            \yyy{\chemname%
            {\chemfig{@{H1}H-[:180]N(-[:-120]H)-[:120]*6(-N(-@{H2}H)-(=@{O3}O)-(*5(-N=-N(-R)-))=-N=)}}
            {\normalsize Guanine}}
          \qquad
          \xxx{\chemname%
            {\chemfig{@{O1}O=[:60]*6(-N(-R)-=-(-N(-[::60]@{H3}H)-[::-60]H)=@{N2}N-)}}
            {\normalsize Cytosine}}
          \chemmove[dashed]{\draw (H1)--(O1) (H2)--(N2) (O3)--(H3) ;}
          \hspace{18pt}
          \yyy{\chemname%
            {\chemfig{-[:120,,,,back]*6(-N(-@{H2}H)=(-NH-[:0]@{H4}H)-(*5(-N=-N(-R)-))=-N=)}}
            {\normalsize Adenine}}
          \qquad
          \xxx{\chemname%
            {\chemfig{O=[:60]*6(-N(-R)-=(-)-(=@{O}O)-@{N2}N-)}}
            {\normalsize Thymine}}
          \chemmove[dashed]{\draw (H2)--(N2) (H4)--(O) ;}
        \end{itemize} 
\end{itemize}


%\endgroup
%%%%%%%%%%% DNA Structure and Function %%%%%%%%%%%

\end{document}