\documentclass[basic]{inVerba-notes}

\definecolor{title-color}{HTML}{f5769c}
\newcommand{\theTitle}{\href{https://github.com/cullyn-inverba/notes/tree/master/bi-455}{Amyloid Staining Worksheet}}
\newcommand{\userName}{Cullyn Newman}
\newcommand{\class}{BI:\@ 455}
\newcommand{\institution}{Portland State University}

\begin{document}

\begin{center}
  \Image{1\columnwidth}{images/amyloid.png}
\end{center}

\section{Observations}
I believe this image is from a nephron, the functional unit of the kidney that is responsible for filtering using renal corpuscles and long renal tubules through which filtrate passes through.
\begin{itemize}
  
  \item \jjj{Renal corpuscle}: spherical structures
  with an average diameter of 200 µm
  distributed throughout the cortex.
  
  \begin{center}
    \Image{0.7\columnwidth}{images/amyloid-r.jpg}
  \end{center}

  \item \jjj{Glomerulus}: blood flowing through a capillary network (or tuft) undergoes filtration to produce the ultrafiltrate. 
  
  \begin{center}
    \Image{0.7\columnwidth}{images/amyloid-g.jpg}
  \end{center}

  \item \jjj{Bowman's capsule (light blue)}: encloses the glomerulus.
  
  \begin{center}
    \Image{0.7\columnwidth}{images/amyloid-b.jpg}
  \end{center}
  
  \begin{multicols}{3}
    \item \jjj{Vascular pole}:  where the afferent and efferent arterioles enter and exit the glomerulus.
  
    \begin{center}
      \Image{0.94\columnwidth}{images/amyloid-v.jpg}
    \end{center}
  \item \jjj{Mucosa densa(s)?}: % chktex 36
  
  \begin{center}
    \Image{0.8\columnwidth}{images/amyloid-mcd.jpg}
    \Image{0.8\columnwidth}{images/amyloid-mcd2.jpg}
  \end{center}
  
  \item \jjj{Collecting ducts}:
  
  \vspace{20pt}

  \begin{center}
    \Image{0.94\columnwidth}{images/amyloid-cd.jpg}
  \end{center}
  \end{multicols}
  
\end{itemize}

\newpage
\section{Questions}
\minor{Compare your stained tissue above to kidney tissue in your lab manual.}
\begin{itemize}
  \minor{\item In what ways is your stained slide different from the digital slides and images?}
  \begin{itemize}
    \item Well, obviously the slide is stained much differently using congo red dye, but that's a given.
    \item Here I'm assuming the red-ish zones in the glomerulus is the accumulating abnormal protein termed amyloid.
    \item The surrounding cortex has much more visible nuclei, which I assume are the purple dots, though it's hard to tell for certain as the image is low quality.
    \item I'm curious to what the very light, cyan colored regions are inside and outside the renal corpuscle. 
    \item The renal corpuscles are not as visible in the slides provided in the worksheet, but I'm comparing to histologyguide's slides, so it's a bit more useful. 
  \end{itemize}
  \minor{\item If you were a pathologist assessing the health of the kidney of the individual who was the source of your stained tissue, what might you note in your report to the physician treating this patient?}
  \begin{itemize}
    \item I say that the patient has amyloidosis due to apparent build up of the protein and that the patient might have a threat of kidney failure.
    \item I'd warn that and that the kidney might have a harder time breaking down wastes and proteins and to be mindful of what is being ingested.
    \item Other than that, I have no idea. I'm really confused by this assignment and the resources to do a good job on it are super lacking. 
  \end{itemize}
\end{itemize}

\end{document}