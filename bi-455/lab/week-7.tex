\chapter{Week 7: Respiratory, Integument, Urinary}

\section{Respiratory}

\begin{center}
  \Image{0.94\columnwidth}{images/week-7-a1.jpg}
\end{center}


\subsection{Nasal and Oral Cavities}
The nasal cavities provide an extensive surface area for removing debris, warming, and humidifying the air. The nasal and oral cavities are separated by the hard and soft palate.
\begin{multicols}{2}
\begin{itemize}
  \item \jjj{Respiratory epithelium}: pseudostratified epithelium with cilia and goblet cells lines the nasal cavity.
  
  \begin{center}
    \Image{0.8\columnwidth}{images/week-7-a3.jpg}
  \end{center}
  
  \item \jjj{Nasal concahe}:  long, narrow and curled bone that protrudes into the nasal cavity.
  
  \begin{center}
    \Image{0.8\columnwidth}{images/week-7-a4.jpg}
  \end{center}
  
  \item \jjj{Nasal Sero-Mucous glands}: 
  
  \begin{center}
    \Image{0.8\columnwidth}{images/week-7-a5.jpg}
  \end{center}
  
  \item \jjj{Nasal pharynx}: nasal portion of the pharynx.
  
  \begin{center}
    \Image{0.8\columnwidth}{images/week-7-a6.jpg}
  \end{center}
  
  \item \jjj{Palate}: roof of the mouth that separates the oral cavity from the nasal cavity. 
  
  \begin{center}
    \Image{0.8\columnwidth}{images/week-7-a7.jpg}
  \end{center}
  
  \item \jjj{Hard palate}: anterior, bony (rigid) portion.
  
  \begin{center}
    \Image{0.8\columnwidth}{images/week-7-a8.jpg}
  \end{center}

  \item \jjj{Soft palate}: anterior, bony (rigid) portion.
  
  \begin{center}
    \Image{0.7\columnwidth}{images/week-7-a9.jpg}
  \end{center}
  
  \item \jjj{Palatine mucous glands}:
  
  \begin{center}
    \Image{0.7\columnwidth}{images/week-7-a10.jpg}
  \end{center}

  \item \jjj{Tooth}:
  
  \begin{center}
    \Image{0.7\columnwidth}{images/week-7-a11.jpg}
  \end{center}
  
  \item \jjj{Tooth Bud}:
  
  \begin{center}
    \Image{0.7\columnwidth}{images/week-7-a12.jpg}
  \end{center}
  
\end{itemize}
\end{multicols}

\subsection{Larynx}
The larynx plays a critical role in speech.

\begin{center}
  \Image{0.5\columnwidth}{images/week-7-b.jpg}
\end{center}

\begin{multicols}{2}
\begin{itemize}
  \item \jjj{False vocal (ventricular) fold}:
  
  \begin{center}
    \Image{0.94\columnwidth}{images/week-7-b1.jpg}
  \end{center}
  
  \item \jjj{Respiratory epithelium}: pseudostratified columnar epithelium with cilia and goblet cells. 
  
  \begin{center}
    \Image{0.8\columnwidth}{images/week-7-b2.jpg}
  \end{center}
  
  \item \jjj{Pseudostratified columnar cells}: basal bodies visible as a dark line at the base of the cilia.
  
  \begin{center}
    \Image{0.8\columnwidth}{images/week-7-b3.jpg}
  \end{center}
  
  \item \jjj{Goblet cells}: basal bodies visible as a dark line at the base of the cilia.
  
  \begin{center}
    \Image{0.8\columnwidth}{images/week-7-b4.jpg}
  \end{center}
  
  \item \jjj{Basement membrane}: basal bodies visible as a dark line at the base of the cilia.
  
  \begin{center}
    \Image{0.8\columnwidth}{images/week-7-b5.jpg}
  \end{center}
  
  \item \jjj{Sero-Mucous glands}: add moisture to air and aid in trapping contaminants.
  
  \begin{center}
    \Image{0.94\columnwidth}{images/week-7-b6.jpg}
  \end{center}
  
  \item \jjj{Laryngeal ventricle}:  a lateral diverticulum that separates false folds above from true vocal cords below.
  
  \begin{center}
    \Image{0.94\columnwidth}{images/week-7-b7.jpg}
  \end{center}
  
  \item \jjj{True vocal cord}:
  
  \begin{center}
    \Image{0.94\columnwidth}{images/week-7-b8.jpg}
  \end{center}
  
  \item \jjj{Vocal ligament}:  thick band of connective tissue within the lamina propria near the surface of the vocal cord. 
  
  \begin{center}
    \Image{0.7\columnwidth}{images/week-7-b9.jpg}
  \end{center}
  
  \item \jjj{Stratified squamous non-keratinized epithelium}: covers this region of the larynx because it is subject to mechanical stress.
  
  \begin{center}
    \Image{0.7\columnwidth}{images/week-7-b10.jpg}
  \end{center}
  
  \item \jjj{Vocalis muscle}: skeletal muscle that underlies and regulates the tension of the vocal ligament. 
  
  \begin{center}
    \Image{0.94\columnwidth}{images/week-7-b11.jpg}
  \end{center}

  \item \jjj{Respiratory epithelium}: covers the true vocal cord except for the region that covers the vocal ligament.
  
  \begin{center}
    \Image{0.94\columnwidth}{images/week-7-b12.jpg}
  \end{center}
  
  \item \jjj{Sero-Mucous glands}: add moisture to air and aid in trapping contaminants.
  
  \begin{center}
    \Image{0.8\columnwidth}{images/week-7-b13.jpg}
  \end{center}
\end{itemize}
\end{multicols}

\subsection{Trachea}
The trachea (windpipe) is a fibromuscular tube supported by C-shaped rings of hyaline cartilage. It extends from the larynx toward the lungs.
\begin{center}
  \Image{0.8\columnwidth}{images/week-7-c.jpg}  
\end{center}
\begin{multicols}{2}
\begin{itemize}
  \item \jjj{Respiratory epithelium}:  the trachea is lined with a pseudostratified columnar epithelium with cilia and goblet cells. 
  
  \begin{center}
    \Image{0.75\columnwidth}{images/week-7-c1.jpg}
  \end{center}
  
  \item \jjj{Cilia}: extend 5 to 7 µm from the surface of the columnar epithelial cells. The dark line at their base is from their basal bodies.
  
  \begin{center}
    \Image{0.7\columnwidth}{images/week-7-c2.jpg}
  \end{center}
  
  \item \jjj{Goblet cells}: secrete mucus. They are difficult to identify in this specimen, but a thick layer of mucus (20 to 30 µm) is seen on the surface of the epithelium.
  
  \begin{center}
    \Image{0.7\columnwidth}{images/week-7-c3.jpg}
  \end{center}
  
  \item \jjj{Basement membrane}: separates the epithelium from the underlying connective tissue. It is seen as a thick, eosinophilic band beneath the epithelium.
  
  \begin{center}
    \Image{0.7\columnwidth}{images/week-7-c4.jpg}
  \end{center}
  
  \item \jjj{Lamina propria}: dense irregular connective tissue supports the epithelium. 
  
  \begin{center}
    \Image{0.7\columnwidth}{images/week-7-c5.jpg}
  \end{center}
  
  \item \jjj{Sero-Mucous glands}: add moisture to air and aid in trapping contaminants. The cilia propel mucus towards the esophagus where it is swallowed.
  
  \begin{center}
    \Image{0.8\columnwidth}{images/week-7-c6.jpg}
  \end{center}
  
  \item \jjj{Hyaline cartilage}: ``C''-shaped cartilage that is open in its posterior aspect.
  
  \begin{center}
    \Image{0.8\columnwidth}{images/week-7-c7.jpg}
  \end{center}
  
  \item \jjj{Trachealis muscle}: smooth muscle that spans the ends of tracheal cartilages. They control the diameter of the trachea.
  
  \begin{center}
    \Image{0.8\columnwidth}{images/week-7-c8.jpg}
  \end{center}
  
\end{itemize}
\end{multicols}

\vspace{30pt}

\subsection{Lung}
The lung consists of airways and structures for gas exchange.

The trachea divides into primary bronchi for each lung. They divide into secondary (lobar) bronchi and then into segmental (terminal) bronchi.

\begin{center}
  \Image{1\columnwidth}{images/week-7-d.jpg}
\end{center}
\begin{multicols}{2}
\begin{itemize}
  \item \jjj{Primary bronchi}:
  
  \begin{center}
    \Image{0.95\columnwidth}{images/week-7-d1.jpg}
    \Image{0.95\columnwidth}{images/week-7-d4.jpg}
  \end{center}

  \item \jjj{Respiratory epithelium}: composed of pseudostratified columnar epithelium.
  
  \begin{center}
    \Image{0.9\columnwidth}{images/week-7-d2.jpg}
  \end{center}
  
  \item \jjj{Pseudostratified columnar epithelium}:
  
  \begin{center}
    \Image{0.9\columnwidth}{images/week-7-d3.jpg}
  \end{center}
  
  \item \jjj{Lamina propria}:
  
  \begin{center}
    \Image{0.85\columnwidth}{images/week-7-d5.jpg}
  \end{center}
  
  \item \jjj{Bronchial cartilage}:
  
  \begin{center}
    \Image{0.94\columnwidth}{images/week-7-d6.jpg}
  \end{center}
  
  \item \jjj{Sero-Mucous glands}:
  
  \begin{center}
    \Image{0.94\columnwidth}{images/week-7-d8.jpg}
  \end{center}
  
  \item \jjj{Primary muscular bronchioles}:
  
  \begin{center}
    \Image{0.85\columnwidth}{images/week-7-d9.jpg}
  \end{center}
  
  \item \jjj{Epithelium}: changes from pseudostratified columnar to simple, ciliated columnar epithelium as they decrease in diameter.
  
  \begin{center}
    \Image{0.94\columnwidth}{images/week-7-d10.jpg}
  \end{center}
  
  \item \jjj{Terminal bronchioles}: conducting airways.
  
  \begin{center}
    \Image{0.94\columnwidth}{images/week-7-d11.jpg}
  \end{center}
  
  \item \jjj{Epithelium}: changes from simple, ciliated columnar epithelium to cuboidal epithelium.
  
  \begin{center}
    \Image{0.94\columnwidth}{images/week-7-d12.jpg}
  \end{center}
  
\end{itemize}
\end{multicols}


\newpage
\section{Integument}

\subsection{Thick Skin}
\begin{multicols}{2}
\begin{itemize}
  \item 
\end{itemize}
\end{multicols}

\subsection{Thin Skin}
\begin{multicols}{2}
\begin{itemize}
  \item 
\end{itemize}
\end{multicols}

\subsection{Pigmented Skin}
\begin{multicols}{2}
\begin{itemize}
  \item 
\end{itemize}
\end{multicols}

\subsection{Hair Follicles}
\begin{multicols}{2}
\begin{itemize}
  \item 
\end{itemize}
\end{multicols}

\subsection{Meissner and Pacinian Corpuscles}
\begin{multicols}{2}
\begin{itemize}
  \item 
\end{itemize}
\end{multicols}

\subsection{Questions}
\begin{itemize}
  \minor{\item What are the cell junctions in the stratum spinosum?}
  \minor{\item Where are the melanocytes located?}
\end{itemize}



\newpage
\section{Urinary}

\subsection{Kidney}
\begin{multicols}{2}
\begin{itemize}
  \item 
\end{itemize}
\end{multicols}

\subsection{Ureter}
\begin{multicols}{2}
\begin{itemize}
  \item 
\end{itemize}
\end{multicols}

\subsection{Urinary Bladder}
\begin{multicols}{2}
\begin{itemize}
  \item 
\end{itemize}
\end{multicols}

\subsection{Questions}
\begin{itemize}
  \minor{\item What are the components of the blood-urinary barrier in the glomerulus?}
\end{itemize}