\chapter{Week 7: Respiratory, Integument, Urinary}

\section{Respiratory}

\begin{center}
  \Image{0.94\columnwidth}{images/week-7-a1.jpg}
\end{center}


\subsection{Nasal and Oral Cavities}
The nasal cavities provide an extensive surface area for removing debris, warming, and humidifying the air. The nasal and oral cavities are separated by the hard and soft palate.
\begin{multicols}{2}
\begin{itemize}
  \item \jjj{Respiratory epithelium}: pseudostratified epithelium with cilia and goblet cells lines the nasal cavity.
  
  \begin{center}
    \Image{0.8\columnwidth}{images/week-7-a3.jpg}
  \end{center}
  
  \item \jjj{Nasal concahe}:  long, narrow and curled bone that protrudes into the nasal cavity.
  
  \begin{center}
    \Image{0.8\columnwidth}{images/week-7-a4.jpg}
  \end{center}
  
  \item \jjj{Nasal Sero-Mucous glands}: 
  
  \begin{center}
    \Image{0.8\columnwidth}{images/week-7-a5.jpg}
  \end{center}
  
  \item \jjj{Nasal pharynx}: nasal portion of the pharynx.
  
  \begin{center}
    \Image{0.8\columnwidth}{images/week-7-a6.jpg}
  \end{center}
  
  \item \jjj{Palate}: roof of the mouth that separates the oral cavity from the nasal cavity. 
  
  \begin{center}
    \Image{0.8\columnwidth}{images/week-7-a7.jpg}
  \end{center}
  
  \item \jjj{Hard palate}: anterior, bony (rigid) portion.
  
  \begin{center}
    \Image{0.8\columnwidth}{images/week-7-a8.jpg}
  \end{center}

  \item \jjj{Soft palate}: anterior, bony (rigid) portion.
  
  \begin{center}
    \Image{0.7\columnwidth}{images/week-7-a9.jpg}
  \end{center}
  
  \item \jjj{Palatine mucous glands}:
  
  \begin{center}
    \Image{0.7\columnwidth}{images/week-7-a10.jpg}
  \end{center}

  \item \jjj{Tooth}:
  
  \begin{center}
    \Image{0.7\columnwidth}{images/week-7-a11.jpg}
  \end{center}
  
  \item \jjj{Tooth Bud}:
  
  \begin{center}
    \Image{0.7\columnwidth}{images/week-7-a12.jpg}
  \end{center}
  
\end{itemize}
\end{multicols}

\subsection{Larynx}
The larynx plays a critical role in speech.

\begin{center}
  \Image{0.5\columnwidth}{images/week-7-b.jpg}
\end{center}

\begin{multicols}{2}
\begin{itemize}
  \item \jjj{False vocal (ventricular) fold}:
  
  \begin{center}
    \Image{0.94\columnwidth}{images/week-7-b1.jpg}
  \end{center}
  
  \item \jjj{Respiratory epithelium}: pseudostratified columnar epithelium with cilia and goblet cells. 
  
  \begin{center}
    \Image{0.8\columnwidth}{images/week-7-b2.jpg}
  \end{center}
  
  \item \jjj{Pseudostratified columnar cells}: basal bodies visible as a dark line at the base of the cilia.
  
  \begin{center}
    \Image{0.8\columnwidth}{images/week-7-b3.jpg}
  \end{center}
  
  \item \jjj{Goblet cells}: basal bodies visible as a dark line at the base of the cilia.
  
  \begin{center}
    \Image{0.8\columnwidth}{images/week-7-b4.jpg}
  \end{center}
  
  \item \jjj{Basement membrane}: basal bodies visible as a dark line at the base of the cilia.
  
  \begin{center}
    \Image{0.8\columnwidth}{images/week-7-b5.jpg}
  \end{center}
  
  \item \jjj{Sero-Mucous glands}: add moisture to air and aid in trapping contaminants.
  
  \begin{center}
    \Image{0.94\columnwidth}{images/week-7-b6.jpg}
  \end{center}
  
  \item \jjj{Laryngeal ventricle}:  a lateral diverticulum that separates false folds above from true vocal cords below.
  
  \begin{center}
    \Image{0.94\columnwidth}{images/week-7-b7.jpg}
  \end{center}
  
  \item \jjj{True vocal cord}:
  
  \begin{center}
    \Image{0.94\columnwidth}{images/week-7-b8.jpg}
  \end{center}
  
  \item \jjj{Vocal ligament}:  thick band of connective tissue within the lamina propria near the surface of the vocal cord. 
  
  \begin{center}
    \Image{0.7\columnwidth}{images/week-7-b9.jpg}
  \end{center}
  
  \item \jjj{Stratified squamous non-keratinized epithelium}: covers this region of the larynx because it is subject to mechanical stress.
  
  \begin{center}
    \Image{0.7\columnwidth}{images/week-7-b10.jpg}
  \end{center}
  
  \item \jjj{Vocalis muscle}: skeletal muscle that underlies and regulates the tension of the vocal ligament. 
  
  \begin{center}
    \Image{0.94\columnwidth}{images/week-7-b11.jpg}
  \end{center}

  \item \jjj{Respiratory epithelium}: covers the true vocal cord except for the region that covers the vocal ligament.
  
  \begin{center}
    \Image{0.94\columnwidth}{images/week-7-b12.jpg}
  \end{center}
  
  \item \jjj{Sero-Mucous glands}: add moisture to air and aid in trapping contaminants.
  
  \begin{center}
    \Image{0.8\columnwidth}{images/week-7-b13.jpg}
  \end{center}
\end{itemize}
\end{multicols}

\subsection{Trachea}
The trachea (windpipe) is a fibromuscular tube supported by C-shaped rings of hyaline cartilage. It extends from the larynx toward the lungs.
\begin{center}
  \Image{0.8\columnwidth}{images/week-7-c.jpg}  
\end{center}
\begin{multicols}{2}
\begin{itemize}
  \item \jjj{Respiratory epithelium}:  the trachea is lined with a pseudostratified columnar epithelium with cilia and goblet cells. 
  
  \begin{center}
    \Image{0.75\columnwidth}{images/week-7-c1.jpg}
  \end{center}
  
  \item \jjj{Cilia}: extend 5 to 7 µm from the surface of the columnar epithelial cells. The dark line at their base is from their basal bodies.
  
  \begin{center}
    \Image{0.7\columnwidth}{images/week-7-c2.jpg}
  \end{center}
  
  \item \jjj{Goblet cells}: secrete mucus. They are difficult to identify in this specimen, but a thick layer of mucus (20 to 30 µm) is seen on the surface of the epithelium.
  
  \begin{center}
    \Image{0.7\columnwidth}{images/week-7-c3.jpg}
  \end{center}
  
  \item \jjj{Basement membrane}: separates the epithelium from the underlying connective tissue. It is seen as a thick, eosinophilic band beneath the epithelium.
  
  \begin{center}
    \Image{0.7\columnwidth}{images/week-7-c4.jpg}
  \end{center}
  
  \item \jjj{Lamina propria}: dense irregular connective tissue supports the epithelium. 
  
  \begin{center}
    \Image{0.7\columnwidth}{images/week-7-c5.jpg}
  \end{center}
  
  \item \jjj{Sero-Mucous glands}: add moisture to air and aid in trapping contaminants. The cilia propel mucus towards the esophagus where it is swallowed.
  
  \begin{center}
    \Image{0.8\columnwidth}{images/week-7-c6.jpg}
  \end{center}
  
  \item \jjj{Hyaline cartilage}: ``C''-shaped cartilage that is open in its posterior aspect.
  
  \begin{center}
    \Image{0.8\columnwidth}{images/week-7-c7.jpg}
  \end{center}
  
  \item \jjj{Trachealis muscle}: smooth muscle that spans the ends of tracheal cartilages. They control the diameter of the trachea.
  
  \begin{center}
    \Image{0.8\columnwidth}{images/week-7-c8.jpg}
  \end{center}
  
\end{itemize}
\end{multicols}

\vspace{30pt}

\subsection{Lung}
The lung consists of airways and structures for gas exchange.

The trachea divides into primary bronchi for each lung. They divide into secondary (lobar) bronchi and then into segmental (terminal) bronchi.

\begin{center}
  \Image{1\columnwidth}{images/week-7-d.jpg}
\end{center}
\begin{multicols}{2}
\begin{itemize}
  \item \jjj{Primary bronchi}:
  
  \begin{center}
    \Image{0.95\columnwidth}{images/week-7-d1.jpg}
    \Image{0.95\columnwidth}{images/week-7-d4.jpg}
  \end{center}

  \item \jjj{Respiratory epithelium}: composed of pseudostratified columnar epithelium.
  
  \begin{center}
    \Image{0.9\columnwidth}{images/week-7-d2.jpg}
  \end{center}
  
  \item \jjj{Pseudostratified columnar epithelium}:
  
  \begin{center}
    \Image{0.9\columnwidth}{images/week-7-d3.jpg}
  \end{center}
  
  \item \jjj{Lamina propria}:
  
  \begin{center}
    \Image{0.85\columnwidth}{images/week-7-d5.jpg}
  \end{center}
  
  \item \jjj{Bronchial cartilage}:
  
  \begin{center}
    \Image{0.94\columnwidth}{images/week-7-d6.jpg}
  \end{center}
  
  \item \jjj{Sero-Mucous glands}:
  
  \begin{center}
    \Image{0.94\columnwidth}{images/week-7-d8.jpg}
  \end{center}
  
  \item \jjj{Primary muscular bronchioles}:
  
  \begin{center}
    \Image{0.85\columnwidth}{images/week-7-d9.jpg}
  \end{center}
  
  \item \jjj{Epithelium}: changes from pseudostratified columnar to simple, ciliated columnar epithelium as they decrease in diameter.
  
  \begin{center}
    \Image{0.94\columnwidth}{images/week-7-d10.jpg}
  \end{center}
  
  \item \jjj{Terminal bronchioles}: conducting airways.
  
  \begin{center}
    \Image{0.94\columnwidth}{images/week-7-d11.jpg}
  \end{center}
  
  \item \jjj{Epithelium}: changes from simple, ciliated columnar epithelium to cuboidal epithelium.
  
  \begin{center}
    \Image{0.94\columnwidth}{images/week-7-d12.jpg}
  \end{center}
  
\end{itemize}
\end{multicols}


\newpage
\section{Integument}

\subsection{Thick Skin}
Thick skin is only found on the palms of the hands, and the soles of the feet, locations subjected to considerable abrasion. It has a thick epidermis and contains sweat glands, but lacks hair follicles and sebaceous glands.
\begin{center}
  \Image{0.9\columnwidth}{images/week-7-e.jpg}
\end{center}
\begin{multicols}{2}
\begin{itemize}
  \item \jjj{Epidermis}: stratified squamous keratinized epithelium divided into five strata (or layers). 
  
  \begin{center}
    \Image{0.94\columnwidth}{images/week-7-e1.jpg}
  \end{center}
  
  \item \jjj{Stratum basale}: single layer of germinal cells resting on the basement membrane which is attached to the dermis.
  
  \begin{center}
    \Image{0.94\columnwidth}{images/week-7-e2.jpg}
  \end{center}
  
  \item \jjj{Stratum spinosum}: keratinocytes attached to each other by desmosomes on spiny processes.
  
  \begin{center}
    \Image{0.94\columnwidth}{images/week-7-e3.jpg}
  \end{center}
  
  \item \jjj{Stratus granulosum}: keratinocytes with numerous basophilic, keratohyalin granules in their cytoplasm.
  
  \begin{center}
    \Image{0.8\columnwidth}{images/week-7-e4.jpg}
  \end{center}
  
  \item \jjj{Stratum lucidum}: highly refractive zone only seen in very thick skin.
  
  \begin{center}
    \Image{0.8\columnwidth}{images/week-7-e5.jpg}
  \end{center}
  
  \item \jjj{Stratum Corneum}: - thick layer of dead cells (squames) devoid of nuclei and organelles.
  
  \begin{center}
    \Image{0.8\columnwidth}{images/week-7-e6.jpg}
  \end{center}
  
  \item \jjj{Dermis}:  dense irregular connective tissue that supports the epidermis. 
  
  \begin{center}
    \Image{0.85\columnwidth}{images/week-7-e7.jpg}
  \end{center}
  
  \item \jjj{Papillary layer}: papillae that project into the dermis. 
  
  \begin{center}
    \Image{0.85\columnwidth}{images/week-7-e8.jpg}
  \end{center}
  
  \item \jjj{Dermal papillae}:  increase adhesion between the epidermis and dermis.
  
  \begin{center}
    \Image{0.94\columnwidth}{images/week-7-e9.jpg}
    \Image{0.94\columnwidth}{images/week-7-e10.jpg}
  \end{center}
  
  \item \jjj{Papillary capillaries}: bring nutrients to the epidermis.
  
  \begin{center}
    \Image{0.94\columnwidth}{images/week-7-e11.jpg}
  \end{center}
  
  \item \jjj{Reticular layer}: dense irregular connective. 
  
  \begin{center}
    \Image{0.94\columnwidth}{images/week-7-e12.jpg}
  \end{center}
  
  \item \jjj{Eccrine sweat glands}: coiled tubular gland with simple or stratified cuboidal epithelium (lightly stained) and duct cells (dark staining).
  
  \begin{center}
    \Image{0.94\columnwidth}{images/week-7-e13.jpg}
  \end{center}
  
  \item \jjj{Hypodermis}: loose connective tissue with adipose tissue.
  
  \begin{center}
    \Image{0.94\columnwidth}{images/week-7-e14.jpg}
  \end{center}
\end{itemize}
\end{multicols}

\newpage
\subsection{Thin Skin}
Thin skin (1 to 2 mm) covers most of the body, whereas thick skin is restricted to the palms of the hands and soles of the feet.

The keratin layers often become dislodged during preparation of thin skin. In the natural state, the keratin layers would be attached to the underlying layers. The thickness of the stratum corneum is less than the cellular layers.
\begin{center}
  \Image{0.94\columnwidth}{images/week-7-f.jpg}
\end{center}
\begin{multicols}{2}
\begin{itemize}
  \item \jjj{Epidermis}: stratified squamous keratinized epithelium divided into four strata (or layers). 
  
  \begin{center}
    \Image{0.8\columnwidth}{images/week-7-f1.jpg}
  \end{center}
  
  \item \jjj{Stratum basale}: single layer of germinal cells resting on the basement membrane which is attached to the dermis.
  
  \begin{center}
    \Image{0.8\columnwidth}{images/week-7-f2.jpg}
  \end{center}
  
  \item \jjj{Stratum spinosum}: keratinocytes attached to each other by desmosomes on spiny processes.
  
  \begin{center}
    \Image{0.7\columnwidth}{images/week-7-f3.jpg}
  \end{center}
  
  \item \jjj{Stratum granulosum}:  keratinocytes with numerous basophilic granules in their cytoplasm.
  
  \begin{center}
    \Image{0.8\columnwidth}{images/week-7-f4.jpg}
  \end{center}
  
  \item \jjj{Stratum corneum}: thin layer of dead cells devoid of nuclei and organelles.
   
  \begin{center}
    \Image{0.94\columnwidth}{images/week-7-f5.jpg}
  \end{center}
  
  \item \jjj{Dermis}:  dense irregular connective tissue that supports the epidermis. 
  
  \begin{center}
    \Image{0.94\columnwidth}{images/week-7-f6.jpg}
  \end{center}
  
  \item \jjj{Dermal papillae}: less prominent than in thick skin. They increase adhesion between the epidermis and dermis.
  
  \begin{center}
    \Image{0.94\columnwidth}{images/week-7-f7.jpg}
    \Image{0.94\columnwidth}{images/week-7-f10.jpg}
  \end{center}
  
  \item \jjj{Eccrine sweat glands}: coiled tubular glands (lightly stained) and ducts (dark stained) with simple or stratified cuboidal epithelium.
  
  \begin{center}
    \Image{0.94\columnwidth}{images/week-7-f8.jpg}
  \end{center}
  
  \item \jjj{Hair follicle}:  thin skin has hair follicles.
  
  \begin{center}
    \Image{0.94\columnwidth}{images/week-7-f9.jpg}
  \end{center}
  
\end{itemize}
\end{multicols}

\subsection{Pigmented Skin}
Thick and thin skin from a dark skinned individual.
\begin{center}
  \Image{0.75\columnwidth}{images/week-7-g.jpg}
\end{center}
\begin{multicols}{2}
\begin{itemize}
  \item \jjj{Thin skin}: covers most of the body and is heavily pigmented.
  
  \begin{center}
    \Image{0.75\columnwidth}{images/week-7-g1.jpg}
  \end{center}
  
  \item \jjj{Melanocytes}: melanin-producing cells located in the stratum basale. 
  
  \begin{center}
    \Image{0.75\columnwidth}{images/week-7-g2.jpg}
  \end{center}
  
  \item \jjj{Stratum basale}: heavily pigmented with dark brown granules of melanin.
  
  \begin{center}
    \Image{0.8\columnwidth}{images/week-7-g4.jpg}
  \end{center}
   
  \item \jjj{Stratum spinosum}: pigment is present in all layers of the epidermis.
  
  \begin{center}
    \Image{0.8\columnwidth}{images/week-7-g5.jpg}
  \end{center}
  
  \item \jjj{Thick skin}:  covers the palms of the hands and soles of the feet and is lightly pigmented.
  
  \begin{center}
    \Image{0.94\columnwidth}{images/week-7-g6.jpg}
  \end{center}
  
  \item \jjj{Melanocytes}:  covers the palms of the hands and soles of the feet and is lightly pigmented.
  
  \begin{center}
    \Image{0.94\columnwidth}{images/week-7-g7.jpg}
  \end{center}
  
  \item \jjj{Stratum basale}: lightly pigmented with dark brown granules of melanin.
  
  \begin{center}
    \Image{0.94\columnwidth}{images/week-7-g8.jpg}
  \end{center}
  
  \item \jjj{Sweat gland}: coiled tubular gland with simple or stratified cuboidal epithelium (lightly stained) and duct cells (dark staining).
  
  \begin{center}
    \Image{0.94\columnwidth}{images/week-7-g9.jpg}
  \end{center}
  
  \item \jjj{Hair follicle and sebaceous gland}: common in thin skin. The sebaceceus glands are large cells with a central nuclei and foamy cytoplasm. These cells produce an oily, waxy substance called sebum that is released onto the surface of the skin.
  
  \begin{center}
    \Image{0.94\columnwidth}{images/week-7-g10.jpg}
  \end{center}
    
\end{itemize}
\end{multicols}


\subsection{Questions}
\begin{itemize}
  \minor{\item What are the cell junctions in the stratum spinosum?} --- Desmosomes, with spiky membrane projections.
  \minor{\item Where are the melanocytes located?} --- Stratum basale.
\end{itemize}

\newpage
\subsection{Meissner and Pacinian Corpuscles}
Meissner and Pacinian corpuscles are two types of touch/pressure receptors that are found in skin.
\begin{multicols}{2}
\begin{itemize}
  \item \jjj{Meissner corpuscles}: nerve endings in skin responsible for sensitivity to light touch.
  
  \begin{center}
    \Image{0.89\columnwidth}{images/week-7-h.jpg}
    \Image{0.89\columnwidth}{images/week-7-h1.jpg}
    \Image{0.89\columnwidth}{images/week-7-h3.jpg}
  \end{center}
  
  \item \jjj{Pacinian corpuscles}:  nerve endings in skin responsible for sensitivity to vibration and pressure.
  
  \begin{center}
    \Image{0.8\columnwidth}{images/week-7-h5.jpg}
    \Image{0.8\columnwidth}{images/week-7-h6.jpg}
  \end{center}

  \item \jjj{Inner bulb}: an unmyelinated axon within a fluid-filled cavity formed by several lamellae of Schwann cells. 
  
  \begin{center}
    \Image{0.8\columnwidth}{images/week-7-h8.jpg}
  \end{center}
  
\end{itemize}
\end{multicols}

\newpage
\section{Urinary}

\subsection{Kidney}
Kidneys filter blood and produce urine. Unlike the human kidney which is multilobed (10 to 12 lobes) separated by renal columns (cortical tissue that extends alongside the margin of pyramids in the medulla), the monkey kidney is unilobular.
\begin{center}
  \Image{0.94\columnwidth}{images/week-7-i.jpg}
\end{center}

\begin{multicols}{2}
\begin{itemize}
  \item \jjj{Cortex}: darker outer region. 
  
  \begin{center}
    \Image{0.85\columnwidth}{images/week-7-i1.jpg}
  \end{center}
  
  \item \jjj{Renal corpuscles}: spherical structures that form ultrafiltrate from blood.
  
  \begin{center}
    \Image{0.85\columnwidth}{images/week-7-i2.jpg}
  \end{center}
  
  \item \jjj{Cortical labyrinths}:  regions between renal corpuscles and medullary rays that contain proximal and distal convoluted tubules.
  
  \begin{center}
    \Image{0.92\columnwidth}{images/week-7-i3.jpg}
  \end{center}
  
  \item \jjj{Medullary rays}: projections of tubules between the cortex and medulla that contains straight tubules and collecting ducts.
  
  \begin{center}
    \Image{0.94\columnwidth}{images/week-7-i4.jpg}
    \Image{0.94\columnwidth}{images/week-7-i44.jpg}
  \end{center}
  
  \item \jjj{Medulla}: lighter inner region. 
  
  \begin{center}
    \Image{0.82\columnwidth}{images/week-7-i5.jpg}
  \end{center}
  
  \item \jjj{Renal papilla}
  
  \begin{center}
    \Image{0.82\columnwidth}{images/week-7-i6.jpg}
  \end{center}
  
  \item \jjj{Renal pelvis}: funnel-shaped origin of the ureter.
  
  \begin{center}
    \Image{0.82\columnwidth}{images/week-7-i7.jpg}
  \end{center}
  
  \item \jjj{Arcuate arteries}:  branches of interlobular arteries that form an arcade over the pyramids at the junction of the cortex and medulla.
  
  \begin{center}
    \Image{0.7\columnwidth}{images/week-7-i8.jpg}
  \end{center}
  
  \item \jjj{Hilum}: concave surface with a deep fissure in which vessels enter and exit the kidney.
  
  \begin{center}
    \Image{0.94\columnwidth}{images/week-7-i10.jpg}
  \end{center}
\end{itemize}
\end{multicols}

\subsection{Nephron}
The nephron is the functional unit of the kidney. Each nephron includes a filter (renal corpuscle), and a single, long tubule (renal tubule) through which the filtrate passes before emerging as urine.

\begin{center}
  \Image{0.84\columnwidth}{images/week-7-k.jpg}
\end{center}

\begin{multicols}{2}
\begin{itemize}
  \item \jjj{Renal corpuscle}: spherical structures with an average diameter of 200 µm distributed throughout the cortex. 
  
  \begin{center}
    \Image{0.8\columnwidth}{images/week-7-k1.jpg}
  \end{center}
  
  \item \jjj{Bowman's capsule}: encloses the glomerulus.
  
  \begin{center}
    \Image{0.8\columnwidth}{images/week-7-k2.jpg}
  \end{center}
  
  \item \jjj{Glomerulus}:  blood flowing through a capillary network (or tuft) undergoes filtration to produce the ultrafiltrate. 
  
  \begin{center}
    \Image{0.9\columnwidth}{images/week-7-k3.jpg}
  \end{center}
  
  \item \jjj{Vascular pole}:
  
  \begin{center}
    \Image{0.65\columnwidth}{images/week-7-k4.jpg}
  \end{center}
  
  \item \jjj{Urinary pole}:
  
  \begin{center}
    \Image{0.65\columnwidth}{images/week-7-k5.jpg}
  \end{center}
  
  \item \jjj{Collecting ducts}: convey urine from nephrons to collecting ducts within medullary rays.
  
  \begin{center}
    \Image{0.65\columnwidth}{images/week-7-k6.jpg}
  \end{center}
  
  \item \jjj{Macula densa}: tightly packed cells where the distal straight tubule contacts the afferent arteriole of the vascular pole of the renal corpuscle.
  
  \begin{center}
    \Image{0.65\columnwidth}{images/week-7-k7.jpg}
  \end{center}

\end{itemize}
\end{multicols}

\subsection{Ureter}
Ureter transport urine from the kidney to the bladder. It is lined with an epithelium that is impermeable to water and ions. Peristaltic contraction of the smooth muscle moves urine from the kidney to the bladder.

\begin{center}
  \Image{0.94\columnwidth}{images/week-7-l.jpg}
\end{center}
\begin{multicols}{2}
\begin{itemize}
  \item \jjj{Cross-section}: composed of four concentric layers.
  
  \begin{center}
    \Image{0.85\columnwidth}{images/week-7-l1.jpg}
  \end{center}
  
  \item \jjj{Transitional epithelium (Urothelium)}: consists of two to three cell layers in the upper ureter with up to ten cell layers near the bladder 
  
  \begin{center}
    \Image{0.85\columnwidth}{images/week-7-l2.jpg}
  \end{center}
  
  \item \jjj{Umbrella cells}: upper layer of cells that change shape depending on the distention of the ureter (relaxed)
  
  \begin{center}
    \Image{0.7\columnwidth}{images/week-7-l3.jpg}
  \end{center}
  
  \item \jjj{Lamina propria}: thick layer of dense irregular connective tissue rich in collagen and elastic fibers 
  
  \begin{center}
    \Image{0.7\columnwidth}{images/week-7-l4.jpg}
  \end{center}
  
  \item \jjj{Muscularis externa}: irregular arrangement of smooth muscle in two layers (inner longitudinal and outer circular) in the upper ureter or three layers (inner longitudinal, middle circular and outer longitudinal) near the bladder.
  
  \begin{center}
    \Image{0.8\columnwidth}{images/week-7-l5.jpg}
  \end{center}
  
  \item \jjj{Longitudinal section}:
  
  \begin{center}
    \Image{0.8\columnwidth}{images/week-7-l6.jpg}
  \end{center}
  
  \item \jjj{Transitional epithelium}:
  
  \begin{center}
    \Image{0.8\columnwidth}{images/week-7-l7.jpg}
  \end{center}
  
  \item \jjj{Umbrella cells}:
  
  \begin{center}
    \Image{0.92\columnwidth}{images/week-7-l8.jpg}
  \end{center}
  
  \item \jjj{Lamina propria}:
  
  \begin{center}
    \Image{0.92\columnwidth}{images/week-7-l9.jpg}
  \end{center}
  
  \item \jjj{Muscularis externa}:
  
  \begin{center}
    \Image{0.92\columnwidth}{images/week-7-l10.jpg}
  \end{center}
  
\end{itemize}
\end{multicols}

\newpage

\subsection{Urinary Bladder}
Bladder is an expandable vessel for the storage of urine. It is lined with an epithelium that is impermeable to water and ions.

Like the ureters, the bladder is composed of four concentric layers.

\begin{center}
  \Image{0.65\columnwidth}{images/week-7-m.jpg}
\end{center}

\begin{multicols}{2}
\begin{itemize}
  \item \jjj{Urothelium}: consists of two to three cell layers in the upper ureter with up to ten cell layers near the bladder. 
  
  \begin{center}
    \Image{0.8\columnwidth}{images/week-7-m1.jpg}
  \end{center}
  
  \item \jjj{Umbrella cells}: the upper layer of cells that change shape depending on the distention of the bladder. Umbrella cells are frequently binucleate. 
  
  \begin{center}
    \Image{0.8\columnwidth}{images/week-7-m2.jpg}
  \end{center}
  
  \item \jjj{Dome-shaped umbrella cells}: rounded and bulge from the surface of the epithelium.
  
  \begin{center}
    \Image{0.94\columnwidth}{images/week-7-m3.jpg}
  \end{center}
  
  \item \jjj{Flattened umbrella cells}: stretch over several underlying epithelial cells.
  
  \begin{center}
    \Image{0.94\columnwidth}{images/week-7-m4.jpg}
  \end{center}
  
  \item \jjj{Lamina propria}: thick layer of dense irregular connective tissue rich in collagen and elastic fibers.
  
  \begin{center}
    \Image{0.85\columnwidth}{images/week-7-m5.jpg}
  \end{center}
   
  \item \jjj{Muscularis externa}: loosely arranged smooth muscle in two layers (inner longitudinal and outer circular) in the upper ureter or three layers (inner longitudinal, middle circular and outer longitudinal) near the bladder.
  
  \begin{center}
    \Image{0.8\columnwidth}{images/week-7-m6.jpg}
  \end{center}
  
  \item \jjj{Adventitia}: loose connective tissue with blood vessels, nerves and adipose cells.
  
  \begin{center}
    \Image{0.8\columnwidth}{images/week-7-m7.jpg}
  \end{center}
  
\end{itemize}
\end{multicols}