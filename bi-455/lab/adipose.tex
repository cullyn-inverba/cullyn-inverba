\documentclass[basic]{inVerba-notes}

\definecolor{title-color}{HTML}{f5769c}
\newcommand{\theTitle}{\href{https://github.com/cullyn-inverba/notes/tree/master/bi-455}{Adipose Tissue Worksheet}}
\newcommand{\userName}{Cullyn Newman}
\newcommand{\class}{BI:\@ 455}
\newcommand{\institution}{Portland State University}

\begin{document}
\begin{enumerate}
  \item In what ways is the below image slide different from these digital images? Compare the image below to adipose tissue in: 
  
  Junqueira skin slide 36 \& Junqueira Aorta Slide 146
 
  with this image of white adipose tissue:
  
  \begin{center}
    \Image{0.9\columnwidth}{images/adipose-1.jpg}
  \end{center}
  
  \bigskip

  \begin{itemize}
    \item Slide 36 seems to have fewer nuclei, blood tissue, and other tissue nearby:
    
    \begin{center}
      \Image{0.7\columnwidth}{images/adipose-2.jpg}
    \end{center}

    \item Although, adipose in the skin does have places where there is more nearby; still less nuclei, but it appears to be in a small pocket, surrounding by, smooth muscle? I'm not sure. Is that a capillary next to it on the right? No, that's far too large, what is that? And there seems to be some brown adipocytes in the middle there as well.
    
    \begin{center}
      \Image{0.7\columnwidth}{images/adipose-3.jpg}
    \end{center}

    \item Slide 146 seems to have the adipocytes broken up? Perhaps that is just due to how the specimen was prepared though. You can see plenty of nearby blood cells, which makes sense since we are looking at part of the aorta.
    
    \begin{center}
      \Image{0.7\columnwidth}{images/adipose-4.jpg}
    \end{center}

    \item I'm not really sure what I'm looking at, but there seems to be a lot more of the white adipose tissue, and little brown adipose if any at all. 
    
    \begin{center}
      \Image{0.7\columnwidth}{images/adipose-5.jpg}
    \end{center}

    \item Here there is a lot of blood flowing through/next to some limited amounts of white adipose tissue. Plus, there are long, (multi-nuclei?) cells that I'm not quite sure of. 
    
    \begin{center}
      \Image{0.7\columnwidth}{images/adipose-6.jpg}
    \end{center}
    
  \end{itemize}

  \item Identify and label as many features as possible from first image of adipose tissue above. Provide details on how to identify structures based on specific visual characteristics. 
    \begin{itemize}
      \item Okay, plate 16 has a perfectly labeled image of the exact same one posted above. I would essentially just use that to recreate another labeled image, but that's truly a waste of time. So hopefully I can use the labeled image here and 
    \end{itemize}
  
    \begin{center}
      \Image{0.9\columnwidth}{images/adipose-7.jpg}
    \end{center}

    \bigskip

    \begin{itemize} 
      \item \jjj{A}: Adipocytes, which main object of interest here. They have a spherical profile with a white center, as the fat is lost during tissue preparation. 
      \item \jjj{Cy}: Cytoplasm, the thin line that lines the adipocytes, which is not lost during slide preparation. 
      \item \jjj{N}: Nucleus, small dark spot on the outer edge of the adipocytes.
      \item \jjj{Thin connective tissue (not labeled)}: between the adipocytes is some extremely thin and delicate connective tissue that holds together the adipocytes. Within this tissue you can find other structures such as blood vessels and mast cells.
      \item \jjj{BV}: Blood vessels, small purple structures that look much like nuclei, but are actually various capillaries and venules. These are found in the thin connective tissue, not within the cytoplasm like the nuclei are.
      \item \jjj{DICT}: Dense irregular connective tissue, slightly wavy, strikingly different (compared to other structures on the slide) connective tissue that separates the lobules containing the adipose tissues. 
      \item \jjj{MC}: Mast cells, which can be found within the delicate connective tissue stroma. Not many are present here, but the have many granules rich in histamine and heparin that lighter purple color that surrounds the nucleus.
    \end{itemize}

\end{enumerate}
\end{document}
