\chapter{Week 9: Female Reproductive, Eye, Ear}

\section{Female Reproductive System}

\subsection{Ovary}
The ovaries are responsible for the production of an oocyte (oogenesis) and secretion of female sex hormones (estrogen and progesterone). When it releases a mature ovum, it travels down the oviduct to the uterus.
\begin{center}
  \Image{0.8\columnwidth}{images/week-9-a.jpg}
\end{center}
\begin{multicols}{2}
\begin{itemize}
  \item \jjj{Capsule}: covers the outer surface. 
  
  \begin{center}
    \Image{0.7\columnwidth}{images/week-9-a1.jpg}
  \end{center}
   
  \item \jjj{Germinal epithelium}: the surface is covered by a simple cuboidal epithelium.
  
  \begin{center}
    \Image{0.7\columnwidth}{images/week-9-a2.jpg}
  \end{center}
  
  \item \jjj{Tunica albuginea}: layer of dense irregular connective tissue that supports the epithelium.
  \begin{center}
    \Image{0.7\columnwidth}{images/week-9-a3.jpg}
  \end{center}
  
  \item \jjj{Cortex}: outer region that is the site of oocyte development. 
  
  \begin{center}
    \Image{0.85\columnwidth}{images/week-9-a4.jpg}
  \end{center}
  
  \item \jjj{Primordial follicles}: oocytes arrested in development are located in the outer cortex. 
  
  \begin{center}
    \Image{0.75\columnwidth}{images/week-9-a5.jpg}
  \end{center}
  
  \item \jjj{Primary oocyte}: large (25 to 30 µm), round to oval cells with a vesicular nucleus.
  
  \begin{center}
    \Image{0.75\columnwidth}{images/week-9-a6.jpg}
  \end{center}
  
  \item \jjj{Zona pellucida}:  layer of glycoproteins between the oocyte and granulosa cells. It is visible a thin, eosinophilic band in many follicles.
  
  \begin{center}
    \Image{0.75\columnwidth}{images/week-9-a7.jpg}
  \end{center}
  
  \item \jjj{Follicular cells}: single layer of flattened cells that surround each oocyte.
  
  \begin{center}
    \Image{0.8\columnwidth}{images/week-9-a8.jpg}
  \end{center}
  
  \item \jjj{Unilaminar}: primary oocytes surrounded by a single layer of granulosa cells.
  
  \begin{center}
    \Image{0.8\columnwidth}{images/week-9-a9.jpg}
  \end{center}
  
  \item \jjj{Multilaminar}:  primary oocytes surrounded by multiple layers of granulosa cells.
  
  \begin{center}
    \Image{0.8\columnwidth}{images/week-9-a10.jpg}
  \end{center}
  
  \item \jjj{Secondary (Antral) Follicles}: characterized by the formation an antrum (a fluid-filled space) containing an oocyte. The antrum increases in size as the follicle matures 
  
  \begin{center}
    \Image{0.61\columnwidth}{images/week-9-a111.jpg}
    \Image{0.61\columnwidth}{images/week-9-a11.jpg}
    \Image{0.61\columnwidth}{images/week-9-a112.jpg}
    \Image{0.61\columnwidth}{images/week-9-a113.jpg}
  \end{center}
   
  \item \jjj{oocyte}: large (50 to 100 µm), round to oval cells with a vesicular nucleus.
  
  \begin{center}
    \Image{0.82\columnwidth}{images/week-9-a12.jpg}
  \end{center}
  
  \item \jjj{Zona pellucida}:  layer of glycoproteins between the oocyte and granulosa cells (eosinophilic).
  
  \begin{center}
    \Image{0.82\columnwidth}{images/week-9-a13.jpg}
  \end{center}
  
  \item \jjj{Corona radiata}: group of cells anchored to the follicle wall that contains the oocyte.
  
  \begin{center}
    \Image{0.82\columnwidth}{images/week-9-a14.jpg}
  \end{center}
  
  \item \jjj{Cumulus oophorus}: group of cells anchored to the follicle wall that contains the oocyte.
  
  \begin{center}
    \Image{0.7\columnwidth}{images/week-9-a15.jpg}
  \end{center}
  
  \item \jjj{Stratum granulosum}: multiple layers of cells that form the follicle wall that surrounds the antrum. 
  
  \begin{center}
    \Image{0.7\columnwidth}{images/week-9-a16.jpg}
  \end{center}
  
  \item \jjj{Theca folliculi}: stromal cells around the follicle develop into a sheath of highly vascularized connective tissue. 
  
  \begin{center}
    \Image{0.7\columnwidth}{images/week-9-a18.jpg}
  \end{center}
  
  \item \jjj{Mature (Graafian) follicles}: usually only one follicle will continue to grow each cycle to form a very large, mature follicle (25 mm or more in diameter). 
  
  \begin{center}
    \Image{0.8\columnwidth}{images/week-9-a17.jpg}
  \end{center}

  \item \jjj{Stratum granulosum}: becomes thinner as the follicle continues to grow in size.
  
  \begin{center}
    \Image{0.8\columnwidth}{images/week-9-a182.jpg}
  \end{center}
  
  \item \jjj{Theca folliculi}: becomes more organized and contains many blood vessels.
  
  \begin{center}
    \Image{0.8\columnwidth}{images/week-9-a19.jpg}
  \end{center}
  
  
  \item \jjj{Stroma}: highly cellular connective tissue with fewer connective tissue fibers in which ovarian follicles are located. 
  
  \begin{center}
    \Image{0.85\columnwidth}{images/week-9-a20.jpg}
  \end{center}
  
  \item \jjj{Medulla}: inner region of fibroelastic connective tissue with many large, tortuous blood vessels, lymph vessels and nerve fibers. 
  
  \begin{center}
    \Image{0.85\columnwidth}{images/week-9-a21.jpg}
  \end{center}
  
\end{itemize}
\end{multicols}

\subsection{Oviduct}
The oviducts (uterine tubes; fallopian tubes) are fibromuscular tubes that transport an ovulated ovum from the ovary to the uterus. Fertilization usually takes place in the oviduct.
\begin{center}
  \Image{0.8\columnwidth}{images/week-9-b.jpg}
\end{center}
\begin{multicols}{2}
\begin{itemize}
  \item \jjj{Infundibulum with fimbria}:  funnel-shaped segment open to the peritoneal cavity with fringed extensions that extend toward the ovary.
  
  \begin{center}
    \Image{0.94\columnwidth}{images/week-9-b2.jpg}
  \end{center}
  
  \item \jjj{Fringed extensions}:
  
  \begin{center}
    \Image{0.94\columnwidth}{images/week-9-b3.jpg}
  \end{center}
  
  \item \jjj{Ampulla}: longest region and site of fertilization.
  
  \begin{center}
    \Image{0.94\columnwidth}{images/week-9-b4.jpg}
  \end{center}
  
  \vspace{40pt}

  \item \jjj{Mucosa}: exhibits thin  longitudinal folds that project into the lumen. 
  
  \begin{center}
    \Image{0.94\columnwidth}{images/week-9-b5.jpg}
  \end{center}
  
  \item \jjj{Longitudinal folds}:
  
  \begin{center}
    \Image{0.94\columnwidth}{images/week-9-b6.jpg}
  \end{center}
  
  \item \jjj{Simple columnar epithelium}: consists of two types of cells.
  
  \begin{center}
    \Image{0.94\columnwidth}{images/week-9-b7.jpg}
  \end{center}

  \item \jjj{Ciliated cells}: secrete fluid that provides nutrients for the ovum or fertilized zygote.
  
  \begin{center}
    \Image{0.9\columnwidth}{images/week-9-b72.jpg}
  \end{center}
   
  \item \jjj{Peg cells}: secrete fluid that provides nutrients for the ovum or fertilized zygote.
  
  \begin{center}
    \Image{0.9\columnwidth}{images/week-9-b73.jpg}
  \end{center}
  
  \item \jjj{Lamina propria}: contains blood vessels and nerves.
  
  \begin{center}
    \Image{0.9\columnwidth}{images/week-9-b8.jpg}
  \end{center}
  
  \item \jjj{Muscularis}: consists of an inner circular or spiral layer and an outer longitudinal layer whose peristaltic contractions help propel the ovum or fertilized zygote to the uterus.
  
  \begin{center}
    \Image{0.9\columnwidth}{images/week-9-b9.jpg}
  \end{center}
  
  \item \jjj{Serosa}: composed of a simple cuboidal epithelium (or mesothelium) supported by a thin layer of connective tissue.
  
  \begin{center}
    \Image{0.9\columnwidth}{images/week-9-b10.jpg}
  \end{center}
  
\end{itemize}
\end{multicols}

\newpage
\subsection{Uterus}
\begin{center}
  \Image{0.65\columnwidth}{images/week-9-c.jpg}
\end{center}
\begin{multicols}{2}
\begin{itemize}
  \item \jjj{Endometrium}: specialized mucosa that undergoes marked changes during the menstrual cycle. 
  
  \begin{center}
    \Image{0.7\columnwidth}{images/week-9-c1.jpg}
  \end{center}
  
  \item \jjj{Functional layer}: the upper two thirds of the mucosa that develops glands and is lost during menstruation. 
  
  \begin{center}
    \Image{0.7\columnwidth}{images/week-9-c2.jpg}
  \end{center}
  
  \item \jjj{Simple columnar epithelium}: ciliated columnar and non-ciliated secretory cells.
  
  \begin{center}
    \Image{0.75\columnwidth}{images/week-9-c3.jpg}
  \end{center}
  
  \item \jjj{Endometrial stroma}:the underlying lamina propria is highly cellular.
  
  \begin{center}
    \Image{0.75\columnwidth}{images/week-9-c4.jpg}
  \end{center}
  
  \item \jjj{Uterine glands}:  during the menstrual cycle the surface epithelium invaginates into the stroma to form simple tubular glands lined with mostly non-ciliated secretory cells.
  
  \begin{center}
    \Image{0.77\columnwidth}{images/week-9-c5.jpg}
  \end{center}
  
  \item \jjj{Simple tubular glands}:
  
  \begin{center}
    \Image{0.77\columnwidth}{images/week-9-c6.jpg}
  \end{center}
  
  \item \jjj{Basal layer (stratum basalis)}: lower third of the mucosa that is retained during menstruation and regenerates the functional layer.
  
  \begin{center}
    \Image{0.77\columnwidth}{images/week-9-c7.jpg}
  \end{center}
  
  \item \jjj{Glycogen}: stored in the base of the epithelial cells is characteristic of early secretory glands. 
  
  \begin{center}
    \Image{0.77\columnwidth}{images/week-9-c8.jpg}
  \end{center}
  
  \item \jjj{Myometrium}: composed of three indistinct layers of smooth muscle. 
  
  \begin{center}
    \Image{0.77\columnwidth}{images/week-9-c9.jpg}
  \end{center}
  
  \item \jjj{Stratum vasculare}: thickest layer of mostly circular or spiral bundles of smooth muscle with numerous blood vessels.
  
  \begin{center}
    \Image{0.77\columnwidth}{images/week-9-c10.jpg}
  \end{center}
  
  \item \jjj{Perimetrium}: covered by an outer serous layer or visceral peritoneum that is continuous with the broad ligament.
  
  \begin{center}
    \Image{0.94\columnwidth}{images/week-9-c11.jpg}
  \end{center}

  \item \jjj{Broad ligament}:
  
  \begin{center}
    \Image{0.8\columnwidth}{images/week-9-c111.jpg}
    \Image{0.8\columnwidth}{images/week-9-c112.jpg}
  \end{center}
  
  \item \jjj{Arcuate arteries}:  6 to 10 branches of the uterine artery that encircle the uterus in the myometrium.
  
  \begin{center}
    \Image{0.85\columnwidth}{images/week-9-c12.jpg}
  \end{center}
  
  \item \jjj{Radial arteries}:  branches of arcuate arteries that ascend into the endometrium and give rise to: 
  
  \begin{center}
    \Image{0.8\columnwidth}{images/week-9-c13.jpg}
  \end{center}
  
  \item \jjj{Straight arteries}: supply the basal layer.
  
  \begin{center}
    \Image{0.8\columnwidth}{images/week-9-c14.jpg}
  \end{center}
  
  \item \jjj{Spiral arteries}: pass through the basal layer and supply the functional layer. 
  
  \begin{center}
    \Image{0.85\columnwidth}{images/week-9-c15.jpg}
  \end{center}
  
  \item \jjj{Terminal capillaries}:  dilated (or ectatic) capillaries that arise from spiral arteries.
  
  \begin{center}
    \Image{0.75\columnwidth}{images/week-9-c16.jpg}
  \end{center}
  
\end{itemize}
\end{multicols}

\subsection{Cervix}
The cervix is the lower end of the uterus that opens into the vagina. During menstruation, it allows the passage of menstrual fluid from the uterus. In childbirth, it widens (dilates) to allow passage of the baby from the uterus to the outside world.
\bigskip
\begin{center}
  \Image{0.94\columnwidth}{images/week-9-d.jpg}
\end{center}
\bigskip
\begin{multicols}{2}
\begin{itemize}
  \item \jjj{Enodcervix}: forms the wall of the cervical canal. 
  
  \begin{center}
    \Image{0.85\columnwidth}{images/week-9-d1.jpg}
  \end{center}
  
  \item \jjj{Cervical glands}:  branched glands of muscus-secreting cells located in the lamina propria.
  
  \begin{center}
    \Image{0.85\columnwidth}{images/week-9-d2.jpg}
  \end{center}
  
  \item \jjj{Mucus-secreting cells}:
  
  \begin{center}
    \Image{0.85\columnwidth}{images/week-9-d3.jpg}
  \end{center}
  
  \item \jjj{Ectocervix}: part of the cervix that protrudes in the vagina and contains the opening of the uterus. 
  
  \begin{center}
    \Image{0.7\columnwidth}{images/week-9-d4.jpg}
  \end{center}

  \item \jjj{Stratified squamous non-kerantized epithelium}: continuous with the lining of the vagina.
  
  \begin{center}
    \Image{0.7\columnwidth}{images/week-9-d5.jpg}
  \end{center}
  
  \item \jjj{Transformation zone}: abrupt junction between the mucus-secreting columnar epithelium of the endocervix and the squamous epithelium of the ectocervix.
  
  \begin{center}
    \Image{0.7\columnwidth}{images/week-9-d6.jpg}
  \end{center}
  
  \item \jjj{Nabothian cysts}: develop as stratified squamous epithelium grows over mucus-secreting simple columnar epithelium and entraps large amounts of mucus.
  
  \begin{center}
    \Image{0.73\columnwidth}{images/week-9-d7.jpg}
  \end{center}
  
  \item \jjj{Cervical wall}: composed o dense connective tissue rich in both collagen and elastic fibers. Unlike the rest of the uterus, it contains little smooth muscle.
  
  \begin{center}
    \Image{0.94\columnwidth}{images/week-9-d8.jpg}
  \end{center}
  
\end{itemize}
\end{multicols}

\subsection{Placenta}
The placenta develops during pregnancy to support the developing fetus by producing hormones, transferring nutrients and waste products between the mother and the fetus. The placenta is composed of 15 to 20 regions called cotyledons. This specimen is a portion of a cotyledon from late pregnancy.
\begin{center}
  \Image{0.94\columnwidth}{images/week-9-e.jpg}
\end{center}
\begin{multicols}{2}
\begin{itemize}
  \item \jjj{Villi}: projections of the fetal chorion that extend into lacunae in which maternal blood flows. Exchange between the two circulations occurs through the villus wall. 
  
  \begin{center}
    \Image{0.8\columnwidth}{images/week-9-e1.jpg}
  \end{center}
  
  \item \jjj{Syntiotrophblasts}: multinucleated cuboidal cells with microvilli.
  
  \begin{center}
    \Image{0.8\columnwidth}{images/week-9-e3.jpg}
  \end{center}
  
  \item \jjj{Stroma}: mesenchymal connective tissue forms the core of villi and contains fetal capillaries and venules.
  
  \begin{center}
    \Image{0.8\columnwidth}{images/week-9-e4.jpg}
  \end{center}
   
  \item \jjj{Placental arteries}:  develop from arteries in the endometrium to supply maternal blood to the lacunae.
  
  \begin{center}
    \Image{0.8\columnwidth}{images/week-9-e5.jpg}
  \end{center}
  
  \item \jjj{Basal plate}:  the part of the uterus to which chorionic villi are anchored.
  
  \begin{center}
    \Image{0.8\columnwidth}{images/week-9-e6.jpg}
  \end{center}
  
  \item \jjj{Decidual cells}:  clusters of large round or oval cells.
  
  \begin{center}
    \Image{0.8\columnwidth}{images/week-9-e7.jpg}
  \end{center}
  
\end{itemize}
\end{multicols}

\subsection{Mammary Gland}
Mammary glands are compound, tubulo-alveolar glands whose structure changes depending on the reproductive status of females.

Lactating (left) --- Resting (right)
\begin{center}
  \Image{0.49\columnwidth}{images/week-9-m.jpg}
  \Image{0.465\columnwidth}{images/week-9-mm.jpg}
\end{center}
\begin{multicols}{2}
\begin{itemize}
  \item \jjj{Lactiferous duct}:  each lobe is drained by a single lactiferous duct that opens into the nipple. It is lined by a double layer of cuboidal or columnar cells surrounded by a sheath of connective tissue with myoid cells.
  
  \begin{center}
    \Image{0.94\columnwidth}{images/week-9-m1.jpg}
  \end{center}
  
  \vspace*{15pt}
  \item \jjj{Double layered cuboidal/columnar cells}: 
  \vspace{20pt}
  \begin{center}
    \Image{0.94\columnwidth}{images/week-9-m2.jpg}
  \end{center}
  
  \item \jjj{Lobules}: enclosed by a thin layer of connective tissue.
  
  \begin{center}
    \Image{0.75\columnwidth}{images/week-9-m22.jpg}
  \end{center}
  
  \item \jjj{Intralobular ducts}: lined by one or two layers of cuboidal cells surrounded by a thin layer of connective tissue.
  
  \begin{center}
    \Image{0.75\columnwidth}{images/week-9-m3.jpg}
  \end{center}
  
  \item \jjj{Terminal ductules}:  branches of intralobular ducts lined with cuboidal secretory cells. 
  
  \begin{center}
    \Image{0.75\columnwidth}{images/week-9-m4.jpg}
  \end{center}
  
  \item \jjj{Alveoli}: grow and expand during pregnancy and lactation. 
  
  \begin{center}
    \Image{0.75\columnwidth}{images/week-9-m5.jpg}
  \end{center}
  
  \item \jjj{Lactiferous ducts}: (resting)
  
  \begin{center}
    \Image{0.75\columnwidth}{images/week-9-m6.jpg}
  \end{center}
  
  \item \jjj{Double layered cuboidal/columnar cells}: (resting)
  
  \begin{center}
    \Image{0.75\columnwidth}{images/week-9-m7.jpg}
  \end{center}
  
  \item \jjj{Terminal ductules}: branches of intralobular ducts. 
  
  \begin{center}
    \Image{0.94\columnwidth}{images/week-9-m8.jpg}
  \end{center}
  
  \item \jjj{Lamina propria}: dense connective tissue with numerous bundles of whose contraction smooth allows for erection of the nipple.
  
  \begin{center}
    \Image{0.8\columnwidth}{images/week-9-m9.jpg}
  \end{center}
  
\end{itemize}
\end{multicols}

\subsection{Questions}
\begin{itemize}\color{minor}
  \item In which portion of the uterine tube does fertilization occur?
  \basec{\begin{itemize}
    \item Ampulla
  \end{itemize}}
  
  \item What is the primary ovarian hormone stimulating the endometrium during each stage?\\
  \basec{\begin{itemize}
    \item Estrogen.
  \end{itemize}}
  
  \item What is the dominant ovarian structure during the secretory stage?
  \basec{\begin{itemize}
    \item Corpus luteum. 
  \end{itemize}}

  \item Which ovarian hormone is necessary for the maintenance of the secretory stage of the endometrium?
  \basec{\begin{itemize}
    \item Progesterone.
  \end{itemize}}
  
  \item Which zones of the endometrium may be lost during menstruation?
  \basec{\begin{itemize}
    \item Functionalis
  \end{itemize}}

  \item What are some possible functions of cervical mucus?
  \basec{\begin{itemize}
    \item Impedes sperm entry except in peri-ovulatory period when the mucus in less viscous
  \end{itemize}}

  \item In what other regions of the body does one observe an abrupt junction between simple columnar and stratified epithelia?
  \basec{\begin{itemize}
    \item Gastro-esophageal junction, colo-rectal junction.
  \end{itemize}}

  \item  What are the major hormones that are responsible for the cyclic changes in the mammary gland?
  \basec{\begin{itemize}
    \item Estrogen and progesterone
  \end{itemize}}
\end{itemize}

\section{Eye}
\begin{center}
  \Image{0.7\columnwidth}{images/week-9-n.jpg}
\end{center}
\begin{multicols}{2}
\begin{itemize}
  \item \jjj{Aqueous chamber}: region anterior to the lens. 
  
  \begin{center}
    \Image{0.65\columnwidth}{images/week-9-n1.jpg}
  \end{center}
  
  \item \jjj{Vitreous chamber}: region posterior to the lens. 
  
  \begin{center}
    \Image{0.65\columnwidth}{images/week-9-n2.jpg}
  \end{center}
  
  \item \jjj{Wall of the eye}: composed of three concentric layers---the \ddd{fibrous tunic} (outer layer), the \ddd{uveal tunic} (pigmented inner layer), and the \ddd{retinal tunic} (innermost layer)
  
  \begin{center}
    \Image{0.9\columnwidth}{images/week-9-n3.jpg}
  \end{center}
  
  \item \jjj{Sclera}:  opaque, heavily vascularized connective tissue that covers the posterior 5/6th of the eye. 
  
  \begin{center}
    \Image{0.8\columnwidth}{images/week-9-n4.jpg}
  \end{center}
  
  \item \jjj{Conjunctiva}: mucous membrane (a stratified squamous epithelium) covering the anterior sclera (and lining the inner surface of eyelids).
  
  \begin{center}
    \Image{0.8\columnwidth}{images/week-9-n5.jpg}
  \end{center}
  
  \item \jjj{Cornea}: transparent, avascular connective tissue that covers the anterior 1/6th of the eye. It is composed of five layers.
  
  \begin{center}
    \Image{0.6\columnwidth}{images/week-9-n6.jpg}
  \end{center}
  
  \item \jjj{Corneal epithelium}: non-keratinized stratified squamous epithelium that covers its anterior surface exposed to air. 
  
  \begin{center}
    \Image{0.8\columnwidth}{images/week-9-n7.jpg}
  \end{center}
  
  \item \jjj{Bowman membrane}:  a distinctive layer of collagen fibers (7 to 12 µm thick).
  
  \begin{center}
    \Image{0.8\columnwidth}{images/week-9-n8.jpg}
  \end{center}
  
  \item \jjj{Stroma}: avascular layer of collagen fibers and fibroblasts. The thickest layer of the cornea.
  
  \begin{center}
    \Image{0.85\columnwidth}{images/week-9-n9.jpg}
  \end{center}
  
  \item \jjj{Descemet's membrane}:  a thick (5 to 10 µm) basement membrane underneath the corneal endothelium.
  
  \begin{center}
    \Image{0.8\columnwidth}{images/week-9-n10.jpg}
  \end{center}
  
  \item \jjj{Corneal endothelium}: simple squamous epithelium that covers the posterior surface exposed to the aqueous humor.
  
  \begin{center}
    \Image{0.8\columnwidth}{images/week-9-n11.jpg}
  \end{center}
   
  \item \jjj{Limbus}: the junction of the opaque sclera and transparent cornea.
  
  \begin{center}
    \Image{0.8\columnwidth}{images/week-9-n12.jpg}
  \end{center}

  \item \jjj{Choroid}: highly vascular, pigmented layer in the posterior 2/3rd of the eye composed of the Bruch membrane (thick basement membrane) and a vascular layer of connective tissue and blood.  
  \begin{center}
    \Image{0.75\columnwidth}{images/week-9-n13.jpg}
  \end{center}
  
  \item \jjj{Ciliary body}:  thickening of the choroid at the junction between the posterior 2/3rd and anterior 1/3rd of the eye. 
  
  \begin{center}
    \Image{0.75\columnwidth}{images/week-9-n14.jpg}
  \end{center}
  
  \item \jjj{Ciliary muscle}: attached to sclera and ciliary body and control the shape of the lens.
  
  \begin{center}
    \Image{0.75\columnwidth}{images/week-9-n15.jpg}
  \end{center}
  
  \item \jjj{Ciliary processes}: ridge-like projections of the choroid into the posterior chamber towards the lens. 
  
  \begin{center}
    \Image{0.72\columnwidth}{images/week-9-n16.jpg}
  \end{center}
  
  \item \jjj{Zonular fibers}: suspensory ligaments that insert into the capsule of the lens (only fragments remain).
  
  \begin{center}
    \Image{0.72\columnwidth}{images/week-9-n17.jpg}
  \end{center}
  
  \item \jjj{Iris}: most anterior extension of the choroid and separates the anterior and the posterior chambers. 
  
  \begin{center}
    \Image{0.72\columnwidth}{images/week-9-n18.jpg}
  \end{center}
  
  \item \jjj{Retina}: posterior surface of the eye responsible for photoreception.
  
  \begin{center}
    \Image{0.93\columnwidth}{images/week-9-n19.jpg}
  \end{center}
  
  \item \jjj{10 layers (outer \to~inner)}:
    \begin{itemize}
      \item Pigmented epithelium
      \item Photoreceptive layer of rods and cones 
      \item External limiting membrane
      \item Outer nuclear layer
      \item Outer plexiform layer
      \item Inner nuclear layer
      \item Ganglion cell layer
      \item Optic nerve fiber layer
      \item Inner limiting membrane
    \end{itemize}
  
  \begin{center}
    \Image{0.93\columnwidth}{images/week-9-n20.jpg}
  \end{center}
  
  \item \jjj{Optic disk}: the site where axons of ganglion cells converge to form the optic nerve by penetrating the sclera.
  
  \begin{center}
    \Image{0.8\columnwidth}{images/week-9-n21.jpg}
  \end{center}
  
  \item \jjj{Ora Serrata}: the junction between the posterior region with photoreceptive neurons and the anterior non-neuronal region.
  
  \begin{center}
    \Image{0.8\columnwidth}{images/week-9-n22.jpg}
  \end{center}
  
  \item \jjj{Lens}:  transparent bi-convex disk composed of lens fibers that are remnants of cells filled with crystallins.
  
  \begin{center}
    \Image{0.6\columnwidth}{images/week-9-n23.jpg}
  \end{center}
  
  \item \jjj{Capsule}: the anterior surface has a thick (10 to 20 µm) basement membrane produced by subcapsular cuboidal cells.
  
  \begin{center}
    \Image{0.8\columnwidth}{images/week-9-n24.jpg}
  \end{center}
  
\end{itemize}
\end{multicols}

\newpage
\section{Ear}
The inner ear detects sound, acceleration and position. It consists of fluid filled sacs (membranous labyrinth) that lie in cavities of the temporal bone (bony labyrinth).

The membranous labyrinth is divided into two inter-connecting regions: the cochlea, which detects sound vibration; and the vesitbular apparatus, which detects acceleration, gravity, and position.
\begin{center}
  \Image{0.94\columnwidth}{images/week-9-i.jpg}
\end{center}
\newpage
\begin{multicols}{2}
\begin{itemize}
  \item \jjj{Cochlea}: a coiled tube that spiral 2 and 2/3 times around a central pillar of bone.
  
  \begin{center}
    \Image{0.8\columnwidth}{images/week-9-i1.jpg}
  \end{center}
  
  \item \jjj{Modiolus}: the central pillar of bone.
  
  \begin{center}
    \Image{0.65\columnwidth}{images/week-9-i2.jpg}
  \end{center}
  
  \item \jjj{Cochlear divisions}: Two ribbons of tissue a lower basilar membrane and upper vestibular membrane, divide the cochlea throughout its length into three compartments: upper (scala vestibuli), middle (scala media), lower (scala tympani)
  
  \begin{center}
    \Image{0.66\columnwidth}{images/week-9-i3.jpg}
  \end{center}
  
  \item \jjj{Helicotrema}: at the apex of the spiral the scala tympani communicates with the scala vestibuli.
  
  \begin{center}
    \Image{0.66\columnwidth}{images/week-9-i4.jpg}
  \end{center}
  
  \item \jjj{Organ of Corti}: the sensory compartment of the cochlea.
  
  \begin{center}
    \Image{0.66\columnwidth}{images/week-9-i5.jpg}
  \end{center}
  
  \item \jjj{Basilar membrane}: extends from a medial projection of bone to a lateral projection of connective tissue.
  
  \begin{center}
    \Image{0.8\columnwidth}{images/week-9-i6.jpg}
  \end{center}
  
  \item \jjj{Cochlear nerves}:
  
  \begin{center}
    \Image{0.8\columnwidth}{images/week-9-i7.jpg}
  \end{center}
  
  \item \jjj{Spiral ganglia}:
  
  \begin{center}
    \Image{0.8\columnwidth}{images/week-9-i8.jpg}
  \end{center}
  
  \item \jjj{Vestibular apparatus}:  detects acceleration (ampulla of semi-circular canals) and gravity direction and static position (macula of utricle and saccule).
  
  \begin{center}
    \Image{0.9\columnwidth}{images/week-9-i9.jpg}
  \end{center}
  
  \item \jjj{Crista ampullaris}:  sensory regions within the ampulla of the semi-circular canals and similar structures in the utricle and saccule. 
  
  \begin{center}
    \Image{0.8\columnwidth}{images/week-9-i10.jpg}
  \end{center}
  
\end{itemize}
\end{multicols}

