%chktex-file 8
\chapter{Week 4: Cardiovascular}
\section{The Heart}
\begin{itemize}
  \begin{multicols}{2}
  \item \jjj{Epicardium}: outer surface of mesothelial cells supported by dense irregular connective tissue.

  \begin{center}
    \Image{0.7\columnwidth}{images/week-4-1a.png}
  \end{center}

  \item \jjj{Myocardium}: thickest, middle layer of cardiac muscle responsible for the pumping action of the heart; contains blood vessels, nerves and adipose cells.
  
  \begin{center}
    \Image{0.7\columnwidth}{images/week-4-1b.png}
  \end{center}
  
  \item \jjj{Endocardium}:  inner surface of a simple squamous epithelium  supported by a thin layer of dense irregular connective tissue.

  \begin{center}
      \Image{0.7\columnwidth}{images/week-4-1c.png}
  \end{center}

  \vspace{60pt}

  \item \jjj{Right Atrium}: receives venous blood from the systemic circulation. 

  \begin{center}
      \Image{0.7\columnwidth}{images/week-4-1g.png}
  \end{center}

  \item \jjj{Cardiac muscle cells}: rounded cross-sections with a centrally located nucleus.

  \begin{center}
      \Image{0.7\columnwidth}{images/week-4-1f.png}
  \end{center}

  \item \jjj{Collagen fibers (Fibrous rings)}: support heart values and contain bands of collagen and elastic fibers that appear dark red.

  \begin{center}
      \Image{0.7\columnwidth}{images/week-4-1e.png}
  \end{center}

  \vspace{42pt}

  \item \jjj{Intercalated discs}: thin, dark stained linear structures dividing adjacent cells that are perpendicular to the direction of the muscle fiber

  \begin{center}
      \Image{0.7\columnwidth}{images/week-4-1d.png}
  \end{center}

  \item \jjj{Pectinate muscles}: bundles of muscle that protrude from the surface that are common in the right atrium.
  
  \begin{center}
    \Image{0.7\columnwidth}{images/week-4-1h.png}
  \end{center}

  \item \jjj{Ventricle}: receives blood from the right atrium and pumps it into the lungs via the pulmonary arteries.
  
  \begin{center}
    \Image{0.7\columnwidth}{images/week-4-1i.png}
  \end{center}

  \item \jjj{Ventricle myocardium}: receives blood from the right atrium and pumps it into the lungs via the pulmonary arteries.
  
  \begin{center}
    \Image{0.7\columnwidth}{images/week-4-1j.png}
  \end{center}

  \end{multicols}

  \item \jjj{Is the myocardium thicker in the atrium or ventricle? Why?}
  \begin{itemize}
    \item The \textbf{ventricles'} myocardium are thicker, since it does more work \textbf{pumping} blood, as the atrium mainly \textbf{receives} blood and only pumps it to the ventricles. Since pumping increases size of muscles, then the myocardium housing said muscles, will be larger.
  \end{itemize}

\end{itemize}

\newpage
\section{Blood Vessels}
\begin{itemize}
  \item []
  
  \subsection{Arteries}
  \begin{itemize}
    \item \jjj{Elastic Artery (Carotid Artery)}: conveys blood from the heart to systemic and pulmonary circulations.
    
    \begin{center}
      \Image{0.9\columnwidth}{images/week-4-2a.png}
    \end{center}
    
    \bigskip

    \begin{multicols}{3}
    \begin{center}
    \jjj{Tunica intima}

    \Image{0.9\columnwidth}{images/week-4-2b.png}

    \jjj{Tunica media}

    \Image{0.9\columnwidth}{images/week-4-2c.png}

    \jjj{Tunica adventitia}

    \Image{0.9\columnwidth}{images/week-4-2d.png}
    
    \end{center}
    \end{multicols} 

    \begin{multicols}{2}
    \item \jjj{Vasa Vasorum}: blood vessels that supply the tunica adventitia and tunica media.
    
    \begin{center}
      \Image{0.7\columnwidth}{images/week-4-2e.png}
    \end{center}

    \item \jjj{Nerves in adventitia}: responsible for regulation of the contraction of the smooth muscle.
    
    \begin{center}
      \Image{0.7\columnwidth}{images/week-4-2f.png}
    \end{center}

    \end{multicols}
  \end{itemize}

  \item \jjj{Comparison of Muscular Artery (Top) and Medium Vein (Bottom)}:
  
  \begin{center}
    \Image{0.8\columnwidth}{images/week-4-2g.png}
  \end{center}

  \newpage

  \subsection{Veins}
  \begin{itemize}
    \item \jjj{Large Vein (Brachiocephalic Vein)}: has thinner walls than arteries with less distinct layers, as well as the adventitia instead of media being the thickest portion.
    
    \begin{center}
      \Image{0.7\columnwidth}{images/week-4-3a.png}
    \end{center}
    
    \bigskip

    \begin{multicols}{3}
    \begin{center}
      \jjj{Tunica intima}

      \Image{0.9\columnwidth}{images/week-4-3b.png}

      \jjj{Tunica media}

      \Image{0.9\columnwidth}{images/week-4-3c.png}

      \jjj{Tunica adventitia}

      \Image{0.9\columnwidth}{images/week-4-3d.png}
    \end{center}
    \end{multicols} 

    \begin{multicols}{3}
    \begin{center}
      \jjj{Vasa Vasorum}

      \Image{0.9\columnwidth}{images/week-4-3e.png}

      \jjj{Nerves}

      \Image{0.9\columnwidth}{images/week-4-3f.png}

      \jjj{White Adipose}

      \Image{0.9\columnwidth}{images/week-4-3g.png}
    \end{center}
    \end{multicols} 

  \end{itemize}

  \newpage
  \subsection{Arterioles and Venules}
  \begin{multicols}{2}
  \begin{itemize}
    \item \jjj{Arteriole}: small diameter blood vessels that allow blood to flow into and out of capillary beds, respectively.
    
    \begin{center}
      \Image{0.7\columnwidth}{images/week-4-4b.png}
    \end{center}

    \item \jjj{Two layers of smooth muscle}: 
    
    \begin{center}
      \Image{0.7\columnwidth}{images/week-4-4a.png}
    \end{center}
    
    \item \jjj{Venules}: small diameter blood vessel that allows blood to return from capillary beds to veins.
    
    \begin{center}
      \Image{0.7\columnwidth}{images/week-4-4c.png}
    \end{center}
    
    \item \jjj{Arterioles vs. Venules}
    
    \begin{center}
      \Image{0.7\columnwidth}{images/week-4-4d.png}
    \end{center}
  
  \end{itemize}
  \end{multicols}
\end{itemize}

\newpage
\section{Capillaries}
\begin{itemize}
  \item []
  
  \subsection{Continuous Capillaries}
  \begin{itemize}
    \item Continuous capillaries have many, tight well-developed occluding junctions between slightly overlapping endothelia cells. They are the most common capillary that provide well-regulated metabolic exchange across the cells.
 
    \begin{center}
      \jjj{Longitudinal}
    
      \Image{0.9\columnwidth}{images/week-4-5a.png}

      \bigskip

      \jjj{Cross-section}
    \end{center}

    \begin{multicols}{3}
    \begin{center}
      \Image{0.9\columnwidth}{images/week-4-5b.png}
      \Image{0.9\columnwidth}{images/week-4-5c.png}
      \Image{0.9\columnwidth}{images/week-4-5d.png}
    \end{center}
    \end{multicols}
  \end{itemize}

  \newpage
  \subsection{Fenestrated Capillaries}
  \begin{itemize}
    \item Similar to continuous, but with endothelia cells that are penetrated by numerous small circular openings. Found in organs with rapid interchange of substances between tissue and the blood. The pore can only be seen with electron microscopy.
    
    
    \begin{center}
      \jjj{SEM}

      \Image{0.8\columnwidth}{images/week-4-6.png}

      \bigskip

      \jjj{TEM}
    \end{center}

    \begin{multicols}{2}
      \begin{center}
        \Image{0.8\columnwidth}{images/week-4-6a.png}
        \Image{0.8\columnwidth}{images/week-4-6b.png}
      \end{center}
    \end{multicols}
  \end{itemize}

  \newpage
  \subsection{Sinusoidal (Discontinuous) Capillaries}
  \begin{itemize}
    \item These capillaries allow for maximal exchange of macromolecules as well as easier movement of cells between tissues and blood. 
    \item The endothelium has large perforations without diaphragms and irregular intercellular clefts.
    \item Unlike other capillaries, sinusoids have highly discontinuous basement membranes and larger diameters, slowing blood flow.
    

    \begin{center}
      \jjj{Network of sinusoids}

      \Image{0.5\columnwidth}{images/week-4-7.png}

      \bigskip

      \jjj{Sinusoid surrounded by endocrine cells}

      \Image{0.5\columnwidth}{images/week-4-7b.png}

    \end{center}

  \end{itemize}

\end{itemize}