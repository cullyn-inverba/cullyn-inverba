%chktex-file 8
\chapter{Week 2: Epithelial Connective Bone Cartilage}

\section{Epithelial Tissue}
\begin{itemize}
  \item[]
  
  \subsection{Examples}
  \begin{multicols}{2}
  \begin{itemize}
    \item \jjj{Simple squamous epithelium \(\uparrow \)}  
    
    \Image{0.9\columnwidth}{images/week-2-1a.png}
    \begin{multicols}{2}
    \begin{itemize}
      \item Cell nuclei
      
      \Image{0.8\columnwidth}{images/week-2-1b.png}
      
      \item Basement membrane (BM)
      
      \Image{0.8\columnwidth}{images/week-2-1c.png}
    \end{itemize}
    \end{multicols}
    \item \jjj{Ciliated pseudostratified columnar epithelium (PsEp)}

     \Image{0.9\columnwidth}{images/week-2-2a.png}

     \begin{multicols}{3}
      \begin{itemize}
        \item Goblet c.
        
        \Image{0.8\columnwidth}{images/week-2-2b.png}
        
        \item Cilia
        
        \Image{0.8\columnwidth}{images/week-2-2c.png}
        
        \item BM
        
        \Image{0.8\columnwidth}{images/week-2-2d.png}

      \end{itemize}
      \end{multicols}

      \bigskip
      \medskip

    \item \jjj{Stratified squamous epithelium}

     \Image{0.85\columnwidth}{images/week-2-3a.png}

     \begin{multicols}{2}
      \begin{itemize}
        \item Cell nuclei
        
        \Image{0.8\columnwidth}{images/week-2-3c.png}
        
        \item BM
        
        \Image{0.8\columnwidth}{images/week-2-3b.png}
      \end{itemize}
      \end{multicols}

    \item \jjj{Simple cuboidal epithelium}

     \Image{0.9\columnwidth}{images/week-2-4a.png}

      \begin{itemize}
        
        \item BM (maybe?)
        
        \begin{center}
          \Image{0.5\columnwidth}{images/week-2-4b.png}
        \end{center}
      \end{itemize}

    \item \jjj{Simple columnar epithelium}
    
     \Image{0.8\columnwidth}{images/week-2-5a.png}

      \begin{itemize}
        \item Microvillous border
        
        \Image{0.35\columnwidth}{images/week-2-5b.png}
        
        \item Goblet cells
        
        \Image{0.35\columnwidth}{images/week-2-5c.png}
      \end{itemize}

  \end{itemize}
  \end{multicols}

  \subsection{Review}
  \begin{itemize}
    \item \jjj{Simple squamous}: wider than their height, hence flat and scale like (squamous). 
      \begin{itemize}
        \item Locations: in mouth, esophagus, blood vessels (endothelium), alveoli of lungs, lymphatic vessels, lining of cavities (mesothlium).
        \item Function: facilitates movements of viscera via diffusion and filtration, active transport by pinocytosis, secretion of molecules and lubricating substances. 
      \end{itemize}
    \item \jjj{Simple cuboidal}: cells with height and width that are approximately the same, hence cuboidal.
      \begin{itemize}
        \item Location: covering the ovary and thyroid, in kidney tubules, and other secretory portions of small glands.
        \item Function: covering, secretion, absorption.
      \end{itemize}
    \item \jjj{Simple columnar}: cells that are taller than they are wide, i.e., column shaped, hence columnar.
      \begin{itemize}
        \item Location: ciliated tissues are in bronchi, uterine tubes, and uterus; smooth line intestine and gallbladder.
        \item Function: protection, lubrication, absorption, and secretion.
      \end{itemize}
    \item \jjj{Pseudostratified columnar}: cells with nuclei that appear at different heights, leading the pseudo impression of stratified columnar cells when viewed in cross-section.
      \begin{itemize}
        \item Location: ciliated tissues lines the trachea and much of the upper respiratory tract, including the nasal cavity.
        \item Function: protection, secretion (mostly mucus), celia-mediated transport of particles trapped in mucus.
      \end{itemize}
    \item \jjj{Stratified squamous}: like simple squamous, but multilayered. Cells generally become more squamous as they become more apical.
      \begin{itemize}
        \item Location: lines the esophagus, mouth, vagina, anal canal, larynx. 
        \item Function: multi layers protects against abrasion, prevents water loss.
      \end{itemize}
    \item \jjj{Stratified cuboidal}: again, same as simple cuboidal, but multi layered.
      \begin{itemize}
        \item Location: sweat, salivary, and mammary glands, also in developing ovarian follicles.
        \item Function: Protection, secretion.
      \end{itemize}
    \item \jjj{Stratified columnar}: multilayered columnar\dots
      \begin{itemize}
        \item Location: the male urethra and the ducts of some glands, and the conjunctiva (mucus membrane in front eye and inside eyelid).
        \item Function: protection, secretion of mucus.
      \end{itemize}
    \item \jjj{Transitional}: cells that can change from squamous to cuboidal, depending on amount of tension in the epithelium.
      \begin{itemize}
        \item Location: bladder, ureters, urethra, and renal calyces (chambers in kidney through which urine passes).
        \item Function: protection, distensibility (ability to swell from inside), stretch.
      \end{itemize}
  \end{itemize}
  
\end{itemize}

\newpage

\section{Connective Tissue Proper}
\begin{itemize}
  \item[]
  
  \subsection{Examples}
  \begin{multicols}{2}
  \begin{itemize}
    \item \jjj{Dense regular connective tissue}:

    \Image{0.9\columnwidth}{images/week-2-6a.png}

    \begin{itemize}
      \item Collagenous tissue
      \begin{center}
        \Image{0.5\columnwidth}{images/week-2-6b.png}
      \end{center}
    \end{itemize}


    \item \jjj{Reticular tissue}

    \Image{0.9\columnwidth}{images/week-2-7a.png}

    \begin{multicols}{2}
    \begin{itemize}
      \item Reticular fibers
      
      \Image{0.8\columnwidth}{images/week-2-7b.png}
      
      \item Collagen (type I)
      
      \Image{0.8\columnwidth}{images/week-2-7c.png}
    \end{itemize}
    \end{multicols}

    \item \jjj{Loose connective tissue}

    \Image{0.9\columnwidth}{images/week-2-8a.png}

    \begin{multicols}{2}
      \begin{itemize}
        \item Nuclei
        
        \Image{0.8\columnwidth}{images/week-2-8b.png}
        
        \item \amp{Elastin} \rrr{fibers}
        
        \Image{0.8\columnwidth}{images/week-2-8c.png}

        \item Matrix
        
        \Image{0.8\columnwidth}{images/week-2-8d.png}
        
        \item \bbb{Collagen} \xxx{fibers}
        
        \Image{0.8\columnwidth}{images/week-2-8e.png}
      \end{itemize}
      \end{multicols}

    \item \jjj{Lymph node reticular fibers}

    \Image{0.9\columnwidth}{images/week-2-9a.png}

    \item \jjj{Muscle tendon junctions}

    \Image{0.9\columnwidth}{images/week-2-10a.png}

    \end{itemize}

    \begin{itemize}
      \item[]
    \begin{multicols}{2}
      \begin{itemize}
        \item Collagen I
        
        \Image{0.82\columnwidth}{images/week-2-10b.png}
        
        \item Fibroblasts
        
        \Image{0.85\columnwidth}{images/week-2-10c.png}
      \end{itemize}
      \end{multicols}

      \hspace{60pt}

    \item \jjj{Thin skin}

    \Image{0.9\columnwidth}{images/week-2-11a.png}

    \begin{itemize}
      \item Thick collagenous fibers type I
      
      \begin{center}
        \Image{0.5\columnwidth}{images/week-2-11b.png}
      \end{center}
    \end{itemize}
  \end{itemize}
  \end{multicols}

  \subsection{Review}
  \begin{itemize}
    \item \jjj{Mesenchyme}: connective tissue found mostly during embryonic development and is generally undifferentiated, sparse, and uniformly distributed in the matrix with sparse collagen fibers.
    \begin{itemize}
      \item Location: mesodermal layer of early embryo, develops into tissues of lymphatic and circulatory system; can migrate easily. Organized into adherent sheets.
      \item Function: contains stem/progenitor cells for all adult connective tissue cells.
    \end{itemize}
    \item \jjj{Areolar}: or loose connective tissue, is the most widely distributed connective tissue type, has much ground substance (gel like), many cells, little collagen, and randomly distributed.
    \begin{itemize}
      \item Location: can be found in the skin and in places that connect the epithelium to other tissues; found beneath the dermis layer and under epithelial tissue of body systems that have external openings. Also surrounds blood vessels and nerves and a component of mucus membranes in the digestive, respiratory, reproductive and urinary systems.
      \item Function: supports microvasculature, nerves, immune defense cells, holds organs in place, and serves as reservoir of water and salts.
    \end{itemize}
    \item \jjj{Loose CT reticular}: a network of reticular fibers (synthesized by fibroblasts called reticular cells) made up of type III collagen.
    \begin{itemize}
      \item Location: bone marrow, spleen, liver, kidney, lymph nodes (not thymus), adrenal glands.
      \item Function: supports blood-forming cells, many secretory cells, and lymphocytes in most lymphoid organs.
    \end{itemize}
    \item \jjj{Dense regular}: filled with mostly parallel bundles of collagen, few fibroblasts, is aligned with collagen, and very strong. Slow to heal due to poor blood supply.
    \begin{itemize}
      \item Location: tendons, ligaments, aponeuroses (layers of flat broad tendons), and corneal stroma.
      \item Function: provide strong connections within musculoskeletal system and strong resistance to force, especially in well in one direction. 
    \end{itemize}
    \item \jjj{Dense elastic fibers}: an essential component of the extracellular matrix composed of bundles of protein (elastin) produced by fibroblasts, endothelial, smooth muscle, and airway epithelial cells. Can stretch many times their length and return to back to original length.
    \begin{itemize}
      \item Location: found in the skin, lungs, arteries, veins, connective tissue proper, elastic cartilage, periodontal ligaments, fetal tissues and more; basically anything that must undergo stretching.
      \item Function: provides the elastic properties to connective tissues.
    \end{itemize}
    \item \jjj{Dense irregular}: fibers not arranged in parallel bundles as in dense regular connective tissue. Still consists of mostly collagen fibers, but has less ground substance than loose connective tissue. Fibroblasts are scattered sparsely across the tissue.
    \begin{itemize}
      \item Location: found mostly in the reticular layer of the dermis, but also in the sclera in the deeper skin layers. Found in the submucosal layer of the digestive tract.
      \item Function: mainly provides strength, similar to regular tissue. Protects organs and helps resist tearing. 
    \end{itemize}
  \end{itemize}
  
\end{itemize}

\newpage
\section{Cartilage and Bone}
\begin{itemize}
  \item[]
  
  \subsection{Examples}
  \begin{multicols}{2}
  \begin{itemize}
    \item \jjj{Hyaline cartilage}

    \Image{0.9\columnwidth}{images/week-2-12a.png}

    \medskip 

    \begin{multicols}{2}
    \begin{itemize}
      \item Incomplete rings???
      
      \Image{0.8\columnwidth}{images/week-2-12b.png}
      
      \item Inner Perichondrium 
      
      \Image{0.8\columnwidth}{images/week-2-12c.png}

      \item Outer Perichondrium
      
      \Image{0.8\columnwidth}{images/week-2-12d.png}

      \item Chondrocytes

      \Image{0.8\columnwidth}{images/week-2-12e.png}
    \end{itemize}
    \end{multicols}


    \item \jjj{Elastic cartilage}

    \Image{0.9\columnwidth}{images/week-2-13a.png}

    \item \jjj{Fibrocartilage}

    \Image{0.9\columnwidth}{images/week-2-14a.png}


    % \item \jjj{Ground bone Cross and longitudinal sections}

    % \Image{0.9\columnwidth}{images/week-2-10a.png}

    % \begin{multicols}{2}
    % \begin{itemize}
    %   \item Reticular fibers
      
    %   \Image{0.8\columnwidth}{images/week-2-7b.png}
      
    %   \item Collagen (type I)
      
    %   \Image{0.8\columnwidth}{images/week-2-7c.png}
    % \end{itemize}
    % \end{multicols}


    % \item \jjj{Decalcified bone}

    % \Image{0.9\columnwidth}{images/week-2-10a.png}

    % \begin{multicols}{2}
    % \begin{itemize}
    %   \item Reticular fibers
      
    %   \Image{0.8\columnwidth}{images/week-2-7b.png}
      
    %   \item Collagen (type I)
      
    %   \Image{0.8\columnwidth}{images/week-2-7c.png}
    % \end{itemize}
    % \end{multicols}


    % \item \jjj{Growth at the epiphyseal plate}

    % \Image{0.9\columnwidth}{images/week-2-10a.png}

    % \begin{multicols}{2}
    % \begin{itemize}
    %   \item Reticular fibers
      
    %   \Image{0.8\columnwidth}{images/week-2-7b.png}
      
    %   \item Collagen (type I)
      
    %   \Image{0.8\columnwidth}{images/week-2-7c.png}
    % \end{itemize}
    % \end{multicols}


  \end{itemize}
  \end{multicols}

  \subsection{Review}
  \begin{itemize}
    \item \jjj{Hyaline Cartilage}: homogeneous with type II collagen, most common of the cartilage types. Covered externally by fibrous membrane known as perichondrium. 
    \begin{itemize}
      \item Location: components of the upper respiratory tact; in the larynx, trachea, bronchi, sternal ends of the ribs, and articular ends and epiphyseal plates of long bones.
      \item Function: gives structures (particularly respiratory) a definite but pliable form, low friction surface to joints.
    \end{itemize}
    \item \jjj{Elastic Cartilage}: similar to hyaline cartilage, but contains many yellow elastic fibers in a solid matrix. 
    \begin{itemize}
      \item Location: external ear, external acoustic meatus (ear canal), auditory tube; epiglottis (leaf shape flap in throat), and more.
      \item Function: provides flexible shape and support of soft tissues.
    \end{itemize}
    \item \jjj{Fibrocartilage}: a mixture of white fibrous tissue and cartilaginous tissue in various proportions; contains type I collagen in addition to type II\@.
    \begin{itemize}
      \item Location: many joints (shoulder, hip, knee), as well secondary joints (pubic symphysis, manubriosternal joint), intervertebral discs, and insertions of tendons.
      \item Function: provides cushioning, tensile strength, and resistance to tearing and compression.
    \end{itemize}
    \item \jjj{Bone}: calcified connective tissue also consisting of cells, fibers, and ground substances. Structurally rigid using calcium phosphate salts within its matrix. There are multiple types of bone, i.e., cortical (hard, outer) and cancellous (trabecular, spongy, internal). 
    \begin{itemize}
      \item Location: like, everywhere. Spooky Scary Skeletons. 
      \item Function: provides solid support for the body, protects organs, encloses internal cavities, server as reservoir of calcium, phosphate, and other ions, form systems of levers that multiply forces generated during skeletal muscle contraction, and transforms energy into bodily movement.
    \end{itemize}
  \end{itemize}
  
\end{itemize}

