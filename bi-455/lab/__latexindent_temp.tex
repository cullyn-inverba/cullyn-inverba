%chktex-file 8
\chapter{Week 2: Epithelial Connective Bone Cartilage}

\section{Epithelial Tissue}
\begin{itemize}
  \item[]
  
  \subsection{Examples}
  \begin{multicols}{2}
  \begin{itemize}
    \item \jjj{Simple squamous epithelium \(\uparrow \)}  
    
    \Image{0.9\columnwidth}{images/week-2-1a.png}
    \begin{multicols}{2}
    \begin{itemize}
      \item Cell nuclei
      
      \Image{0.8\columnwidth}{images/week-2-1b.png}
      
      \item Basement membrane (BM)
      
      \Image{0.8\columnwidth}{images/week-2-1c.png}
    \end{itemize}
    \end{multicols}
    \item \jjj{Ciliated pseudostratified columnar epithelium (PsEp)}

     \Image{0.9\columnwidth}{images/week-2-2a.png}

     \begin{multicols}{3}
      \begin{itemize}
        \item Goblet c.
        
        \Image{0.8\columnwidth}{images/week-2-2b.png}
        
        \item Cilia
        
        \Image{0.8\columnwidth}{images/week-2-2c.png}
        
        \item BM
        
        \Image{0.8\columnwidth}{images/week-2-2d.png}

      \end{itemize}
      \end{multicols}

      \bigskip
      \medskip

    \item \jjj{Stratified squamous epithelium}

     \Image{0.85\columnwidth}{images/week-2-3a.png}

     \begin{multicols}{2}
      \begin{itemize}
        \item Cell nuclei
        
        \Image{0.8\columnwidth}{images/week-2-3c.png}
        
        \item BM
        
        \Image{0.8\columnwidth}{images/week-2-3b.png}
      \end{itemize}
      \end{multicols}

    \item \jjj{Simple cuboidal epithelium}

     \Image{0.9\columnwidth}{images/week-2-4a.png}

      \begin{itemize}
        
        \item BM (maybe?)
        
        \begin{center}
          \Image{0.5\columnwidth}{images/week-2-4b.png}
        \end{center}
      \end{itemize}

    \item \jjj{Simple columnar epithelium}
    
     \Image{0.8\columnwidth}{images/week-2-5a.png}

      \begin{itemize}
        \item Microvillous border
        
        \Image{0.35\columnwidth}{images/week-2-5b.png}
        
        \item Goblet cells
        
        \Image{0.35\columnwidth}{images/week-2-5c.png}
      \end{itemize}

  \end{itemize}
  \end{multicols}

  \subsection{Review}
  \begin{itemize}
    \item \jjj{Simple Squamous}: wider than their height, hence flat and scale like (squamous). 
      \begin{itemize}
        \item Locations: in mouth, esophagus, blood vessels (endothelium), alveoli of lungs, lymphatic vessels, lining of cavities (mesothlium).
        \item Function: facilitates movements of viscera via diffusion and filtration, active transport by pinocytosis, secretion of molecules and lubricating substances. 
      \end{itemize}
    \item \jjj{Simple Cuboidal}: cells with height and width that are approximately the same, hence cuboidal.
      \begin{itemize}
        \item Location: covering the ovary and thyroid, in kidney tubules, and other secretory portions of small glands.
        \item Function: covering, secretion, absorption.
      \end{itemize}
    \item \jjj{Simple Columnar}: cells that are taller than they are wide, i.e., column shaped, hence columnar.
      \begin{itemize}
        \item Location: ciliated tissues are in bronchi, uterine tubes, and uterus; smooth line intestine and gallbladder.
        \item Function: protection, lubrication, absorption, and secretion.
      \end{itemize}
    \item \jjj{Pseudostratified Columnar}: cells with nuclei that appear at different heights, leading the pseudo impression of stratified columnar cells when viewed in cross section.
      \begin{itemize}
        \item Location: ciliated tissues lines the trachea and much of the upper respiratory tract, including the nasal cavity.
        \item Function: protection, secretion (mostly mucus), celia-mediated transport of particles trapped in mucus.
      \end{itemize}
    \item \jjj{Stratified Squamous}: like simple squamous, but multilayered. Cells generally become more squamous as they become more apical.
      \begin{itemize}
        \item Location: lines the esophagus, mouth, vagina, anal canal, larynx. 
        \item Function: multi layers protects against abrasion, prevents water loss.
      \end{itemize}
    \item \jjj{Stratified Cuboidal}: again, same as simple cuboidal, but multi layered.
      \begin{itemize}
        \item Location: sweat, salivary, and mammary glands, also in developing ovarian follicles.
        \item Function: Protection, secretion.
      \end{itemize}
    \item \jjj{Stratified Columnar}: multilayered columnar\dots
      \begin{itemize}
        \item Location: the male urethra and the ducts of some glands, and the conjunctiva (mucus membrane in front eye and inside eyelid).
        \item Function: protection, secretion of mucus.
      \end{itemize}
    \item \jjj{Transitional}: cells that can change from squamous to cuboidal, depending on amount of tension in the epithelium.
      \begin{itemize}
        \item Location: bladder, ureters, urethra, and renal calyces (chambers in kidney through which urine passes).
        \item Function: protection, distensibility (ability to swell from inside), stretch.
      \end{itemize}
  \end{itemize}
  
\end{itemize}

\newpage

\section{Connective Tissue Proper}
\begin{itemize}
  \item[]
  
  \subsection{Examples}
  \begin{multicols}{2}
  \begin{itemize}
    \item \jjj{Dense Regular Connective Tissue}:

    \Image{0.9\columnwidth}{images/week-2-6a.png}

    \begin{itemize}
      \item Collagenous tissue
      \begin{center}
        \Image{0.5\columnwidth}{images/week-2-6b.png}
      \end{center}
    \end{itemize}


    \item \jjj{Reticular Tissue}

    \Image{0.9\columnwidth}{images/week-2-7a.png}

    \begin{multicols}{2}
    \begin{itemize}
      \item Reticular fibers
      
      \Image{0.8\columnwidth}{images/week-2-7b.png}
      
      \item Collagen (type I)
      
      \Image{0.8\columnwidth}{images/week-2-7c.png}
    \end{itemize}
    \end{multicols}

    \item \jjj{Loose Connective Tissue}

    \Image{0.9\columnwidth}{images/week-2-6a.png}

    \begin{multicols}{2}
      \begin{itemize}
        \item Nuclei
        
        \Image{0.8\columnwidth}{images/week-2-7b.png}
        
        \item Matrix
        
        \Image{0.8\columnwidth}{images/week-2-7c.png}

        \item Elastin fibers
        
        \Image{0.8\columnwidth}{images/week-2-7b.png}
        
        \item Collagen fibers
        
        \Image{0.8\columnwidth}{images/week-2-7c.png}
      \end{itemize}
      \end{multicols}

    \item \jjj{Lymph Node}

    \Image{0.9\columnwidth}{images/week-2-6a.png}

    \item \jjj{Muscle tendon junctions}

    \Image{0.9\columnwidth}{images/week-2-6a.png}

    \item \jjj{Thin skin}

    \Image{0.9\columnwidth}{images/week-2-6a.png}

  \end{itemize}
  \end{multicols}

  \subsection{Review}
  \begin{itemize}
    \item \jjj{Mesenchyme}
    \item \jjj{Areolar}
    \item \jjj{Loose CT Reticular}
    \item \jjj{Dense Regular}
    \item \jjj{Dense Elastic}
    \item \jjj{Dense Irregular}
  \end{itemize}
  
\end{itemize}

\section{Cartilage and Bone}
\begin{itemize}
  \item[]
  
  \subsection{Examples}
  \begin{itemize}
    \item \jjj{Hyaline cartilage}
    \item \jjj{Elastic cartilage}
    \item \jjj{Fibrocartilage}
    \item \jjj{Ground bone Cross and longitudinal sections}
    \item \jjj{Decalcified bone}
    \item \jjj{Growth at the epiphyseal plate}
  \end{itemize}
  
  \subsection{Review}
  \begin{itemize}
    \item \jjj{Hyaline Cartilage}
    \item \jjj{Elastic Cartilage}
    \item \jjj{Fibrocartilage}
    \item \jjj{Bone}
  \end{itemize}
  
\end{itemize}

