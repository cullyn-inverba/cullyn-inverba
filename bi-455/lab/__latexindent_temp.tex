\chapter{Week 9: Female Reproductive, Eye, Ear}

\section{Female Reproductive System}

\subsection{Ovary}
The ovaries are responsible for the production of an oocyte (oogenesis) and secretion of female sex hormones (estrogen and progesterone). When it releases a mature ovum, it travels down the oviduct to the uterus.
\begin{center}
  \Image{0.8\columnwidth}{images/week-9-a.jpg}
\end{center}
\begin{multicols}{2}
\begin{itemize}
  \item \jjj{Capsule}: covers the outer surface. 
  
  \begin{center}
    \Image{0.7\columnwidth}{images/week-9-a1.jpg}
  \end{center}
   
  \item \jjj{Germinal epithelium}: the surface is covered by a simple cuboidal epithelium.
  
  \begin{center}
    \Image{0.7\columnwidth}{images/week-9-a2.jpg}
  \end{center}
  
  \item \jjj{Tunica albuginea}: layer of dense irregular connective tissue that supports the epithelium.
  \begin{center}
    \Image{0.7\columnwidth}{images/week-9-a3.jpg}
  \end{center}
  
  \item \jjj{Cortex}: outer region that is the site of oocyte development. 
  
  \begin{center}
    \Image{0.85\columnwidth}{images/week-9-a4.jpg}
  \end{center}
  
  \item \jjj{Primordial follicles}: oocytes arrested in development are located in the outer cortex. 
  
  \begin{center}
    \Image{0.75\columnwidth}{images/week-9-a5.jpg}
  \end{center}
  
  \item \jjj{Primary oocyte}: large (25 to 30 µm), round to oval cells with a vesicular nucleus.
  
  \begin{center}
    \Image{0.75\columnwidth}{images/week-9-a6.jpg}
  \end{center}
  
  \item \jjj{Zona pellucida}:  layer of glycoproteins between the oocyte and granulosa cells. It is visible a thin, eosinophilic band in many follicles.
  
  \begin{center}
    \Image{0.75\columnwidth}{images/week-9-a7.jpg}
  \end{center}
  
  \item \jjj{Follicular cells}: single layer of flattened cells that surround each oocyte.
  
  \begin{center}
    \Image{0.8\columnwidth}{images/week-9-a8.jpg}
  \end{center}
  
  \item \jjj{Unilaminar}: primary oocytes surrounded by a single layer of granulosa cells.
  
  \begin{center}
    \Image{0.8\columnwidth}{images/week-9-a9.jpg}
  \end{center}
  
  \item \jjj{Multilaminar}:  primary oocytes surrounded by multiple layers of granulosa cells.
  
  \begin{center}
    \Image{0.8\columnwidth}{images/week-9-a10.jpg}
  \end{center}
  
  \item \jjj{Secondary (Antral) Follicles}: characterized by the formation an antrum (a fluid-filled space) containing an oocyte. The antrum increases in size as the follicle matures 
  
  \begin{center}
    \Image{0.61\columnwidth}{images/week-9-a111.jpg}
    \Image{0.61\columnwidth}{images/week-9-a11.jpg}
    \Image{0.61\columnwidth}{images/week-9-a112.jpg}
    \Image{0.61\columnwidth}{images/week-9-a113.jpg}
  \end{center}
   
  \item \jjj{oocyte}: large (50 to 100 µm), round to oval cells with a vesicular nucleus.
  
  \begin{center}
    \Image{0.82\columnwidth}{images/week-9-a12.jpg}
  \end{center}
  
  \item \jjj{Zona pellucida}:  layer of glycoproteins between the oocyte and granulosa cells (eosinophilic).
  
  \begin{center}
    \Image{0.82\columnwidth}{images/week-9-a13.jpg}
  \end{center}
  
  \item \jjj{Corona radiata}: group of cells anchored to the follicle wall that contains the oocyte.
  
  \begin{center}
    \Image{0.82\columnwidth}{images/week-9-a14.jpg}
  \end{center}
  
  \item \jjj{Cumulus oophorus}: group of cells anchored to the follicle wall that contains the oocyte.
  
  \begin{center}
    \Image{0.7\columnwidth}{images/week-9-a15.jpg}
  \end{center}
  
  \item \jjj{Stratum granulosum}: multiple layers of cells that form the follicle wall that surrounds the antrum. 
  
  \begin{center}
    \Image{0.7\columnwidth}{images/week-9-a16.jpg}
  \end{center}
  
  \item \jjj{Theca folliculi}: stromal cells around the follicle develop into a sheath of highly vascularized connective tissue. 
  
  \begin{center}
    \Image{0.7\columnwidth}{images/week-9-a18.jpg}
  \end{center}
  
  \item \jjj{Mature (Graafian) follicles}: usually only one follicle will continue to grow each cycle to form a very large, mature follicle (25 mm or more in diameter). 
  
  \begin{center}
    \Image{0.8\columnwidth}{images/week-9-a17.jpg}
  \end{center}

  \item \jjj{Stratum granulosum}: becomes thinner as the follicle continues to grow in size.
  
  \begin{center}
    \Image{0.8\columnwidth}{images/week-9-a182.jpg}
  \end{center}
  
  \item \jjj{Theca folliculi}: becomes more organized and contains many blood vessels.
  
  \begin{center}
    \Image{0.8\columnwidth}{images/week-9-a19.jpg}
  \end{center}
  
  
  \item \jjj{Stroma}: highly cellular connective tissue with fewer connective tissue fibers in which ovarian follicles are located. 
  
  \begin{center}
    \Image{0.85\columnwidth}{images/week-9-a20.jpg}
  \end{center}
  
  \item \jjj{Medulla}: inner region of fibroelastic connective tissue with many large, tortuous blood vessels, lymph vessels and nerve fibers. 
  
  \begin{center}
    \Image{0.85\columnwidth}{images/week-9-a21.jpg}
  \end{center}
  
\end{itemize}
\end{multicols}

\subsection{Oviduct}
The oviducts (uterine tubes; fallopian tubes) are fibromuscular tubes that transport an ovulated ovum from the ovary to the uterus. Fertilization usually takes place in the oviduct.
\begin{center}
  \Image{0.8\columnwidth}{images/week-9-b.jpg}
\end{center}
\begin{multicols}{2}
\begin{itemize}
  \item \jjj{Infundibulum with fimbria}:
  
  \begin{center}
    \Image{0.8\columnwidth}{images/week-9-b2.jpg}
  \end{center}
  
  \item \jjj{Fringed extensions}:
  
  \begin{center}
    \Image{0.8\columnwidth}{images/week-9-b3.jpg}
  \end{center}
  
  \item \jjj{Ampulla}:
  
  \begin{center}
    \Image{0.8\columnwidth}{images/week-9-b4.jpg}
  \end{center}
  
  \item \jjj{Mucosa}:
  
  \begin{center}
    \Image{0.8\columnwidth}{images/week-9-b5.jpg}
  \end{center}
  
  \item \jjj{Longitudinal folds}:
  
  \begin{center}
    \Image{0.8\columnwidth}{images/week-9-b6.jpg}
  \end{center}
  
  \item \jjj{Simple columnar epithelium}:
  
  \begin{center}
    \Image{0.8\columnwidth}{images/week-9-b7.jpg}
  \end{center}

  \item \jjj{Ciliated cells}:
  
  \begin{center}
    \Image{0.8\columnwidth}{images/week-9-b72.jpg}
  \end{center}
  
  \item \jjj{}:
  
  \begin{center}
    \Image{0.8\columnwidth}{images/week-9-.jpg}
  \end{center}
  
  
  
  \item \jjj{Lamina propria}: contains blood vessels and nerves.
  
  \begin{center}
    \Image{0.8\columnwidth}{images/week-9-b8.jpg}
  \end{center}
  
  \item \jjj{Muscularis}: consists of an inner circular or spiral layer and an outer longitudinal layer whose peristaltic contractions help propel the ovum or fertilized zygote to the uterus.
  
  \begin{center}
    \Image{0.8\columnwidth}{images/week-9-b9.jpg}
  \end{center}
  
  \item \jjj{Serosa}: composed of a simple cuboidal epithelium (or mesothelium) supported by a thin layer of connective tissue.
  
  \begin{center}
    \Image{0.8\columnwidth}{images/week-9-b10.jpg}
  \end{center}
  
\end{itemize}
\end{multicols}

\subsection{Uterus}
\begin{center}
  % \Image{0.94\columnwidth}{images/week-9-c.jpg}
\end{center}
\begin{multicols}{2}
\begin{itemize}
  \item 
\end{itemize}
\end{multicols}

\subsection{Cervix}
\begin{center}
  % \Image{0.94\columnwidth}{images/week-9-d.jpg}
\end{center}
\begin{multicols}{2}
\begin{itemize}
  \item 
\end{itemize}
\end{multicols}

\subsection{Placenta}
\begin{center}
  % \Image{0.94\columnwidth}{images/week-9-e.jpg}
\end{center}
\begin{multicols}{2}
\begin{itemize}
  \item 
\end{itemize}
\end{multicols}

\subsection{Breast}
\begin{center}
  % \Image{0.94\columnwidth}{images/week-9-m.jpg}
\end{center}
\begin{multicols}{2}
\begin{itemize}
  \item 
\end{itemize}
\end{multicols}

\subsection{Questions}
\begin{itemize}\color{minor}
  \item In which portion of the uterine tube does fertilization occur?
  \basec{\begin{itemize}
    \item 
  \end{itemize}}
  
  \item What is the primary ovarian hormone stimulating the endometrium during each stage?
  \basec{\begin{itemize}
    \item 
  \end{itemize}}
  
  \item What is the dominant ovarian structure during the secretory stage?
  \basec{\begin{itemize}
    \item 
  \end{itemize}}

  \item Which ovarian hormone is necessary for the maintenance of the secretory stage of the endometrium?
  \basec{\begin{itemize}
    \item 
  \end{itemize}}
  
  \item Which zones of the endometrium may be lost during menstruation?
  \basec{\begin{itemize}
    \item 
  \end{itemize}}

  \item What are some possible functions of cervical mucus?
  \basec{\begin{itemize}
    \item 
  \end{itemize}}

  \item In what other regions of the body does one observe an abrupt junction between simple columnar and stratified epithelia?
  \basec{\begin{itemize}
    \item 
  \end{itemize}}

  \item  What are the major hormones that are responsible for the cyclic changes in the mammary gland?
  \basec{\begin{itemize}
    \item 
  \end{itemize}}
\end{itemize}

\section{Eye}
\begin{center}
  % \Image{0.94\columnwidth}{images/week-9-n.jpg}
\end{center}
\begin{multicols}{2}
\begin{itemize}
  \item 
\end{itemize}
\end{multicols}

\section{Ear}
\begin{center}
  % \Image{0.94\columnwidth}{images/week-9-i.jpg}
\end{center}
\begin{multicols}{2}
\begin{itemize}
  \item 
\end{itemize}
\end{multicols}

