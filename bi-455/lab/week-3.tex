% %chktex-file 8
% \chapter{Week 3: Adipose Blood Muscle Nerve}

% \section{Adipose}
% \begin{itemize}
%   \item []
  
%   \subsection{White Adipose}
%   \begin{center}
%     \Image{0.9\columnwidth}{images/week-3-white.png}
%     \Image{0.9\columnwidth}{images/week-3-white-key.png}
%   \end{center}
%   \bigskip
%   \begin{itemize}
%     \item Location: subcutaneous layer, mammary gland, greater omentum, mesenteries, retroperitoneal space, visceral pericardium, orbits, bone marrow cavity.
%     \item Function: energy storage, insulation, cushioning, hormone production, source of metabolic water.
%     \item Adipocyte Morphology: unilocular, spherical, flatten nuclear, rim of cytoplasm, large diameter (15--\SI{150}{\micro\meter})
%     \item Factors inducing differentiation: PPAR-\(\gamma \)/RXR
%     \item UCP-1 gene expression: None
%     \item Mitochondria: Few, poorly developed
%     \item Response to environmental stress (cold): Decreased lipogenesis, increased lipoprotein lipase activity.
%     \item Growth and differentiation: entire life, starting form stromal-vascular cells
%   \end{itemize}
  
%   \subsection{Brown Adipose}
%   \begin{center}
%     \Image{0.9\columnwidth}{images/week-3-brown.png} 
%     \Image{0.9\columnwidth}{images/week-3-white-key.png}
%   \end{center}
%   \bigskip
%   \begin{itemize}
%     \item Location: large amounts in newborn; in adults: retroperitoneal space, deep cervical and supraclavicular regions of neck, interscapular, paravertebral regions of the back, mediastinum.
%     \item Function: thermogenesis.
%     \item Adipocyte Morphology: multiocular, spherical, round eccentric nucleus, smaller diameter (10--\SI{25}{\micro\meter}).
%     \item Factors inducing differentiation: PRDM16/PGC-1
%     \item UCP-1 gene expression: Yes (unique to brown fat)
%     \item Mitochondria: many, well-developed.
%     \item Response to environmental stress (cold): increased lipogenesis, decreased lipoprotein lipase activity.
%     \item Growth and differentiation: only during fetal period, decreases in adult life, expect those with pheochromocytoma and hibernoma.
%   \end{itemize}
  
%   \newpage
%   \subsection{White Adipose Examples}
%   \begin{multicols}{2}
%   \begin{itemize}
%     \item \jjj{Adipocyte}: the largest component of loose connective tissue.
    
%     \begin{center}
%       \Image{0.8\columnwidth}{images/week-3-wa.png}
%     \end{center}
%     \item \jjj{EM of adipose (blue)}
    
%     \begin{center}
%       \Image{0.8\columnwidth}{images/week-3-wb.png}
%     \end{center}
%     \item \jjj{Reticular Fibers (purple)}
    
%     \begin{center}
%       \Image{0.8\columnwidth}{images/week-3-we.png}
%     \end{center}
    
%     \item \jjj{Small blood vessel (red)}
    
%     \begin{center}
%       \Image{0.8\columnwidth}{images/week-3-wc.png}
%     \end{center}

%     \item \jjj{Nerve Fiber (cyan)}
  
%     \begin{center}
%       \Image{0.5\columnwidth}{images/week-3-wd.png}
%     \end{center}

%     \item \jjj{Nuclei} 
    
%     \begin{center}
%       \Image{0.8\columnwidth}{images/week-3-wf.png}
%     \end{center}
%   \end{itemize}
%   \end{multicols}

%   \newpage
%   \subsection{Brown Adipose Examples}
%   \begin{itemize}
%     \item Brown adipose tissue
    
%     \begin{center}
%       \Image{0.4\columnwidth}{images/week-3-ba.png}
%     \end{center}
    
%     \item Brown adipocyte
    
%     \begin{center}
%       \Image{0.4\columnwidth}{images/week-3-bb.png}
%     \end{center}
    
%     \item Contrast between white unilocular and brown multiocular adipocyte
    
%     \begin{center}
%       \Image{0.4\columnwidth}{images/week-3-bc.png}
%     \end{center}
%   \end{itemize}
% \end{itemize}

% \newpage
% \section{Blood}
% \begin{multicols}{2}
% \begin{itemize}
%   \item \jjj{RBC (erythrocytes):}  light red, biconcave discs without nuclei that carry oxygen and return CO\(_2\).
%   Size: \approx~\SI{7.5}{\micro\meter}
  
%   \begin{center}
%     \Image{0.8\columnwidth}{images/week-3-ca.png}
%   \end{center}

%   \item \jjj{Platelet}: small, basophilic discs that help prevent bleeding via blood clotting. Size: 2--\SI{4}{\micro\m}
  
%   \begin{center}
%     \Image{0.7\columnwidth}{images/week-3-cb.png}
%   \end{center}

%   \item \jjj{Eosinophil}: contain distinctive large, eosinophilic granules and usually nuclei with two lobes. Aids in anti-parasitic and bacterial activity, allergic reactions, and modulating inflammation.

%   \begin{center}
%     \Image{0.7\columnwidth}{images/week-3-cc.png}
%   \end{center}

%   \item \jjj{Neutrophil}: have distinctive nuclei with 2 to 5 lobes and azure granules that form essential part of innate immune system with various functions.

%   \begin{center}
%     \Image{0.64\columnwidth}{images/week-3-cd.png}
%   \end{center}

%   \item \jjj{Basophil}: rare cells with distinctive large, basophilic granules. Aids immune responses by releasing cytokines, leuktrienes, and histamine. 

%   \begin{center}
%     \Image{0.64\columnwidth}{images/week-3-ce.png}
%   \end{center}

%   \item \jjj{Monocyte}:  large cells with ``kidney-shaped'' or notched nuclei. Precursors of macrophages and other cells of the mononculear phagocyte system.

%   \begin{center}
%     \Image{0.64\columnwidth}{images/week-3-cf.png}
%   \end{center}
% \end{itemize}
% \end{multicols}
% \begin{itemize}
%   \item \jjj{Lymphocyte}: occur in a range of sizes (small:left, large:right). Have many functions relating to immune system, as well as subtypes (T and B), helps produce antibodies.

%   \begin{center}
%     \Image{0.44\columnwidth}{images/week-3-ch.png}
%     \Image{0.4\columnwidth}{images/week-3-cg.png}
%   \end{center}
% \end{itemize}

% \newpage
% \section{Muscle}
% \begin{itemize}
%   \item []

%   \subsection{Skeletal Muscle}
%   \begin{multicols}{2}
%   \begin{itemize}
%   \item \jjj{Tendon}
  
%   \begin{center}
%     \Image{0.85\columnwidth}{images/week-3-3a.png}
%   \end{center}

%   \item \jjj{Skeletal Muscle}:
  
%   \begin{center}
%     \Image{0.85\columnwidth}{images/week-3-3b.png}
%   \end{center}

%   \item \jjj{Skeletal Muscle Cross-Section}
  
%   \begin{center}
%     \Image{0.85\columnwidth}{images/week-3-3c.png}
%   \end{center}

%   \vspace{45pt}

%   \item \jjj{Fibroblasts Nuclei and Sacromere}
  
%   \begin{center}
%     \Image{0.8\columnwidth}{images/week-3-3e.png}
%   \end{center}

%   \item \jjj{Sarcomere} (Em)

%    \begin{center}
%     \Image{0.9\columnwidth}{images/week-3-3d.png}
%   \end{center}

%   \item \jjj{Fascicles} (F)

%    \begin{center}
%     \Image{0.8\columnwidth}{images/week-3-3f.png}
%   \end{center}

%   \vspace{20pt}

%   \item \jjj{Perimysium}: thick layer of connective tissue that surrounds a group of muscle cells to form fascicles. Blood and nerves are found in this layer.

%    \begin{center}
%     \Image{0.9\columnwidth}{images/week-3-3h.png}
%   \end{center}

%   \item \jjj{Endomysium}: thin layer of connective tissue that surrounds each muscle cell.

%    \begin{center}
%     \Image{0.9\columnwidth}{images/week-3-3g.png}
%   \end{center}

%   \item \jjj{Epimysium}: dense connective tissue that surrounds the entire muscle and is usually continuous with a tendon.

%    \begin{center}
%     \Image{0.9\columnwidth}{images/week-3-3i.png}
%   \end{center}

%   \vspace{60pt}

%   \item \jjj{Satellite Cells}: occur on the surface of muscle cells

%    \begin{center}
%     \Image{0.8\columnwidth}{images/week-3-3j.png}
%   \end{center}

%   \item \jjj{Capillaries}: a collapsed capillary containing flattened red blood cells is present on the left edge of the center muscle cell.

%    \begin{center}
%     \Image{0.8\columnwidth}{images/week-3-3k.png}
%   \end{center}
%   \end{itemize}
%   \end{multicols}

%   \newpage 

%   \subsection{Cardiac Muscle}
%   \begin{multicols}{2}
%   \begin{itemize}
%     \item \jjj{Cross-striations}: cardiac muscles cells have rounded cross-sections (< \SI{25}{\micro\meter}) with a centrally located nucleus.

%     \begin{center}
%       \Image{0.8\columnwidth}{images/week-3-4a.png}
%       \Image{0.8\columnwidth}{images/week-3-4b.png}
%     \end{center}
    
%     \item \jjj{Longitudinal section}: cardiac muscle cells are smaller than skeletal muscle.
    
%     \begin{center}
%       \Image{0.8\columnwidth}{images/week-3-4c.png}
%     \end{center}
    
%     \vspace{30pt}

%     \item \jjj{Branched cells}: cells are joined end-to-end and are often branched
    
%     \begin{center}
%       \Image{0.8\columnwidth}{images/week-3-4d.png}
%     \end{center}

%     \item \jjj{Straight intercalated disks}: specialized junctions that connect the individual cells 
    
%     \begin{center}
%       \Image{0.8\columnwidth}{images/week-3-4e.png}
%     \end{center}

%     \item \jjj{Stepped intercalated disks}
    
%     \begin{center}
%       \Image{0.8\columnwidth}{images/week-3-4f.png}
%     \end{center}


%   \end{itemize}
%   \end{multicols}
  
%   \subsection{Colon}

%   \begin{center}
%     \Image{0.9\columnwidth}{images/week-3-5a.png}
%     \Image{0.9\columnwidth}{images/week-3-5b.png}
%     \Image{0.9\columnwidth}{images/week-3-5key.png}
%   \end{center}

%   \newpage

%   \begin{itemize}
%     \item Smooth muscle in small intestine:
    
%     \begin{center}
%       \Image{0.5\columnwidth}{images/week-3-5c.png}
%     \end{center}

%     \item Inner layer:  individual cells vary in diameter depending on their location within the cell. Cross-sections through the middle of cells have centrally located nuclei, usually surrounded by an unstained region.
    
%     \begin{center}
%       \Image{0.4\columnwidth}{images/week-3-5d.png}
%     \end{center}

%     \item Outer layer: in relaxed smooth muscle, the nuclei are elongated with rounded ends. When contracted, the nuclei spiral, kink, or twist. The cytoplasm is pink, unstriated and with little detail.
    
%     \begin{center}
%       \Image{0.4\columnwidth}{images/week-3-5e.png}
%     \end{center}

%   \end{itemize}

%   \subsection{Uterus}
%   \begin{itemize}
%     \item \jjj{Myometrium (left) Fascicles of smooth muscle (right)}

%     \begin{center}
%       \Image{0.4\columnwidth}{images/week-3-6a.png}
%       \Image{0.43\columnwidth}{images/week-3-6d.png}
%     \end{center}

%     \item \jjj{Blood vessels within myometrium}

%     \begin{center}
%       \Image{0.4\columnwidth}{images/week-3-6b.png}
%       \Image{0.42\columnwidth}{images/week-3-6c.png}
%     \end{center}

%     \begin{center}
%       \Image{0.5\columnwidth}{images/week-3-6e.png}
%     \end{center}
%   \end{itemize}
% \end{itemize}

% \newpage
% \section{Central Nervous System}
% \begin{itemize}
%   \item []
  
%   \subsection{Spinal Cord}
%   \begin{multicols}{2}
%   \begin{itemize}
%     \item \jjj{Gray matter}
    
%     \begin{center}
%       \Image{0.8\columnwidth}{images/week-3-7a.png}
%     \end{center}

%     \item \jjj{White matter}
    
%     \begin{center}
%       \Image{0.8\columnwidth}{images/week-3-7b.png}
%     \end{center}

%     \item \jjj{Dorsal structure}
    
%     \begin{center}
%       \Image{0.8\columnwidth}{images/week-3-7c.png}
%     \end{center}

%     \vspace{30pt}

%     \item \jjj{Ventral structure}
    
%     \begin{center}
%       \Image{0.8\columnwidth}{images/week-3-7d.png}
%     \end{center}

%     \item \jjj{Motor neurons in anterior horn}
    
%     \begin{center}
%       \Image{0.8\columnwidth}{images/week-3-7e.png}
%     \end{center}

%     \item \jjj{Perikaryon, Nucleus of ventral horn cell, and neuropil}
    
%     \begin{center}
%       \Image{0.8\columnwidth}{images/week-3-7f.png}
%     \end{center}

%     \item \jjj{Surrounding meninges}
    
%     \begin{center}
%       \Image{0.72\columnwidth}{images/week-3-7h.png}
%     \end{center}

%     \item \jjj{Central canal of spinal cord}
    
%     \begin{center}
%       \Image{0.8\columnwidth}{images/week-3-7i.png}
%     \end{center}

%   \end{itemize}
%   \end{multicols}

%   \newpage

%   \subsection{Cerebellum}
%   \begin{multicols}{2}
%   \begin{itemize}
%     \item \jjj{Cerebellum
%     }
    
%     \begin{center}
%       \Image{0.9\columnwidth}{images/week-3-8a.png}
%     \end{center}

%     \item \jjj{Molecular (Mol) and granular (Gr) layer}
    
%     \begin{center}
%       \Image{0.9\columnwidth}{images/week-3-8b.png}
%     \end{center}

%     \item \jjj{Granular nuclei count}
    
%     \begin{center}
%       \Image{0.9\columnwidth}{images/week-3-8c.png}
%     \end{center}

%     \vspace{30pt}

%     \item \jjj{Molecular nuclei count}
    
%     \begin{center}
%       \Image{0.8\columnwidth}{images/week-3-8f.png}
%     \end{center}

%     \item \jjj{Purkinje cells}: large neurons found at the interface between the two layers of the cerebellum.
    
%     \begin{center}
%       \Image{0.8\columnwidth}{images/week-3-8e.png}
%     \end{center}

%     \item \jjj{Extension of Purkinje axons}: 
    
%     \begin{center}
%       \Image{0.8\columnwidth}{images/week-3-8d.png}
%     \end{center}

%   \end{itemize}
%   \end{multicols}

%   \subsection{Sympathetic Nerve}
%   \begin{multicols}{2}
%   \begin{itemize}
%     \item \jjj{Satellite cells}: glial cells with small nuclei at the periphery of nerve cell bodies.

%      \begin{center}
%       \Image{0.8\columnwidth}{images/week-3-9a.png}
%     \end{center}
    
%     \item \jjj{Ganglion Cells}: large, nerve cell bodies with prominent nuclei and nucleoli.

%      \begin{center}
%       \Image{0.8\columnwidth}{images/week-3-9b.png}
%     \end{center}

%     \item \jjj{Nerves}

%     \begin{center}
%       \Image{0.8\columnwidth}{images/week-3-9c.png}
%     \end{center}

%     \item \jjj{Lipofuscin}: yellow-brown pigment located in the cytoplasm. It is end-stage lysosomes that accumulates with age.
    
%     \begin{center}
%       \Image{0.8\columnwidth}{images/week-3-9d.png}
%     \end{center}
  
%   \end{itemize}
%   \end{multicols}

%   \newpage

%   \subsection{Enteric Nervous System}
%   \begin{multicols}{2}
%   \begin{itemize}
%     \item \jjj{Mucosa}
    
%     \begin{center}
%       \Image{0.8\columnwidth}{images/week-3-10aa.png}
%     \end{center}

%     \item \jjj{Epithelium}

%      \begin{center}
%       \Image{0.8\columnwidth}{images/week-3-10a.png}
%     \end{center}
    
%     \item \jjj{Submucosa}: dense irregular connective tissue.

%      \begin{center}
%       \Image{0.8\columnwidth}{images/week-3-10b.png}
%     \end{center}

%     \item \jjj{Meissner's Plexus}: provides secretory innervation of goblet cells and motor innervation of the muscularis mucosae.
    
%     \begin{center}
%       \Image{0.8\columnwidth}{images/week-3-10c.png}
%     \end{center}
    
%     \item \jjj{Muscularis Externa}: two orthogonal layers of smooth muscle.

%      \begin{center}
%       \Image{0.8\columnwidth}{images/week-3-10d.png}
%     \end{center}

%     \item \jjj{Auerbach's Plexus}: provides motor innervation of the muscularis externa.

%      \begin{center}
%       \Image{0.8\columnwidth}{images/week-3-10f.png}
%     \end{center}
%   \end{itemize}
%   \end{multicols}
  
% \end{itemize}