\chapter{Week 8: Endocrine, Male Reproductive}

\section{Endocrine Glands}

\begin{center}
  \Image{0.5\columnwidth}{images/week-8-eno.jpg}
\end{center}

\subsection{Pituitary Gland}
The pituitary is often called the ``master gland'' of the body because it produces hormones that regulate other endocrine glands, as well as, have direct effects on target tissues.
\begin{center}
  \Image{0.65\columnwidth}{images/week-8-hormones.jpg}
\end{center}
\begin{center}
  \Image{0.495\columnwidth}{images/week-8-a.jpg}
  \Image{0.49\columnwidth}{images/week-8-aa.jpg}
\end{center}
\bigskip
\begin{multicols}{2}
\begin{itemize}
  \item \jjj{Anterior pituitary}
  
  \begin{center}
    \Image{0.94\columnwidth}{images/week-8-a1.jpg}
  \end{center}
  
  \item \jjj{Chromophils}: stain with H\&E and secrete hormones. 
  
  \begin{center}
    \Image{0.94\columnwidth}{images/week-8-a2.jpg}
  \end{center}
  
  \vspace{50pt}

  \item \jjj{Acidophils}: stain dark red with azan.
  
  \begin{center}
    \Image{0.6\columnwidth}{images/week-8-a3.jpg}
  \end{center}
  
  \item \jjj{Basophils}: stain bluish-purple with H\&E
  
  \begin{center}
    \Image{0.6\columnwidth}{images/week-8-a4.jpg}
  \end{center}
  
  \item \jjj{Chromophobes}: stain light blue with azan and do not secrete hormones.
  
  \begin{center}
    \Image{0.6\columnwidth}{images/week-8-a5.jpg}
  \end{center}
  
  \item \jjj{Sinusoidal capillaries}: extensive network that receives hormones from acidophils and basophils.
  
  \begin{center}
    \Image{0.85\columnwidth}{images/week-8-a6.jpg}
  \end{center}
  
  \item \jjj{Pars intermedia}:  thin remnant (<2\%) at interface between the anterior and posterior lobes that contains numerous colloid (protein)-filled cysts (Rathke's cysts)
  
  \begin{center}
    \Image{0.85\columnwidth}{images/week-8-a7.jpg}
  \end{center}

  \item \jjj{Rathke's Cysts}:
  
  \begin{center}
    \Image{0.85\columnwidth}{images/week-8-a8.jpg}
  \end{center}
  
  \item \jjj{Posterior pituitary}: axons from the hypothalamus that release hormones into the capillaries of the pars nervosa. 
  
  \begin{center}
    \Image{0.85\columnwidth}{images/week-8-a9.jpg}
  \end{center}
  
  \item \jjj{Herring bodies}: dilations of axons filled with neuro-secretion vesicles.
  
  \begin{center}
    \Image{0.85\columnwidth}{images/week-8-a10.jpg}
  \end{center}

  \item \jjj{Pituicytes}: most nuclei belong to these glial cells.
  
  \begin{center}
    \Image{0.85\columnwidth}{images/week-8-a11.jpg}
  \end{center}
  
  
\end{itemize}
\end{multicols}

\subsection{Adrenal Gland}
\begin{center}
  \Image{0.94\columnwidth}{images/week-8-b.jpg}
\end{center}
\begin{multicols}{2}
\begin{itemize}
  \item \jjj{Capsule}: enclosed by a thin layer of connective tissue.
  
  \begin{center}
    \Image{0.94\columnwidth}{images/week-8-b1.jpg}
  \end{center}
  
  \item \jjj{Afferent blood vessels}:  penetrate the capsule and branch into sinusoids that supply the cortex and medulla.
  
  \begin{center}
    \Image{0.94\columnwidth}{images/week-8-b2.jpg}
  \end{center}
  
  \vspace{20pt}

  \item \jjj{Cortex}:  cells that synthesize and secrete steroid hormones. 
  
  \begin{center}
    \Image{0.85\columnwidth}{images/week-8-b3.jpg}
  \end{center}
  
  \item \jjj{Zona glomerulosa}:  outer zone (15\%) of glomerular-like clusters of cells and secrete aldosterone.
  
  \begin{center}
    \Image{0.85\columnwidth}{images/week-8-b4.jpg}
  \end{center}
  
  \item \jjj{Zona fasciculata}: middle zone (80\%) of two-cell wide vertical cords that secrete cortisol. 
  
  \begin{center}
    \Image{0.85\columnwidth}{images/week-8-b5.jpg}
  \end{center}
  
  \item \jjj{Zona reticularis}: inner zone (7\%) of one-cell wide anastomosing rows that secrete precursors of testosterone. 
  
  \begin{center}
    \Image{0.85\columnwidth}{images/week-8-b6.jpg}
  \end{center}
  
  \item \jjj{Lipofuscin pigment}: 
  
  \begin{center}
    \Image{0.85\columnwidth}{images/week-8-b62.jpg}
  \end{center}

  \item \jjj{Sinusoidal capillaries}: rich network of blood vessels.
  
  \begin{center}
    \Image{0.85\columnwidth}{images/week-8-b7.jpg}
  \end{center}
  
  \item \jjj{Medulla}: 
  
  \begin{center}
    \Image{0.85\columnwidth}{images/week-8-b8.jpg}
  \end{center}
  
  \item \jjj{Chromaffin cells}:  modified postganglionic sympathetic neurons that secrete catecholamines 
  
  \begin{center}
    \Image{0.85\columnwidth}{images/week-8-b9.jpg}
  \end{center}
  
  \item \jjj{Ganglion cells}:  infrequent sympathetic ganglion cells.
  
  \begin{center}
    \Image{0.8\columnwidth}{images/week-8-b10.jpg}
  \end{center}
  
  \item \jjj{Medullary vessels}: large veins that drain the organ.
  
  \begin{center}
    \Image{0.8\columnwidth}{images/week-8-b11.jpg}
  \end{center}
  
\end{itemize}
\end{multicols}


\subsection{Questions}
\begin{itemize}\color{minor}
  \item What hormone is produced by the zona glomerulosa?
  \basec{\begin{itemize}
    \item Aldosterone, which plays a central role in regulating blood pressure and certain electrolytes.
  \end{itemize}}

  \item By the zona fasciculata?
  \basec{\begin{itemize}
    \item Cortisol, involved in response to stress.
  \end{itemize}}
  
  \item By the zona reticularis?
  \basec{\begin{itemize}
    \item Androgens, various sex hormones.
  \end{itemize}}
  
  \item By the medulla?
  \basec{\begin{itemize}
    \item Epinephrine and norepinephrine. 
  \end{itemize}}
  
  \item What hormones regulate the function of the cortex?
  \basec{\begin{itemize}
    \item Corticotropin-releasing hormone (CRH) and vasopressin. 
  \end{itemize}}
  
  \item How is medullary function regulated?
  \basec{\begin{itemize}
    \item CRH and adrenocorticotropin hormone (ACTH). 
  \end{itemize}}
\end{itemize}

\newpage
\subsection{Thyroid}
The thyroid gland is a bilobed endocrine gland. It is unique in that it stores its hormones bound to an extracellular pool of protein (colloid)
\begin{center}
  \Image{0.94\columnwidth}{images/week-8-c.jpg}
\end{center}
\begin{multicols}{2}
\begin{itemize}
  \item \jjj{Capsule}: enclosed by a thin layer of connective tissue.
  
  \begin{center}
    \Image{0.55\columnwidth}{images/week-8-c1.jpg}
  \end{center}
  
  \item \jjj{Trabeculae}: connective tissue extends inwards from the capsule to partially outline irregular lobes and lobules.
  
  \begin{center}
    \Image{0.55\columnwidth}{images/week-8-c2.jpg}
  \end{center}
  
  \item \jjj{Thyroid follicles}: spherical follicles of varying size (50 to 500 µm) in which thyroid hormones are stored.
   
  \begin{center}
    \Image{0.94\columnwidth}{images/week-8-c3.jpg}
  \end{center}
  
  \item \jjj{Colloid}: lumen of each follicle is filled with the gel-like mass called colloid. It is mostly the protein thyroglobulin (pink) and bound thyroid hormones (triiodothyronine and tetraiodothyronine (or thyroxin)). The clear space around the colloid is a shrinkage artifact.
  
  \begin{center}
    \Image{0.94\columnwidth}{images/week-8-c5.jpg}
  \end{center}
  
  \item \jjj{Follicular cells}: follicles are lined by a simple cuboidal to columnar epithelium depending on functional activity. Secrete thyroid hormones when active.

  
  \begin{center}
    \Image{0.94\columnwidth}{images/week-8-c6.jpg}
  \end{center}
  
  \item \jjj{Capillaries}: a rich network surrounds each follicle.
  
  \begin{center}
    \Image{0.8\columnwidth}{images/week-8-c7.jpg}
  \end{center}
  
  \item \jjj{Parafollicular cells}: small numbers of larger cells located at the periphery of follicles that secrete calcitonin. They stain poorly with H\&E making identification difficult.
  
  \begin{center}
    \Image{0.8\columnwidth}{images/week-8-c8.jpg}
  \end{center}
  
\end{itemize}
\end{multicols}

\newpage

\section{Male Reproductive System}

\subsection{Testis (Adult and Neonatal)}
Testes are responsible for the production of sperm (spermatogenesis) and secretion of male sex hormones (testosterone). The production of sperm occurs within the seminiferous tubules that make up most of the testis.
\begin{center}
  \Image{0.94\columnwidth}{images/week-8-d.jpg}
  \Image{0.94\columnwidth}{images/week-8-dd.jpg}
\end{center}
\begin{multicols}{2}
\begin{itemize}
  \item \jjj{Seminiferous tubules}:  each lobule contains 1 to 4 highly-coiled seminiferous tubules lined by a germinal epithelium that is the site of sperm production. 
  
  \begin{center}
    \Image{0.73\columnwidth}{images/week-8-d1.jpg}
  \end{center}
  
  \item \jjj{Sertoli cells}: large, columnar cells that extend the full thickness of the germinal epithelium that separate the basal epithelial compartment (of spermatogonia) from the luminal compartment (of spermatocytes, spermatids and sperm).
  
  \begin{center}
    \Image{0.73\columnwidth}{images/week-8-d2.jpg}
  \end{center}
  
  \item \jjj{Spermatogonia}: single layer of germ cells resting on the basement membrane.
  
  \begin{center}
    \Image{0.73\columnwidth}{images/week-8-d3.jpg}
  \end{center}
  
  \item \jjj{Primary spermatocytes}: arise from spermatogonia and cross from the basal epithelial to luminal compartment of the germinal epithelium.
  
  \begin{center}
    \Image{0.94\columnwidth}{images/week-8-d4.jpg}
  \end{center}
  
  \item \jjj{Spermatids}: Small, spherical cells (8 µm or less) with intensely stained nuclei near the lumen that arise from secondary spermatocytes and undergo spermiogenesis to transform into sperm and embedded in the cytoplasm of Sertoli cells.
  
  \begin{center}
    \Image{0.94\columnwidth}{images/week-8-d5.jpg}
  \end{center}
  
  \item \jjj{Leydig cells}:  Large, round cells (20 to 30 µm diameter; usually clustered) with vesicular nuclei and eosinophilic cytoplasm found in the connective tissue (or interstitium) between seminiferous tubules that secrete testosterone.
  
  \begin{center}
    \Image{0.94\columnwidth}{images/week-8-d6.jpg}
  \end{center}
  
  \item \jjj{Mediastinum}: region in which seminiferous tubules converge and sperm exits the testis. 
  
  \begin{center}
    \Image{0.6\columnwidth}{images/week-8-d7.jpg}
  \end{center}
  
  \item \jjj{Straight tubules (tubuli recti)}: short, terminal section of each seminiferous tubule lined only by Sertoli cells.
  
  \begin{center}
    \Image{0.6\columnwidth}{images/week-8-d8.jpg}
  \end{center}
  
  \item \jjj{Rete testis}: straight tubules empty in an anastomosing labyrinth lined by a simple cuboidal or columnar epithelium.
  
  \begin{center}
    \Image{0.85\columnwidth}{images/week-8-d9.jpg}
  \end{center}
  
  \item \jjj{Tunica albuginea}:  capsule of thick connective tissue.
  
  \begin{center}
    \Image{0.85\columnwidth}{images/week-8-d10.jpg}
  \end{center}
  
  \item \jjj{Lobules}:  pyramid shaped lobules separated septae by of connective tissue that extend inward from the capsule.
  
  \begin{center}
    \Image{0.85\columnwidth}{images/week-8-d11.jpg}
  \end{center}
  
  \item \jjj{Seminiferous tubules}: each lobule contains 1 to 4 highly-coiled seminiferous tubules with an immature germinal epithelium and small lumen. 
  
  \begin{center}
    \Image{0.8\columnwidth}{images/week-8-d12.jpg}
  \end{center}
  
  \item \jjj{Sertoli cells}: most of the cells within the tubule. The columnar cells have dark, round to oval nuclei.
  
  \begin{center}
    \Image{0.8\columnwidth}{images/week-8-d13.jpg}
  \end{center}
  
  \item \jjj{Spermatogonia}: single layer of germ cells resting on the basement membrane. Large cells with a thin rim of lightly stained cytoplasm.
  
  \begin{center}
    \Image{0.8\columnwidth}{images/week-8-d14.jpg}
  \end{center}
  
  \item \jjj{Leydig cells}: found in the connective tissue (or interstitium) between seminiferous tubules. 
  
  \begin{center}
    \Image{0.8\columnwidth}{images/week-8-d15.jpg}
  \end{center}
  
\end{itemize}
\end{multicols}

\newpage

\subsection{Epididymis}
Sperm leave the testes and enter the epididymis. Each epididymis is a long, tightly coiled duct in which sperm undergo maturation as they move through it. Mature sperm are stored in the tail of the epididymis.
\begin{center}
  % \Image{0.94\columnwidth}{images/week-8-e.jpg}
\end{center}
\begin{multicols}{2}
\begin{itemize}
  \item 
\end{itemize}
\end{multicols}

\subsection{Ductus Deferens}
Ductus deferens (vas deferens) is a thick walled, fibromuscular tube that is continuous with the epididymis. Peristaltic movements propel sperm through the duct.
\begin{center}
  % \Image{0.94\columnwidth}{images/week-8-k.jpg}
\end{center}
\begin{multicols}{2}
\begin{itemize}
  \item 
\end{itemize}
\end{multicols}

\subsection{Seminal Vesicle}
The seminal vesicles are unbranched, highly-coiled tubular glands. Their secretions make up 60 percent of the volume of the semen. This fluid is high in fructose that acts as the main energy source for sperm outside the body.
\begin{center}
  % \Image{0.94\columnwidth}{images/week-8-l.jpg}
\end{center}
\begin{multicols}{2}
\begin{itemize}
  \item 
\end{itemize}
\end{multicols}

\subsection{Prostate Gland}
The prostate is composed of compound tubuloalveolar glands that contributes a slightly alkaline fluid to semen. These secretions help neutralize the acidity of the vagina, prolonging the lifespan of sperm.
\begin{center}
  % \Image{0.94\columnwidth}{images/week-8-m.jpg}
\end{center} 
\begin{multicols}{2}
\begin{itemize}
  \item 
\end{itemize}
\end{multicols}

\subsection{Penis With Penile Urethra}
The penis is composed of three cylindrical bodies of erectile tissue.
\begin{center}
  % \Image{0.94\columnwidth}{images/week-8-n.jpg}
\end{center} 
\begin{multicols}{2}
\begin{itemize}
  \item 
\end{itemize}
\end{multicols}

\subsection{Questions}
\begin{itemize}\color{minor}
  \item Give an overview of the process of spermatogenesis and spermiogenesis.
  \basec{\begin{itemize}
    \item 
  \end{itemize}}

  \item What is the effect of castration on the accessory sex glands (prostate and seminal vesicles)?
  \basec{\begin{itemize}
    \item 
  \end{itemize}}

  \item Where is the primary source of testosterone?
  \basec{\begin{itemize}
    \item 
  \end{itemize}}

  \item Where is the principal site of storage of spermatozoa in the male reproductive system?
  \basec{\begin{itemize}
    \item 
  \end{itemize}}

  \item Follow the passage of spermatozoa from the seminiferous tubules of the testis up to ejaculation.
  \basec{\begin{itemize}
    \item 
  \end{itemize}}

  \item Which organs are the major sources of seminal fluid?
  \basec{\begin{itemize}
    \item 
  \end{itemize}}

  \item What are the components of the blood-testis barrier, and what is its significance?
  \basec{\begin{itemize}
    \item 
  \end{itemize}}
\end{itemize}

