\chapter{Week 6: Lymphatic, Digestive}
% chktex-file 8
\section{Lymphatic}
\begin{itemize}
  \item[]

  \subsection{Tonsils}
  \begin{itemize}
    \item Tonsils an example of mucosa-associated lymphoid tissue (\jjj{MALT}). The lymphocytes are distributed as diffuse, non-encapsulated nodules in the underlying connective tissue.
  \end{itemize}
  \begin{multicols}{2}
  \begin{itemize}
    
    \item \jjj{Stratified Squamous Non-Kerantized Epithelium}:  covers the numerous nodules that compromise the palatine tonsil.
    
    \begin{center}
      \Image{0.75\columnwidth}{images/week-6-1a.jpg}
    \end{center}
    
    \item \jjj{Lymph Nodules}: spherical aggregations of lymphocytes that usually have germinal centers.
    
    \begin{center}
      \Image{0.7\columnwidth}{images/week-6-2a.jpg}
    \end{center}

    \item \jjj{Submucosa}
    
    \begin{center}
      \Image{0.7\columnwidth}{images/week-6-8a.jpg}
    \end{center}
    
    
    \item \jjj{Germinal centers}
    
    \begin{center}
      \Image{0.7\columnwidth}{images/week-6-3a.jpg}
    \end{center}

    \item \jjj{Crypts}: infoldings of the epithelium into the underlying connective tissue. 
    
    \begin{center}
      \Image{0.7\columnwidth}{images/week-6-4a.jpg}
    \end{center}
    
    \item \jjj{Lymphocytes}
    
    \begin{center}
      \Image{0.7\columnwidth}{images/week-6-5a.jpg}
    \end{center}

    \item \jjj{Sequestered crypts}: usually inflamed and filled with debris and lymphocytes 
    
    \begin{center}
      \Image{0.85\columnwidth}{images/week-6-6a.jpg}
    \end{center}
    
    \item \jjj{Plasma cells}:  large numbers of plasma cells are usually seen in the underlying connective tissue near the epithelium.
    
    \begin{center}
      \Image{0.75\columnwidth}{images/week-6-7a.jpg}
    \end{center}
    
  \end{itemize}
  \end{multicols}

  \subsection{Lymph Nodes}\label{Lymph Nodes}
  \begin{multicols}{2}
  \begin{itemize}
    \item \jjj{Capsule}: dense connective tissue enclosing the node.
    
    \begin{center}
      \Image{0.6\columnwidth}{images/week-6-2b.jpg}
      \Image{0.6\columnwidth}{images/week-6-1b.jpg}
      \Image{0.6\columnwidth}{images/week-6-4b.jpg}
    \end{center}
    
    \item \jjj{Subcapsular Sinus}: space underneath the capsule that receives lymph from afferent lymphatic vessels.
    
    \begin{center}
      \Image{0.7\columnwidth}{images/week-6-3b.jpg}
      \Image{0.7\columnwidth}{images/week-6-6b.jpg}
      \Image{0.7\columnwidth}{images/week-6-5b.jpg}
    \end{center}
    
    \item \jjj{Trabeculae}: connective tissue that extends inward from the capsule.
    
    \begin{center}
      \Image{0.7\columnwidth}{images/week-6-9b.jpg}
      \Image{0.7\columnwidth}{images/week-6-7b.jpg}
      \Image{0.7\columnwidth}{images/week-6-8b.jpg}
    \end{center}

    \item \jjj{Cortex}: reticular fibers form an irregular, anastomosing network in the outer region of the node. Nodules are enclosed by reticular fibers.
    
    \begin{center}
      \Image{0.5\columnwidth}{images/week-6-10b.jpg}
      \Image{0.7\columnwidth}{images/week-6-12b.jpg}
    \end{center}

    \item \jjj{Inner Cortex}: region between the outer cortex and the medulla that is free of nodules. 
    
    \begin{center}
      \Image{0.7\columnwidth}{images/week-6-11b.jpg}
    \end{center}
    
    \item \jjj{Medulla}: inner part of the node. 
    
    \begin{center}
      \Image{0.7\columnwidth}{images/week-6-13b.jpg}
      \Image{0.7\columnwidth}{images/week-6-14b.jpg}
    \end{center}
  \end{itemize}
\end{multicols}

  \subsection{Thymus}\label{Thymus}
  \begin{multicols}{2}
  \begin{itemize}
    \item \jjj{Capsule (neonatal/adult)}: thin connective tissue layer surrounding the thymus that extends inwards to form incomplete lobules.
    
    \begin{center}
      \Image{0.7\columnwidth}{images/week-6-1c.jpg}
      \Image{0.7\columnwidth}{images/week-6-2c.jpg}
    \end{center}

    \item \jjj{Cortex (neonatal/adult)}: outer darker, region of small lymphocytes.
    
    \begin{center}
      \Image{0.55\columnwidth}{images/week-6-3c.jpg}
      \Image{0.55\columnwidth}{images/week-6-4c.jpg}
    \end{center}
    
    \item \jjj{T Lymphocytes}: small nuclei of condensed chromatin.
    
    \begin{center}
      \Image{0.85\columnwidth}{images/week-6-5c.jpg}
    \end{center}
    
    \item \jjj{Epithelial Reticular Cells}
    
    \begin{center}
      \Image{0.85\columnwidth}{images/week-6-6c.jpg}
    \end{center}
    
    \item \jjj{Macrophages}: large cells that phagocytize T cells marked for removal.
    
    \begin{center}
      \Image{0.85\columnwidth}{images/week-6-7c.jpg}
    \end{center}
    
    \item \jjj{Medulla}: inner, lighter region of larger lymphocytes. 
    
    \begin{center}
      \Image{0.92\columnwidth}{images/week-6-8c.jpg}
    \end{center}
    \
    \item \jjj{Hassal's Corpuscles}: closely packed, concentrically arranged epithelial reticular cells.
    
    \begin{center}
      \Image{0.555\columnwidth}{images/week-6-9c.jpg}
      \Image{0.555\columnwidth}{images/week-6-10c.jpg}
    \end{center}
    
  \end{itemize}
  \end{multicols}

  \subsection{Spleen}\label{Spleen}
  \begin{multicols}{2}
  \begin{itemize}
    \item \jjj{Capsule}: dense connective tissue enclosing the organ.
    
    \begin{center}
      \Image{0.8\columnwidth}{images/week-6-1d.jpg}
      \Image{0.8\columnwidth}{images/week-6-2d.jpg}
    \end{center}
    
    \item \jjj{Trabeculae}:  connective tissue that extends inward from the capsule through which blood vessels enter the pulp
    
    \begin{center}
      \Image{0.65\columnwidth}{images/week-6-3d.jpg}
      \Image{0.65\columnwidth}{images/week-6-4d.jpg}
    \end{center}
    
    \item \jjj{Blood vessels}
    
    \begin{center}
      \Image{0.6\columnwidth}{images/week-6-5d.jpg}
      \Image{0.6\columnwidth}{images/week-6-6d.jpg}
    \end{center}
    
    \item \jjj{White Pulp}: composed of lymphatic tissue. It appears basophilic due to the large number of nuclei.
    
    \begin{center}
      \Image{0.7\columnwidth}{images/week-6-7d.jpg}
      \Image{0.7\columnwidth}{images/week-6-8d.jpg}
    \end{center}
    
    \item \jjj{Splenic Nodules}: clusters of B lymphocytes located on central arterioles. They usually contain a germinal center of activated B lymphocytes.
    
    \begin{center}
      \Image{0.9\columnwidth}{images/week-6-9d.jpg}
      \Image{0.9\columnwidth}{images/week-6-10d.jpg}
      \Image{0.9\columnwidth}{images/week-6-11d.jpg}
    \end{center}

    \item \jjj{Central Aretrioles} branches of trabecular arteries coated by PALS and adjacent to nodules.
    
    \begin{center}
      \Image{0.7\columnwidth}{images/week-6-12d.jpg}
      \Image{0.7\columnwidth}{images/week-6-13d.jpg}
      \Image{0.7\columnwidth}{images/week-6-14d.jpg}
    \end{center}
    
    \item \jjj{Marginal Zone}: region between white and red pulp where macrophages, dendritic cells, and lymphocytes interact.
    
    \begin{center}
      \Image{0.6\columnwidth}{images/week-6-15d.jpg}
    \end{center}
    
    \item \jjj{Red pulp}: filters and degrades red blood cells (RBCs). It appears eosinophilic due to the large number of RBCs.
    
    \begin{center}
      \Image{0.85\columnwidth}{images/week-6-16d.jpg}
      \Image{0.85\columnwidth}{images/week-6-17d.jpg}
      \Image{0.85\columnwidth}{images/week-6-18d.jpg}
    \end{center}
    
    \newpage

    \item \jjj{Splenic Sinusoids}: vascular spaces lined by specialized endothelial cells that filter RBCs. 
    
    \begin{center}
      \Image{0.945\columnwidth}{images/week-6-19d.jpg}
      \Image{0.945\columnwidth}{images/week-6-20d.jpg}
      \Image{0.945\columnwidth}{images/week-6-21d.jpg}
    \end{center}
    
    \item \jjj{Specialized Endothelial Cells}
    
    \begin{center}
      \Image{0.8\columnwidth}{images/week-6-22d.jpg}
    \end{center}

    \item \jjj{Splenic Cords}: loose connective tissue supported by a meshwork of reticular fibers, and contains loose connective tissue supported by a meshwork of reticular fibers.
    
    \begin{center}
      \Image{0.94\columnwidth}{images/week-6-23d.jpg}
      \Image{0.94\columnwidth}{images/week-6-24d.jpg}
    \end{center}
    
    \item \jjj{Pulp Arterioles}: not surrounded by lymphocytes like central arterioles in white pulp and surrounded by layer of reticular fibers.
    
    \begin{center}
      \Image{0.94\columnwidth}{images/week-6-25d.jpg}
      \Image{0.7\columnwidth}{images/week-6-26d.jpg}
      \Image{0.7\columnwidth}{images/week-6-27d.jpg}
    \end{center}
    
    
  \end{itemize}
  \end{multicols}

  \subsection{Questions}\label{Questions}
  \begin{enumerate}\color{minor}
    \item Which lymphatic organs have afferent lymphatic vessels?
      \begin{itemize}\color{text}
        \item Found only in lymph nodes; efferent are found in thymus and spleen.
      \end{itemize}
    \item How do lymphocytes enter:
      \begin{enumerate}
        \item Lymph nodes?
          \begin{itemize}\color{text}
            \item Via blood vessel walls or afferent lymphatic vessels.
          \end{itemize}
        \item MALT\@?
          \begin{itemize}\color{text}
            \item Via efferent lymphatic vessels.
          \end{itemize}
      \end{enumerate}
    \item What are the components of the blood thymic barrier?
    \begin{itemize}\color{text}
      \item Epithelial reticular cells, basal laminae, and endothelial cells joined by tight junctions.
    \end{itemize}
    \item Which of the lymphatic organs filters blood?
      \begin{itemize}\color{text}
        \item Spleen.
      \end{itemize}
  \end{enumerate}
  
\end{itemize}

\newpage
\section{Digestive}\label{Digestive}
\begin{itemize}
  \item \jjj{Gastrointestinal Tract}
  
  \Image{0.9\columnwidth}{images/week-6-1e.jpg}
  

  \subsection{Tongue}\label{Tongue}
  \begin{multicols}{2}
  \begin{itemize}
    \item \jjj{Overview of the Tongue}
    
    \begin{center}
      \Image{0.7\columnwidth}{images/week-6-e.jpg}
    \end{center}
    
    \item \jjj{Stratified Squamous Non-Kerantized Epithelium}
    
    \begin{center}
      \Image{0.5\columnwidth}{images/week-6-2e.jpg}
    \end{center}

    \item \jjj{Dermal Papillae}: ridges of connective tissue that project into the epithelium that reduce its mobility and brings blood vessels in close contact with the epithelial cells. 
    
    \begin{center}
      \Image{0.7\columnwidth}{images/week-6-3e.jpg}
    \end{center}
    
    \item \jjj{Foliate Papillae}: parallel ridges on the lateral edges of the tongue separated by deep mucosal furrows.
    
    \begin{center}
      \Image{0.8\columnwidth}{images/week-6-4e.jpg}
    \end{center}

    \item \jjj{Furrows}: separate each papillae and receive saliva from the minor lingual glands.
    
    \begin{center}
      \Image{0.8\columnwidth}{images/week-6-5e.jpg}
    \end{center}
    
    \item \jjj{Taste Buds}: elliptical structures found in the epithelium of the furrows that contain cells with taste receptors. The circular opening at its apex is the taste pore.
    
    \begin{center}
      \Image{0.8\columnwidth}{images/week-6-6e.jpg}
    \end{center}

    \item \jjj{Taste Pore}
    
    \begin{center}
      \Image{0.8\columnwidth}{images/week-6-7e.jpg}
    \end{center}
    
    \item \jjj{Skeletal Muscle}: arranged in three bundles at right angles to each other to allow flexibility and precision in movements of the tongue.
    
    \begin{center}
      \Image{0.8\columnwidth}{images/week-6-8e.jpg}
    \end{center}
    
    \item \jjj{Serous Glands}: secrete a fluid that contains digestive enzymes.
    
    \begin{center}
      \Image{0.9\columnwidth}{images/week-6-9e.jpg}
    \end{center}
    
    \item \jjj{Mucous Glands}: secrete a fluid that contains mucus (glycoproteins known as mucins).
    
    \begin{center}
      \Image{0.8\columnwidth}{images/week-6-10e.jpg}
    \end{center}
    
  \end{itemize}
  \end{multicols}

  \subsection{Esophagus}\label{Esophagus}
  
  \begin{center}
    \Image{0.5\columnwidth}{images/week-6-f.jpg}
  \end{center}
  
  \begin{multicols}{2}
  \begin{itemize}
    \item \jjj{Stratified Squamous Non-Kerantized Epithelium}: 
    
    \begin{center}
      \Image{0.75\columnwidth}{images/week-6-1f.jpg}
    \end{center}
    
    \item \jjj{Lamina Propria}: dense irregular connective tissue.
    
    \begin{center}
      \Image{0.8\columnwidth}{images/week-6-2f.jpg}
    \end{center}
    
    \item \jjj{Muscularis Mucosae}: smooth muscle.
    
    \begin{center}
      \Image{0.9\columnwidth}{images/week-6-3f.jpg}
    \end{center}
    
    \item \jjj{Submucosa}: dense irregular connective tissue.
    
    \begin{center}
      \Image{0.9\columnwidth}{images/week-6-4f.jpg}
    \end{center}

    \item \jjj{Mucus Glands}: only the esophagus and the duodenum have glands in the submucosa.
    
    \begin{center}
      \Image{0.9\columnwidth}{images/week-6-5f.jpg}
    \end{center}
    
    \item \jjj{Ducts}: usually have cuboidal or stratified cuboidal epithelium.
    
    \begin{center}
      \Image{0.94\columnwidth}{images/week-6-6f.jpg}
    \end{center}
    
    \item \jjj{Muscularis Externa}: contains both smooth and skeletal muscle because this specimen is from the middle third of the esophagus.   
    
    \begin{center}
      \Image{0.85\columnwidth}{images/week-6-7f.jpg}
    \end{center}
    
    \item \jjj{Inner Layer}
    
    \begin{center}
      \Image{0.85\columnwidth}{images/week-6-8f.jpg}
    \end{center}

    \item \jjj{Outer Layer}
    
    \begin{center}
      \Image{0.9\columnwidth}{images/week-6-9f.jpg}
    \end{center}
    
    \item \jjj{Auerbach's plexus}:  is found between the inner and outer layers of the muscularis externa. Ganglia with prominent capsule cells can be seen.
    
    \begin{center}
      \Image{0.8\columnwidth}{images/week-6-10f.jpg}
    \end{center}
    
  \end{itemize}
  \end{multicols}
  
  \subsection{Gastroesophageal Junction}\label{Gastroesophageal Junction}
  \begin{multicols}{2}
  \begin{itemize}
    \item \jjj{Gastroesophageal Junction}
    
    \begin{center}
      \Image{0.94\columnwidth}{images/week-6-g.jpg}
    \end{center}

    \item \jjj{Change in Epithelium}
    
    \begin{center}
      \Image{0.83\columnwidth}{images/week-6-1g.jpg}
    \end{center}
  \end{itemize}
  \end{multicols}
  \begin{itemize}
    \item How can you diagnose whether you are  
    looking at the upper or lower portion of the esophagus?
      \begin{itemize}
        \item The best way to tell is to look at the epithelium; the upper portion of the esophagus has stratified squamous while the lower portion (stomach) has simple columnar epithelium with long linear or coiled glands.
      \end{itemize}
  \end{itemize}

  \newpage

  \subsection{Stomach}\label{Stomach}
  \begin{multicols}{2}
  \begin{itemize}
    \item \jjj{Stomach}
    
    \begin{center}
      \Image{0.8\columnwidth}{images/week-6-h.jpg}
    \end{center}
    
    \item \jjj{Mucosa}: composed of the epithelium, lamina propria, and muscularis mucosae. 
    
    \begin{center}
      \Image{0.8\columnwidth}{images/week-6-1h.jpg}
    \end{center}

    \item \jjj{Gastric Pits}: invaginations of the surface epithelium.
    
    \begin{center}
      \Image{0.8\columnwidth}{images/week-6-2h.jpg}
    \end{center}
    
    \item \jjj{Gastric Glands}: tubular glands that extend from the base of the gastric pits to the muscularis mucosae.
    
    \begin{center}
      \Image{0.75\columnwidth}{images/week-6-3h.jpg}
    \end{center}

    \item \jjj{Lamina Propria}: reduced to small amounts of connective tissue found between gastric pits and glands.
    
    \begin{center}
      \Image{0.75\columnwidth}{images/week-6-3-2h.jpg}
    \end{center}
    

    \item \jjj{Muscularis Mucosae}: narrow layer of smooth muscle cells at the base of the mucosa.
    
    \begin{center}
      \Image{0.75\columnwidth}{images/week-6-4h.jpg}
    \end{center}
    
    \item \jjj{Submucosa}: dense irregular connective tissue.
    
    \begin{center}
      \Image{0.8\columnwidth}{images/week-6-5h.jpg}
    \end{center}
    
    \item \jjj{Muscularis Externa}:  three layers of muscle cells rather than two layers found elsewhere in the GI tract.
    
    \begin{center}
      \Image{0.8\columnwidth}{images/week-6-6h.jpg}
    \end{center}
    
    \item \jjj{Inner Oblique Layer}: this layer is unique to the stomach and is found near the boundary with the submucosa.
    
    \begin{center}
      \Image{0.8\columnwidth}{images/week-6-7h.jpg}
    \end{center}
    
    \item \jjj{Middle Circular Layer}: 
    
    \begin{center}
      \Image{0.94\columnwidth}{images/week-6-8h.jpg}
    \end{center}
    
    \item \jjj{Outer Longitudinal Layer}: 
    
    \begin{center}
      \Image{0.94\columnwidth}{images/week-6-9h.jpg}
    \end{center}
    
    \item \jjj{Adventitia}: loose irregular connective tissue.
    
    \begin{center}
      \Image{0.94\columnwidth}{images/week-6-10h.jpg}
    \end{center}
    
    \item \jjj{Cardiac Stomach}: the narrow region surrounding the opening of the esophagus that contains cardiac glands in the mucosa.
    
    \begin{center}
      \Image{0.9\columnwidth}{images/week-6-11h.jpg}
    \end{center}
    
    \item \jjj{Cardiac Glands}: appear as cross-sections of the coiled tubular glands of mostly mucus secreting cells that empty into the bottom of gastric pits. 
    
    \begin{center}
      \Image{0.7\columnwidth}{images/week-6-12h.jpg}
    \end{center}

  \end{itemize}
  \end{multicols}
  \begin{itemize}
    \item To what ultrastructural feature does the brush border correspond?
      \begin{itemize}
        \item The presence of mucous-secreting duodenal glands in its submucosa.
      \end{itemize}
  \end{itemize}

  \newpage

  \subsection{Small Intestine}\label{Small Intestine}
  \begin{multicols}{2}
  \begin{itemize}
    \item \jjj{Duodenum}: proximal portion adjacent to the stomach. 
    
    \begin{center}
      \Image{0.75\columnwidth}{images/week-6-i.jpg}
    \end{center}

    \item \jjj{Villi (Duodenum)}:  tall, slender finger-like projections that extend into the lumen.
    
    \begin{center}
      \Image{0.75\columnwidth}{images/week-6-1i.jpg}
    \end{center}
    
    \item \jjj{Brunner's Glands}: only region of the gastrointestinal tract (along with the esophagus) with glands in the submucosa.
    
    \begin{center}
      \Image{0.75\columnwidth}{images/week-6-2i.jpg}
    \end{center}
    
    \item \jjj{Jejunum}: middle portion. 
    
    \begin{center}
      \Image{0.75\columnwidth}{images/week-6-3i.jpg}
    \end{center}
    
    \item \jjj{Villi (Jejunum)}
    
    \begin{center}
      \Image{0.75\columnwidth}{images/week-6-4i.jpg}
    \end{center}
    
    \item \jjj{Ileum}: distal portion adjacent to the large intestine. 
    
    \begin{center}
      \Image{0.75\columnwidth}{images/week-6-5i.jpg}
    \end{center}
    
    \item \jjj{Villi (Ileum)}: short, broad finger-like projections with blunt ends that extend into the lumen.
    
    \begin{center}
      \Image{0.7\columnwidth}{images/week-6-6i.jpg}
    \end{center}
    
    \item \jjj{Peyer's Patches}: diffuse aggregations of lymphoid cells in the lamina propria.
    
    \begin{center}
      \Image{0.9\columnwidth}{images/week-6-7i.jpg}
    \end{center}
    
  \end{itemize}
  \end{multicols}
  
  \subsection{Large Intestine}\label{Large Intestine}
  \begin{itemize}
    \item \jjj{Colon}: The colon is composed of the four layers characteristic of the gastrointestinal tract. However, neither villi nor plicae circularis are present and goblet cells become more frequent.
    
    \begin{center}
      \Image{0.9\columnwidth}{images/week-6-k.jpg}
    \end{center}
    
  \end{itemize}

  \bigskip

  \begin{multicols}{2}
  \begin{itemize}
    \item \jjj{Mucosa}: mucous membrane.
    
    \begin{center}
      \Image{0.9\columnwidth}{images/week-6-1k.jpg}
    \end{center}
    
    \item \jjj{Epithelium}: with enterocytes (simple columnar cells with microvilli; i.e., the brush border) and goblet cells.
    
    \begin{center}
      \Image{0.72\columnwidth}{images/week-6-2k.jpg}
    \end{center}
    
    \item \jjj{Intestinal Crypts of Lieberkuhn}:  straight, unbranched, tubular glands.
    
    \begin{center}
      \Image{0.9\columnwidth}{images/week-6-3k.jpg}
    \end{center}
    
    \item \jjj{Lamina Propria}: abundant between cross-sections of the crypts. Many including cell types plasma cells, lymphocytes, eosinophils and macrophages can be seen.
    
    \begin{center}
      \Image{0.9\columnwidth}{images/week-6-4k.jpg}
    \end{center}
    
    \item \jjj{Muscularis Mucosae}: layer of smooth muscle.
    
    \begin{center}
      \Image{0.9\columnwidth}{images/week-6-5k.jpg}
    \end{center}
    
    \item \jjj{Submucosa}: dense irregular connective tissue. 
    
    \begin{center}
      \Image{0.86\columnwidth}{images/week-6-6k.jpg}
    \end{center}
    
    \item \jjj{Meissner's Plexus}: provides secretory innervation of goblet cells and motor innervation of the muscularis mucosae.
    
    \begin{center}
      \Image{0.86\columnwidth}{images/week-6-7k.jpg}
    \end{center}
    
    \item \jjj{Muscularis Externa}: two orthogonal layers of smooth muscle. 
    
    \begin{center}
      \Image{0.86\columnwidth}{images/week-6-8k.jpg}
    \end{center}
    
    \item \jjj{Inner Circular Layer}: smooth muscle.
    
    \vspace{20pt}

    \begin{center}
      \Image{0.94\columnwidth}{images/week-6-9k.jpg}
    \end{center}

    \item \jjj{Auerbach's Plexus}: provides motor innervation of the muscularis externa.
    
    \begin{center}
      \Image{0.58\columnwidth}{images/week-6-10k.jpg}
    \end{center}
    
    \item \jjj{Outer Longitudinal Layer}: smooth muscle.
    
    \begin{center}
      \Image{0.6\columnwidth}{images/week-6-11k.jpg}
    \end{center}
    
  \end{itemize}
  \end{multicols}
  
  \subsection{Appendix}\label{Appendix}
  \begin{itemize}
    \item \jjj{Appendix}: the appendix is composed of the four layers characteristic of the gastrointestinal tract.
    
    \hspace{15pt}\Image{0.8\columnwidth}{images/week-6-l.jpg}
  \end{itemize}


  \begin{multicols}{2}
  \begin{itemize}
    \item \jjj{Mucosa}
    
    \begin{center}
      \Image{0.9\columnwidth}{images/week-6-1l.jpg}
    \end{center}
    
    \item \jjj{Enterocytes}: simple columnar cells with microvilli (or brush border).
    
    \begin{center}
      \Image{0.9\columnwidth}{images/week-6-2l.jpg}
    \end{center}
    
    \item \jjj{M-Cells}: cover nodules and have a lower profile than absorptive cells (small folds on their surface versus microvilli on absorptive cells).
    
    \begin{center}
      \Image{0.9\columnwidth}{images/week-6-3l.jpg}
    \end{center}
    
    \item \jjj{Goblet Cells}: secrete mucus for lubrication
    
    \begin{center}
      \Image{0.9\columnwidth}{images/week-6-4l.jpg}
    \end{center}
    
    \item \jjj{Crypts}: very few.
    
    \begin{center}
      \Image{0.9\columnwidth}{images/week-6-5l.jpg}
    \end{center}
    
    \item \jjj{Lamina Propria}: comprises almost the entire mucosa.
    
    \begin{center}
      \Image{0.9\columnwidth}{images/week-6-6l.jpg}
    \end{center}
    
    \item \jjj{Nodules}: fill the lamina propria.
    
    \begin{center}
      \Image{0.94\columnwidth}{images/week-6-7l.jpg}
    \end{center}
    
    \item \jjj{Submucosa}: nodules may extend into the submucosa
    
    \begin{center}
      \Image{0.94\columnwidth}{images/week-6-8l.jpg}
    \end{center}
    
    \item \jjj{Muscularis Externa}: two orthogonal layers of smooth muscle (inner circular and outer longitudinal).
    
    \begin{center}
      \Image{0.75\columnwidth}{images/week-6-9l.jpg}
    \end{center}
    
    \item \jjj{Serosa}: covers the outer surface of the appendix.
    
    \begin{center}
      \Image{0.75\columnwidth}{images/week-6-10l.jpg}
    \end{center}
    
  \end{itemize}
  \end{multicols}

  \subsection{Rectum}\label{Rectum}
  \begin{center}
    \Image{0.8\columnwidth}{images/week-6-m.jpg}
  \end{center}
  \begin{multicols}{2}
  \begin{itemize}
    \item \jjj{Mucosa}:
    
    \begin{center}
      \Image{0.75\columnwidth}{images/week-6-1m.jpg}
    \end{center}
    
    \item \jjj{Villi}: cover the surface of the mucosa.
    
    \begin{center}
      \Image{0.75\columnwidth}{images/week-6-2m.jpg}
    \end{center}
    
    \item \jjj{Crypts}:
    
    \begin{center}
      \Image{0.7\columnwidth}{images/week-6-3m.jpg}
    \end{center}
    
    \item \jjj{Lamina Propria}: loose connective tissue that supports the epithelium and forms the core of villi.
    
    \begin{center}
      \Image{0.7\columnwidth}{images/week-6-4m.jpg}
    \end{center}
    
    \item \jjj{Muscularis Mucosae}: layer of smooth muscle
    
    \begin{center}
      \Image{0.75\columnwidth}{images/week-6-5m.jpg}
    \end{center}
    
    \item \jjj{Submucosa}: dense irregular connective tissue.
    
    \begin{center}
      \Image{0.75\columnwidth}{images/week-6-6m.jpg}
    \end{center}
    
    \item \jjj{Anal Columns}: vertical folds of the mucosa and submucosa that project into the lumen.
    
    \begin{center}
      \Image{0.75\columnwidth}{images/week-6-7m.jpg}
    \end{center}
    
    \item \jjj{Muscularis Externa}: two orthogonal layers of smooth muscle (inner circular and outer longitudinal).
    
    \begin{center}
      \Image{0.75\columnwidth}{images/week-6-8m.jpg}
    \end{center}
  \end{itemize}
  \end{multicols}

  \subsection{Recto-Anal Junction}
  \begin{center}
    \Image{0.94\columnwidth}{images/week-6-n.jpg}
  \end{center}
  
  \begin{multicols}{2}
  \begin{itemize}
      \item \jjj{Colorectal Zone}: left side of the specimen
      
      \begin{center}
        \Image{0.84\columnwidth}{images/week-6-1n.jpg}
      \end{center}
      
      \item \jjj{Anal Glands}: secrete mucus into the anal canal
      
      \begin{center}
        \Image{0.84\columnwidth}{images/week-6-2n.jpg}
      \end{center}
      
      \item \jjj{Internal Anal Sphincter}: an expansion of the inner circular layer of the muscularis externa
      
      \begin{center}
        \Image{0.84\columnwidth}{images/week-6-3n.jpg}
      \end{center}
      
      \item \jjj{Pectinate Line}: junction between the simple columnar epithelium of the colon and the stratified squamous epithelium of the skin
      
      \begin{center}
        \Image{0.75\columnwidth}{images/week-6-4n.jpg}
      \end{center}
      
      \item \jjj{Squamous Zone of the Anal Canal}: right side of the specimen. 
      
      \begin{center}
        \Image{0.75\columnwidth}{images/week-6-5n.jpg}
      \end{center}
      
      \item \jjj{Stratified Squamous Epithelium}:  initially is non-keratinized but becomes keratinized within a few millimeters.
      
      \begin{center}
        \Image{0.75\columnwidth}{images/week-6-6n.jpg}
      \end{center}
      
      \item \jjj{External Sphincter}: skeletal muscle that is part of the pelvic floor.
      
      \vspace{25pt}

      \begin{center}
        \Image{0.94\columnwidth}{images/week-6-7n.jpg}
      \end{center}

      \item \jjj{Circumanal Glands}: apocrine glands.
      
      \begin{center}
        \Image{0.6\columnwidth}{images/week-6-8n.jpg}
      \end{center}
      
      \item \jjj{Sebaceous Glands}: 
      
      \begin{center}
        \Image{0.6\columnwidth}{images/week-6-9n.jpg}
      \end{center}
      
  \end{itemize}
  \end{multicols}

  \subsection{Parotid Gland}\label{Parotid Gland}
  
  \Image{0.94\columnwidth}{images/week-6-o.jpg}
  
  \begin{multicols}{2}
  \begin{itemize}
    \item \jjj{Capsule}: connective tissue that encapsulates the gland.
    
    \begin{center}
      \Image{0.85\columnwidth}{images/week-6-1o.jpg}
    \end{center}
    
    \item \jjj{Lobules}:  connective tissue further divides lobes into lobules the smallest functional unit.
    
    \begin{center}
      \Image{0.85\columnwidth}{images/week-6-2o.jpg}
    \end{center}
    
    \item \jjj{Serous Cells}: arranged in acini of pyramidal serous cells. These polarized cells have rough endoplasmic reticulum at their base (basophilic) and secretion granules (eosinophilic) at their apex.
    
    \begin{center}
      \Image{0.85\columnwidth}{images/week-6-3o.jpg}
    \end{center}
    
    \item \jjj{Mucous Cells}: polarized cells with flattened nuclei at the bottom of the cells. They are very lightly stained with a ``foamy'' appearance (mucous has been extracted).
    
    \begin{center}
      \Image{0.94\columnwidth}{images/week-6-4o.jpg}
    \end{center}
    
    \item \jjj{Intercalated Ducts}: the smallest ducts that insert into and drain individual acini. They are more lightly stained than acini cells and are low cuboidal.
    
    \begin{center}
      \Image{0.8\columnwidth}{images/week-6-5o.jpg}
      \Image{0.8\columnwidth}{images/week-6-52o.jpg}
    \end{center}
    
    \item \jjj{Striate Ducts}: arise from intercalated ducts. They are columnar with basal striations and are surrounded by capillaries.
    
    \begin{center}
      \Image{0.7\columnwidth}{images/week-6-6o.jpg}
    \end{center}
    
    \item \jjj{Interlobular Ducts}: found outside of lobules. 
    
    \bigskip

    \begin{center}
      \Image{0.94\columnwidth}{images/week-6-7o.jpg}
    \end{center}
    
  \end{itemize}
  \end{multicols}
  
  \subsection{Liver}\label{Liver}
  \begin{center}
    \Image{0.5\columnwidth}{images/week-6-p.jpg}
  \end{center}
  \begin{multicols}{2}
  \begin{itemize}
    \item \jjj{Lobules}: individual lobules are seen as lighter areas with darker edges at low magnification
    
    \begin{center}
      \Image{0.7\columnwidth}{images/week-6-1q.jpg}
    \end{center}
    
    \item \jjj{Lobule Structure}: roughly hexagonal structure with a central vein at its center and six portal triads at its periphery.
    
    \begin{center}
      \Image{0.7\columnwidth}{images/week-6-2q.jpg}
    \end{center}
    
    \item \jjj{Central Vein}: large venule at the center of the lobule.
    
    \begin{center}
      \Image{0.8\columnwidth}{images/week-6-3q.jpg}
    \end{center}
    
    \item \jjj{Hepatocytes}: anastomosing plates, one cell thick, radiate outward from the central vein separated by sinusoidal capillaries and supported by reticular fibers.
    
    \begin{center}
      \Image{0.8\columnwidth}{images/week-6-4q.jpg}
    \end{center}
    
    \item \jjj{Portal Triads}: at the corners of each lobule.
    
    \begin{center}
      \Image{0.8\columnwidth}{images/week-6-5q.jpg}
    \end{center}
    
    \item \jjj{Bile Ducts}: lined with a simple cuboidal epithelium.
    
    \begin{center}
      \Image{0.8\columnwidth}{images/week-6-6q.jpg}
    \end{center}
    
  \end{itemize}
  \end{multicols}
  
  \subsection{Gall Bladder}\label{Gall Bladder}
  \begin{center}
    \Image{0.6\columnwidth}{images/week-6-r.jpg}  
  \end{center}
  \begin{multicols}{2}
  \begin{itemize}
    \item \jjj{Mucosa}: the empty bladder has numerous deep folds (or rugae) often resulting in the appearance of cross bridges.
    
    \begin{center}
      \Image{0.7\columnwidth}{images/week-6-1r.jpg}
    \end{center}
    
    \item \jjj{Cross Bridges}
    
    \begin{center}
      \Image{0.7\columnwidth}{images/week-6-2r.jpg}
    \end{center}
    
    \item \jjj{Simple Columnar epithelium}: similar in appearance to absorptive cells in the intestines.
    
    \begin{center}
      \Image{0.65\columnwidth}{images/week-6-3r.jpg}
    \end{center}
    
    \item \jjj{Lamina Propria}: dense irregular connective tissue that supports the epithelium. It is rich in fenestrated capillaries and small venules.
    
    \begin{center}
      \Image{0.65\columnwidth}{images/week-6-4r.jpg}
    \end{center}
    
    \item \jjj{Tunica Muscularis}: randomly oriented bundles of smooth muscle containing numerous collagen and elastic fibers. Its contraction results in emptying of the gallbladder.
    
    \begin{center}
      \Image{0.8\columnwidth}{images/week-6-5r.jpg}
    \end{center}
    
    \item \jjj{Serosa}: where the gallbladder is unattached to the liver. It is composed of a surface layer of mesothelium supported by loose connective tissue.
    
    \begin{center}
      \Image{0.85\columnwidth}{images/week-6-6r.jpg}
    \end{center}
    
    
  \end{itemize}
  \end{multicols}
  
  \subsection{Exocrine Pancreas}\label{Exocrine Pancreas}
  \begin{itemize}
    \item The pancreas is the largest exocrine gland and is 95\% exocrine tissue and 1--2\% endocrine tissue. The exocrine portion is a purely serous gland which produces digestive enzymes that are released into the duodenum. The duct cells also secrete bicarbonate to neutralize acid from the stomach.
    \item The exocrine pancreas is compound tubuloacinar in structure. Centroacinar cells are epithelial cells from the beginning of ducts that protrude into the acinar lumen.
    
    \Image{0.94\columnwidth}{images/week-6-s.jpg}
    
  \end{itemize}
  \begin{multicols}{2}
  \begin{itemize}
    \item \jjj{Capsule}: connective tissue covers the exterior surface.
    
    \begin{center}
      \Image{0.94\columnwidth}{images/week-6-1s.jpg}
    \end{center}
    
    \item \jjj{Blood Vessels}
    
    \begin{center}
      \Image{0.94\columnwidth}{images/week-6-2s.jpg}
    \end{center}
    
    \item \jjj{Nerves}
    
    \begin{center}
      \Image{0.94\columnwidth}{images/week-6-3s.jpg}
    \end{center}
    
    \item \jjj{Exocrine Cells}: arranged as acini of pyramidal serous cells. 
    
    \begin{center}
      \Image{0.7\columnwidth}{images/week-6-4s.jpg}
    \end{center}
    
    \item \jjj{Acini}
    
    \begin{center}
      \Image{0.8\columnwidth}{images/week-6-5s.jpg}
    \end{center}
    
    \item \jjj{Intercalated Ducts}
    
    \begin{center}
      \Image{0.8\columnwidth}{images/week-6-6s.jpg}
    \end{center}
    
    \item \jjj{Centroacinar Cells}: duct cells located within an acinus.
    
    \begin{center}
      \Image{0.7\columnwidth}{images/week-6-7s.jpg}
    \end{center}
    
    \item \jjj{Interlobular Ducts}: leave the lobule and drain into interlobular ducts.
    
    \begin{center}
      \Image{0.8\columnwidth}{images/week-6-8s.jpg}
    \end{center}

    \item \jjj{Interlobular Ducts}: ducts located outside a lobule. 
    
    \begin{center}
      \Image{0.8\columnwidth}{images/week-6-9s.jpg}
    \end{center}
  \end{itemize}
  \end{multicols}

  \subsection{Endocrine Pancreas}\label{Endocrine Pancreas}
  \begin{itemize}
    \item Pancreatic islets (or islets of Langerhans) are ``islands'' of endocrine cells located within the pancreas. They secrete hormones (insulin and glucagon) important in the regulation of glucose in blood.
  \end{itemize}

  \begin{center}
    \Image{0.94\columnwidth}{images/week-6-t.jpg}  
  \end{center}

  \begin{multicols}{2}
  \begin{itemize}
    \item \jjj{Islets of Langerhans}:
    
    \begin{center}
      \Image{0.7\columnwidth}{images/week-6-1u.jpg}
      \Image{0.7\columnwidth}{images/week-6-2u.jpg}
      \Image{0.7\columnwidth}{images/week-6-3u.jpg}
    \end{center}
    
    \item \jjj{Parasympathetic Ganglion}: that is easily confused with small islets at low magnification, however at higher magnification shows typical structure of automatic ganglia.
    
    \begin{center}
      \Image{0.8\columnwidth}{images/week-6-5u.jpg}
      \Image{0.8\columnwidth}{images/week-6-4u.jpg}
    \end{center}
    
    
    
  \end{itemize}
  \end{multicols}
  
  \begin{enumerate}\color{minor}
    \item Why can the liver be characterized as both an exocrine and endocrine organ?
      \begin{itemize}\color{text}
        \item Exocrine, since it primarily produces digestive bile for the small intestine, but is also endocrine since it releases hormones into the blood.
      \end{itemize}
    \item What are the secretory products of the exocrine pancreas?
      \begin{itemize}\color{text}
        \item Digestive enzymes: trypsinogen, lipase, amylase, etc., in inactive state
      \end{itemize}
    \item What is the major factor controlling insulin secretion?
      \begin{itemize}\color{text}
        \item Modulating blood glucose levels.
      \end{itemize}
  \end{enumerate}


\end{itemize}
