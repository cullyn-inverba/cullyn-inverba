\chapter{Week 6: Lymphatic, Digestive}
% chktex-file 8
\section{Lymphatic}
\begin{itemize}
  \item[]

  \subsection{Tonsils}
  \begin{itemize}
    \item Tonsils an example of mucosa-associated lymphoid tissue (\jjj{MALT}). The lymphocytes are distributed as diffuse, non-encapsulated nodules in the underlying connective tissue.
  \end{itemize}
  \begin{multicols}{2}
  \begin{itemize}
    
    \item \jjj{Stratified Squamous Non-Kerantized Epithelium}:  covers the numerous nodules that compromise the palatine tonsil.
    
    \begin{center}
      \Image{0.75\columnwidth}{images/week-6-1a.jpg}
    \end{center}
    
    \item \jjj{Lymph Nodules}: spherical aggregations of lymphocytes that usually have germinal centers.
    
    \begin{center}
      \Image{0.7\columnwidth}{images/week-6-2a.jpg}
    \end{center}

    \item \jjj{Submucosa}
    
    \begin{center}
      \Image{0.7\columnwidth}{images/week-6-8a.jpg}
    \end{center}
    
    
    \item \jjj{Germinal centers}
    
    \begin{center}
      \Image{0.7\columnwidth}{images/week-6-3a.jpg}
    \end{center}

    \item \jjj{Crypts}: infoldings of the epithelium into the underlying connective tissue. 
    
    \begin{center}
      \Image{0.7\columnwidth}{images/week-6-4a.jpg}
    \end{center}
    
    \item \jjj{Lymphocytes}
    
    \begin{center}
      \Image{0.7\columnwidth}{images/week-6-5a.jpg}
    \end{center}

    \item \jjj{Sequestered crypts}: usually inflamed and filled with debris and lymphocytes 
    
    \begin{center}
      \Image{0.85\columnwidth}{images/week-6-6a.jpg}
    \end{center}
    
    \item \jjj{Plasma cells}:  large numbers of plasma cells are usually seen in the underlying connective tissue near the epithelium.
    
    \begin{center}
      \Image{0.75\columnwidth}{images/week-6-7a.jpg}
    \end{center}
    
  \end{itemize}
  \end{multicols}

  \subsection{Lymph Nodes}\label{Lymph Nodes}
  \begin{multicols}{2}
  \begin{itemize}
    \item \jjj{Capsule}: dense connective tissue enclosing the node.
    
    \begin{center}
      \Image{0.6\columnwidth}{images/week-6-2b.jpg}
      \Image{0.6\columnwidth}{images/week-6-1b.jpg}
      \Image{0.6\columnwidth}{images/week-6-4b.jpg}
    \end{center}
    
    \item \jjj{Subcapsular Sinus}: space underneath the capsule that receives lymph from afferent lymphatic vessels.
    
    \begin{center}
      \Image{0.7\columnwidth}{images/week-6-3b.jpg}
      \Image{0.7\columnwidth}{images/week-6-6b.jpg}
      \Image{0.7\columnwidth}{images/week-6-5b.jpg}
    \end{center}
    
    \item \jjj{Trabeculae}: connective tissue that extends inward from the capsule.
    
    \begin{center}
      \Image{0.7\columnwidth}{images/week-6-9b.jpg}
      \Image{0.7\columnwidth}{images/week-6-7b.jpg}
      \Image{0.7\columnwidth}{images/week-6-8b.jpg}
    \end{center}

    \item \jjj{Cortex}: reticular fibers form an irregular, anastomosing network in the outer region of the node. Nodules are enclosed by reticular fibers.
    
    \begin{center}
      \Image{0.5\columnwidth}{images/week-6-10b.jpg}
      \Image{0.7\columnwidth}{images/week-6-12b.jpg}
    \end{center}

    \item \jjj{Inner Cortex}: region between the outer cortex and the medulla that is free of nodules. 
    
    \begin{center}
      \Image{0.7\columnwidth}{images/week-6-11b.jpg}
    \end{center}
    
    \item \jjj{Medulla}: inner part of the node. 
    
    \begin{center}
      \Image{0.7\columnwidth}{images/week-6-13b.jpg}
      \Image{0.7\columnwidth}{images/week-6-14b.jpg}
    \end{center}
  \end{itemize}
\end{multicols}

  \subsection{Thymus}\label{Thymus}
  \begin{multicols}{2}
  \begin{itemize}
    \item \jjj{Capsule (neonatal/adult)}: thin connective tissue layer surrounding the thymus that extends inwards to form incomplete lobules.
    
    \begin{center}
      \Image{0.7\columnwidth}{images/week-6-1c.jpg}
      \Image{0.7\columnwidth}{images/week-6-2c.jpg}
    \end{center}

    \item \jjj{Cortex (neonatal/adult)}: outer darker, region of small lymphocytes.
    
    \begin{center}
      \Image{0.55\columnwidth}{images/week-6-3c.jpg}
      \Image{0.55\columnwidth}{images/week-6-4c.jpg}
    \end{center}
    
    \item \jjj{T Lymphocytes}: small nuclei of condensed chromatin.
    
    \begin{center}
      \Image{0.85\columnwidth}{images/week-6-5c.jpg}
    \end{center}
    
    \item \jjj{Epithelial Reticular Cells}
    
    \begin{center}
      \Image{0.85\columnwidth}{images/week-6-6c.jpg}
    \end{center}
    
    \item \jjj{Macrophages}: large cells that phagocytize T cells marked for removal.
    
    \begin{center}
      \Image{0.85\columnwidth}{images/week-6-7c.jpg}
    \end{center}
    
    \item \jjj{Medulla}: inner, lighter region of larger lymphocytes. 
    
    \begin{center}
      \Image{0.92\columnwidth}{images/week-6-8c.jpg}
    \end{center}
    \
    \item \jjj{Hassal's Corpuscles}: closely packed, concentrically arranged epithelial reticular cells.
    
    \begin{center}
      \Image{0.6\columnwidth}{images/week-6-9c.jpg}
      \Image{0.6\columnwidth}{images/week-6-10c.jpg}
    \end{center}
    
  \end{itemize}
  \end{multicols}

  \subsection{Spleen}\label{Spleen}
  \begin{multicols}{2}
  \begin{itemize}
    \item 
  \end{itemize}
  \end{multicols}

  \subsection{Questions}\label{Questions}
  \begin{enumerate}
    \item Which lymphatic organs have afferent lymphatic vessels
    \item How do lymphocytes enter:
      \begin{enumerate}
        \item Lymph nodes
        \item MALT
      \end{enumerate}
    \item What are the components of the blood thymic barrier?
    \item Which of the lymphatic organs filters blood?
  \end{enumerate}
  
\end{itemize}

\newpage
\section{Digestive}\label{Digestive}
\begin{itemize}
  \item[]

  \subsection{Tongue}\label{Tongue}
  \begin{multicols}{2}
  \begin{itemize}
    \item 
  \end{itemize}
  \end{multicols}

  \subsection{Esophagus}\label{Esophagus}
  \begin{multicols}{2}
  \begin{itemize}
    \item 
  \end{itemize}
  \end{multicols}
  
  \subsection{Junction Esophagus and Stomach}\label{Junction Esophagus and Stomach}
  \begin{multicols}{2}
  \begin{itemize}
    \item How can you diagnose whether you are  
    looking at the upper or lower portion of the esophagus?
  \end{itemize}
  \end{multicols}

  \subsection{Stomach}\label{Stomach}
  \begin{multicols}{2}
  \begin{itemize}
    \item To what ultrastructural feature does the brush border correspond?
  \end{itemize}
  \end{multicols}

  \subsection{Small Intestine}\label{Small Intestine}
  \begin{multicols}{2}
  \begin{itemize}
    \item 
  \end{itemize}
  \end{multicols}
  
  \subsection{Large Intestine}\label{Large Intestine}
  \begin{multicols}{2}
  \begin{itemize}
    \item 
  \end{itemize}
  \end{multicols}

  \subsection{Rectum and Anal Canal}\label{Rectum and Anal Canal}
  \begin{multicols}{2}
  \begin{itemize}
    \item 
  \end{itemize}
  \end{multicols}
  
  \subsection{Appendix}\label{Appendix}
  \begin{multicols}{2}
  \begin{itemize}
    \item 
  \end{itemize}
  \end{multicols}

  \subsection{Parotid Gland}\label{Parotid Gland}
  \begin{multicols}{2}
  \begin{itemize}
    \item 
  \end{itemize}
  \end{multicols}
  
  \subsection{Liver}\label{Liver}
  \begin{multicols}{2}
  \begin{itemize}
    \item 
  \end{itemize}
  \end{multicols}
  
  \subsection{Gall Bladder}\label{Gall Bladder}
  \begin{multicols}{2}
  \begin{itemize}
    \item 
  \end{itemize}
  \end{multicols}
  
  \subsection{Pancreas}\label{Pancreas}
  \begin{multicols}{2}
  \begin{itemize}
    \item Why can the liver be characterized as both an exocrine and endocrine organ?
    \item What are the secretory products of the exocrine pancreas?
    \item What is the major factor controlling insulin secretion?
  \end{itemize}
  \end{multicols}

\end{itemize}
