\chapter{Week 6: Lymphatic, Digestive}
% chktex-file 8
\section{Lymphatic}
\begin{itemize}
  \item[]

  \subsection{Tonsils}
  \begin{itemize}
    \item Tonsils an example of mucosa-associated lymphoid tissue (\jjj{MALT}). The lymphocytes are distributed as diffuse, non-encapsulated nodules in the underlying connective tissue.
  \end{itemize}
  \begin{multicols}{2}
  \begin{itemize}
    
    \item \jjj{Stratified Squamous Non-Kerantized Epithelium}:  covers the numerous nodules that compromise the palatine tonsil.
    
    \begin{center}
      \Image{0.75\columnwidth}{images/week-6-1a.jpg}
    \end{center}
    
    \item \jjj{Lymph Nodules}: spherical aggregations of lymphocytes that usually have germinal centers.
    
    \begin{center}
      \Image{0.7\columnwidth}{images/week-6-2a.jpg}
    \end{center}

    \item \jjj{Submucosa}
    
    \begin{center}
      \Image{0.7\columnwidth}{images/week-6-8a.jpg}
    \end{center}
    
    
    \item \jjj{Germinal centers}
    
    \begin{center}
      \Image{0.7\columnwidth}{images/week-6-3a.jpg}
    \end{center}

    \item \jjj{Crypts}: infoldings of the epithelium into the underlying connective tissue. 
    
    \begin{center}
      \Image{0.7\columnwidth}{images/week-6-4a.jpg}
    \end{center}
    
    \item \jjj{Lymphocytes}
    
    \begin{center}
      \Image{0.7\columnwidth}{images/week-6-5a.jpg}
    \end{center}

    \item \jjj{Sequestered crypts}: usually inflamed and filled with debris and lymphocytes 
    
    \begin{center}
      \Image{0.85\columnwidth}{images/week-6-6a.jpg}
    \end{center}
    
    \item \jjj{Plasma cells}:  large numbers of plasma cells are usually seen in the underlying connective tissue near the epithelium.
    
    \begin{center}
      \Image{0.75\columnwidth}{images/week-6-7a.jpg}
    \end{center}
    
  \end{itemize}
  \end{multicols}

  \subsection{Lymph Nodes}\label{Lymph Nodes}
  \begin{multicols}{2}
  \begin{itemize}
    \item \jjj{Capsule}: dense connective tissue enclosing the node.
    
    \begin{center}
      \Image{0.6\columnwidth}{images/week-6-2b.jpg}
      \Image{0.6\columnwidth}{images/week-6-1b.jpg}
      \Image{0.6\columnwidth}{images/week-6-4b.jpg}
    \end{center}
    
    \item \jjj{Subcapsular Sinus}: space underneath the capsule that receives lymph from afferent lymphatic vessels.
    
    \begin{center}
      \Image{0.7\columnwidth}{images/week-6-3b.jpg}
      \Image{0.7\columnwidth}{images/week-6-6b.jpg}
      \Image{0.7\columnwidth}{images/week-6-5b.jpg}
    \end{center}
    
    \item \jjj{Trabeculae}: connective tissue that extends inward from the capsule.
    
    \begin{center}
      \Image{0.7\columnwidth}{images/week-6-9b.jpg}
      \Image{0.7\columnwidth}{images/week-6-7b.jpg}
      \Image{0.7\columnwidth}{images/week-6-8b.jpg}
    \end{center}

    \item \jjj{Cortex}: reticular fibers form an irregular, anastomosing network in the outer region of the node. Nodules are enclosed by reticular fibers.
    
    \begin{center}
      \Image{0.5\columnwidth}{images/week-6-10b.jpg}
      \Image{0.7\columnwidth}{images/week-6-12b.jpg}
    \end{center}

    \item \jjj{Inner Cortex}: region between the outer cortex and the medulla that is free of nodules. 
    
    \begin{center}
      \Image{0.7\columnwidth}{images/week-6-11b.jpg}
    \end{center}
    
    \item \jjj{Medulla}: inner part of the node. 
    
    \begin{center}
      \Image{0.7\columnwidth}{images/week-6-13b.jpg}
      \Image{0.7\columnwidth}{images/week-6-14b.jpg}
    \end{center}
  \end{itemize}
\end{multicols}

  \subsection{Thymus}\label{Thymus}
  \begin{multicols}{2}
  \begin{itemize}
    \item \jjj{Capsule (neonatal/adult)}: thin connective tissue layer surrounding the thymus that extends inwards to form incomplete lobules.
    
    \begin{center}
      \Image{0.7\columnwidth}{images/week-6-1c.jpg}
      \Image{0.7\columnwidth}{images/week-6-2c.jpg}
    \end{center}

    \item \jjj{Cortex (neonatal/adult)}: outer darker, region of small lymphocytes.
    
    \begin{center}
      \Image{0.55\columnwidth}{images/week-6-3c.jpg}
      \Image{0.55\columnwidth}{images/week-6-4c.jpg}
    \end{center}
    
    \item \jjj{T Lymphocytes}: small nuclei of condensed chromatin.
    
    \begin{center}
      \Image{0.85\columnwidth}{images/week-6-5c.jpg}
    \end{center}
    
    \item \jjj{Epithelial Reticular Cells}
    
    \begin{center}
      \Image{0.85\columnwidth}{images/week-6-6c.jpg}
    \end{center}
    
    \item \jjj{Macrophages}: large cells that phagocytize T cells marked for removal.
    
    \begin{center}
      \Image{0.85\columnwidth}{images/week-6-7c.jpg}
    \end{center}
    
    \item \jjj{Medulla}: inner, lighter region of larger lymphocytes. 
    
    \begin{center}
      \Image{0.92\columnwidth}{images/week-6-8c.jpg}
    \end{center}
    \
    \item \jjj{Hassal's Corpuscles}: closely packed, concentrically arranged epithelial reticular cells.
    
    \begin{center}
      \Image{0.555\columnwidth}{images/week-6-9c.jpg}
      \Image{0.555\columnwidth}{images/week-6-10c.jpg}
    \end{center}
    
  \end{itemize}
  \end{multicols}

  \subsection{Spleen}\label{Spleen}
  \begin{multicols}{2}
  \begin{itemize}
    \item \jjj{Capsule}: dense connective tissue enclosing the organ.
    
    \begin{center}
      \Image{0.8\columnwidth}{images/week-6-1d.jpg}
      \Image{0.8\columnwidth}{images/week-6-2d.jpg}
    \end{center}
    
    \item \jjj{Trabeculae}:  connective tissue that extends inward from the capsule through which blood vessels enter the pulp
    
    \begin{center}
      \Image{0.65\columnwidth}{images/week-6-3d.jpg}
      \Image{0.65\columnwidth}{images/week-6-4d.jpg}
    \end{center}
    
    \item \jjj{Blood vessels}
    
    \begin{center}
      \Image{0.6\columnwidth}{images/week-6-5d.jpg}
      \Image{0.6\columnwidth}{images/week-6-6d.jpg}
    \end{center}
    
    \item \jjj{White Pulp}: composed of lymphatic tissue. It appears basophilic due to the large number of nuclei.
    
    \begin{center}
      \Image{0.7\columnwidth}{images/week-6-7d.jpg}
      \Image{0.7\columnwidth}{images/week-6-8d.jpg}
    \end{center}
    
    \item \jjj{Splenic Nodules}: clusters of B lymphocytes located on central arterioles. They usually contain a germinal center of activated B lymphocytes.
    
    \begin{center}
      \Image{0.9\columnwidth}{images/week-6-9d.jpg}
      \Image{0.9\columnwidth}{images/week-6-10d.jpg}
      \Image{0.9\columnwidth}{images/week-6-11d.jpg}
    \end{center}

    \item \jjj{Central Aretrioles} branches of trabecular arteries coated by PALS and adjacent to nodules.
    
    \begin{center}
      \Image{0.7\columnwidth}{images/week-6-12d.jpg}
      \Image{0.7\columnwidth}{images/week-6-13d.jpg}
      \Image{0.7\columnwidth}{images/week-6-14d.jpg}
    \end{center}
    
    \item \jjj{Marginal Zone}: region between white and red pulp where macrophages, dendritic cells, and lymphocytes interact.
    
    \begin{center}
      \Image{0.6\columnwidth}{images/week-6-15d.jpg}
    \end{center}
    
    \item \jjj{Red pulp}: filters and degrades red blood cells (RBCs). It appears eosinophilic due to the large number of RBCs.
    
    \begin{center}
      \Image{0.85\columnwidth}{images/week-6-16d.jpg}
      \Image{0.85\columnwidth}{images/week-6-17d.jpg}
      \Image{0.85\columnwidth}{images/week-6-18d.jpg}
    \end{center}
    
    \newpage

    \item \jjj{Splenic Sinusoids}: vascular spaces lined by specialized endothelial cells that filter RBCs. 
    
    \begin{center}
      \Image{0.945\columnwidth}{images/week-6-19d.jpg}
      \Image{0.945\columnwidth}{images/week-6-20d.jpg}
      \Image{0.945\columnwidth}{images/week-6-21d.jpg}
    \end{center}
    
    \item \jjj{Specialized Endothelial Cells}
    
    \begin{center}
      \Image{0.8\columnwidth}{images/week-6-22d.jpg}
    \end{center}

    \item \jjj{Splenic Cords}: loose connective tissue supported by a meshwork of reticular fibers, and contains loose connective tissue supported by a meshwork of reticular fibers.
    
    \begin{center}
      \Image{0.94\columnwidth}{images/week-6-23d.jpg}
      \Image{0.94\columnwidth}{images/week-6-24d.jpg}
    \end{center}
    
    \item \jjj{Pulp Arterioles}: not surrounded by lymphocytes like central arterioles in white pulp and surrounded by layer of reticular fibers.
    
    \begin{center}
      \Image{0.94\columnwidth}{images/week-6-25d.jpg}
      \Image{0.7\columnwidth}{images/week-6-26d.jpg}
      \Image{0.7\columnwidth}{images/week-6-27d.jpg}
    \end{center}
    
    
  \end{itemize}
  \end{multicols}

  \subsection{Questions}\label{Questions}
  \begin{enumerate}\color{minor}
    \item Which lymphatic organs have afferent lymphatic vessels?
      \begin{itemize}\color{text}
        \item Found only in lymph nodes; efferent are found in thymus and spleen.
      \end{itemize}
    \item How do lymphocytes enter:
      \begin{enumerate}
        \item Lymph nodes?
          \begin{itemize}\color{text}
            \item Via blood vessel walls or afferent lymphatic vessels.
          \end{itemize}
        \item MALT\@?
          \begin{itemize}\color{text}
            \item Via efferent lymphatic vessels.
          \end{itemize}
      \end{enumerate}
    \item What are the components of the blood thymic barrier?
    \begin{itemize}\color{text}
      \item Epithelial reticular cells, basal laminae, and endothelial cells joined by tight junctions.
    \end{itemize}
    \item Which of the lymphatic organs filters blood?
      \begin{itemize}\color{text}
        \item Spleen.
      \end{itemize}
  \end{enumerate}
  
\end{itemize}

\newpage
\section{Digestive}\label{Digestive}
\begin{itemize}
  \item \jjj{Gastrointestinal Tract}
  
  \Image{0.9\columnwidth}{images/week-6-1e.jpg}
  

  \subsection{Tongue}\label{Tongue}
  \begin{multicols}{2}
  \begin{itemize}
    \item \jjj{Overview of the Tongue}
    
    \begin{center}
      \Image{0.7\columnwidth}{images/week-6-e.jpg}
    \end{center}
    
    \item \jjj{Stratified Squamous Non-Kerantized Epithelium}
    
    \begin{center}
      \Image{0.5\columnwidth}{images/week-6-2e.jpg}
    \end{center}

    \item \jjj{Dermal Papillae}: ridges of connective tissue that project into the epithelium that reduce its mobility and brings blood vessels in close contact with the epithelial cells. 
    
    \begin{center}
      \Image{0.7\columnwidth}{images/week-6-3e.jpg}
    \end{center}
    
    \item \jjj{Foliate Papillae}: parallel ridges on the lateral edges of the tongue separated by deep mucosal furrows.
    
    \begin{center}
      \Image{0.8\columnwidth}{images/week-6-4e.jpg}
    \end{center}

    \item \jjj{Furrows}: separate each papillae and receive saliva from the minor lingual glands.
    
    \begin{center}
      \Image{0.8\columnwidth}{images/week-6-5e.jpg}
    \end{center}
    
    \item \jjj{Taste Buds}: elliptical structures found in the epithelium of the furrows that contain cells with taste receptors. The circular opening at its apex is the taste pore.
    
    \begin{center}
      \Image{0.8\columnwidth}{images/week-6-6e.jpg}
    \end{center}

    \item \jjj{Taste Pore}
    
    \begin{center}
      \Image{0.8\columnwidth}{images/week-6-7e.jpg}
    \end{center}
    
    \item \jjj{Skeletal Muscle}: arranged in three bundles at right angles to each other to allow flexibility and precision in movements of the tongue.
    
    \begin{center}
      \Image{0.8\columnwidth}{images/week-6-8e.jpg}
    \end{center}
    
    \item \jjj{Serous Glands}: secrete a fluid that contains digestive enzymes.
    
    \begin{center}
      \Image{0.9\columnwidth}{images/week-6-9e.jpg}
    \end{center}
    
    \item \jjj{Mucous Glands}: secrete a fluid that contains mucus (glycoproteins known as mucins).
    
    \begin{center}
      \Image{0.8\columnwidth}{images/week-6-10e.jpg}
    \end{center}
    
  \end{itemize}
  \end{multicols}

  \subsection{Esophagus}\label{Esophagus}
  
  \begin{center}
    \Image{0.5\columnwidth}{images/week-6-f.jpg}
  \end{center}
  
  \begin{multicols}{2}
  \begin{itemize}
    \item \jjj{Stratified Squamous Non-Kerantized Epithelium}: 
    
    \begin{center}
      \Image{0.75\columnwidth}{images/week-6-1f.jpg}
    \end{center}
    
    \item \jjj{Lamina Propria}: dense irregular connective tissue.
    
    \begin{center}
      \Image{0.8\columnwidth}{images/week-6-2f.jpg}
    \end{center}
    
    \item \jjj{Muscularis Mucosae}: smooth muscle.
    
    \begin{center}
      \Image{0.9\columnwidth}{images/week-6-3f.jpg}
    \end{center}
    
    \item \jjj{Submucosa}: dense irregular connective tissue.
    
    \begin{center}
      \Image{0.9\columnwidth}{images/week-6-4f.jpg}
    \end{center}

    \item \jjj{Mucus Glands}: only the esophagus and the duodenum have glands in the submucosa.
    
    \begin{center}
      \Image{0.9\columnwidth}{images/week-6-5f.jpg}
    \end{center}
    
    \item \jjj{Ducts}: usually have cuboidal or stratified cuboidal epithelium.
    
    \begin{center}
      \Image{0.94\columnwidth}{images/week-6-6f.jpg}
    \end{center}
    
    \item \jjj{Muscularis Externa}: contains both smooth and skeletal muscle because this specimen is from the middle third of the esophagus.   
    
    \begin{center}
      \Image{0.85\columnwidth}{images/week-6-7f.jpg}
    \end{center}
    
    \item \jjj{Inner Layer}
    
    \begin{center}
      \Image{0.85\columnwidth}{images/week-6-8f.jpg}
    \end{center}

    \item \jjj{Outer Layer}
    
    \begin{center}
      \Image{0.9\columnwidth}{images/week-6-9f.jpg}
    \end{center}
    
    \item \jjj{Auerbach's plexus}:  is found between the inner and outer layers of the muscularis externa. Ganglia with prominent capsule cells can be seen.
    
    \begin{center}
      \Image{0.8\columnwidth}{images/week-6-10f.jpg}
    \end{center}
    
  \end{itemize}
  \end{multicols}
  
  \subsection{Gastroesophageal Junction}\label{Gastroesophageal Junction}
  \begin{multicols}{2}
  \begin{itemize}
    \item \jjj{Gastroesophageal Junction}
    
    \begin{center}
      \Image{0.94\columnwidth}{images/week-6-g.jpg}
    \end{center}

    \item \jjj{Change in Epithelium}
    
    \begin{center}
      \Image{0.83\columnwidth}{images/week-6-1g.jpg}
    \end{center}
  \end{itemize}
  \end{multicols}
  \begin{itemize}
    \item How can you diagnose whether you are  
    looking at the upper or lower portion of the esophagus?
      \begin{itemize}
        \item The best way to tell is to look at the epithelium; the upper portion of the esophagus has stratified squamous while the lower portion (stomach) has simple columnar epithelium with long linear or coiled glands.
      \end{itemize}
  \end{itemize}

  \newpage

  \subsection{Stomach}\label{Stomach}
  \begin{multicols}{2}
  \begin{itemize}
    \item \jjj{Stomach}
    
    \begin{center}
      \Image{0.8\columnwidth}{images/week-6-h.jpg}
    \end{center}
    
    \item \jjj{Mucosa}: composed of the epithelium, lamina propria, and muscularis mucosae. 
    
    \begin{center}
      \Image{0.8\columnwidth}{images/week-6-1h.jpg}
    \end{center}

    \item \jjj{Gastric Pits}: invaginations of the surface epithelium.
    
    \begin{center}
      \Image{0.8\columnwidth}{images/week-6-2h.jpg}
    \end{center}
    
    \item \jjj{Gastric Glands}: tubular glands that extend from the base of the gastric pits to the muscularis mucosae.
    
    \begin{center}
      \Image{0.75\columnwidth}{images/week-6-3h.jpg}
    \end{center}

    \item \jjj{Lamina Propria}: reduced to small amounts of connective tissue found between gastric pits and glands.
    
    \begin{center}
      \Image{0.75\columnwidth}{images/week-6-3-2h.jpg}
    \end{center}
    

    \item \jjj{Muscularis Mucosae}: narrow layer of smooth muscle cells at the base of the mucosa.
    
    \begin{center}
      \Image{0.75\columnwidth}{images/week-6-4h.jpg}
    \end{center}
    
    \item \jjj{Submucosa}: dense irregular connective tissue.
    
    \begin{center}
      \Image{0.8\columnwidth}{images/week-6-5h.jpg}
    \end{center}
    
    \item \jjj{Muscularis Externa}:  three layers of muscle cells rather than two layers found elsewhere in the GI tract.
    
    \begin{center}
      \Image{0.8\columnwidth}{images/week-6-6h.jpg}
    \end{center}
    
    \item \jjj{Inner Oblique Layer}: this layer is unique to the stomach and is found near the boundary with the submucosa.
    
    \begin{center}
      \Image{0.8\columnwidth}{images/week-6-7h.jpg}
    \end{center}
    
    \item \jjj{Middle Circular Layer}: 
    
    \begin{center}
      \Image{0.94\columnwidth}{images/week-6-8h.jpg}
    \end{center}
    
    \item \jjj{Outer Longitudinal Layer}: 
    
    \begin{center}
      \Image{0.94\columnwidth}{images/week-6-9h.jpg}
    \end{center}
    
    \item \jjj{Adventitia}: loose irregular connective tissue.
    
    \begin{center}
      \Image{0.94\columnwidth}{images/week-6-10h.jpg}
    \end{center}
    
    \item \jjj{Cardiac Stomach}: the narrow region surrounding the opening of the esophagus that contains cardiac glands in the mucosa.
    
    \begin{center}
      \Image{0.9\columnwidth}{images/week-6-11h.jpg}
    \end{center}
    
    \item \jjj{Cardiac Glands}: appear as cross-sections of the coiled tubular glands of mostly mucus secreting cells that empty into the bottom of gastric pits. 
    
    \begin{center}
      \Image{0.7\columnwidth}{images/week-6-12h.jpg}
    \end{center}

  \end{itemize}
  \end{multicols}
  \begin{itemize}
    \item To what ultrastructural feature does the brush border correspond?
      \begin{itemize}
        \item The presence of mucous-secreting duodenal glands in its submucosa.
      \end{itemize}
  \end{itemize}

  \subsection{Small Intestine}\label{Small Intestine}
  \begin{multicols}{2}
  \begin{itemize}
    \item Mucosa: 
  \end{itemize}
  \end{multicols}
  
  \subsection{Large Intestine}\label{Large Intestine}
  \begin{multicols}{2}
  \begin{itemize}
    \item 
  \end{itemize}
  \end{multicols}

  \subsection{Rectum and Anal Canal}\label{Rectum and Anal Canal}
  \begin{multicols}{2}
  \begin{itemize}
    \item 
  \end{itemize}
  \end{multicols}
  
  \subsection{Appendix}\label{Appendix}
  \begin{multicols}{2}
  \begin{itemize}
    \item 
  \end{itemize}
  \end{multicols}

  \subsection{Parotid Gland}\label{Parotid Gland}
  \begin{multicols}{2}
  \begin{itemize}
    \item 
  \end{itemize}
  \end{multicols}
  
  \subsection{Liver}\label{Liver}
  \begin{multicols}{2}
  \begin{itemize}
    \item 
  \end{itemize}
  \end{multicols}
  
  \subsection{Gall Bladder}\label{Gall Bladder}
  \begin{multicols}{2}
  \begin{itemize}
    \item 
  \end{itemize}
  \end{multicols}
  
  \subsection{Pancreas}\label{Pancreas}
  \begin{multicols}{2}
  \begin{itemize}
    \item Why can the liver be characterized as both an exocrine and endocrine organ?
    \item What are the secretory products of the exocrine pancreas?
    \item What is the major factor controlling insulin secretion?
  \end{itemize}
  \end{multicols}

\end{itemize}
