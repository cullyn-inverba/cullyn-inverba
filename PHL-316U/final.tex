\documentclass[12pt,a4paper]{article}
\usepackage{inverba, apacite}

\newcommand{\userName}{Cullyn Newman} 
\newcommand{\class}{PHL 316U} 
\newcommand{\institution}{\hyperlink{home}{\color{G-Moon}Social and Political Philosphy}}
\newcommand{\theTitle}{Argumentative Meta-Analysis of Contemporary Liberal Democracy and Freedom}
\setlength\headheight{20pt}
\setlength{\parindent}{0pt}
\title{\hypertarget{home}\theTitle\vspace*{-0.4cm}}
\author{\normalsize\userName}
\date{\vspace*{-0.5cm}\small\today}
\rfoot{}
\lfoot{}
\renewcommand{\baselinestretch}{1.618} 
\begin{document}
%%%%%%%%%%%%%%%%%%%%%%%%%%%%%%%%%%%%%%%%%%%%%%%%%%%%%%%%%%%%%%%%%%%%%
\maketitle
\begin{abstract}
   \cite{neo}

   \cite{craft}

   \cite{direct}

   \cite{systems}

   \cite{people}

   \cite{tom}

   In this essay I analyze and argue the views of both David Harvey and Richard Sennett in regards to the dynamic intereaction between contemporary liberal democracy and freedom in various scales. While the arguments of the such authors are focused, the works of others are included in order to help establish the constraints on such broad subjets. This initial meta-analysis will be used as a means to generate a more nuanced discussion designed to allow more significant argumentative claims to be made. The goals of my arguments will used as a means to advance or refute the views of Harvey and Sennett until sufficient support is estalished such that they may lead to actionable claims on what society ought to do. 


\end{abstract}
\fancyhead{}
\fancyhead[R]{\hyperlink{home}{\nouppercase\leftmark}}
\fancyhead[L]{\theTitle}
\rhead{\hyperlink{home}{\thepage}}%%%%%%%%%%%%%%%%%%%%%%%%%%%%%%%%%%%%%%%%%%%%%%%%%%%%%%%%


\textbf{Thesis}

\textbf{Counter-arguments}

\textbf{Rebuttals}

\textbf{Conclusion}




\clearpage
\bibliographystyle{apacite}
\bibliography{citations.bib}
%\endgroup
\end{document}