\documentclass[basic]{inVerba-notes}
\usepackage{inVerba-math}

\newcommand{\userName}{Cullyn Newman}
\newcommand{\class}{BI:\@ 430}
\newcommand{\theTitle}{Assingment 4}
\newcommand{\institution}{Portland State University}

\begin{document}

\begin{enumerate}[align=left, leftmargin=0pt, labelindent=\parindent, listparindent=\parindent, labelwidth=0pt, itemindent=!]\color{minor}
  \item \pts{5} See the paper posted on D2L\@: “Subclone wars 2017” by Chanock (2017) Nature 545 p. 160.
  \begin{enumerate}
    \item \pts{1} In the study described by this article, how were human embryonic stem cell (hESCs) lines analyzed for mutations in protein coding genes?
    \basec{\begin{itemize}
      \item The authors extracted 140 lines of hESCs and used next-generation sequencing of the protein coding regions in order to compare base-pair changes and their influence on commonly reported amino-acid substitutions in adult cancer genomes.
    \end{itemize}}
    
    \item \pts{2} Which particular gene did the authors detect mutations in? Why are mutations in this gene a concern for the use of pluripotent cell lines in regenerative medicine?
    \basec{\begin{itemize}
      \item The \textit{TP53} mutation caused by a single-nucleotide mutation was gene of interest. The cells with the mutation were thought to confer a growth advantage that enabled mutant cells populations to expand at the expense of normal cells. This mutation is common is Li-Fraumeni syndrome, which is associated with predisposition to multiple cancers.
      \item The cause of concern for \textit{TP53} was due to the fact that the mutation arose independently in different laboratories that used the cells from the same unmutated source, indicating that \textit{TP53} to arise can potentially cause cancer. Additionally, there is evidence of subpopulations of \textit{TP53}-mutant cells replacing normal cell populations, indicating that use of pluripotent cells may risk proliferation of mutated cell lines and thus cancer as well.
    \end{itemize}}
    
    \item \pts{2} Suggest a potential change or specific protocol for human pluripotent stem cell culture that could help improve safety of these kinds of cells in medical procedures. Why would this make the procedures safer?
    \basec{\begin{itemize}
      \item I support the authors' idea for increase sample frequency of cultured cells and sequencing on whole genomes to see how \textit{in vitro} conditions may influence rates of mutations in order to identify practices that may prevent such occurrences.
      \item In the meantime, a specific protocol not discussed that may improve safety might involve the development some kind of screening procedures that can identify mosaic populations of cells that include the \textit{TP53} gene. This way, if hESCs were needed then at least there might be a potential to remove mutated cell populations and still recover some use of the hESCs. However, since the cells exhibit qualities similar to those of aging cells, I'm not sure how practical this protocol would be, as it might require some long-term screening solution. 
    \end{itemize}}
    
  \end{enumerate}

  \newpage

  \item \pts{5} Genome engineering has been revolutionized by the use of the CRISPR-Cas9 sequence-specific nuclease. As you answer the following questions, consult the class notes on CRISPR-Cas9, as well as the posted paper “Crispr muscular dystrophy 2016” by Long et al.
  \begin{enumerate}
    \item \pts{1} In the mouse genome engineering paper (“Crispr muscular dystrophy 2016”), how were the Cas9 proteins and single guide RNAs (sgRNAs) delivered to animals that were being engineered?
    \basec{\begin{itemize}
      \item Myoediting, i.e., CRISP/Cas9-mediated nonhomologous endjoining (NHEJ), aimed at gene correction in postnatal tissues was done using adeno-associated virus-9 (AAV9), as it displays high tropism (high tendency to target) cardiac and skeletal muscle cells. 
      \item AAV-guide RNAs were generated by cloning the sgRNA-mdx and sgRNA-R3 into AAV-sgRNA vector containing a human U6 promoter and green fluorescent protein.
      \item Then, different modes of AAV9 delivery, as well as variations in timing of expression, were compared. The most optimal method. Each method restored dystrophin protein expression in target tissues with varying degrees between 3--12 weeks after injection.
    \end{itemize}}
    
    \item \pts{2} How did the action of the Cas9-sgRNAs help to restore the correct expression of the dystrophin (Dmd) gene?
    \basec{\begin{itemize}
      \item Previous studies have indicated that the actin binding and cystein-rich domains of the Dmd gene are essential for function and expression, while many regions are dispensable.
      \item The use of Cas9-sgRNAs helped restore correct expression by skipping those regions and avoiding frameshift mutations that sometimes effect the essential regions. For example, the authors did this to bypass the premature termination codon in exon 23 that is responsible for a particular phenotype of \textit{mdx} mice, leading to possibility of restoring the \textit{Dmd} open reading frame.
    \end{itemize}}
    
    \item \pts{2} In the Discussion section of the paper (the last three paragraphs), the relative efficiency of gene correction by the method the authors used was compared to different methods that relied on homology directed repair (HDR). Which approach is more efficient, and why is it more efficient?
    \basec{\begin{itemize}
      \item The Myoediting method previously described was about 10 times more efficient as gene correction by HDR\@. 
      \item The superiority was thought to be due to the fact that exon skipping allows for permanent removal of the harmful mutation since the NHEJ does not require precise genetic modification, unlike HDR\@.
    \end{itemize}}
    
  \end{enumerate}

  \item \pts{5} Two highly effective vaccines against SARS CoV2 include lipid nanoparticle delivery of mRNA that encodes the spike protein of the virus, which when expressed trains the immune system to recognize this important viral surface antigen. Please read the posted PDFs “Kariko Shields the World” and “Fauci Story of the SARS CoV2 Vaccines” to answer the following questions. It will also be helpful to consult the paper “RNA mods and immune response”, especially the abstract and introduction.
  \begin{enumerate}
    \item \pts{1} Dr.\ Katalin Kariko and co-workers were able to cause cultured mammalian cells to make new proteins when transfected with mRNA\@. However, they were unable to provoke the same response in mice---what happened to the mice that were tested, and why did it happen (what organismal response was activated)?
    \basec{\begin{itemize}
      \item The mice exhibited an immune response and displayed signs of sickness and inflammation. This is because the mRNA injections appeared to be pathogens to the mice's immune system.
    \end{itemize}}
    
    \item \pts{2} What changes were made to the mRNA, and how did these changes help to prevent the response seen previously?
    \basec{\begin{itemize}
      \item An experiment with tRNA showed that RNA itself wasn't the cause, instead critical difference was that tRNA contained pseudouridine. This slightly different change in nucleotide allowed for the tRNA to evade the immune system, and sure enough, synthesizing mRNA with pseudouridine allowed mRNA to elude the immune system and even synthesize more of target protein.
    \end{itemize}}
    
    \item \pts{2} Why are the advances pioneered by Kariko, Weismann, and co-workers likely to have a broader significance than simply immunization against SARS CoV2 infection? How might this technology be useful in other disease or gene therapy scenarios?
    \basec{\begin{itemize}
      \item The use mRNA vaccines can allow for easy synthesis of protein that let our immune systems learn to recognize many other forms of diseases, all we need to do is isolate the mRNA associated with the protein of interest. This means rapid and fast development of vaccines for rapidly changing, or novel diseases. Previous use of mRNAs as a therapeutic was too difficult and often failed, but this is a significant step forward. 
      \item Since many gene therapies focus on dysfunctional proteins, or lack of crucial proteins, then the easy introduction of mRNA that solve or replacing such proteins offer a very attractive field of interest to pursue. 
      \item I'm glad you brought Kati Kariko to my attention, definitely worth the read! Very exciting stuff. 
    \end{itemize}}
    
  \end{enumerate}
\end{enumerate}

\end{document}