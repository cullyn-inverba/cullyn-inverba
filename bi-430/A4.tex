\documentclass[basic]{inVerba-notes}
\usepackage{inVerba-math}

\newcommand{\userName}{Cullyn Newman}
\newcommand{\class}{BI:\@ 430}
\newcommand{\theTitle}{Assingment 4}
\newcommand{\institution}{Portland State University}

\begin{document}

\begin{enumerate}[align=left, leftmargin=0pt, labelindent=\parindent, listparindent=\parindent, labelwidth=0pt, itemindent=!]\color{minor}
  \item \pts{5} See the paper posted on D2L\@: “Subclone wars 2017” by Chanock (2017) Nature 545 p. 160.
  \begin{enumerate}
    \item \pts{1} In the study described by this article, how were human embryonic stem cell lines analyzed for mutations in protein coding genes?
    \basec{\begin{itemize}
      \item 
    \end{itemize}}
    
    \item \pts{2} Which gene of concern did the authors detect mutations in? Why are mutations in this gene a concern for the use of pluripotent cell lines in regenerative medicine?
    \basec{\begin{itemize}
      \item 
    \end{itemize}}
    
    \item \pts{2} Suggest a potential change or specific protocol for human pluripotent stem cell culture that could help improve safety of these kinds of cells in medical procedures. Why would this make the procedures safer?
    \basec{\begin{itemize}
      \item 
    \end{itemize}}
    
  \end{enumerate}

  \item \pts{5} Genome engineering has been revolutionized by the use of the CRISPR-Cas9 sequence-specific nuclease. As you answer the following questions, consult the class notes on CRISPR-Cas9, as well as the posted paper “Crispr muscular dystrophy 2016” by Long et al.
  \begin{enumerate}
    \item \pts{1} In the mouse genome engineering paper (“Crispr muscular dystrophy 2016”), how were the Cas9 proteins and single guide RNAs (sgRNAs) delivered to animals that were being engineered?
    \basec{\begin{itemize}
      \item 
    \end{itemize}}
    
    \item \pts{2} How did the action of the Cas9-sgRNAs help to restore the correct expression of the dystrophin (Dmd) gene?
    \basec{\begin{itemize}
      \item 
    \end{itemize}}
    
    \item \pts{2} In the Discussion section of the paper (the last three paragraphs), the relative efficiency of gene correction by the method the authors used was compared to different methods that relied on homology directed repair (HDR). Which approach is more efficient, and why is it more efficient?
    \basec{\begin{itemize}
      \item 
    \end{itemize}}
    
  \end{enumerate}

  \item \pts{5} Two highly effective vaccines against SARS CoV2 include lipid nanoparticle delivery of mRNA that encodes the spike protein of the virus, which when expressed trains the immune system to recognize this important viral surface antigen. Please read the posted PDFs “Kariko Shields the World” and “Fauci Story of the SARS CoV2 Vaccines” to answer the following questions. It will also be helpful to consult the paper “RNA mods and immune response”, especially the abstract and introduction.
  \begin{enumerate}
    \item \pts{1} Dr.\ Katalin Kariko and co-workers were able to cause cultured mammalian cells to make new proteins when transfected with mRNA\@. However, they were unable to provoke the same response in mice---what happened to the mice that were tested, and why did it happen (what organismal response was activated)?
    \basec{\begin{itemize}
      \item 
    \end{itemize}}
    
    \item \pts{2} What changes were made to the mRNA, and how did these changes help to prevent the response seen previously?
    \basec{\begin{itemize}
      \item 
    \end{itemize}}
    
    \item \pts{2} Why are the advances pioneered by Kariko, Weismann, and co-workers likely to have a broader significance than simply immunization against SARS CoV2 infection? How might this technology be useful in other disease or gene therapy scenarios?
    \basec{\begin{itemize}
      \item 
    \end{itemize}}
    
  \end{enumerate}
\end{enumerate}

\end{document}