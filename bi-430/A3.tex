\documentclass[basic]{inVerba-notes}
\usepackage{inVerba-math}
\usepackage{inVerba-chem}

\newcommand{\userName}{Cullyn Newman}
\newcommand{\class}{BI:\@ 430}
\newcommand{\theTitle}{Assingment 3}
\newcommand{\institution}{Portland State University}

\begin{document}

\begin{enumerate}[align=left, leftmargin=0pt, labelindent=\parindent, listparindent=\parindent, labelwidth=0pt, itemindent=!]\color{minor}
  \item \pts{5} See the paper posted on D2L\@: “RNAi porphyria 2014” (Yasuda et al. 2014, PNAS 111, p. 7777) to answer the following. In this paper, a simulation of the disease porphyria is treated in a mouse model using RNAi directed to the liver.
  \begin{enumerate}[align=left, leftmargin=0pt, labelindent=\parindent, listparindent=\parindent, labelwidth=0pt, itemindent=!]
    \item \pts{2} What causes the disease porphyria---what gene is affected in the disease state, and how does the gene contribute to the disease state?
    \basec{\begin{itemize}
      \item There are several forms of porphyria, though in general, porphyrias are inherited defects in the biosynthesis of heme, which precursor to hemoglobin, and  synthesized in both bone marrow and the liver. The 2014 paper provided describes three autosomal dominant and one autosomal recessive diseases:
        \begin{itemize}
          \item Acute Intermittent Porphyria (AIP),
          \item Hereditary Copropohyria (HP),
          \item Variegate Porphyria (VP),
          \item and the recessive, Acid dehydratase Deficiency Porphyria (ADP).
        \end{itemize}
      
      Various factors can induce the over expression of hepatic 5-aminoevulinic acid synthase (ALAS1), which is the first rate-limiting enzyme of the heme biosynthetic pathway. Normally this isn't an issue, and functional genes provide negative feedback inhibition of ALAS1.
      
      The various inherited diseases disrupt enzymes in the biosynthetic pathway and reduce production capacity of the heme pool. This creates a metabolic bottleneck when ALAS1 activity is increase, as the depletion of the free heme pool, the loss of negative feedback inhibition of heme on ALAS1, and thus \emph{further up-regulation of hepatic ALAS1 expression} starts to inhibit functionality of other enzymes in the pathway that are responsible for the breakdown neurotoxic hepatic metabolites. 
      
      Consequently, the neurotoxic porphyrin precursors are overproduced in the liver and accumulate in the plasma and urine during acute attacks.
    \end{itemize}}
    
    \item \pts{1} Why is the liver a good tissue to target when treating this disease?
    \basec{\begin{itemize}
      \item The liver is the primary site of the pathology, as demonstrated by liver transplants recipients with recurrent and incapacitating attacks resulting in rapid normalization of neurotoxic metabolites (ALA, PBG) and abrupt remission of attacks.
      
      Essentially, \emph{inhibition of the induced over expression of ALAS1} is the ideal solution to prevent or treat the acute attacks in all four of the acute hepatic porphyrias. 
    \end{itemize}}
    
    \item \pts{2} In this study, how was the RNAi packaged in order to direct it to liver tissue? What is the underlying mechanism that gets the RNA into cells?
    \basec{\begin{itemize}
      \item First, potent Alas1-siRNA was identified using a panel of siRNAs targeting the \textit{Alas1} was made in order to screen for their ability to inhibit \textit{Alas1} mRNA expression.
      \item The compounds were then \emph{put into lipid nanoparticles (LNPs)} for efficient hepatic delivery, as the liver is responsible for filtering particles from circulation, meaning the RNAi compound is very likely to reach it once put into these lipid nanoparticles and allowed to circulate.
      \item Mechanically speaking, the rapid endocytosis of is achieved via binding of the nanoparticles to the highly expressed liver-specific receptors (e.g., GalNac) on hepatocytes.
    \end{itemize}}
    
  \end{enumerate}

  \item \pts{5} Some of the first color-shifted mutants of green fluorescent protein (GFP) were first isolated by Roger Tsien and colleagues in the 1990s. Roger Tsien’s work was recognized in 2008 with the Nobel Prize in Chemistry (alongside Osamu Shimamura and Martin Chalfie). For more information, see Roger Tsien’s \link{https://www.nobelprize.org/prizes/chemistry/2008/tsien/lecture/}{Nobel lecture} and additional \link{https://www.nobelprize.org/prizes/chemistry/2008/tsien/biographical/}{biographical info}. See the paper posted on D2L\@: “GFP mut 1994” (Heim et al. 2006, PNAS 91, p. 12501) to answer the following.

  \begin{enumerate}
    \item \pts{2} In one part of the study, new mutants of green fluorescent protein were isolated. Which techniques were used to make the mutations in the gene, and how do they cause mutations?
    \basec{\begin{itemize}
      \item \emph{Random mutagenesis by PCR} of the GFP cDNA was done by \ch{MnCl2} + various concentrations of deoxyribonucleosides (dATP, dGTP, dCTP, dTTP), or by hydroxylamine treatment. These treatments were done to increase the error rate, which then were made into colonies on agar and screened.
    \end{itemize}}
    
    \item \pts{1} How was the library of mutant proteins screened, and how many mutants were screened in this study?
    \basec{\begin{itemize}
      \item The colonies made by the previous step were \emph{screened visually}, based on emission colors and ratios of brightness when excited at various wavelengths of light (max 475, min 397 nm). There were \emph{five different mutants} with various cite directed mutagenesis leading to different amino acids expressed at the critical position of interest.
    \end{itemize}}
    
    \item \pts{2} One of the mutations caused the GFP protein to fluoresce blue rather than green. How did this mutation affect the amino acids that make up the GFP chromophore?
    \basec{\begin{itemize}
      \item Sequencing of the blue mutant revealed five amino acid substitutions, with only one that proved to be critical. The critical change was \emph{Tyr-66 \to~His} in the center of the chromophore. The authors then tested site directed mutagenesis with tryptophan and phenylalanine at the same position, with tryptophan producing intermediate fluorescence between Tyr-66 and His, and phenylalanine producing none. 
      
      The authors hypothesized that the inefficiency of folding or poor chromophore formation due to the change at that key site was the cause of the observations.
    \end{itemize}}
  \end{enumerate}

  \bigskip

  \item The enzyme cytochrome P450 has been engineered for many biotech applications, in work led by Frances Arnold and others. This work led in part to the Nobel Prize received by Frances Arnold in 2018 (alongside George Smith and Gregory Winter). For more information, see Frances Arnold’s \link{https://www.youtube.com/watch?v=6hOZ5e0g9Uo}{Nobel lecture} and \link{https://www.nobelprize.org/womenwhochangedscience/stories/frances-arnold}{biographical story}.
  
  See the review article by Jung, Lauchli, and Arnold posted on 
  D2L\@: “Cytochrome P450: taming a wild type enzyme”. Use this article to answer the following questions.

  \begin{enumerate}
    \item \pts{3} Read the Introduction section of the paper.
    \begin{enumerate}[align=left, leftmargin=0pt, labelindent=\parindent, listparindent=\parindent, labelwidth=0pt, itemindent=!]
      \item What are some biotechnological applications for cytochrome P450?
      \basec{\begin{itemize}
        \item According to the paper, P450s could have many biotechnological applications---including, but not limited to,
          \begin{itemize}
            \item applications in productions of drugs and drug metabolites,
            \item catalyst in a variety of processes,
            \item biological sensors or as bioremediation agents,
            \item and other critical biosynthetic mechanisms.
          \end{itemize}
      \end{itemize}}
      
      \item What are some difficulties associated with adapting P450s for industrial chemical processes?
      \basec{\begin{itemize}
        \item Many natural P450s are insoluble; often membrane-associated.
        \item P450s are often expressed at low levels or lack the level of activity sufficient for practical biocatalysis.
        \item Current challenges involving solving required selectivity (or lack of), substrate scope, coupling efficiency, or activity inhibition.
      \end{itemize}}
      
      \item Which is preferable for engineering P450s: a rational mutagenesis or directed evolution approach? Why?
      \basec{\begin{itemize}
        \item Rational approaches are currently very difficult due to the wide range of potential changes, minimal information about current state of P450s, and difficulties obtaining new information. Thus, \emph{directed evolution} is currently much more preferable and favored. 
        
        Directed evolution allows for semi-random mutations and the subsequent creative iterative screening processes/recombination allows for hopefully functional goals to be met without rational understanding how reach such goals.
      \end{itemize}}
      
    \end{enumerate}
    \item \pts{2} Read the section “Properties of the P450s that facilitate their ability to adapt.”  Describe at least two properties of P450s that allow them to be adapted to a variety of substrates.
    \basec{\begin{itemize}
      \item \emph{Functional diversity}: the diversity of the P450 enzyme family (more than 11,000 known members) allows for a wide range of `evolvable' proteins, as the chances of a new, useful traits increase with the diversity of the mechanisms made possible by the enzymes as they help lead to the production various proteins that evolution to act upon.
      \item \emph{Mutational robustness}: diversity alone isn't always enough, as the ability to accept such differences is just as essential. Most mutations are negative, or neutral, and the ability to temporarily accept new mutations without breaking function completely allows for more potential routes (i.e., more diversity) for evolution to act upon.
      
      P450s are not that stable, but they are very accepting to mutations, allowing for new structures to evolve while still maintaining catalytic competence. 
      
      The authors hypothesize that the ability of P450s to tolerate conformational and mutational changes may be a result of the non-polar nature of amino acids in the active sites (at least in examples examined), as non-polar changes typically only affect weaker forces that are less likely to disrupt global functions. 
    \end{itemize}}
    
  \end{enumerate}
\end{enumerate}

\end{document}