\documentclass[basic]{inVerba-notes}
\usepackage{inVerba-math}

\newcommand{\userName}{Cullyn Newman}
\newcommand{\class}{BI:\@ 430}
\newcommand{\theTitle}{Assingment 3}
\newcommand{\institution}{Portland State University}

\begin{document}

\begin{enumerate}[align=left, leftmargin=0pt, labelindent=\parindent, listparindent=\parindent, labelwidth=0pt, itemindent=!]\color{minor}
  \item \pts{5} See the paper posted on D2L\@: “RNAi porphyria 2014” (Yasuda et al. 2014, PNAS 111, p. 7777) to answer the following. In this paper, a simulation of the disease porphyria is treated in a mouse model using RNAi directed to the liver.
  \begin{enumerate}
    \item \pts{2} What causes the disease porphyria---what gene is affected in the disease state, and how does the gene contribute to the disease state?
    \item \pts{1} Why is the liver a good tissue to target when treating this disease?
    \item \pts{2} In this study, how was the RNAi packaged in order to direct it to liver tissue? What is the underlying mechanism that gets the RNA into cells?
  \end{enumerate}

  \item \pts{5} Some of the first color-shifted mutants of green fluorescent protein (GFP) were first isolated by Roger Tsien and colleagues in the 1990s. Roger Tsien’s work was recognized in 2008 with the Nobel Prize in Chemistry (alongside Osamu Shimamura and Martin Chalfie). For more information, see Roger Tsien’s \link{https://www.nobelprize.org/prizes/chemistry/2008/tsien/lecture/}{Nobel lecture} and additional \link{https://www.nobelprize.org/prizes/chemistry/2008/tsien/biographical/}{biographical info}. 
  
  
  See the paper posted on D2L\@: “GFP mut 1994” (Heim et al. 2006, PNAS 91, p. 12501) to answer the following.

  \begin{enumerate}
    \item \pts{2} In one part of the study, new mutants of green fluorescent protein were isolated. Which techniques were used to make the mutations in the gene, and how do they cause mutations?
    \item \pts{1} How was the library of mutant proteins screened, and how many mutants were screened in this study?
    \item \pts{2} One of the mutations caused the GFP protein to fluoresce blue rather than green. How did this mutation affect the amino acids that make up the GFP chromophore?
  \end{enumerate}

  \item The enzyme cytochrome P450 has been engineered for many biotech applications, in work led by Frances Arnold and others. This work led in part to the Nobel Prize received by Frances Arnold in 2018 (alongside George Smith and Gregory Winter). For more information, see Frances Arnold’s \link{https://www.youtube.com/watch?v=6hOZ5e0g9Uo}{Nobel lecture} and \link{https://www.nobelprize.org/womenwhochangedscience/stories/frances-arnold}{biographical story}.
  
  See the review article by Jung, Lauchli, and Arnold posted on 
  D2L: “Cytochrome P450: taming a wild type enzyme”. Use this article to answer the following questions.

  \begin{enumerate}
    \item \pts{3} Read the Introduction section of the paper.
    \begin{enumerate}[align=left, leftmargin=0pt, labelindent=\parindent, listparindent=\parindent, labelwidth=0pt, itemindent=!]
      \item What are some biotechnological applications for cytochrome P450?
      \item What are some difficulties associated with adapting P450s for industrial chemical processes?
      \item Which is preferable for engineering P450s: a rational mutagenesis or directed evolution approach? Why?
    \end{enumerate}
    \item \pts{2} Read the section “Properties of the P450s that facilitate their ability to adapt.”  Describe at least two properties of P450s that allow them to be adapted to a variety of substrates.
  \end{enumerate}
\end{enumerate}

\end{document}