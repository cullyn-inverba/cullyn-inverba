\documentclass[12pt,letterpaper]{article}
\usepackage{inverba}
\newcommand{\userName}{Cullyn Newman} 
\newcommand{\class}{BI 341} 
\newcommand{\institution}{Portland State University} 
\newcommand{\thetitle}{\hypertarget{home}{Essential Genetics and Genomics}}
\rfoot{\hyperlink{home}{\thepage}}

\begin{document}
\setcounter{section}{39}

%%%%%%%%%%%%%%%%%%%%%%%%%%%%%%%%%%%%%%%%%%%%%%%%%%%%%%%%%%%%%%%%%%%%%%%%%%%%%%%%%%%%%%%%%
%                               %   %   %   %   %   %   %                               %
%                           %   %   %   %   %   %   %   %   %                           %
%                       %   %                               %   %                       %
%   %   %   %   %   %   %   %   O   U   T   L   I   N   E   %   %   %   %   %   %   %   %
%                       %   %                               %   %                       %
%                           %   %   %   %   %   %   %   %   %                           %
%                               %   %   %   %   %   %   %                               %
%%%%%%%%%%%%%%%%%%%%%%%%%%%%%%%%%%%%%%%%%%%%%%%%%%%%%%%%%%%%%%%%%%%%%%%%%%%%%%%%%%%%%%%%%

%\endgroup
\begin{contbox}{Genetics and Genomics}{ 
\begin{enumerate}[font=\bfseries, wide]
\item \hyperlink{1}{\textbf{The Genetic Code of Genes and Genomes}}
    \begin{itemize}
        \item \hyperlink{1.r}{Review}
    \end{itemize}
    \item \hyperlink{2}{\textbf{Transmission Genetics: Heritage from Mendel}}
    \begin{itemize}
        \item \hyperlink{2.r}{Review}
    \end{itemize}
    \item \hyperlink{3}{\textbf{The Chromosomal Basis of Heredity}}
    \begin{itemize}
        \item \hyperlink{3.1}{Each species has a characteristic set of chromosomes}
        \item \hyperlink{3.2}{The daughter cells of mitosis have identical chromosomes}
        \item \hyperlink{3.3}{Meiosis results in gametes that differ genetically}
        \item \hyperlink{3.4}{Eukaryotic chromosomes are highly coiled complexes of DNA and protein}
        \item \hyperlink{3.5}{The centromere and telomere are essential parts of chromosomes}
        \item \hyperlink{3.6}{Genes are located in chromosomes}
        \item \hyperlink{3.7}{Genetic data analysis makes use of probability and statistics}
    \end{itemize}
    \item \hyperlink{4}{\textbf{Gene Linkage and Genetic Mapping}}
    \begin{itemize}
        \item \hyperlink{4.1}{Linked alleles tend to stay together in meiosis}
        \item \hyperlink{4.2}{Recombination results from crossing-over between linked alleles}
        \item \hyperlink{4.3}{Polymorphic DNA sequences are used in human genetic mapping}
        \item \hyperlink{4.4}{Double crossovers are revealed in three-point crosses}
        \item \hyperlink{4.5}{Tetrads contain all four products of meiosis}
        \item \hyperlink{4.6}{Recombination is initiated by a double-stranded break in DNA}
    \end{itemize}
    \item \hyperlink{5}{\textbf{Human Chromosomes and Chromosome Behavior}}
    \begin{itemize}
        \item 
    \end{itemize}
    \item \hyperlink{6}{\textbf{DNA Structure, Replication, and Manipulation}}
    \begin{itemize}
        \item 
    \end{itemize}
    \item \hyperlink{7}{\textbf{The Genetics of Bacteria and Their Viruses}}
    \begin{itemize}
        \item 
    \end{itemize}
    \item \hyperlink{8}{\textbf{The Molecular Genetics of Gene Expression}}
    \begin{itemize}
        \item 
    \end{itemize}
    \item [10.] \hyperlink{10}{\textbf{Genomics, Proteomics, and Genetic Engineering}}
    \begin{itemize}
        \item 
    \end{itemize}
    \item [13.] \hyperlink{13}{\textbf{Molecular Genetics of the Cell Cycle and Cancer}}
    \begin{itemize}
        \item 
    \end{itemize}    
    \item [14.] \hyperlink{14}{\textbf{Molecular Evolution and Population Genetics}}
    \begin{itemize}
        \item 
    \end{itemize}
    \item [15.] \hyperlink{15}{\textbf{The Genetic Basis of Complex Traits}}
    \begin{itemize}
        \item 
    \end{itemize}
\end{enumerate}
}\end{contbox}
%\endgroup

%%%%%%%%%%%%%%%%%%%%%%%%%%%%%%%%%%%%%%%%%%%%%%%%%%%%%%%%%%%%%%%%%%%%%%%%%%%%%%%%%%%%%%%%%
%                               %   %   %   %   %   %   %                               %
%                           %   %   %   %   %   %   %   %   %                           %
%                       %   %                               %   %                       %
%   %   %   %   %   %   %   %       N   O   T   E   S       %   %   %   %   %   %   %   %
%                       %   %                               %   %                       %
%                           %   %   %   %   %   %   %   %   %                           %
%                               %   %   %   %   %   %   %                               %
%%%%%%%%%%%%%%%%%%%%%%%%%%%%%%%%%%%%%%%%%%%%%%%%%%%%%%%%%%%%%%%%%%%%%%%%%%%%%%%%%%%%%%%%%


%%%%%%%%%%%%%%%%%%%%%%%%%%%%%%%%%%%%%%%%%%%%%%%%%%%%%%%%%%%%%%%%%%%%%%%%%%%%%%%%%%%%%%%%%
%  vvvvvvvvvvvvvvvvvvvvvvvvvvvvvvvvv   Chapter 1   vvvvvvvvvvvvvvvvvvvvvvvvvvvvvvvvvvv  %
%\begingroup

\clearpage

\renewcommand{\thetitle}{\hypertarget{1}{The Genetic Code of Genes
and Genomes}}
\rfoot{\hyperlink{1}{1 --- \thepage}}
\hypertarget{1}{} 
\hypertarget{1.r}{}
\begin{chapbox}{\hyperlink{home}{Chapter 1: The Genetic Code of Genes
    and Genomes}}
\begin{probbox}{Chapter 1: Summary}{
    \begin{itemize}
        \item Inherited traits are affected by genes. 
        \item Genes are composed of the chemical deoxyribonucleic acid (DNA). 
        \item DNA replicates to form (usually identical) copies of itself. 
        \item DNA contains a code specifying what types of enzymes and other proteins are made in cells. 
        \item DNA occasionally mutates, and the mutant forms specify altered proteins. 
        \item A mutant enzyme is an “inborn error of metabolism” that blocks one step in a biochemical pathway for the metabolism of small molecules. 
        \item Genetic analysis of mutants of the fungus Neurospora unable to synthesize an essential nutrient led to the one gene–one enzyme hypothesis. 
        \item Different mutations in the same gene can be identified by means of a complementation test, in which the mutants are brought together in the same cell or organism. Mutations in the same gene fail to complement one another, whereas mutations in different genes show complementation. 
        \item Most traits are complex traits affected by multiple genes as well as by environmental factors. 
        \item Organisms change genetically through generations in the process of biological evolution. 
        \item Because of their common descent, organisms share many features of their genetics and biochemistry.
    \end{itemize}
}\end{probbox}
\end{chapbox}
%\endgroup
%  ^^^^^^^^^^^^^^^^^^^^^^^^^^^^^^^^^   Chapter 1  ^^^^^^^^^^^^^^^^^^^^^^^^^^^^^^^^^^^^  %  
%%%%%%%%%%%%%%%%%%%%%%%%%%%%%%%%%%%%%%%%%%%%%%%%%%%%%%%%%%%%%%%%%%%%%%%%%%%%%%%%%%%%%%%%%

%%%%%%%%%%%%%%%%%%%%%%%%%%%%%%%%%%%%%%%%%%%%%%%%%%%%%%%%%%%%%%%%%%%%%%%%%%%%%%%%%%%%%%%%%
%  vvvvvvvvvvvvvvvvvvvvvvvvvvvvvvvvv   Chapter 2   vvvvvvvvvvvvvvvvvvvvvvvvvvvvvvvvvvv  %
%\begingroup

\clearpage

\renewcommand{\thetitle}{\hypertarget{2}{The Genetic Code of Genes
and Genomes}}
\rfoot{\hyperlink{2}{2 --- \thepage}}
\hypertarget{2}{} 
\hypertarget{2.r}{}
\begin{chapbox}{\hyperlink{home}{Chapter 2: The Genetic Code of Genes and Genomes}}
\begin{probbox}{Chapter 2: Summary}{
    \begin{itemize}
        \item Inherited traits are determined by the genes present in the reproductive cells united in fertilization. 
        \item Genes are usually inherited in pairs, one from the mother and one from the father. 
        \item The genes in a pair may differ in DNA sequence and in their effect on the expression of a particular inherited trait. 
        \item The maternally and paternally inherited genes are not changed by being together in the same organism. 
        \item In the formation of reproductive cells, the paired genes separate again into different cells. 
        \item Random combinations of reproductive cells containing different genes result in Mendel’s ratios of traits appearing among the progeny. 
        \item Simple Mendelian inheritance results in characteristic patterns in human pedigrees for both dominant and recessive traits. 
        \item When two possible outcomes of a cross are mutually exclusive, they cannot occur together. In this case, the probability that either one or the other outcome occurs is given by the sum of their respective probabilities (the addition rule). 
        \item When two possible outcomes of a cross are independent, then knowledge that one has occurred provides no information whether the other has occurred. In this case, the probability that both outcomes occur together is given by the product of their respective probabilities (the multiplication rule). 
        \item The ratios actually observed for any traits are determined by the types of dominance and gene interaction (epistasis).
    \end{itemize}
}\end{probbox}
\end{chapbox}

%\endgroup
%  ^^^^^^^^^^^^^^^^^^^^^^^^^^^^^^^^^   Chapter 2   ^^^^^^^^^^^^^^^^^^^^^^^^^^^^^^^^^^^  %
%%%%%%%%%%%%%%%%%%%%%%%%%%%%%%%%%%%%%%%%%%%%%%%%%%%%%%%%%%%%%%%%%%%%%%%%%%%%%%%%%%%%%%%%%

%%%%%%%%%%%%%%%%%%%%%%%%%%%%%%%%%%%%%%%%%%%%%%%%%%%%%%%%%%%%%%%%%%%%%%%%%%%%%%%%%%%%%%%%%
%  vvvvvvvvvvvvvvvvvvvvvvvvvvvvvvvvv   Chapter 3   vvvvvvvvvvvvvvvvvvvvvvvvvvvvvvvvvvv  %
%\begingroup

\clearpage

\renewcommand{\thetitle}{\hypertarget{3}{The Chromosomal Basis of Heredity}}
\rfoot{\hyperlink{3}{3 --- \thepage}}
\hypertarget{3}{} 

\begin{chapbox}{\hyperlink{home}{Chapter 3: The Chromosomal Basis of Heredity}}
    \begin{enumerate}
        \item \hyperlink{3.1}{Each species has a characteristic set of chromosomes}
        \item \hyperlink{3.2}{The daughter cells of mitosis have identical chromosomes}
        \item \hyperlink{3.3}{Meiosis results in gametes that differ genetically}
        \item \hyperlink{3.4}{Eukaryotic chromosomes are highly coiled complexes of DNA and protein}
        \item \hyperlink{3.5}{The centromere and telomere are essential parts of chromosomes}
        \item \hyperlink{3.6}{Genes are located in chromosomes}
        \item \hyperlink{3.7}{Genetic data analysis makes use of probability and statistics}
    \end{enumerate}
\end{chapbox}


\hypertarget{3.1}{}
\begin{secbox}{\hyperlink{3}{Each species has a characteristic set of chromosomes}}{

}\end{secbox}

\hypertarget{3.2}{}
\begin{secbox}{\hyperlink{3}{The daughter cells of mitosis have identical chromosomes}}{

}\end{secbox}

\hypertarget{3.3}{}
\begin{secbox}{\hyperlink{3}{Meiosis results in gametes that differ genetically}}{

}\end{secbox}

\hypertarget{3.4}{}
\begin{secbox}{\hyperlink{3}{Eukaryotic chromosomes are highly coiled complexes of DNA and protein}}{

}\end{secbox}

\hypertarget{3.5}{}
\begin{secbox}{\hyperlink{3}{The centromere and telomere are essential parts of chromosomes}}{

}\end{secbox}

\hypertarget{3.6}{}
\begin{secbox}{\hyperlink{3}{Genes are located in chromosomes}}{

}\end{secbox}

\hypertarget{3.7}{}
\begin{secbox}{\hyperlink{3}{Genetic data analysis makes use of probability and statistics}}{

}\end{secbox}

\begin{probbox}{Chapter 3: Summary}{
    \begin{itemize}
        \item
    \end{itemize}
}\end{probbox}

%\endgroup
%  ^^^^^^^^^^^^^^^^^^^^^^^^^^^^^^^^^   Chapter 3   ^^^^^^^^^^^^^^^^^^^^^^^^^^^^^^^^^^^  %
%%%%%%%%%%%%%%%%%%%%%%%%%%%%%%%%%%%%%%%%%%%%%%%%%%%%%%%%%%%%%%%%%%%%%%%%%%%%%%%%%%%%%%%%%

%%%%%%%%%%%%%%%%%%%%%%%%%%%%%%%%%%%%%%%%%%%%%%%%%%%%%%%%%%%%%%%%%%%%%%%%%%%%%%%%%%%%%%%%%
%  vvvvvvvvvvvvvvvvvvvvvvvvvvvvvvvvv   Chapter 4   vvvvvvvvvvvvvvvvvvvvvvvvvvvvvvvvvvv  %
%\begingroup

\clearpage

\renewcommand{\thetitle}{\hypertarget{4}{Gene Linkage and Genetic Mapping}}
\rfoot{\hyperlink{4}{4 --- \thepage}}
\hypertarget{4}{} 

\begin{chapbox}{\hyperlink{home}{Chapter 4: Gene Linkage and Genetic Mapping}}
    \begin{enumerate}
        \item \hyperlink{4.1}{Linked alleles tend to stay together in meiosis}
        \item \hyperlink{4.2}{Recombination results from crossing-over between linked alleles}
        \item \hyperlink{4.3}{Polymorphic DNA sequences are used in human genetic mapping}
        \item \hyperlink{4.4}{Double crossovers are revealed in three-point crosses}
        \item \hyperlink{4.5}{Tetrads contain all four products of meiosis}
        \item \hyperlink{4.6}{Recombination is initiated by a double-stranded break in DNA}
    \end{enumerate}
\end{chapbox}


\hypertarget{4.1}{}
\begin{secbox}{\hyperlink{4}{Linked alleles tend to stay together in meiosis}}{

}\end{secbox}

\hypertarget{4.2}{}
\begin{secbox}{\hyperlink{4}{Recombination results from crossing-over between linked alleles}}{

}\end{secbox}

\hypertarget{4.3}{}
\begin{secbox}{\hyperlink{4}{Double crossovers are revealed in three-point crosses}}{

}\end{secbox}

\hypertarget{4.4}{}
\begin{secbox}{\hyperlink{4}{Polymorphic DNA sequences are used in human genetic mapping}}{

}\end{secbox}

\hypertarget{4.5}{}
\begin{secbox}{\hyperlink{4}{Tetrads contain all four products of meiosis}}{

}\end{secbox}

\hypertarget{4.6}{}
\begin{secbox}{\hyperlink{4}{Recombination is initiated by a double-stranded break in DNA}}{

}\end{secbox}

\begin{probbox}{Chapter 4: Summary}{
    \begin{itemize}
        \item
    \end{itemize}
}\end{probbox}

%\endgroup
%  ^^^^^^^^^^^^^^^^^^^^^^^^^^^^^^^^^   Chapter 4   ^^^^^^^^^^^^^^^^^^^^^^^^^^^^^^^^^^^  %
%%%%%%%%%%%%%%%%%%%%%%%%%%%%%%%%%%%%%%%%%%%%%%%%%%%%%%%%%%%%%%%%%%%%%%%%%%%%%%%%%%%%%%%%%


\end{document}