\documentclass[12pt,a4paper]{article}
\usepackage{inverba}

\newcommand{\userName}{Cullyn Newman} 
\newcommand{\class}{CH 334} 
\newcommand{\institution}{Portland State University} 
\newcommand{\theTitle}{\color{B-Cold} Apres Lecture Quizzes}

\begin{document}
%%%%%%%%%%%%%%%%%%%%%%%%%%%%%%%%%%%%%%%%%%%%%%%%%%%%%%%%%%%%%%%%%%%%%
\tableofcontents
\cleardoublepage
\fancyhead{}
%%%%%%%%%%%%%%%%%%%%%%%%%%%%%%%%%%%%%%%%%%%%%%%%%%%%%%%%%%%%%%%%%%%%%


%%%%%%%%%%%%%%%%%%%%%%%%%%%%% Week 3 %%%%%%%%%%%%%%%%%%%%%%%%%%%%%
%\begingroup
\clearpage
\section*{Week 3}\phantomsection
\addcontentsline{toc}{section}{\textbf{Week 3}}
\fancyhead[R]{\hyperlink{home}{Week 3}}

\fancyhead[L]{\hyperlink{home}{Wednesday, October 14}}
\subsection{Wednesday, October 14}
\begin{enumerate}
    {\color{G-Moon}\item The reason that there is free rotation about any single sigma bond is because}
    \begin{itemize}
        \item {\color{o-Sun}\textbf{orbital overlap doesn't change with rotation}}
            \begin{itemize}
                \item Electron density lies along the axis that conects the two nuclei forming the $\sigma$ bond; no issue during rotation as one atom can freely rotate and stay connected.
            \end{itemize}
    \end{itemize}
    {\color{G-Moon}\item The reason there is restricted rotation about a C=C double bond is because}
        \begin{itemize}
            \item {\color{o-Sun}\textbf{rotation would eliminate the parallel overlap of the p orbitals}}
                \begin{itemize}
                    \item $\pi$ bonds lie above and below the axis and must break the overlap between the p orbitals during rotation, {\color{y-Sun}\textit{unless they rotated in unison\(^{?}\)}}. 
                \begin{center}
                    \includegraphics[scale=0.15]{images/pi-bond.pdf}
                \end{center}
                \end{itemize}
        \end{itemize}
    {\color{G-Moon}\item What is a nucleophile?}
        \begin{itemize}
            \item {\color{o-Sun}\textbf{all of the above}}
                \begin{itemize}
                    \item a nucleophile is an electron rich atom that is capable of {\color{o-Sun}donating a pair of electrons}, which is the same definition as a {\color{neg}Lewis base}.
                    \item Electron {\color{o-Sun}rich} means {\color{neg}Lewis bases are negative} and thus attracted to a {\color{pos}positive charged} center (nuclei are generally positively charged) and capable of forming a bond due to excess of electrons. 
                \end{itemize}
        \end{itemize}
    {\color{G-Moon}\item What is an electrophile?}
        \begin{itemize}
        {\color{G-Moon}\item[a.] a species that is attracted to a negative charge
            \item[b.] a species that contains extra electrons 
            \item[c.] a Lewis acid}
            {\color{o-Sun}\item[d.] \textbf{a and c}}
            \begin{itemize}
                \item An electrophile is an {\color{o-Sun}electron-deficient} atom that is capable of {\color{o-Sun}accepting a pair of electrons}, which is the same as a {\color{pos}Lewis acid}.
            \end{itemize}
        \end{itemize}
    \newpage
    {\color{G-Moon}\item When drawing curved arrows, the arrow head always points towards the Lewis acid.}
        \begin{itemize}
            \item {\color{o-Sun}\textbf{True}}
            \subsubsection{Notes on Drawing Curved Arrows}
            \begin{itemize}
                \item \textbf{Tails} must be placed on either a bond or a lone pair.
                    \begin{itemize}
                        \item Shows the {\color{o-Sun}source}, i.e., the electron donor (base).
                        \item Electrons can only be found in lone pairs or bonds, so {\color{o-Sun}never place the tail} of a curved arrow on a {\color{pos}positive charge}.
                    \end{itemize}
                \item \textbf{Heads} must be placed so that it shows either the formation of a bond or the formation of a lone pair.
                    \begin{itemize}
                        \item Shows the {\color{o-Sun}destination}, i.e., the electron acceptor (acid).
                        \item Avoid drawing an arrow that violates the octet rule, so never draw an arrow that gives more than four orbitals to a second-row element.
                    \end{itemize}
                \end{itemize}
        \end{itemize}
    {\color{G-Moon}\item What factors contribute to making something a good EWG?}
        \begin{itemize}
            \item {\color{o-Sun}\textbf{all of the above}}
                \begin{itemize}
                    \item Both inductive and resonance effects can have an impact on electronegativity, which governs substituent effects, such as the electron withdrawing group.
                \end{itemize}
        \end{itemize}
    {\color{G-Moon}\item Given either the presence of a highly electronegative element that results in good inductive effects, or a group that, through resonance can delocalize positive charge, which has the greater effect as an EWG?}
        \begin{itemize}
            \item {\color{o-Sun}\textbf{a group that is capable of delocalizing positive charge through resonance}}
                \begin{itemize}
                    \item The spreading of $\pi$ bonds is called \textbf{delocalization}, {\color{o-Sun}which is a major stabilizing factor} since the electrons are shared between multipile atoms.
                    \item Inductive effect is due to electrons being shifted towards the more electronegative atom, but staying in the same place, provding less stability compared to resonance electron sharing.
                \end{itemize}
        \end{itemize}
    \newpage
    {\color{G-Moon}\item On your own, you can show how electron distribution in the structure below results in the delocalization of positive charge, which increases the electronegativity at the terminal carbon atom.
    \begin{center}
        \chemfig{=_[:-30]-[:30]=_[:-30]-[:30]N{\color{pos}^\oplus}(-[:90]O)=[:-30]O}
    \end{center}}
    \begin{itemize}
        \item I briefly read some sections on resonance, not expecting it to be on this quiz much since he talked about it very little. I'm not super confident in my explanation but I'll give it shot.
        \item There are some patterns for identifying lone pairs of oxygen and nitrogen, excerpt from todays notes:
            \begin{itemize}
                \item \textbf{Associated Patterns for Oxygen}
                \begin{itemize}
                    \item A {\color{neg}negative ($\circleddash$)} charge corresponds with {\color{o-Sun}one bond} and {\color{o-Sun}three lone pairs}.
                    \item The {\color{G-Moon}absence} of charge corresponds with {\color{o-Sun}two bonds} and {\color{o-Sun}two lone pairs}.
                    \item A {\color{pos}positive ($\oplus$)} charge corresponds with {\color{o-Sun}three bonds} and {\color{o-Sun}one lone pair} 
                \end{itemize}
            \item \textbf{Associated Patterns for Nitrogen}
                \begin{itemize}
                    \item A {\color{neg}negative} charge corresponds with {\color{o-Sun}two bond} and {\color{o-Sun}two lone pairs}.
                    \item The {\color{G-Moon}absence} of charge corresponds with {\color{o-Sun}three bonds} and {\color{o-Sun}one lone pair}.
                    \item A {\color{pos}positive} charge corresponds with {\color{o-Sun}four bonds} and {\color{o-Sun}no lone pairs}
                \end{itemize}
            \end{itemize}
        \item Thus, the nitrogen should be have a {\color{pos}positive formal} charge, due to the four bonds (including the $\pi$  bond) and no lone pairs.
        \item I don't know how to draw resonance arrows yet in the bond-line diagrams... but that charge is {\color{o-Sun}delocalized} down the carbon chain, in reality it's being shared with {\color{y-Sun}\textit{the four carbons and thus the electronegativity at the terminal carbon.\(^{?}\)}} (? Less confident about the validity of this statement, but seems reasonable)
    \end{itemize}
\end{enumerate}
\newpage
\fancyhead[L]{\hyperlink{home}{Monday, October 12}}
\subsection{Monday, October 12}
\begin{enumerate}
    {\color{G-Moon}\item How many isomers exist for a molecule with the molecular formula \ch{C4H10O}?}
        \begin{itemize}
            \item {\color{o-Sun}\textbf{7}}
                \begin{itemize}
                    \item {\tiny\chemfig{-[:30]-[:-30]-[:30]-[:-30]OH}
                          \hspace{12pt}
                          \chemfig{-[:30](-[:90])-[:-30]-[:30]OH}
                          \hspace{12pt}
                          \chemfig{-[:30]-[:-30](-[:-90])-[:30]OH}
                          \hspace{12pt}
                          \chemfig{-[:30](-[:-90])(-[:120])-[:0]OH}}
                        \begin{itemize}
                            \item These are the alcohols, or {\color{y-Sun}\textit{all the possbile combinations where the oxygen is on the end}\(^{?}\)} ({\color{y-Sun}?} I'm sure when we get into nomenclature this will make more senses and lead to better explanation)
                        \end{itemize}
                    \item {\tiny\chemfig{-[:30]-[:-30]-[:30]O-[:-30]}
                          \hspace{12pt} 
                          \chemfig{-[:30](-[:90])-[:-30]O-[:30]}
                          \hspace{12pt} 
                          \chemfig{-[:30]-[:-30]O-[:30]-[:-30]}}
                        \begin{itemize}
                            \item These are the ethers, or {\color{y-Sun}\textit{all the possbile combinations where the oxygen is within the chain}\(^{?}\)} ({\color{y-Sun}?} same disclaimer as above)
                        \end{itemize}
                    \item Seems like drawing is only way to get good at this. Eventually we will probably get some intuition or memorize certain compounds with continual practice, similar to practicing math problems.
                \end{itemize}
        \end{itemize}
    {\color{G-Moon}\item How many degrees of unsaturation are there in \ch{C4H10O}?}
        \begin{itemize}
            \item {\color{o-Sun}\textbf{0}}
                \begin{itemize}
                    \item Determining saturation using molecular formula: {\color{o-Sun}C\(_{n}\)H\(_{2n+2}\)}\\\(n=\) carbon atoms.
                        \begin{itemize}
                            \item \textbf{Halogens}: takes the place of a hydrogen atom; {\color{o-Sun}add one H} for each halogen.
                            \item \textbf{Oxygen}: no affect on saturation; {\color{o-Sun}ignore}.
                            \item \textbf{Nitrogen}: needs an extra hydrogen; {\color{o-Sun}subtract one H} for each nitrogen. 
                        \end{itemize}
                    \item Or use HDI formula: {\color{o-Sun}HDI = \(\frac{1}{2}(2C + 2 + N - H - X)~~\)} \(X\): halogen atoms.
                        \begin{itemize}
                            \item HDI: hydrogen deficiency index, which is the measure of degrees of freedom.
                            \item For \ch{C4H10O}:  \(\frac{1}{2}(8 + 2 + 0 - 10 - 0) = \frac{1}{2}(0) = 0\) 
                        \end{itemize}
                \end{itemize}
        \end{itemize}
    \newpage
    {\color{G-Moon}\item Based upon the previous question, which structure(s) can be eliminated as possible for \ch{C4H10O}?}
        \begin{itemize}
            \item {\color{o-Sun}\textbf{all of the above}}
                \begin{itemize}
                    \item {\color{o-Sun}Zero degrees of freedom} means {\color{false}no double/triple bonds or rings} are possbile.
                    \item Degrees of freedom help represent possbile structures, indicating possible double bounds, triple bounds, rings, or various combinations of each.
                \end{itemize}
        \end{itemize}
    {\color{G-Moon}\item How many degrees of unsaturation are there in \ch{C5H13}?}
        \begin{itemize}
            \item {\color{o-Sun}\textbf{0}}
                \begin{itemize}
                    \item Again, general formula: {\color{o-Sun}HDI = \(\frac{1}{2}(2C + 2 + N - H - X)~~\)} \(X\): halogen atoms.
                    \item For \ch{C5H13}{\color{true}N}:  \(\frac{1}{2}(2(5) + 2~ {\color{true}+~1} - 13 - 0) = \frac{1}{2}(0) = 0\)
                \end{itemize}
        \end{itemize}
    {\color{G-Moon}\item Based upon the previous question, what is a possible structure for \ch{C5H13N}?}
        \begin{itemize}
            \item {\color{o-Sun}\chemfig{-[:30]-[:-30]-[:30](-[:90]NH_2)-[:-30]}}
                \begin{itemize}
                    \item Same as question three, {\color{o-Sun}zero degrees of freedom} means {\color{false}no double/triple bonds or rings} are possbile.
                \end{itemize}
        \end{itemize}
    {\color{G-Moon}\item How many degrees of unsaturation exist for a molecule with the formula \ch{C3H6O2}?}
        \begin{itemize}
            \item {\color{o-Sun}\textbf{1}}
                \begin{itemize}
                    \item \(\frac{1}{2}(2(3) + 2 + 0 - 6 - 0) = \frac{1}{2}(2) = 1\)
                \end{itemize}
        \end{itemize}
    {\color{G-Moon}\item What structures are possible for the \ch{C3H6O2} molecule?}
        \begin{itemize}
            \item {\color{o-Sun}\textbf{both the first and second structures are possible}}
                \begin{itemize}
                    \item For reference: 
                    \hspace{8pt}
                    {\tiny\chemfig{-[:-30]-[:30](=[:90]O)-[:-30]OH}
                    \hspace{12pt}
                    \chemfig{-[:30](=[:90]O)-[:-30]O-[:30]}}
                    \item Has one degree of freedom so it has to have at least a double/triple bound or ring structure.
                        \begin{itemize}
                            \item Makes ~~{\tiny\chemfig{HO-[:30]-[:-30]-[:30]-[:-30]OH}} ~~invalid due to lack of any double/triple/rings.
                        \end{itemize}
                \end{itemize}
        \end{itemize}
\end{enumerate}
%\endgroup
%%%%%%%%%%%%%%%%%%%%%%%%%%%%% Week 3 %%%%%%%%%%%%%%%%%%%%%%%%%%%%%

%%%%%%%%%%%%%%%%%%%%%%%%%%%%% Week 2 %%%%%%%%%%%%%%%%%%%%%%%%%%%%%
%\begingroup
\clearpage
\section*{Week 2}\phantomsection
\addcontentsline{toc}{section}{\textbf{Week 2}}
\fancyhead[R]{\hyperlink{home}{Week 2}}


\subsection{Friday, October 9}
\fancyhead[L]{\hyperlink{home}{Wednesday, October 9}}
\begin{enumerate}
    {\color{G-Moon}\item What is the condensed formula for the following molecule?
    \begin{align*}
        \chemfig{
            {\color{b-Ocean}C}H(-[5]{\color{r-Sun}CH_3})(-[2]{\color{r-Sun}CH_3})
            (-[::-20]{\color{b-Ocean}C}H_2
            (-[::40]{\color{b-Ocean}C}(-[2]{\color{r-Sun}CH_3})(-[::-40]{\color{p-Haze}Br})(-[6]{\color{r-Sun}CH_3})
            ))
        }
    \end{align*}}
    \begin{itemize}
        \item {\color{o-Sun}\textbf{\ch{(CH3)2CHCH2C(CH3)2Br}}}
            \begin{itemize}
                \item Breaking down answer: \({\color{r-Sun}(CH_3)_2}{\color{b-Ocean}C}H{\color{b-Ocean}C}H_2{\color{b-Ocean}C}{\color{r-Sun}(CH_3)_2}{\color{p-Haze}Br}\)
            \end{itemize}
    \end{itemize}
    {\color{G-Moon}\item What is the structural formula for \ch{(CH3)3CCH(OH)CH3}? }
        \begin{align*}
            {\color{o-Sun}\chemfig{-[:30](-[:90])(-[:-90])(-[:-30](-[:-90]OH)(-[:30]))}}
        \end{align*}
        \begin{itemize}
            \item Each line with an end is a \ch{CH3} (need 4)
            \item Three points touching is a \ch{CH}. (need 1)
            \item \ch{(OH)} is on the CH
        \end{itemize}
    {\color{G-Moon}\item For any two molecules to be constitutional isomers of each other, at the very least, they:}
        \begin{itemize}
            \item {\color{o-Sun}\textbf{must have the same chemical formula}}
                \begin{itemize}
                    \item isomers: same chemical formula.
                    \item constitution: ways it connects; connect must be {\color{o-Sun}different} to be a constitutional isomer.
                \end{itemize}
        \end{itemize}
    {\color{G-Moon}\item How many constitutional isomers can be formed with the molecular formula \ch{C4H10O}?}
        \begin{itemize}
            \item {\color{o-Sun}\textbf{7}}
                \begin{itemize}
                    \item See Monday, October 12 quiz. Same question asked and a better explanation given there.
                \end{itemize}
        \end{itemize}
    \newpage
    {\color{G-Moon}\item The SO2 molecule has two resonance forms, each of which is a constitutional isomer of the other. Which of the two structures shown below is the most stable?  (The S$\rightarrow$O bond is a dative bond in which both bonding electrons come from only one atom.) }
        \begin{itemize}
            \item {\color{o-Sun}\textbf{the structure on the right}}
                \begin{itemize}
                    \item {\color{y-Sun}\textit{Delocalized electrons being shared in multiple bonds are more stable than a the dative bond.}\(^{?}\)} (? Not sufficient enough information yet to be confident of this explanation, though I'd be willing to bet that its along the right lines)
                \end{itemize}
        \end{itemize}
\end{enumerate}

\newpage
\subsection{Wednesday, October 7}
\fancyhead[L]{\hyperlink{home}{Wednesday, October 7}}
\begin{enumerate}
    {\color{G-Moon}\item What is the hybridization of the orbitals on C in ethylene?}
        \begin{itemize}
            \item {\color{o-Sun}\textbf{sp\(\bm{^{2}}\)}}
            \begin{itemize}
                \item $\chemfig{C(-[5]H)(-[3]H)(=[0]C(-[-1]H)(-[9]H))}$
                \item Each C has 3 groups (number of bonds, $\pi$ bonds count as 1, or lone pairs)
                \item For atoms with groups between 1 and 4; sp{\color{o-Sun}\(^{x}\)}; x = groups - 1 = ({\color{o-Sun}2})
            \end{itemize}
        \end{itemize}
    {\color{G-Moon}\item What is the hybridization of the orbitals on C in acetylene?}
        \begin{itemize}
            \item {\color{o-Sun}\textbf{sp}}
                \begin{itemize}
                    \item $\chemfig{C(-[4]H)(~[0]C(-[0]H))}$
                    \item 2 groups, thus x = 1.
                \end{itemize}
        \end{itemize}
    {\color{G-Moon}\item What type of bonds combine to make a C=C double bond?}
        \begin{itemize}
            \item {\color{o-Sun}\textbf{one sigma and one pi bond.}}
                \begin{itemize}
                    \item First bond is a $\sigma$ bond.
                    \item Additional bonds are $\pi$ bonds that go above and below the axis that connects the nuclei.
                \end{itemize}
        \end{itemize}
    {\color{G-Moon}\item What is the H-C-H bond angle in ethylene?}
        \begin{itemize}
            \item {\color{o-Sun}\textbf{120}$^{\bm{\,\circ}}$}
                \begin{itemize}
                    \item sp\(^{2}\) with no lone pairs, thus trigonal planar (according to VSEPR).
                    \item \(\frac{360}{3}\) = 120; 3 angles maximized on a single plane. 
                \end{itemize}
        \end{itemize}
    {\color{G-Moon}\item Why is the geometry around the CH2 fragment in ethylene, trigonal  planar?}
        \begin{itemize}
            \item {\color{o-Sun}\textbf{there are three regions of electron density}}
                \begin{itemize}
                    \item \textit{regions electron density =\(^{?}\) groups} (? I'm combining a past youtube video with his lecture...)
                \end{itemize}
        \end{itemize}
    {\color{G-Moon}\item There is completely free rotation around a sigma bond such as a \ch{C-C} single bond. Propose a reason for why there is no free rotation around a \ch{C=C}.}
        \begin{itemize}
            \item {\color{o-Sun}\textbf{the rotation would require the \ch{C-C} pi bond to break}}
                \begin{itemize}
                    \item $\sigma$ bonds can rotate on the axis between the nuclei, but the $\pi$ bonds are above and below and would break during rotation since \textit{there are other orbitals that would block\(^{?}\) their rotation}. (? not completely sure on the mechanism that doesn't allow for rotation)
                \end{itemize}
        \end{itemize}
    {\color{G-Moon}\item As the percent s character increases in a bond, what happens to the bond angle?}
        \begin{itemize}
            \item {\color{o-Sun}\textbf{it increases}}
                \begin{itemize}
                    \item The more {\color{o-Sun}\textit{s} character}, the {\color{pos}shorter} and {\color{pos}stronger} the bond, and the {\color{pos}larger} the bond angle.
                \end{itemize}
        \end{itemize}
    {\color{G-Moon}\item As the percent p character in a bond increases, what happens to the bond angle?}
        \begin{itemize}
            \item {\color{o-Sun}\textbf{it decreases}}
                \begin{itemize}
                    \item the more p character, the less s character; inverse of s character.
                \end{itemize}
        \end{itemize}
    {\color{G-Moon}\item Why is a pi bond weaker than a sigma bond?}
        \begin{itemize}
            \item {\color{o-Sun}\textbf{all of the above}}
        \end{itemize}
    {\color{G-Moon}\item Which statement(s) is(are) correct?}
        \begin{itemize}
            {\color{G-Moon}\item the shorter the bond, the stronger the bond
            \item the more s character in the hybridization, the stronger the bond
            \item the more s character in the hybridization, the greater the bond angle
            \item sigma bonds are stronger than pi bonds
            \item the geometry around the C in the ethylene is trigonal planar
            \item a \ch{C=C} bond is composed of one sigma and one pi bond
            \item the hybridization around C in the methyl anion, isoelectronic with \ch{CH4}, is sp\(^{3}\)
            \item when 2s and 2p orbitals mix to form hybrid orbitals, the hybrid orbitals are higher in energy than the 2s orbitals, but lower in energy than the 2p orbitals
            \item the shapres of the hybrid orbitals match the electron domain geometry shapes predicted by VSEPR
            \item the potential energy of a covalent bond is lower than that of the potential energies of the free atoms from which it was formed}
            \item {\color{o-Sun}\textbf{ALL OR THE ABOVE}}
            \begin{itemize}
                \item This seems to be a review of important concepts.
            \end{itemize}
        \end{itemize}
\end{enumerate}

\newpage
\fancyhead[L]{\hyperlink{home}{Monday, October 5}}
\subsection{Monday, October 5}
\begin{enumerate}
    {\color{G-Moon}\item The concept of orbital shapes comes directly from the wave model of the atom.  What is the shape of an s orbital?}
        \begin{itemize}
            \item {\color{o-Sun}\textbf{Spherical}}
                \begin{itemize}
                    \item S orbital is the most simple orbital, with only two electrons. 
                    \item Alternative shapes come from \textit{nodes}; i.e. when \textit{destructive interferences} cancels out the wave function.
                    \item Not circular, orbitals are three-dimensional.
                \end{itemize}
        \end{itemize}
    {\color{G-Moon}\item What is the shape of a p orbital?}
        \begin{itemize}
            \item {\color{o-Sun}\textbf{Dumbell shaped}}
                \begin{itemize}
                    \item P orbital can hold 6 electrons (3 pairs).
                    \item Each pair has one angular node, squeezing shape into dumbell in each direction (x, y, z).
                \end{itemize}
        \end{itemize}
    {\color{G-Moon}\item When atomic orbitals overlap to form a covalent bond, the resultant bonding orbital is:}
        \begin{itemize}
            \item {\color{o-Sun}\textbf{Lower in energy than the atomic orbitals from which it was formed}}
                \begin{itemize}
                    \item Electrons in a covalent bond are in a more stable (lower energy) state due to multiple nuclei hodling them in place.
                    \item Only when nodes are present do the electrons create a destabalized molecular orbit, incraseing the energy.
                \end{itemize}
        \end{itemize}
    {\color{G-Moon}\item Why can't pure p orbitals be used in forming four equivalent bonds as in methane?}
        \begin{itemize}
            {\color{G-Moon}\item the three 2p orbitals can only hold 6 electrons.}
                \begin{itemize}
                    \item True, we need to make four bonds for methane.
                \end{itemize}
                {\color{G-Moon}\item the bonds would have to be \ang{90} apart.}
                \begin{itemize}
                    \item If p had enough space then it would result in planar geometry with \ang{90}, but the true arrangement is tetrahedral with angles of \ang{109.5}
                \end{itemize}
                {\color{G-Moon}\item electron-electron repulsion would not be minimized.}
                \begin{itemize}
                    \item Planar minimization would be \ang{90}, but we have 3d space to work with, so it's not minimized.
                \end{itemize}
            \item {\color{o-Sun}\textbf{all of the above}}
        \end{itemize}
    {\color{G-Moon}\item When s and p orbitals combine to form hybrid orbitals, the resultant hybridized orbitals are:}
        \begin{itemize}
           {\color{G-Moon}\item lower in energy than the p orbitals
            \item higher in energy than the s orbitals}
            \item {\color{o-Sun}\textbf{both of the above}}
                \begin{itemize}
                    \item It takes energy to move the electron up from the \textit{s} orbital and hybridize the \textit{p} orbitals. 
                    \item The new hybridized sp orbital also has more energy than the s orbital.
                \end{itemize}
        \end{itemize}
    {\color{G-Moon}\item What is the difference between a sigma bond and a pi bond?}
        \begin{itemize}
            \item {\color{o-Sun}\textbf{in a $\pi$ bond, electron density lies above and below the axis that conects the two nuclei; in a $\sigma$ bond, the electronegative density lies along the axis that connets the two nuclei}}
                \begin{itemize}
                    \item $\sigma$ bond has circular symmetry with respect to the bond axis (axis that connets the two nuclei). i.e. it's along the axis.
                \end{itemize}
        \end{itemize}
    {\color{G-Moon}\item What is the hybridization of the C in CH2Cl2?}
        \begin{itemize}
            \item {\color{o-Sun}\textbf{sp\(\bm{^{3}}\)}}
                \begin{itemize}
                    \item $\chemfig{C(-[4]Cl)(-[0]Cl)(-[2]H)(-[6]H)}$
                    \item Look at central atom --- C
                    \item Determin groups (number of bonds, $\pi$ bonds count as 1, and lone pairs attached) --- 4 
                    \item for groups 1-4; sp{\color{o-Sun}\(^{x}\)}; x = groups - 1 ({\color{o-Sun}3})
                \end{itemize}
        \end{itemize}
    {\color{G-Moon}\item What is the hybridization of each C in benzene (shown below)?}
        \begin{itemize}
            \item {\color{o-Sun}\textbf{sp\(\bm{^{2}}\)}}
                \begin{itemize}
                    \item Each carbon has 2H and $\pi$ bond between, so groups = 3.
                    \item Groups - 1 = 2, so sp\(^{2}\)
                    \item Though, each $\pi$ bond is delocalized, or free to spread across all the carbons. Still counts as 1 group.
                \end{itemize}
        \end{itemize}
\end{enumerate}

%\endgroup
%%%%%%%%%%%%%%%%%%%%%%%%%%%%% Week 2 %%%%%%%%%%%%%%%%%%%%%%%%%%%%%

%%%%%%%%%%%%%%%%%%%%%%%%%%%%% Week 1 %%%%%%%%%%%%%%%%%%%%%%%%%%%%%
%\begingroup
\clearpage
\section*{Week 1}\phantomsection
\addcontentsline{toc}{section}{\textbf{Week 1}}
\fancyhead[R]{\hyperlink{home}{Week 1}}
\subsection{Friday, October 2}
\begin{itemize}
    \item Determing formal charge:
    \begin{itemize}
        \item Formula: {\color{o-Sun}\(FC = V - N - \dfrac{B}{2}\)}
        \item V = valance electrons of element
        \item N = lone pair electrons; B = bonded electrons
    \end{itemize}
    \item[1.] What is the formal charge on P in the following structure?  Each F and O has three lone pair of electrons.
        \begin{itemize}
            \item P = 5 - O - 8(0.5); P = {\color{pos}+1}
        \end{itemize}
    \item[2.] What is the formal charge on O in the structure above?
        \begin{itemize}
            \item O = 6 - 6 - 2(0.5); O = {\color{neg}-1}
        \end{itemize}
    \item[3.] What is the formal charge on P in the following structure?  Each F still has three lone pairs of electrons, and O had the tow pairs indicated.
        \begin{itemize}
            \item P = 5 - 0 - 10(0.5); P = \textbf{0}
        \end{itemize}
    \item[4.] Of the two structures shown for \ch{POF3}, which is the most stable, and will, therefore, be the most abundant form?
        \begin{itemize}
            \item \textbf{Structure II}
            \item \ch{O} has formal charge of \textbf{0} and is the 
            {\color{neg}most electronegative} element with difference in charge between the resonance structures.
            \item \ch{F} has greater electronegativity, but remains the same between both structures, so it's not relevant.
            \item Key difference: the double bond in structure II gives oxygen the {\color{o-Sun}lower magnitude} formal charge between the two.
        \end{itemize}
    \item[5.] The fundamental concept upon which VSEPR, and hence molecular shapes, is based is that:
        \begin{itemize}
            \item Electrons pairs repel each other;
                \begin{itemize}
                    \item negative charge repels other negative charges.
                \end{itemize}
            \item Electron repulsion is minimized by maximum angular separation; 
                \begin{itemize}
                    \item in other words, angular separation maximizes distance between electrons.
                \end{itemize}
            \item Bonding pair electrons and lone pair electrons both occupy regions around the central atom;
                \begin{itemize}
                    \item if they didn't occupy the same space than they wouldn't interact and thus wouldn't affect shape.
                \end{itemize}
            \item The electron dommain geometry and the molecular geometry is identical if there all of the electrons are bonding electrons;
                \begin{itemize}
                    \item the lone pairs are have a greater influence than bonded pairs, resulting in less space for bonded pairs.
                \end{itemize}
            \item \textbf{All of the above}
        \end{itemize}
    \item General method of determining structure:
    \begin{itemize}
        \item[1.] Count steric number---the total number of electron pairs in a molecule. Can be bonds or lone pairs.
        \item[2.] Determine predicted geomterical structure predicted (EDG) by VSEPR using steric number.
            \begin{itemize}
                \item Octahedral:6, Bipyramid:5, Tetrahedral:4, Trigonal:3, Linear:2
            \end{itemize}
        \item[3.] Determin impact (the MG) of lone pairs; more lone pairs results in less space between bonded pairs. Shape depends on EDG.
    \end{itemize}
    \item[6.] A resonance form of \ch{SOF2}, completely consistent with the octet rule,  is shown below.  What is the electron domain geometry (EDG), and molecular geometry (MG) of this molecule?
        \begin{itemize}
            \item \textbf{Tetrahedral EDG and trigonal pyramidal MG}
        \end{itemize}
    \item[7.] Draw a Lewis dot structure of formaldehyde (\ch{CH2O}): what is the molecular shape of this molecule?
        \begin{align*}
            \chemfig{C(=[2,]O)(-[5]H)(-[7]H)}
        \end{align*}
        \begin{itemize}
            \item Steric number = 3
                \begin{itemize}
                    \item Double bonds count as 1 for steric number.
                \end{itemize}
            \item No lone pairs on central atom, C, so it's shape planar. 
            \item \textbf{Trigonal planar}
        \end{itemize}
    \item[8.] The EDG for \ch{CH3-} (a carbanion) is tetrahedral, and the MG is trigonal pyramidal.  Why are the \ch{H-C-H} bond angles less than \ang{109.5} as in a perfect tetrahedron? 
        \begin{itemize}
            \item \textbf{The lone pair electrons take up more space than bonding pair electrons.}
        \end{itemize}
\end{itemize}
%\endgroup
%%%%%%%%%%%%%%%%%%%%%%%%%%%%% Week 1 %%%%%%%%%%%%%%%%%%%%%%%%%%%%%


\end{document}