\documentclass[12pt,a4paper]{article}
\usepackage{inverba}

\newcommand{\userName}{Cullyn Newman} 
\newcommand{\class}{CH 334} 
\newcommand{\institution}{Portland State University} 
\newcommand{\theTitle}{\color{B-Cold} Organic Chemisty}

\begin{document}
%%%%%%%%%%%%%%%%%%%%%%%%%%%%%%%%%%%%%%%%%%%%%%%%%%%%%%%%%%%%%%%%%%%%%
\tableofcontents
\cleardoublepage
\fancyhead{}
\fancyhead[R]{\hyperlink{home}{\nouppercase\leftmark}}
%%%%%%%%%%%%%%%%%%%%%%%%%%%%%%%%%%%%%%%%%%%%%%%%%%%%%%%%%%%%%%%%%%%%%

\clearpage
\fancyhead[L]{Organic Chemistry I}
%\begingroup
%%%%%%%%%%%%%%%%%%%%%%%%%%%%% Chapter 1 %%%%%%%%%%%%%%%%%%%%%%%%%%%%%
%\begingroup
\clearpage
\section{General Chemistry Review}\phantomsection
\subsection{Structural Theory of Matter}
\begin{itemize}
    \item \textbf{Constitutional isomers}: same molecular formula, but different in the way the atoms are connect, i.e. their constitution is different.
    \item Each element forms a predictable number of bonds, from one to four.
    \item \textbackslash ch\{x-x\} single: -, double: =, triple: +. e.g. \ch{CH3-CH3}, \ch{CH2=CH2}, \ch{CH+CH}
\end{itemize}

\subsection{Electrons, Bonds, and Lewis Structures}
\begin{itemize}
    \item \textbf{Covalent bond}: two atoms sharing a pair of electrons.
    \item The lowest energy (most stable) state of two atoms is determined both by bond length and bond strength.
    \item \textbf{Lewis structures}: drawings that show free electrons.
    \item Valence electrons are determined by the group, 1A--8A, of the periodic table.
    \item \textbf{Lone pair}: unshared, or nonbonding, electrons.
    \item \textbf{F, O, N, Cl} (Br, I). Most electronegative elements, from left to right; hydrogen needs to bond to these elements.
    \item Examples: \ch{COF2}, \ch{H2O}, \ch{NO3-}, \ch{N2O}
        \begin{itemize}
            \item \chemfig{C(=[2,.8]O)(-[5]F)(-[7]F)}
            \hspace{1cm}
            \chemfig{[:40]H-\lewis{13,O}-[::-80]H} 
        \end{itemize}
\end{itemize}

\subsection{Identifying Formal Charges}
\begin{itemize}
    \item \textbf{Formal charge}: any atom that does not exhibit the appropriate number of valance electrons. 
    \item {\color{pos}Less} than expected results in {\color{pos} positive} charge.
    \item {\color{neg}More} than expected results in {\color{neg}negative} charge.
\end{itemize}

\subsection{Induction and Polar Covalent Bonds}
\begin{itemize}
    \item Bonds are classified into three categories: covalent, polar covalent, ionic.
    \item The categories emerge from the electronegativity values of the atoms sharing a bond.
    \item \textbf{Electronegativity}: a measure of the ability of an atom to attract electrons.
    \item Electronegativity generally {\color{o-Sun}increases left to right}, and from the {\color{o-Sun}bottom to top} of the periodic table.
    \item If the difference in electronegativity is {\color{o-Sun}less than 0.5}, then the electrons are considered equally shared, which is a {\color{o-Sun}\textbf{covalent bond}}.
    \item If the difference in electronegativity is {\color{o-Sun}between 0.5 and 1.7}, then the electrons are not equally shared and thus a {\color{o-Sun}\textbf{polar covalent bond}}.
    \item \textbf{Induction}: the withdrawl of electrons towards to more electronegative atom. {\color{pos}\ch{$\delta$^+}} represnets partial positive charged gained when electrons are pulled away, while {\color{neg}\ch{$\delta$^-}} represnets the partial negative charge pulled closer.
    \item If the difference in electronegativity is {\color{o-Sun}greater than 1.7} then the electrons are not shared and results in an {\color{o-Sun}\textbf{ionic bond}}, which is really just a result of the force between two oppositely charged ions. 
\end{itemize}

\subsection{Atomic Orbitals}
\begin{itemize}
    \item \textbf{Atomic orbital (AO)}: \textit{s(1), p(3), d(5), f(7)}. 
    \item Locations where $\psi$ is zero are called \textbf{nodes}.
    \item The more nodes that an orbital has, the greater it's energy.
    \item \textbf{Degenerate orbitals}: orbitals with the same energy level.
    \item Order in which orbitals are filled is determined by three principles:
        \begin{itemize}
            \item \textbf{Aufbau principle}: lowest energy orbital is filled first.
            \item \textbf{Pauli exclusion principle}: each orbital can accommodate a maximum of two electrons that have opposite spin.
            \item \textbf{Hund's rule}: electrons are placed in each degenerate orbital before being paired up.
        \end{itemize}
    \item Describing the nature of atomic orbital is done with two commoly used theories: \textit{Valence Bond Theory} and \textit{Molecular Orbital Theory}.
    \item The commonly used theories give a deeper understanding of covalent bonds, which is essentially {\color{o-Sun}overlap of atomic orbitals}.
    \item \textbf{Constructive/destructive interference}: the result of two waves that approach each other, or overlap.
        \begin{itemize}
            \item Constructive interference produces a wave with larger amplitude.
            \item Destructive interference cancel each other out and produes a node.
        \end{itemize}
\end{itemize}

\subsection{Valence Bond Theory}
\begin{itemize}
    \item \textbf{Valence bond theory}: the sharing of electron density between two atoms as a result of the constructive interference of their atomic orbitals.
    \item \textit{Bond axis}: the line that can be drawn between two hydrogen atoms.
    \item \textbf{Sigma bond (\(\bm{\sigma}\))}: a particular type of covalent bond that has circular symmertry with respect to the bond axis.
        \begin{itemize}
            \item All single bonds are $\sigma$ bonds.
            \item The strongest type of covalent bond.
        \end{itemize}
    \item \textbf{Pi bond ($\bm{\pi}$)}: covalent bonds where two lobes of an orbital overlap with two lobes of another atom. 
        \begin{itemize}
            \item Each atomic orbital has zero electron density at a shared nodal plane, passing through the two bonded nuclei.
            \item $\pi$ bonds from double and triple bonds but generally do not form single bonds.
        \end{itemize}
\end{itemize}

\subsection{Molecular Orbital Theory}
\begin{itemize}
    \item \textbf{Molecular orbital theory (MO)}: uses linear combinations of atomic orbitals to model and explore the consequences of orbital overlap.
        \begin{itemize}
            \item The newly described orbitals are called {\color{o-Sun}molecular orbitals} accroding to MO theory.
        \end{itemize}
    \item Atomic orbitals refer to an individual atom, while molecular orbitals is associated with an entire molecular.
    \item In other words, MO theory states that atomic orbitals cease to exist when they overlap. Instead they are replaced with multiple molecular orbitals which span the entire molecule.
\end{itemize}

\subsection{Hybridized Atomic Orbitals}
\begin{itemize}
    \item \textbf{\(\bm{sp^3}\)-hybridized orbitals}: produced by averaging one \textit{s} orbital and {\color{o-Sun}three} \textit{p} orbitals.
        \begin{itemize}
            \item Hybridized orbitals explains to geomtry of methane, which results form the {\color{o-Sun}now four degenerate} orbitals pushing apart to achieve tetrahedral geometry.
            \item Hybridized orbitals become {\color{o-Sun}unsymmetrical}, producing a larger front lobe that is more efficient than standard \textit{p} orbitals in the ability to form bonds.
            \item All bonds in are {\color{o-Sun}$\sigma$ bonds}, and thus can be individually represented by the overlap of atomic orbitals.
        \end{itemize}
    \item \textbf{\(\bm{sp^2}\)-hybridized orbitals}: produced by averaging the \textit{s} orbital with only {\color{o-Sun}two} of \textit{p} orbitals.
        \begin{itemize}
            \item The remaining \textit{p} orbital is unaffected, and free multiple p orbitals results in a $\pi$ bond.
            \item This is done to expain geometry of compounds bearing a double bond.
            \item A double bond if formed from one $\sigma$ bond and one $\pi$ bond.
            \item Associated with \textit{trigonal planar geometry}.
        \end{itemize}
    \item \textbf{\(\bm{sp}\)-hybridized orbitals}: produced by averaging of one \textit{s} orbital and {\color{o-Sun}one} \textit{p} orbital.
        \begin{itemize}
            \item Leaves two \textit{p} orbitals and resulting in two $\pi$ bonds.
            \item A triple bond is formed with the addition of one $\sigma$ bond due to the overlap of the sp orbitals.
            \item Geometry of a triple bond has \textit{linear geometry}.
        \end{itemize}
    \item Finding the hybridization of any atom can be done simply: 
        \begin{itemize}
            \item[1.] Look at the central item.
            \item[2.] Determin groups (number of atoms and lone pairs attached) of atom.
            \item[3.] For groups 1-4: sp\(^{x}\); x = groups - 1  
            \item[4.] For groups 5-6: sp\(^{3}\)d\(^{x}\); x = groups - 4 
        \end{itemize}
    \item Bond Strength and Bond Length:
        \begin{itemize}
            \item Bond length {\color{neg}decreases} with more bonds.
            \item Bond strength {\color{pos}increases} with more bonds.
        \end{itemize}
\end{itemize}

\subsection{Molecular Geometry}
\begin{itemize}
    \item \textbf{Valence shell electron pair repulsion (VSEPR) theory}: enables the {\color{o-Sun}prediction of molecular geometry} due to the pressumption that all electron pairs repel each other; resulting in a three-dimensional space that {\color{o-Sun}maximizes distance} from each other.
    \item \textbf{Steric number}: the total number of electron pairs in a molecule. Can be bonds or lone pairs.
    \item \textbf{Tetrahedral geometry}: result of four $\sigma$ bonds and zero lone pairs. 
        \begin{itemize}
            \item produces a tetrahendron with bond angles of \ang{109.5}.
        \end{itemize}
    \item \textbf{Trigonal pyramidal geometry}: three $\sigma$ bonds and one lone pair.
        \begin{itemize}
            \item The lone pair occupy more space than bonded electron pairs, so the remaining angles are slightly less than a tetrahedral, at \ang{107}.
            \item The lone pair sits atop the base forming a pyramid like structure.
        \end{itemize}
    \item \textbf{Bent geometry}: two $\sigma$ bonds and two lone pairs.
        \begin{itemize}
            \item VSEPR predicts the lone pairs to be in two corners of the tetrahedral, producing bond angles of \ang{105}.
            \item VSEPR predicts geometry \ch{H2O} correctly, but for wrong reasons.
                \begin{itemize}
                    \item The lone pairs in \ch{H2O} have different energy levels, suggesting one pair occupies a \textit{p} orbital with the other in a lower-energy hybridized orbital.
                \end{itemize}
        \end{itemize}
    \item VSEPR theory is best used for a first approximation and is mostly accurate for most small molecules.
    \item \textbf{Trigonal planar geometry}: three electron pairs forming three bond angles of \ang{120} and lie on the same plan. 
    \item \textbf{Linear geometry}: two electron pairs that oppose each other at \ang{180}, forming a linear structure.
    \item General method of determining structure:
        \begin{itemize}
            \item[1.] Count steric number (electron pairs, or the bonds/lonepairs).
            \item[2.] Determine predicted geomterical structure predicted by VSEPR using steric number (tetrahedral:4, trigonal:3, linear:2).
            \item[3.] Determin impact of lone pairs; more lone pairs results in less space between bonded pairs.
        \end{itemize}
\end{itemize}

\subsection{Dipole Moments and Molecular Polarity}
\begin{itemize}
    \item 
\end{itemize}

\subsection{Intermolecular Forces and Physical Properties}
\begin{itemize}
    \item 
\end{itemize}

\subsection{Solubility}
\begin{itemize}
    \item 
\end{itemize}
%\endgroup
%%%%%%%%%%%%%%%%%%%%%%%%%%%%% Chapter 1 %%%%%%%%%%%%%%%%%%%%%%%%%%%%%
%\endgroup
\end{document}