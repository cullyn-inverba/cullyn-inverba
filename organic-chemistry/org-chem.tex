\documentclass[12pt,a4paper]{article}
\usepackage{inverba}

\newcommand{\userName}{Cullyn Newman} 
\newcommand{\class}{CH 334} 
\newcommand{\institution}{Portland State University} 
\newcommand{\theTitle}{\color{B-Cold} Organic Chemisty}

\begin{document}
%%%%%%%%%%%%%%%%%%%%%%%%%%%%%%%%%%%%%%%%%%%%%%%%%%%%%%%%%%%%%%%%%%%%%
\tableofcontents
\cleardoublepage
\fancyhead{}
\fancyhead[R]{\hyperlink{home}{\nouppercase\leftmark}}
%%%%%%%%%%%%%%%%%%%%%%%%%%%%%%%%%%%%%%%%%%%%%%%%%%%%%%%%%%%%%%%%%%%%%

\clearpage
\fancyhead[L]{Organic Chemistry I}
%\begingroup
%%%%%%%%%%%%%%%%%%%%%%%%%%%%% Chapter 1 %%%%%%%%%%%%%%%%%%%%%%%%%%%%%
%\begingroup
\clearpage
\section{General Chemistry Review}\phantomsection
\subsection{Structural Theory of Matter}
\begin{itemize}
    \item \textbf{Constitutional isomers}: same molecular formula, but different in the way the atoms are connect, i.e. their constitution is different.
    \item Each element forms a predictable number of bonds, from one to four.
    \item \textbackslash ch\{x-x\} single: -, double: =, triple: +. e.g. \ch{CH3-CH3}, \ch{CH2=CH2}, \ch{CH+CH}
\end{itemize}

\subsection{Electrons, Bonds, and Lewis Structures}
\begin{itemize}
    \item \textbf{Covalent bond}: two atoms sharing a pair of electrons.
    \item The lowest energy (most stable) state of two atoms is determined both by bond length and bond strength.
    \item \textbf{Lewis structures}: drawings that show free electrons.
    \item Valence electrons are determined by the group, 1A--8A, of the periodic table.
    \item \textbf{Lone pair}: unshared, or nonbonding, electrons.
    \item \textbf{F, O, N, Cl} (Br, I). Most electronegative elements, from left to right; hydrogen needs to bond to these elements.
    \item Examples: \ch{COF2}, \ch{H2O}, \ch{NO3-}, \ch{N2O}
        \begin{itemize}
            \item \chemfig{C(=[2,.8]O)(-[5]F)(-[7]F)}
            \hspace{1cm}
            \chemfig{[:40]H-\lewis{13,O}-[::-80]H} 
        \end{itemize}
\end{itemize}

\subsection{Identifying Formal Charges}
\begin{itemize}
    \item \textbf{Formal charge}: any atom that does not exhibit the appropriate number of valance electrons. 
    \item {\color{pos}Less} than expected results in {\color{pos} positive} charge.
    \item {\color{neg}More} than expected results in {\color{neg}negative} charge.
\end{itemize}

\subsection{Induction and Polar Covalent Bonds}
\begin{itemize}
    \item Bonds are classified into three categories: covalent, polar covalent, ionic.
    \item The categories emerge from the electronegativity values of the atoms sharing a bond.
    \item \textbf{Electronegativity}: a measure of the ability of an atom to attract electrons.
    \item Electronegativity generally {\color{o-Sun}increases left to right}, and from the {\color{o-Sun}bottom to top} of the periodic table.
    \item If the difference in electronegativity is {\color{o-Sun}less than 0.5}, then the electrons are considered equally shared and this {\color{o-Sun}\textbf{covalent}}.
    \item If the difference in electronegativity is {\color{o-Sun}between 0.5 and 1.7}, then the electrons are not equally shared and thus a {\color{o-Sun}\textbf{polar covalent bond}}.
    \item \textbf{Induction}: the withdrawl of electrons towards to more electronegative atom. {\color{pos}\ch{$\delta$^+}} represnets partial positive charged gained when electrons are pulled away, while {\color{neg}\ch{$\delta$^-}} represnets the partial negative charge pulled closer.
    \item If the difference in electronegativity is {\color{o-Sun}greater than 1.7} then the electrons are not shared and results in an {\color{o-Sun}\textbf{ionic bond}} which is just a result of the force between two oppositely charged ions. 
\end{itemize}

\subsection{Atomic Orbitals}
\begin{itemize}
    \item \textbf{Atomic orbital (AO)}: \textit{s(1), p(3), d(5), f(7)}. 
    \item Locations where $\psi$ is zero are called \textbf{nodes}.
    \item The more nodes that an orbital has, the greater it's energy.
    \item \textbf{Degenerate orbitals}: orbitals with the same energy level.
    \item Order in which orbitals are filled is determined by three principles:
        \item \textbf{Aufbau principle}: lowest energy orbital is filled first.
        \item \textbf{Pauli exclusion principle}: each orbital can accommodate a maximum of two electrons that have opposite spin.
        \item \textbf{Hund's rule}: electrons are placed in each degenerate orbital before being paired up.
\end{itemize}

\subsection{Valence Bond Theory}
\begin{itemize}
    \item 
\end{itemize}

\subsection{Molecular Orbital Theory}
\begin{itemize}
    \item 
\end{itemize}

\subsection{Hybridized Atomic Orbitals}
\begin{itemize}
    \item 
\end{itemize}

\subsection{Molecular Geometry}
\begin{itemize}
    \item 
\end{itemize}

\subsection{Dipole Moments and Molecular Polarity}
\begin{itemize}
    \item 
\end{itemize}

\subsection{Intermolecular Forces and Physical Properties}
\begin{itemize}
    \item 
\end{itemize}

\subsection{Solubility}
\begin{itemize}
    \item 
\end{itemize}
%\endgroup
%%%%%%%%%%%%%%%%%%%%%%%%%%%%% Chapter 1 %%%%%%%%%%%%%%%%%%%%%%%%%%%%%
%\endgroup
\end{document}