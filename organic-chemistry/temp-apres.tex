\documentclass[12pt,a4paper]{article}
\usepackage{inverba}

\newcommand{\userName}{Cullyn Newman} 
\newcommand{\class}{[Subject]} 
\newcommand{\institution}{[Institution]} 
\newcommand{\theTitle}{\color{B-Cold} [Subject Title]}

\begin{document}
%%%%%%%%%%%%%%%%%%%%%%%%%%%%%%%%%%%%%%%%%%%%%%%%%%%%%%%%%%%%%%%%%%%%%
\tableofcontents
\cleardoublepage
\fancyhead{}
\fancyhead[R]{\hyperlink{home}{\nouppercase\leftmark}}
%%%%%%%%%%%%%%%%%%%%%%%%%%%%%%%%%%%%%%%%%%%%%%%%%%%%%%%%%%%%%%%%%%%%%
\clearpage
\section*{Week x}\phantomsection
\addcontentsline{toc}{section}{\textbf{Week x}}
\fancyhead[R]{\hyperlink{home}{Week x}}

\fancyhead[L]{\hyperlink{home}{Friday, November 13 - Quiz 18}}
\subsection{Friday, November 13 - Quiz 18}
\begin{enumerate}
    {\color{G-Moon}\item Which parameter, kinetic or thermodynamic, gives us information about the rate and mechanism of the reaction?}
        \begin{itemize}
            \item {\color{o-Sun}\textbf{Kinetics}} 
            \begin{itemize}
                \item Kinetics refers to the rate of a reaction, while thermodynamics refers to the equilibrium concentrations of the reactants and products.
            \end{itemize}
        \end{itemize}
    {\color{G-Moon}\item For a reaction to be spontaneous, \(\Delta G^\circ\) must be}
        \begin{itemize}
            \item {\color{o-Sun}\textbf{Negative}} 
            \begin{itemize}
                \textbf{Gibbs free energy (\(\Delta G^\circ\))}: the maximum amount of non-expansion work that can be extracted from a closed system.
                    \begin{itemize}
                        \item Essentially a repackaged way of expressing entropy in a closed system.
                        \item \(\Delta G^\circ\) is just \(\Delta S^\circ\) multiplied by the negative temperature in order to measure the entropy of the surroundings.
                        \item Thus, {\color{o-Sun}\(\Delta G^\circ\) must be {\color{neg}negative} for  a reaction to be spontaneous.} (second law of thermodynamics)
                    \end{itemize}
            \end{itemize}
        \end{itemize}
    {\color{G-Moon}\item Which part of the energy level diagram shown below, represents the kinetic component, and which represents the the thermodynamic component? }
        \begin{itemize}
            \item {\color{o-Sun}\textbf{A is kinetic, B is thermodynamic}} 
            \begin{itemize}
                \item \textbf{A} represents the activation energy (\(E_a, \Delta G_{\text{act}}\)) required for a reaction to form products.
                \item \textbf{B} represents the difference in free energy (\(\Delta G^\circ\)) of the products from the initial reactants.
                    \begin{itemize}
                        \item {\color{pos}\textbf{Endergonic}}: {\color{o-Sun}nonspontaneous} ({\color{pos}+\(\Delta G^\circ\)}) processes.
                        \item {\color{neg}\textbf{Exergonic}}: {\color{o-Sun}spontaneous} ({\color{neg}-\(\Delta G^\circ\)}) processes.
                    \end{itemize}
            \end{itemize}
        \end{itemize}
    {\color{G-Moon}\item Which factors affect the equilibrium of a reaction?}
        \begin{itemize}
            \item {\color{o-Sun}\textbf{a and b} (temperature and concentration)} 
            \begin{itemize}
                \item Rate of the reaction and catalysts effect the kinetics (rate), not equilibrium (thermodynamics)
            \end{itemize}
        \end{itemize}
    \newpage
    {\color{G-Moon}\item Which factors affect the rate of a reaction?}
        \begin{itemize}
            \item {\color{o-Sun}\textbf{a, c, and c} (concentration, temperature, and presences of catalysts)} 
            \begin{itemize}
                \item There are more, such as, substrate type, physical state, surface area, concentration, temperature, catalysts, pressure, and light absorption. 
                \item Essentially concentration effects the \textit{rate order}, while everything else effects the rate constant \(k\) (\(\text{rate} = k[\text{reactants}]\))
                \item As described above, endergonic and exergonic has to do with \(\Delta G^\circ\) (equilibrium).
            \end{itemize}
        \end{itemize}
    {\color{G-Moon}\item A small activation energy, \(\Delta G_{\text{act}}\), corresponds to a}
        \begin{itemize}
            \item {\color{o-Sun}\textbf{a fast reaction}} 
            \begin{itemize}
                \item Lower the activation energy, the less potential energy is needed for a reaction to occur, and thus more likely (faster) that a larger portion of molecules will undergo the reaction upon colliison. 
            \end{itemize}
        \end{itemize}
    {\color{G-Moon}\item Which parameter, kinetic or thermodynamic, determines the equilibrium constant and how much product will form?}
        \begin{itemize}
            \item {\color{o-Sun}\textbf{Thermodynamic}} 
            \begin{itemize}
                \item Kinetics refers to the rate of a reaction, while thermodynamics refers to the equilibrium concentrations of the reactants and products. 
            \end{itemize}
        \end{itemize}
    {\color{G-Moon}\item A large and negative \(\Delta G^\circ\) will correspond to}
        \begin{itemize}
            \item {\color{o-Sun}\textbf{a large \(K_{\text{eq}}\)}} 
            \begin{itemize}
                \item A large {\color{neg}-\(\Delta G^\circ\)} means the {\color{neg}products} are favored. (\(1 < K_{\text{eq}}\))
                \item A large {\color{pos}+\(\Delta G^\circ\)}: the {\color{pos}reactants} are favored. (\(K_{\text{eq}} < 1\))
                \item \(K_{\text{eq}}\) says nothing about rate, that is determined by the kinetics, instead it just determins the direction and spontaneity.
            \end{itemize}
        \end{itemize}
\end{enumerate}
\end{document}