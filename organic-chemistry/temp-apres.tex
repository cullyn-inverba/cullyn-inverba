\documentclass[12pt,a4paper]{article}
\usepackage{inverba}

\newcommand{\userName}{Cullyn Newman} 
\newcommand{\class}{[Subject]} 
\newcommand{\institution}{[Institution]} 
\newcommand{\theTitle}{\color{B-Cold} [Subject Title]}

\begin{document}
%%%%%%%%%%%%%%%%%%%%%%%%%%%%%%%%%%%%%%%%%%%%%%%%%%%%%%%%%%%%%%%%%%%%%
\tableofcontents
\cleardoublepage
\fancyhead{}
\fancyhead[R]{\hyperlink{home}{\nouppercase\leftmark}}
%%%%%%%%%%%%%%%%%%%%%%%%%%%%%%%%%%%%%%%%%%%%%%%%%%%%%%%%%%%%%%%%%%%%%
\clearpage
\section*{Week 6}\phantomsection
\addcontentsline{toc}{section}{\textbf{Week 6}}
\fancyhead[R]{\hyperlink{home}{Week 6}}

\fancyhead[L]{\hyperlink{home}{Tuesday, November 10 - Quiz 17}}
\subsection{Tuesday, November 10 - Quiz 17}
\begin{enumerate}
    {\color{G-Moon}\item Based on the Cahn Ingold Prelog rules, rank the following in order of lowest to highest priority.
    
    \ch{CH3}
    \hspace*{20pt}
    \ch{CH2CH3}
    \hspace*{20pt}
    \ch{CH2CH2OH}
    \hspace*{20pt}
    \ch{CH2CH2NH2}
    \hspace*{20pt}
    \ch{CH3CHO}
    }
        \begin{itemize}
            \item {\color{o-Sun}\textbf{\ch{CH3} < \ch{CH2CH3} < \ch{CH2CH2NH} < \ch{CH2CH2OH} < \ch{CH3CHO}}}
            \begin{itemize}
                \item Relevant notes (highlighted applies to this problem):
                    \begin{itemize}
                        \item \textbf{Chan-Ingold-Prelog system}: a system of nomenclature for Identifying each enantiomer individually.
                        \begin{enumerate}
                            {\color{o-Sun}\item Assign priorties to each of the four groups based on atomic number; the highest atomic number has the highest priority.}
                            \item Rotate the molecule so that the fourth priority group is on a dash (behind)
                            \item Determin the configuration, i.e., sequence of 1-2-3 groups.
                                \begin{itemize}
                                    \item {\color{true}clockwise (R)} or {\color{false}counterclockwise (S)}.
                                \end{itemize}
                        \end{enumerate}
                    {\color{o-Sun}\item If there is a tie between the atoms connected, then continue outward until a difference is found.}
                        \begin{itemize}
                            \item Do not add the sum all atomic numbers attached to each atom, just the first in which the atoms differ.
                           {\color{o-Sun}\item Any multiple bonded atom, (2 or 3) is treated as if connected to multiple atoms equal to number of bonds.}
                        \end{itemize} 
                    \end{itemize}
                \item First difference is the \ch{NH2} vs \ch{OH2}; oxygen has more mass.
                \item Leaves difference between \ch{CH2OH} vs \ch{CHO}; oxygen is double bonded in the latter, so really {CHOO} (O beats H).
            \end{itemize}
        \end{itemize}
    {\color{G-Moon}\item Determine the absolute configuration of the following molecule.
    
    \chemfig{\ch{H3C}-[:30](-[:91]\ch{OH})(<[:-60]\ch{CH2CH3})(<:[:-10]H)}
    
    }
        \begin{itemize}
            \item {\color{o-Sun}\textbf{R}}
            \begin{itemize}
                \item {\color{true}\chemfig{\ch{3}-[:30](-[:91]\ch{1})(<[:-60]\ch{2})(<:[:-10]4)}\hspace*{20pt} clockwise(R)}
                \item H = 4, \ch{CH3} = 3, \ch{CH2CH3} = 2, \ch{OH} = 1.
                \item Lowest priority is already in the back, leave as is.
            \end{itemize}
        \end{itemize}
    {\color{G-Moon}\item The molecule shown below is the enantiomer of the molecule shown in the previous question.  What is its absolute configuration?
    
    \chemfig{\ch{H3C}-[:30](-[:91]\ch{CH2CH3})(<[:-60]\ch{OH})(<:[:-10]H)}
    }
        \begin{itemize}
            \item {\color{o-Sun}\textbf{S}}
            \begin{itemize}
                \item \schemestart
                {\color{true}\chemfig{
                    \ch{3}-[:30](-[:91]\ch{1})(<[:-60]\ch{2})(<:[:-10]4)
                    }}
                \arrow{->}
                {\color{false}\chemfig{
                    \ch{3}-[:30](-[:91]\ch{2})(<[:-60]\ch{1})(<:[:-10]4)
                    }}
                \schemestop
                \item Swap 1 and 2, swap the configuration; \ch{R -> S}
            \end{itemize}
        \end{itemize}
    {\color{G-Moon}\item What is the absolute configuration of the molecule shown below?
    
    \chemfig{
        \ch{Br}-[:30](-[:91]\ch{CH3})(<[:-60]\ch{OH})(<:[:-10]\ch{CHO})
    }
    }
        \begin{itemize}
            \item {\color{o-Sun}\textbf{R}}
            \begin{itemize}
                \item \schemestart
                \chemfig{
                    \ch{1}-[:30](-[:91]\ch{4})(<[:-60]\ch{3})(<:[:-10]\ch{2})
                    }
                \arrow{->}
                {\color{true}\chemfig{
                    \ch{1}-[:30](-[:91]\ch{2})(<[:-60]\ch{3})(<:[:-10]\ch{4})
                    }}
                \schemestop
                \item Once lowest priority is assigned to dashed wedge(aka cram), then one can tell sequence. Original is an S.
            \end{itemize}
        \end{itemize}
\end{enumerate}
\end{document}