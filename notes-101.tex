\documentclass{inVerba-notes}

\definecolor{title-color}{HTML}{78b07a} % #78b07a default contents/class name color
\newcommand{\theTitle}{Contents/Subject Name}


\begin{document}
\hypertarget{ToC}
\tableofcontents

\chapter{Introductory Chapter}

\begin{adjustwidth}{1cm}{1cm}
    Most notes I make will likely use textbooks as the primary source while I'm in university, though I am becoming more inclined to seek out better textbooks and resources. Not all textbooks are structured the same way, so I will attempt to make my structure constant regardless of source. To do this I may restructure whole chapters and sections, possibly even compressing or expanding on content if I find it necessary. I am creating this introductory guide in an attempt to clearly explain my formatting choices while also providing an example of how my notes will be structured so that others may find them useful. I admit they won't look as close to normal notes, instead being closer to a broken up essay, but I believe it can still serve an illustrative function.

    Each chapter will have a brief introduction to the content that will be discussed in the form of one or two short paragraphs. This is designed to be an abstract of sorts, clearly defining what will be discussed. The more explicit goal here is to explain my philosophy on note taking, explain and clearly define the meaning behind my formatting choices, and suggest how one might find the best use from the notes I create.
\end{adjustwidth}

\section{My Philosophy on Notes}\label{philosophy of notes}
\begin{itemize}
    \item The primary goal I am trying to achieve is to create a hierarchical note structure that \emph{can easily be referenced} and actually used in the future.
    \begin{itemize}
        \item Ideas are represented as nested lists to reduce total words needed to express ideas.
        \item Why waste time write lot word when few do trick?
            \begin{itemize}
                \item Obviously I will try to use the best grammar possible, joking aside.
            \end{itemize}
    \end{itemize}
    \item These notes (and I argue notes in general) are \fff{not the primary means} one should use for increasing an understanding of one's subject of interest. 
        \begin{itemize}
            \item Notes are just one step in the process of learning.
            \item The harder, more time-consuming, and often much more rewarding step is using the notes to solve difficult problems that challenge one's understanding.
        \end{itemize}
    \item Changing the way information is compressed forces me to recontextualize the information and prevents mindless copying of text.
    \item \textbfl{Chunking}: a cognitive psychology term defined as the processes in which pieces of information are broken down and then grouped together in a meaningful whole.
        \begin{itemize}
            \item E.g., 971 772 6835 is broken down into chunks that easier to remember, rather than 9717726835. 
                \begin{itemize}
                    \item Yes, that's my number, text me if you want---I'm curious to see if anyone actually will to be honest.
                \end{itemize}
            \item The use of this idea is what I am attempting to maximize; if I later find a previous note isn't helping or lacks info, then I can quickly update my ``chunk.''
            \item I was originally made aware of chunking thanks to an extremely helpful and \link{https://www.coursera.org/learn/learning-how-to-learn}{free online course} from Coursera taught by Dr.\ Barbara Oakley.
        \end{itemize}
    \subsection{On Typed vs.\ Handwritten Notes}
    \begin{itemize}
        \item I think you might be able to guess where I stand on this subject. However, I think it's important to discuss this, because I feel that I am in the minority on this argument. 
        \item I often hear claims that handwritten notes leads to better comprehension.
            \begin{itemize}
                \item In a traditional setting, I'd agree, but most of the arguments I hear seem to attribute the cause to the wrong reason, often talking about some special hand-mind phenomenon. 
                \item I'd argue the primary reason is that it's far too easy to copy and paste, or just type word for word, rather than form concepts with your own concentrated attention.
                    \begin{itemize}
                        \item You can't do be as lazy by hand, and handwritten note often avoid such problems due to time constraints.
                        \item As I mentioned above though, this recontextualization via change in structure is what I use to help fight this tendency.
                    \end{itemize}
            \end{itemize}
        \item The failures of handwritten notes often are underrepresented.
            \begin{itemize}
                \item It's very difficult to share them;
                \item I honestly can't read my own handwriting half the time;
                \item They take unnecessary amounts of extra effort to create, leading to no notes being taken at all;
                \item They are wasteful (save paper!), albeit the difference here might be negligible;
                \item And they are difficult to transport, backup, duplicate, edit, organize, and actually reuse in the future.
            \end{itemize}
        \item Unfortunately, some major benefits from typed notes might not be able to be capitalized on using more traditional methods.
            \begin{itemize}
                \item Using Google Docs or Word to write notes can work, but one they get past a certain length then they can be very hard to manage.
                \item Courses like math and chemistry will quickly challenge that speed and ease of use that typing brings. However, one can learn \LaTeX, which solves this problem (and many others).
                    \begin{itemize}
                        \item \LaTeX~also makes writing essays and scientific papers an absolute breeze---references and in paper citations become so damn easy. 
                        \item I will discuss \LaTeX~in more depth when I explain my \hyperref[design choices]{\dlink{design choices}}.
                    \end{itemize}
                \item Being able to use hyperlinks and the search function (F3/Ctrl+f) makes using the notes later much more helpful.
                \item Typing helps makes beautiful notes, which produces a sense of pride and value behind the craft that goes into making them---leading to even more beautiful notes---creating a positive feedback loop centered around the acquisition of knowledge.
            \end{itemize}
        \item The importance of what is ultimately is the better option isn't that important for my purposes. I've mainly included it as fun filler reading the give a more illustrative example of the structure of my notes.
    \end{itemize}
    \item There will often be a brief summary at the end of sections, meant to make sure important concepts are not missed. In summary:
        \begin{itemize}
            \item I use a hierarchical structure to recontextualize information.
            \item Notes are meant to be a tool used to make further learning more efficient.
            \item The creation of easily modifiable chunks of information act to serve both organizational and cognitive functions.
        \end{itemize}
    \item Essentially my goal follows very basic presentational approach of, ``Tell them what you are going to tell them, tell them, and then tell them what you said.''
        \begin{itemize}
            \item ``Them'' is really just myself in reality probably, but it's astonishing how fast you can forget something without meaningful practice.
        \end{itemize}
    \item Some disclaimers:
            \begin{itemize}
                \item Sections tend to have many more definitions, as well as clickable links to and from other sections with other relevant.
                    \begin{itemize}
                        \item I'm thinking of a way to make a web based application of notes that does not follow a linear approach to note taking, but that's a project I haven't started yet.
                    \end{itemize}
                \item Often each bullet point is just one sentence.
                \item Typically, subsections are shorter and there are more of them.
                \item There is normally more use of \emph{emphasis}, but this falls under the realm of my design philosophy, so I will elaborate on next section. 
            \end{itemize}
    \end{itemize}

\section{Design Choices: Formatting and Theming}\label{design choices}
\begin{itemize}
    \item Each section will also contain a few introductory sentences in form of bulleted list, with some possible supporting notes.
    \begin{itemize}
        \item This is where my formatting comes into play, which will be expanded on when I explain my \hyperref[design choices]{\dlink{formatting choices}}.
        \item \hyperref[Chunking]{\ulink{Chunking}}
    \end{itemize}
        \begin{itemize}
                        \item It's taken some time to learn, but now I it is hard to imagine not knowing how use \LaTeX. 
                        \item Not for everyone, but I highly encourage using it. I'm even working on a tutorial to help anyone get into it easier, since most sucked when I learned.
                        \item With enough use, you can make (and manipulate) beautiful organic chemistry molecules, math equations, graphs of all kinds, and pretty much anything you'd need to use often even \textit{faster} than by hand.
                        \item \LaTeX~can also be a gentle introduction to programming, which can lead to extremely beneficial skills for any profession.
                    \end{itemize}
\end{itemize}

 

\end{document}