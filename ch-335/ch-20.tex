\chapter{20: Conjugation and Resonance}\label{20: Conjugation and Resonance}
\section{Conjugation Basics/Review}\label{}
\begin{itemize}
  \item \ddd{Delocalization}: electrons not associated with a single atom or covalent bond.
    \begin{itemize}
      \item Refers to resonance in conjugated systems and aromatic compounds.
    \end{itemize}
  \item Conjugation: a system of connected p orbitals with delocalized electrons that generally lowers overall energy and increases stability.
    \begin{itemize}
      \item Represented as alternating \(\sigma \) and \(\pi \) bonds, i.e., the overlap of a p orbital with another across an adjacent \(\sigma \) bond.
      \item Most efficient when the p orbitals are coplanar.
    \end{itemize}
  \item General rules of contributing significance:
    \begin{enumerate}
      \item The greatest number of filled octets.
      \item The greatest number of covalent bonds.
      \item Minimize formally charged atoms.
      \item Minimization of unlike charged and maximization of like charges.
      \item Negative charges are on the most electronegative atoms, positive charges are on the most electropositive atoms.
      \item Do not deviate substantially from idealized bond lengths and angles.
      \item Maintain aromatic substructures locally while avoiding anti-aromatic molecules.
    \end{enumerate}


  \subsection{Conjugated Dienes}\label{Conjugated Dienes}
  \begin{itemize}
    \item Additions reaction of with HX and dienes result in both 1,2-addition and 1,4-addition pathways:
    
    \medskip
    \schemestart{}
    \chemfig{=[:30]-[:-30]=[:30]}
    \arrow{->[\ch{H-Br}]}
    \chemfig{4=[:30]3-[:-30]\pbe{2}-[:30]1(-[:90]H)}
    \+
    \chemfig{\pbe{4}-[:30]3=[:-30]2-[:30]1(-[:90]H)}
    \arrow{->[\bbb{\ch{Br^{-}}}]}
    \dots
    \schemestop{}
    \bigskip
    
    \medskip
    \schemestart{}
    \dots
    \qquad
    \chemfig{4=[:30]3-[:-30]2(-[:-90]Br)-[:30]1(-[:90]H)}
    \+
    \chemfig{4(-[:-90]Br)-[:30]3=[:-30]2-[:30]1(-[:90]H)}
    \schemestop{}
    \bigskip
    
    \item \ddd{Kinetic product}: the 1,2-product, which is formed more rapidly due to low activation energy, but it is less thermodynamically stable.
    \item \ddd{Thermodynamic product}: the 1,4-product, which has a higher activation energy, but is more thermodynamically stable.
  \end{itemize}
\end{itemize}

\section{Pericyclic Reactions}\label{Pericyclic Reactions}
\begin{itemize}
  \item \ddd{Pericyclic reaction}: a concerted reaction with a transition state that has cyclic geometry and where the bond orbitals involved overlap in a continuous cycle.
    \begin{itemize}
      \item No solvent effect; mediums make no difference.
      \item No electrophile or nucleophile involved.
      \item Products are highly stereospecific.
      \item Conditions are either thermal (\(\Delta \)) or photochemical (\(h\nu \)).
    \end{itemize}
  \item There are many types of pericyclic reactions, usually being rearrangement or addition reactions. 
    \begin{itemize}
      \item \ddd{Cycloaddition}: a reaction in which two or more unsaturated molecules (or parts of same molecule) combine to form a cyclic derivative in which there is a \emph{net reduction of bond multiplicity}, resulting in cyclization. 
    \end{itemize}
  
  \subsection{Diels-Alder Reaction}\label{Diels-Alder Reaction}
  \begin{itemize}
      \item \ddd{Diels-Alder reaction}: a concerted reaction between a conjugated diene and a substituted alkene (dienophile) to form a substituted cyclohexene. 
        \begin{itemize}
          \item \ddd{Diene}: a conjugated 4\(\pi \) electron system.
          \item \ddd{Dienophiles}: the 2\(\pi \) electron system (generally substituted).
        \end{itemize}
      \begin{center}
        
        \medskip
        \schemestart{}
        \chemfig{(-[:-90]=[:-30])=[:30]}
        \+{,,-15pt}
        \quad
        \chemfig{-[:-90,,,,draw=none]=[:90]}
        \arrow{}
        {\footnotesize\chemfig{*6(--[,,,,emph]--[,,,,emph]-=[,,,,emph]-)}}
        \schemestop{}
        \medskip
      \end{center}
        
      \item \rrr{Good dienophiles} are \rrr{electron poor}, i.e., they have substituents that contain an electron withdrawing group (\rrr{EWG}).
      \item \bbb{Good dienes} are \bbb{electron rich}, i.e., they have substituents that contain an electron donating group (\bbb{EDG}). 
      \item A diene is locked in \textit{cis} form has a lower activation energy than one that is not locked (alternating \textit{cis-trans}), leading to a faster reaction.
        \begin{itemize}
          \item S-\textit{trans} form is locked (closed system), meaning it cannot react. 
        \end{itemize}
      \item \xxx{Inward} facing bonds on the substituted dienophile results in the \xxx{dashed (axial)} position if a chiral carbon is created, while \yyy{outward} facing bonds results in the \yyy{wedge (equatorial)} position. 
        \begin{itemize}
          \item \ddd{Exo form (thermodynamic product)}: when the substituent approaches form the \yyy{outside}.
          \item \ddd{Endo form (kinetic product)}: when the substituent approaches from the \xxx{inside}. 
        \end{itemize}
      \item Reactions typically occur at moderate temperatures; not much activation energy is needed to reach the more stable state.
        \begin{itemize}
          \item \ddd{Endo-Rule}: the endo form (kinetic product) is thus generally the major product, due to better overlap of \(\pi \) electron system in the intermediate step.
        \end{itemize}
      \item Diels-Alder reactions are \emph{stereospecific} depending on approach condition, i.e.,  the \textit{cis} form of dienophile leads to \textit{cis}-type of product and the \textit{trans} form of dienophile leads to \textit{trans}-product.
      \item Diels-Alder reactions are also \emph{regiospecific}, with the ortho product being favored in case of substituted dienes, and the para product being favored int the case of substituted dienophiles.
        \begin{itemize}
          \item The 1-position is favored in 1,-substituted dienes.
          \item The avoidance of the meta-position is due to the preference for the most \bbb{nucleophilic} of the \bbb{diene} lining up with the most \rrr{electrophilic} part of the \rrr{dienophile}.
        \end{itemize}
      \item When reacting with \emph{alkynes}, then the diene has inward and outward methods of approach.
        \begin{itemize}
          \item The \yyy{outward} position of dines leads to the \yyy{dashed} position.
          \item The \xxx{inward} position of dienes, which typically forms the bridge, results in the \xxx{wedge} position. 
        \end{itemize}
  \end{itemize}
\end{itemize}
