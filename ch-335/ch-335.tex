\documentclass{inVerba-notes}

\definecolor{title-color}{HTML}{5f89f5}
\newcommand{\theTitle}{\href{https://github.com/cullyn-inverba/notes/tree/master/ch-335} {Organic Chemistry II}}

\begin{document}
\hypertarget{ToC}{\tableofcontents}
% chktex-file 36

%%%%%%%%%%% Reference Material %%%%%%%%%%%
%\begingroup
\chapter{Reference Material}\label{Reference Material}
\begin{adjustwidth}{1cm}{1cm}
  This section contains material from organic chemistry I. Any material I had to go back and reference at least once in order to complete a problem will be added, as well as some things I think might be useful. 
\end{adjustwidth}

\section{Common Patterns}
    \begin{itemize}
        \item \textbf{Associated Patterns for \emph{Oxygen}}
            \begin{itemize}
                \item A \bbb{negative (\(\circleddash \))} charge corresponds with \bbb{1 bond} and \bbb{3 lone pairs}.
                \item The \minor{absence} of charge corresponds with \minor{2 bonds} and \minor{2 lone pairs}.
                \item A \rrr{positive (\( \oplus \))} charge corresponds with \rrr{3 bonds} and \rrr{1 lone pair}.
            \end{itemize}
        \item \textbf{Associated Patterns for \emph{Nitrogen}}
            \begin{itemize}
                \item A \bbb{negative} charge corresponds with \bbb{2 bonds} and \bbb{2 lone pairs}.
                \item The \minor{absence} of charge corresponds with \minor{3 bonds} and \minor{1 lone pair}.
                \item A \rrr{positive} charge corresponds with \rrr{4 bonds} and \rrr{0 lone pairs}.
            \end{itemize}
        \subsection{Hydrogen Deficiency Index} 
            \begin{itemize}
                \item \ddd{Hydrogen deficiency index (HDI)}: the measure of degrees of unsaturation.
                \begin{itemize}
                    \item E.g., two degrees of unsaturation results in a HDI of 2.
                    \item Degrees of freedom help represent possbile structures, indicating possible double bounds, triple bounds, rings, or various combinations of each.
                    \item Only helpful when molecular formula is known for certainty.
                \end{itemize}
                \item Formula: \emph{HDI = \( \frac{1}{2}(2C + 2 + N - H - X) \)}
                \begin{itemize}
                    \item \(X\): halogen atoms.
            \end{itemize}
        \end{itemize}
    \end{itemize}

\section{Nomenclature of Alkanes}
\begin{itemize}
    \item \ddd{Alkane}: acyclic (linear structure) saturated hydrocarbons (no \(\pi \) bonds).
        \begin{itemize}
            \item General chemical formula: \emph{\ch{C_nH_{2n+2}}}
        \end{itemize}
    \subsection{Selecting the Parent Chain}
    \begin{itemize}
        \item \ddd{Parent chain}: the longest carbon chain in an alkane.
        \begin{table}[ht]
            \centering
            \caption{Parent Names for Alkanes\strut}\label{tab:alkanes}
            \begin{tabular}{ccc}
                \toprule
                Number of Carbons & Parent & Name \\
                \midrule
                1 & meth & methane \\
                2 & eth & ethane \\
                3 & pro & propane \\
                4 & but & butane \\
                5 & pent & pentane\\
                6 & hex & hexane \\
                7 & hept & heptane \\
                8 & oct & octane \\
                9 & non & nonane \\
                10 & dec & decane \\
                11 & undec & undecane \\
                12 & dodec & dodecane \\
                13 & tridec & tridecane \\
                14 & tetradec & tetradecane \\
                15 & pentadec & pentadecane\\
                20 & eicos & eicosane \\
                30 & triacont & triacontane \\
                40 & tetracont & tetracontane \\
                50 & pentacont & hectane \\
                100 & hect & hectane \\
                \bottomrule
                \end{tabular}
        \end{table}   
    \item \ddd{Substituents}: branches connected to the parent chain, can be a single atom, groups of atoms, that replace one or more hydrogen atoms. 
        \begin{itemize}
            \item If there is competition between chains of \emph{equal length}, then \emph{choose the chain with the greatest number of substituents}.
        \end{itemize}
    \end{itemize}
    \subsection{Naming Substituents}
    \begin{itemize}
        \item \ddd{Alkyl groups}: Substituents that are named the same as the parents, but with the added letters ``ly''.
        \begin{table}[ht]
            \centering
            \caption{Names of Alkyl Groups\strut}\label{tab:substituents}
            \begin{tabular}{ccc}
                \toprule
                Substituent Carbons&  Terminology\\
                \midrule
                1 & methyl\\
                2 & ethyl \\
                3 & propyl \\
                4 & butyl \\
                5 & pentyl \\
                6 & hexyl \\
                7 & heptyl \\
                8 & octyl \\
                9 & nonyl \\
                10 & decyl \\
                \bottomrule
                \end{tabular}
        \end{table}
        \item When a group is connected to the ring, then the ring is generally treated as the parent.
            \begin{itemize}
                \item If the ring has fewer atoms than the rest of the structure, then it becomes a substituent.
            \end{itemize}
    \end{itemize}
    \subsection{Naming Complex Substituents}
    \begin{itemize}
        \item \ddd{Complex substituents}: branched alkyl substituents.
        \item Begin by numbering carbons going \emph{away} from the parent chain, then name it as if it's a parent chain itself.
            \begin{itemize}
                \item Complex substituent are placed in parentheses, indicating it as a single substituent of the parent chain.
            \end{itemize}
        \item Some complex substituents have common names that are so well established and allowed by IUPAC\@.
            \begin{itemize}
                \item An alkyl group bearing \emph{three} carbon atoms; only one way to branch it.
                    \begin{itemize}
                        \item \ddd{Isopropyl group}: (1-methylethyl): {\tiny\chemfig{-[:0](-[::60])-[::-60]}}
                    \end{itemize}
                \item Alkyl groups bearing \emph{four} carbon atoms, which can be branched three different ways:
                    \begin{itemize}
                        \item \ddd{sec-butyl} (1-methylpropyl): {\tiny\chemfig{-[:0](-[::60]-[:0])-[::-60]}}
                        \item \ddd{isobutyl} (2-methylpropyl): {\tiny\chemfig{-[:0](-[::60](-[:120])-[:0])}}
                        \item \ddd{tert-butyl} (1,1-dimethylethyl): {\tiny\chemfig{-[:0](-[::60])(-[:0])(-[::-60])}} 
                    \end{itemize}
                \item Alkyl groups bearing \emph{five} carbons, which can be branched many more ways. Two common ways:
                    \begin{itemize}
                        \item \ddd{isopentyl (isoamyl)} (3-methylbutyl): 
                    {\tiny\chemfig{-[:0]-[::60]-[:0](-[:60])-[:-60]}}
                        \item \ddd{neopentyl} (2,2-dimethylpropyl):
                    {\tiny\chemfig{-[:0]-[::60](-[:-15])(-[:50])(-[:120])}}
                    \end{itemize}
            \end{itemize}
    \end{itemize}
    \subsection{Assembling the Systematic Name}
    \begin{itemize}
        \item \ddd{Locant}: the location of a carbon numbered parent chain.
        \item Rules for assigning locant:
            \begin{itemize}
                \item If one substituent is present, then assign the lowest number possible.
                \item When multiple substituents are present, then the first substituent receives the lowest number. 
                    \begin{itemize}
                        \item If there is a tie, the second locant should be as low as possible.
                        \item If tie cannot be broken, then lowest number is assigned alphabetically.
                    \end{itemize}
                \item Prefixes are used when the same substituent appears more than once. 
                    \begin{itemize}
                        \item di: 2, tri: 3, tetra: 4, penta: 5, hexa: 6.
                    \end{itemize}
                \item Hyphens are used to separate numbers from letters, while commas are used to separate numbers from each other.
                \item Substituents are alphabeticalized after all locants are correctly assigned.
                    \begin{itemize}
                        \item Prefixes are ignored during alphabeticalization.
                    \end{itemize}
            \end{itemize}
        \item Summary of discrete steps:
            \begin{enumerate}
                \item \textbf{Identify parent chain.}
                \item \textbf{Identify and name substituents.}
                \item \textbf{Number the parent chain and assign a locant to each substituent.}
                \item \textbf{Arrange the substituents alphabetically.}
            \end{enumerate}
    \end{itemize}
\end{itemize}

\section{Drawing Chair Conformations}
    \begin{center}
        \chemfig{?-[:-50]-[:10]-[:-10]-[:130]-[:190]?} \hspace{20pt} \chemfig{?
        (-[:90,0.7,,,rrr, line width=1pt])
        (-[:190,0.7,,,bbb, line width=1pt])
        -[:-50](-[:-90,0.7,,,rrr, line width=1pt])
        (-[:170,0.7,,,bbb, line width=1pt])
        -[:10](-[:90,0.7,,,rrr, line width=1pt])
        (-[:300,0.7,,,bbb, line width=1pt])
        -[:-10](-[:-90,0.7,,,rrr, line width=1pt])
        (-[:10,0.7,,,bbb, line width=1pt])
        -[:130](-[:90,0.7,,,rrr, line width=1pt])
        (-[:-10,0.7,,,bbb, line width=1pt])
        -[:190]
        ?(-[:-90,0.7,,,rrr, line width=1pt])
        (-[:120,0.7,,,bbb, line width=1pt])}
    \end{center}
    \begin{itemize}
        \item \ddd{Axial position}: parallel to a vertical axis passing through the center of the ring.
            \begin{itemize}
                \item Axial positions are less stable than equatorial due to steric strain.
            \end{itemize}
        \item \ddd{Equatorial}: positioned approximately along the equator of the ring.
        \item The chair is more stable when the methyl (substituent) group is in the \emph{equatorial} position.
        \item The larger the substituent, the more equatorial-substituted conformer is favored.
    \end{itemize}

    \section{Overview of Stereoisomerism}
    \begin{itemize}
        \item \ddd{Constitutional isomers}: aka structural isomers; same \emph{chemical formula}, but different in the way the \emph{atoms are connected}, i.e., their constitution is different.
        \item \ddd{Stereoisomers}: isomers that differ in \emph{spatial arrangement} of atoms, rather than connectivity.
            \begin{itemize}
                \item \ddd{Geometric isomerism}: aka cis--trans; \emph{locked into spatial positions} due to double bonds or a ring structure.
                    \begin{itemize}
                        \item \ddd{Cis}: functional groups that are on the \emph{ same side} of the carbon chain.
                        \item \ddd{Trans}: functional groups on \emph{opposite sides} of the carbon chain.
                        \item \textit{Cis-trans} terminology is used to describe disubstituted alkenes (carbon chain with \(\pi \) bond), even when the two substituents are different from each other.
                            \begin{itemize}
                                \item Does not apply to disubstituted alkenes in which the substituents are connected in the same position.
                            \end{itemize}
                    \end{itemize}
            \end{itemize}
        \subsection{Chirality}
        \begin{itemize}
            \item \ddd{Superimposable (achiral)}: when an object's \emph{mirrored version} is identical to the actual object.
            \item \ddd{Chiral}: objects that are not superimposable.
                \begin{itemize}
                    \item The most common source of molecular chirality is the presence of a carbon bearing \emph{four different groups}. 
                \end{itemize}
            \item \ddd{Enantiomer}: the nonsuperimposable mirror image of a chiral compound.
                \begin{itemize}
                    \item Can be used in speech the same way \emph{twin} is used
                    \item The easiest way to draw enantiomers is to just change wedges and dashes, but there are multiples ways to mirror a molecule, so it can be more complex.
                \end{itemize}
            \item \ddd{Diastereomers}: non-identical stereoisomers (nonsuperimposable) that are \emph{not mirror images} of one another. 
                \begin{itemize}
                        \item  Enantiomers have the same physical properties, while diastereomers have different physical properties.
                        \item Differences between enantiomers and diastereomers are especially relevant when comparing compounds with \emph{more than one chiral center}.
                        \item Maximum number of stereoisomers: \emph{\(2^n\)}
                    \begin{itemize}
                        \item \(n\): number of chiral centers
                        \item \(\dfrac{2^n}{2}\): pairs of enantiomers.
                    \end{itemize}
                \end{itemize}
        \end{itemize}
    \end{itemize}

\section{Cahn-Ingold-Prelog System}
\begin{itemize}
    \item \ddd{Cahn-Ingold-Prelog system}: a system of nomenclature for Identifying each enantiomer individually.
        \begin{enumerate}
            \item Assign priorities to each of the four groups based on atomic number; the highest atomic number has the highest priority.
            \item Rotate the molecule so that the fourth priority group is on a dash (behind)
            \item Determine the configuration, i.e., sequence of 1--2--3 groups.
                \begin{itemize}
                    \item \ttt{Clockwise (R)} or \fff{counterclockwise (S)}.
                \end{itemize}
        \end{enumerate}
    \item If there is a tie between the atoms connected, then continue outward until a difference is found.
        \begin{itemize}
            \item Do not add the sum all atomic numbers attached to each atom, just the first in which the atoms differ.
            \item Any multiple bonded atom, (2 or 3) is treated as if connected to multiple atoms equal to number of bonds.
        \end{itemize}
    \item Switching any two groups on a chiral center will invert the configuration, e.g.,
        \begin{itemize}
            \item 
            \schemestart{}
            \fff{\chemfig{2-[:30](-[:-30]3)(<[:130]1)(<:[:50]4)}}
            \arrow{->}
            \ttt{\chemfig{2-[:30](-[:-30]3)(<[:130]4)(<:[:50]1)}}
            \schemestop{}
            \item Can be done twice without changing configuration, e.g.,
            
            \schemestart{}
            \ttt{\chemfig{
                2-[:30](<[:130]1)(<:[:50]3)-[:-30]4
                }}
            \arrow{->}
            \fff{\chemfig{
                2-[:30](<[:130]1)(<:[:50]4)-[:-30]3
                }}
            \arrow{->}
            \ttt{\chemfig{
                1-[:30](<[:130]2)(<:[:50]4)-[:-30]3
                }}
            \schemestop{}
        \end{itemize}
    \subsection{Configuration in IUPAC nomenclature}
    \begin{itemize}
        \item The configuration of the chiral center is indicated at the beginning of the name, italicized, and surrounded by parentheses.
        \item When multiple centers are present, then each must be preceded by a locant.
    \end{itemize}
\end{itemize}

\section{Optical Activity}
\begin{itemize}
    \item Enantiomers exhibit identical physical properties, but different behavior to plane-polarized light.
        \begin{itemize}
            \item \ddd{Polarization}: the orientation of electric field of the electromagnetic wave.
            \item \ddd{Plane-polarized light}: a filter that only allows light of a particular polarization through.
        \end{itemize}
    \item \ddd{Optically active}: property of compounds that rotate the plane-polarized light.
        \begin{itemize}
            \item Can be measured using a polarimeter.
            \item \emph{Chiral} compounds are \emph{optically active}, while \emph{achiral} compounds \emph{are not} optically active.
        \end{itemize}
    \item \ddd{Observed rotation}:  \emph{(\(\alpha \))}; the rotation of light due to chiral compounds, which depends on the number of molecules light encounters.
        \begin{itemize}
            \item Doubling the concentration or pathlength both double the observed rotation.
        \end{itemize}
    \item \ddd{Specific rotation}: \emph{[\(\alpha \)]}; a standard concentration (\SI{1}{g\per\mL}) and a standard pathlength (\SI{1}{dm}) that allows for meaningful comparison between compounds.
        \begin{itemize}
            \item \([\alpha]=\dfrac{\alpha}{cl} \)
            \item c: concentration, l: pathlength.
            \item Temperature (T) and wavelength (\(\lambda \)) both have a nonlinear relationship, so it is often noted as: \( [\alpha]^T_\lambda \)
        \end{itemize}
    \item Specific rotation for enantiomers are \emph{equal in magnitude} but \emph{opposite in direction}.
        \begin{itemize}
            \item \rrr{\textbf{Dextrorotaory}}: a compound exhibiting \rrr{positive} rotation.
            \item \bbb{\textbf{Levorotatory}}: a compound exhibiting \bbb{negative} rotation.
            \item No direct relationship between R/S system of nomenclature, as that is independent of conditions, but dependent on observation angle.
            \item The \emph{direction} of polarized light, however, is \emph{dependent on conditions}, and can change based on temperature or wavelength even with the same given configuration.
        \end{itemize}
    \subsection{Enantiomeric Excess}
    \begin{itemize}
        \item \ddd{Optically (enantiomerically) pure}: a solution containing a single enantiomer.
        \item \ddd{Racemic mixture}: a solution containing equal amounts of both enantiomers, resulting in an optically inactive appearance.
        \item \ddd{Enantiomeric excess (ee)}: when a solution containing both enantiomers in unequal amounts, appearing optically active.
            \begin{itemize}
                \item \( \%~ee = \dfrac{|\text{observed}~(\alpha)|}{|\text{specific}~[\alpha]|} \times 100\% \)
            \end{itemize}
    \end{itemize}
\end{itemize}

\section{Symmetry and Chirality}
\begin{itemize}
    \item Any compound with a single chiral center must be chiral, however, the same it not always true for two or more.
    \item \ddd{Reflectional symmetry}: when an object has a \emph{plane of symmetry} that can be reflected across and still generate the same image.
        \begin{itemize}
            \item Any compound that posses a \emph{plane of symmetry} in any conformation will be \emph{achiral}.
        \end{itemize}
    \item \ddd{Rotational symmetry}: when an object has an \emph{axis of symmetry}, i.e., it can be rotated around a single point and appear 2 or more times.
        \begin{itemize}
            \item Order: how many times the object appears.
            \item Chirality is \emph{not dependent} on rotational symmetry.
        \end{itemize}
    \item \ddd{Point symmetry (inversion)}: when every part of an object has a matching part, i.e., \emph{equal distance} from the \emph{central point}, but in \emph{opposite direction}.
        \begin{itemize}
            \item Absence of a plane of symmetry does not mean it is chiral.
            \item If a compound exhibits \emph{inversion}, then it is \emph{achiral}.
        \end{itemize}
    \item Summary of relationship between symmetry and chirality:
        \begin{itemize}
            \item The presence of absence of rotational symmetry is irrelevant to chirality.
            \item A compound that has a plane of symmetry will be achiral.
            \item A compound that lacks a plane of symmetry will most likely be chiral, with some exceptions.
        \end{itemize}
    \item \ddd{Meso compounds}: achiral compounds (posses plane of symmetry or can be inverted) that have multiple chiral centers.
        \begin{itemize}
            \item A family of stereoisomers containing a meso compound will have less than \(2^n\) stereoisomers.
        \end{itemize}
    \subsection{Fischer Projections}
    \begin{itemize}
        \item \ddd{Fischer projections}: two-dimensional representations of organic molecules by projection.
            \begin{itemize}
                \item Limited to carbohydrates/sugars with multiple chiral centers.
            \end{itemize}
        \item \emph{Horizontal} lines are considered to be coming \emph{out} of the page.
        \item \emph{Vertical} lines are considered to be going \emph{behind} the page.
        \item Helpful for quickly comparing relationship between stereoisomers with multiple chiral centers;
            \begin{itemize}
                \item Enantiomers will have opposite configurations, while diastereomers will not.
                \item Assuming the north is priority 2, and the south is priority 3:
                    \begin{itemize}
                        \item If the atom with priority 1 is on the \ttt{right}, then it will have an \ttt{R} configuration.
                        \item If the atom with priority 1 is on the \fff{left}, then it will have an \fff{S} configuration.
                    \end{itemize}
            \end{itemize}
    \end{itemize}
\end{itemize}


%%%%%%%%%%% Reference Material %%%%%%%%%%%
\end{document}