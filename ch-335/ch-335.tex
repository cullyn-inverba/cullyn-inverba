\documentclass{inVerba-notes}

\definecolor{title-color}{HTML}{5f89f5}
\newcommand{\theTitle}{\href{https://github.com/cullyn-inverba/notes/tree/master/ch-335} {Organic Chemistry II}}

\begin{document}
\hypertarget{ToC}{\tableofcontents}
% chktex-file 36

%%%%%%%%%%% Chapter 14: Alkenes %%%%%%%%%%%
%\begingroup
\chapter{Chapter 14: Alkenes}\label{Chapter 14: Alkenes}


\section{Nomenclature of Alkenes}\label{Nomenclature of Alkenes}

\medskip
\begin{center}
  \schemestart{}
    \chemname{\chemfig{H-!\hch-!\hch-H}}{Saturated\\alk\emph{anes} eth\emph{ane}}
    \qquad
    \chemname{\chemfig{!\hhc=!\chh}}{Unsaturateed\\alk\emph{enes} eth\emph{ene}}
    \qquad
    \chemname{\chemfig{!\cu!\cd}}{prop\emph{ane}}
    \qquad
    \chemname{\chemfig{!\cu=[:-30]}}{prop\emph{ene}}
  \schemestop{}
\end{center}
\bigskip

\begin{itemize}
  \item Generally prepared through beta elimination, which results in the formation of alkenes (series of unsaturated hydrocarbons contain that a \(\pi \) bond).
  \item Alkenes are named using the same four steps in the previously used nomenclature, though the suffix of \minimal{``ane''} is replaced with \emph{``ene.''}
  \item When choosing the parent chain, choose the parent chain that \emph{includes} the \(\pi \) bond.
  \item When numbering the parent chain, the \(\pi \) bond should receive the \emph{lowest} number possible.
  \item The locant of the \(\pi \) bond should be place right before the suffix of ``ene,'' though, it was previously recommended before the parent (both are acceptable).
  \item Commonly recognized alternative names:

    \medskip
    \begin{center}
    \schemestart{}
      \chemname{\chemfig{=[:30]}}{Ethylene}
      \qquad
      \chemname{\chemfig{=[:30]-[:-30]}}{Propylene}
      \qquad
      \chemname{{\tiny\chemfig{*6(-=-(-=[:-30])=-=)}}}{Styrene}
    \schemestop{}
    \end{center}
    \bigskip
    
  \item \textbf{Degree of substitution}: not a substitution reaction, but the \emph{number of groups} connected to the double bond.
    \begin{itemize}
      \item 
        \chemname{\chemfig{R-[:45]=}}{Monosubstituted}
        \qquad
        \chemname{\chemfig{R-[:45]=-[:45]R}}{Disubstituted}
        \qquad
        \chemname{\chemfig{R-[:45]=(-[:-45]R)-[:45]R}}{Trisubstituted}
        \qquad
        \chemname{\chemfig{R-[:45](-[:135]R)=(-[:-45]R)-[:45]R}}{Tetrasubstituted}
        \bigskip
    \end{itemize}

  \subsection{Practice and Review}
  \begin{itemize}
    \item \textbf{Electronegativity}: negative charges on atoms with lower hybridization result in greater stability due to proximity (overlap) to positive nucleus. More s character results in greater stability.
      \begin{itemize}
        \item I.e., \(\bbb{sp~(50\%~s)} > sp^2~(33\%~s) > \rrr{sp^3~(25\%~s)}\)
        \item E.g., ethene has two carbons that are both \(sp^2\) due to one unhybridized p-orbital. This gives ethene a trigonal planar geometry.
      \end{itemize}

      \bigskip
      \begin{center}
      \hspace{-20pt}
      \schemestart{}
        \chemfig{!\hhc=!\chh}
        \arrow{<->}
        \quad
        \chemfig{
          \orbital{s}
            -[:-30,.85]
              {\orbital[scale=1.8]{p}}
              {\orbital[angle=0,scale=1.6,half,color=dark]{p}}
              {\orbital[angle=150,half]{p}}
              {\orbital[angle=-150,half]{p}}
              (-[:-150,.85]\orbital{s})
            =[,.75]
            =[,.75]
            {\orbital[angle=180,scale=1.6, half, color=dark]{p}}
            {\orbital[scale=1.8]{p}}
            {\orbital[angle=30,half]{p}}
            {\orbital[angle=-30,half]{p}}
            (-[::30,.85]{\orbital{s}})(-[::-30,.85]{\orbital{s}})
        }
        \arrow{0}
        \hspace{-20pt}
        \orbital[scale=1.6]{p}
        \quad (unhybridized p-orbital)
      \schemestop{}
      \end{center}
      \bigskip
      \bigskip

    \item \textbf{Hydrogen deficiency index (HDI)}: the measure of degrees of unsaturation. 
      \begin{itemize}
        \item E.g., two degrees of unsaturation results in a HDI of 2.
        \item Degrees of freedom help represent possible structures, indicating possible double bounds, triple bounds, rings, or various combinations of each.
        \item Only helpful when molecular formula is known for certainty.
        \item Formula: \emph{HDI = \(\frac{1}{2}(2C + 2 + N - H - X)\)}
        \begin{itemize}
            \item \(X\): halogen atoms.
        \end{itemize}
      \end{itemize}
    \item What is the HDI for the following molecules?
    
    \begin{multicols}{2}  
    \begin{itemize}
        \item[i] \chemfig{!\cu=[:-30]!\cu=[:-30]!\cu=[:-30]}
        \item[ii]~~{\footnotesize\chemfig{*6(-=-=-=)}}
      \end{itemize}
    \end{multicols}
  \end{itemize}

  \subsection{Types of Alkenes}\label{Types of Alkenes}
  \begin{itemize}
    \item Basic types of alkenes:
    
    \schemestart{}
      \chemname{\chemfig{=[:30,,,,emph]!\cd!\cu}}{Terminal Alkene}
      \qquad
      \chemname{\chemfig{!\cu=[:-30,,,,emph]!\cu}}{Internal Alkene}
      \qquad
      \chemname{{\footnotesize\chemfig{*6(---=[,,,,emph]--)}}}{Cyloalkene}
    \schemestop{}
    \bigskip

    \item Types of terminal alkenes:  
    
    \schemestart{}
      \chemname{\chemfig{R=(-[:45]H)(-[:-45]H)}}{Methylene}
      \qquad
      \chemname{\chemfig{R-=[:-45]}}{Vinyl}
      \qquad
      \chemname{\chemfig{R--[:-45]=}}{Allyl}
    \schemestop{}
    \bigskip

    \begin{itemize}
      \item ``R'' always tells you it's a carbon containing functional group, or hydrogen. 
      \item ``A'' can be used to represent any functional group. 
    \end{itemize}
  \end{itemize}
\end{itemize}
%\endgroup
%%%%%%%%%%% Chapter 14: Alkenes %%%%%%%%%%%
\end{document}