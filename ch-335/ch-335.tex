\documentclass{inVerba-notes}

\definecolor{title-color}{HTML}{5f89f5}
\newcommand{\theTitle}{\href{https://github.com/cullyn-inverba/notes/tree/master/ch-335} {Organic Chemistry II}}

\begin{document}
\hypertarget{ToC}{\tableofcontents}
% chktex-file 36

%%%%%%%%%%% Chapter 14: Alkenes %%%%%%%%%%%
%\begingroup
\chapter{Chapter 14: Alkenes}\label{Chapter 14: Alkenes}

\section{Alkenes}\label{Alkenes}
\medskip
  \begin{center}
    \schemestart{}
    \chemname{\chemfig{H-!\hch-!\hch-H}}{Saturated\\alk\emph{anes} eth\emph{ane}}
    \qquad
    \chemname{\chemfig{!\hhc=!\chh}}{Unsaturateed\\alk\emph{enes} eth\emph{ene}}
    \qquad
    \chemname{\chemfig{!\cu!\cd}}{prop\emph{ane}}
    \qquad
    \chemname{\chemfig{!\cu=[:-30]}}{prop\emph{ene}}
    \schemestop{}
  \end{center}
\bigskip

\begin{itemize}
  \item \textbf{Electronegativity}: negative charges on atoms with lower hybridization result in greater stability due to proximity (overlap) to positive nucleus. More s character results in greater stability.
  \begin{itemize}
    \item I.e., \(\bbb{sp~(50\%~s)} > sp^2~(33\%~s) > \rrr{sp^3~(25\%~s)}\)
    \item E.g., ethene has two carbons that are both \(sp^2\) due to one unhybridized p-orbital. This gives ethene a trigonal planar geometry.
    \end{itemize}

        \bigskip
        \schemestart{}
        \chemfig{!\hhc=!\chh}
        \arrow{<->}
        \quad
        \chemfig{
          \orbital{s}
            -[:-30,.85]
              {\orbital[scale=1.8]{p}}
              {\orbital[angle=0,scale=1.6,half,color=dark]{p}}
              {\orbital[angle=150,half]{p}}
              {\orbital[angle=-150,half]{p}}
              (-[:-150,.85]\orbital{s})
            =[,.75]
            =[,.75]
            {\orbital[angle=180,scale=1.6, half, color=dark]{p}}
            {\orbital[scale=1.8]{p}}
            {\orbital[angle=30,half]{p}}
            {\orbital[angle=-30,half]{p}}
            (-[::30,.85]{\orbital{s}})(-[::-30,.85]{\orbital{s}})
        }
        \arrow{0}
        \hspace{-8pt}
        \orbital[scale=1.6]{p}
        \quad (unhybridized p-orbital)
        \schemestop{}
        \medskip

  \subsection{Practice}
  \begin{itemize}
    \item \textbf{Hydrogen deficiency index (HDI)}: the measure of degrees of unsaturation. 
      \begin{itemize}
        \item E.g., two degrees of unsaturation results in a HDI of 2.
        \item Degrees of freedom help represent possible structures, indicating possible double bounds, triple bounds, rings, or various combinations of each.
        \item Only helpful when molecular formula is known for certainty.
        \item Formula: \emph{HDI = \(\frac{1}{2}(2C + 2 + N - H - X)\)}
        \begin{itemize}
            \item \(X\): halogen atoms.
        \end{itemize}
      \end{itemize}
    \item 
  \end{itemize}

\end{itemize}
%\endgroup
%%%%%%%%%%% Chapter 14: Alkenes %%%%%%%%%%%
\end{document}