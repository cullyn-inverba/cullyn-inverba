\documentclass[quiz]{inVerba-notes}

\definecolor{title-color}{HTML}{5f89f5}
\newcommand{\theTitle}{\href{https://github.com/cullyn-inverba/notes/tree/master/ch-335}{Mini Quizzes}}

\begin{document}
\hypertarget{ToC}{\tableofcontents}

\clearpage
\section{Week 1 --- Chapter 14}\label{Week 1}
\begin{enumerate}
    \item Name the structure: 
    
    \medskip
    \schemestart{}
        \chemfig{*7((-Cl)--=----)}
    \schemestop{}
    \bigskip
    
    \begin{itemize}
      \item 1-chloro-3-cycloheptene
      \item 4-chloro-1-cycloheptene
      \item 4-chloro-1-cyclohexene
      \item 6-chloro-1-cycloheptene
      \basec{\begin{itemize}
        \item When numbering the parent chain, the double bond should receive the lowest number possible; \emph{k=1}
          \begin{itemize}
            \item Note: define the location \(k\) of the double bond as being the number of its first carbon, not at the end.
          \end{itemize}
        \item The locant (\(k\)) of the double bond should be placed right before the suffix of ``ene,'' though, it was previously recommended before the parent (both are acceptable), e.g., 2-pentene = pent-2-ene; \emph{1-cycloheptene}
        \item Name and the side groups (other than hydrogen) according to the appropriate rules; \emph{chloro}
        \item Define the position of each side group as the number of the chain carbon it is attached to; \emph{4-}
      \end{itemize}}
    \end{itemize}

    \item Name the structure: 
    
    \medskip
    \schemestart{}
      \chemfig{ClCH_2CH_2-[:-30](-[:210]H)=(-[:-30]H)
      (-[:30]C(-[:150]H)=(-[:-30]H)-[:30]CH_3)
      }
    \schemestop{}
    \bigskip

    \begin{itemize}
      \item (2E,4E)-7-chloro-2,4-heptadiene
      \item (2Z,4Z)-7-chloro-2,4-heptadiene
      \item (2Z,4E)-7-chloro-2,4-heptadiene
      \item (2E,4Z)-7-chloro-2,4-heptadiene
      \basec{
        \begin{itemize}
          \item \ddd{E-Z notation}: recommended instead of \textit{cis} and \textit{trans} in order to account for cases that has more than two different groups attached to the double bond by first determining the priority using the \link{https://en.wikipedia.org/wiki/Cahn\%E2\%80\%93Ingold\%E2\%80\%93Prelog_priority_rules}{Cahn-Ingold-Prelog System}.
          \begin{itemize}
            \item \fff{E, entgegen, ``opposite''}.
            \item \ttt{Z, zusammen, ``together''}; ``on ze zame zide.''
        \end{itemize}
        \item When numbering the parent chain, the double bond should receive the lowest number possible; \emph{k=2}
          \begin{itemize}
            \item The two highest priority groups are on \fff{opposite} sides; \emph{2E}
          \end{itemize}
        \item There is more than one double bond; \emph{\(k_2=4\)}
          \begin{itemize}
            \item The two highest priority groups are on \ttt{zame} side; \emph{4Z}
          \end{itemize}
      \end{itemize}
      }
    \end{itemize}
    
    \item How many stereoisomeric product(s) do you get in the reaction below.
    
    \medskip
    \schemestart{}
      \chemfig{=[:30]-[:-30](-[:30]-[:-30])-[:-90]-[:-30]}
      \arrow{->[\ch{Hg(OAc)2}, \ch{H2O}, THF][\ch{NaBH4}]}[0,2.2]
    \schemestop{}
    \bigskip
    
    \basec{\begin{itemize}
      \item Oxymercuration-demercuration reactions follow Markovnikov's rule, i.e., \rrr{\(H^+\)} is added to the carbon with the \rrr{greatest} number of hydrogen atoms while the \bbb{\(X^-\) component} is added to the carbon with the \bbb{fewest} hydrogen atoms.
      \item Drawing the intermediate is not necessary, and no chiral centers are found in the products:
      
      \medskip
      \schemestart{}
        \dots 
        \arrow(--[braces]){->[\rrr{\ch{Hg(OAc)2}}, \bbb{\ch{H2O}}, THF][\ch{NaBH4}]}[0,2.2]
        \chemfig{\rrr{H}-[:30](<[:90]\bbb{OH})-[:-30](-[:30]-[:-30])-[:-90]-[:-30]}
        \+
        \chemfig{\rrr{H}-[:30](<:[:90]\bbb{OH})-[:-30](-[:30]-[:-30])-[:-90]-[:-30]}
      \schemestop{}
      \bigskip
      
    \end{itemize}}
    
    \item Which reaction intermediate is formed when Br2/CCl4 reacts with cyclohexene?
    
    \begin{multicols}{2}
    \begin{enumerate}[label=\roman*]
    
    \item 

    \medskip
    \schemestart{}
      \chemfig{*6(--(-[,.3,,,,draw=none]\cat)-(-Br)---)}
    \schemestop{}
    \bigskip

    \item 

    \medskip
    \schemestart{}
      \chemfig{*6(--(-Br)-(-[,.3,,,,draw=none]\cat)---)}
    \schemestop{}
    \bigskip

    \item 

    \medskip
    \schemestart{}
      \chemfig{*6(--(-[,.3,,,,draw=none]\cdot)-(-Br)---)}
    \schemestop{}
    \bigskip

    \item 

    \medskip
    \schemestart{}
      \chemfig{*6(--(-[:30]Br\,\cat)-(-[:-30,.55])---)}
    \schemestop{}
    \bigskip

    \item 

    \medskip
    \schemestart{}
      \chemfig{*6(--(-Br)-(-Br)---)}
    \schemestop{}
    \bigskip

    \end{enumerate}
    \end{multicols}

    \basec{\begin{itemize}
      \item \ddd{Halogenation}: a reaction that involves the addition of one or more halogens to a compound or material.
    \begin{itemize}
      \item The addition of halogens to alkenes proceeds via \emph{intermediate halonium ions}.
      \item \ddd{Halonium ion}: any onium ion containing a halogen atom carrying a positive charge. This cation has the general structure: \chemfig{R-[,.7]\rrr{+X}-[,.7]R'}
      \item \ddd{Onium ion}: a \rrr{cation} formally obtained by the protonation of mononuclear parent hydride of a pnictogen (group 15 of the periodic table), chalcogen (group 16), or halogen (group 17); \rrr{\(Br^\cat\)} in our case.
    \end{itemize}
    \end{itemize}}
  \end{enumerate}
\end{document}