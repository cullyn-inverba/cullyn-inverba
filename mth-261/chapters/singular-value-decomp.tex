\chapter{Singular Value Decomposition}\label{Singular Value Decomposition}

\section{Singular Value Decomposition Fundamentals}\label{Singular Value Decomposition Fundamentals}
\begin{itemize}
  \item \dd{Singular value decomposition (SVD)}: an extension of \hyperref[Eigendecomposition]{\ulink{eigendecomposition}} to any \(\xxx{m} \times \yyy{n} \) matrix via polar decomposition.
    \begin{itemize}
      \item \dd{Polar decomposition}: the factorization of a matrix \tbm{A} in the form \(\bm{A}=\bm{U}{\bm
      {P}}\), where \tbm{U} is a unitary matrix and \(\bm{{P}}\) is a positive-semidefinite \hyperref[Diagonal and Trace]{\ulink{Hermitian matrix}}, both square and the same size.
        \begin{itemize}
          \item \dd{Unitary}: when a complex square matrix has a \hyperref[Conjugate Transpose]{\ulink{conjugate transpose}} \(\bm{A}^* \) that is also it's inverse, i.e.,
          \[%%%%%%%%%%%%%%%%%%%%%%%%%%%%%%%%%%%%%%%%%%%%%%%
          \bm{U}^*\bm{U} = \bm{U}\bm{U}^* = \bm{I}
          \]%%%%%%%%%%%%%%%%%%%%%%%%%%%%%%%%%%%%%%%%%%%%%%%
          \item The real analogue of a unitary matrix is an \hyperref[Orthogonal Matrices]{\ulink{orthogonal matrix}}.
        \end{itemize}
      \item The rest of the details regarding polar decomposition will not be covered, as it detracts from understanding of single value decomposition; for now we will only deal with real matrices.
      \item Other important notes before continuing:
      \begin{itemize}
        \item Recall how \hyperref[Hadamard Multiplication]{\ulink{creating symmetric matrices}} is done, i.e.,
        \begin{itemize}
          \item \(\bm{\bbb{A^T}A}=\bm{\yyy{S_{n\times n}}}\) \quad and \quad \(\bm{A\rrr{A^T}} = \xxx{\bm{S}_{m \times m}}\)
        \end{itemize}
        \item And that the \hyperref[Column Space]{\ulink{column and row space}} of matrices multiplied with their transpose are the same the original matrix column/row space, i.e., 
        \[%%%%%%%%%%%%%%%%%%%%%%%%%%%%%%%%%%%%%%%%%%%%%%%
        \yyy{C}(\bm{A\rrr{A^T}}) = \yyy{C}(\bm{\xxx{S}}) = \yyy{C}(\bm{A}), \quad \xxx{R}(\bm{\bbb{A^T} A}) = \xxx{R}(\yyy{\bm{S}}) = \xxx{R}(\bm{A})
        \]%%%%%%%%%%%%%%%%%%%%%%%%%%%%%%%%%%%%%%%%%%%%%%%
      \end{itemize}
      \item Finally, the singular value decomposition in more detail:
      \[%%%%%%%%%%%%%%%%%%%%%%%%%%%%%%%%%%%%%%%%%%%%%%%
      \underbrace{\bm{A}}_{\xxx{m} \times \yyy{n}} = \bbb{\underbrace{\U}_{\xxx{m} \times \xxx{m}}} \underbrace{\S}_{\xxx{m} \times \yyy{n}} \rrr{\underbrace{\Vt}_{\yyy{n} \times \yyy{n}}}
      \]%%%%%%%%%%%%%%%%%%%%%%%%%%%%%%%%%%%%%%%%%%%%%%%
      \begin{itemize}
        \item \(\U\)~--- the orthogonal basis for \yyy{column space} of \bm{A}.
          \begin{itemize}
            \item Termed the \bbb{\textbf{left singular vectors}} as \(\bbb{\bm{U}^T}\bm{A = \S \Vt}\)
            \item In the form of eigenvalue equation \(\bbb{\bm{u}^T}\bm{A}=\s \xxx{\bm{v}^T}\).
          \end{itemize}
        \item \(\S\)~--- a diagonal matrix consisting of the \ttt{\textbf{singular values} \(\bm{\sigma}\)} of \bm{A}. 
          \begin{itemize}
            \item The number of non-zero singular values is equal to the \hyperref[Matrix Rank]{\ulink{rank}} of \tbm{A}.
          \end{itemize}
        \item \(\Vt\)~--- the orthogonal basis for \xxx{row space} of \bm{A}.
          \begin{itemize}
            \item Termed the \rrr{\textbf{right singular vectors}} as \(\bm{A}\rrr{\bm{V}} = \U \S\)
            \item In the form of eigenvalue equation \(\bm{A}\rrr{\bm{v}}= \yyy{\bm{u}} \s\)
          \end{itemize}
      \end{itemize}
    \end{itemize}
  
  \newpage
  \subsection{Computing SVD and Singular Values vs. Eigenvalues}\label{Computing SVD and Singular Values vs. Eigenvalues}
  \begin{itemize}
    \item Assuming \tbm{A} is an \(\xxx{m} \times \yyy{n} \) matrix, then multiplying by the transpose yields two results that are the similar to eigendecomposition and dependent on the order of multiplication:
    \begin{align*}
      \bm{A} &= \U \S \Vt &&& \bm{A} &= \U \S \Vt \\
      \bm{\bbb{A^T}A} &= (\U\S\Vt)^T \U\S\Vt &&& \bm{A\rrr{A^T}} &=  \U\S\Vt (\U\S\Vt)^T \\
      \bm{\yyy{S}} &= \Vt^T\S^T\yyy{\bm{U}^T} \U\S\Vt &&& \bm{\xxx{S}} &=  \U\S\Vt \Vt^T\S^T\yyy{\bm{U}^T} \\
      \bm{\yyy{S}} &= \bm{\xxx{V}}\S^T\bm{I}\S\Vt &&& \bm{\xxx{S}} &=  \U\S\bm{I}\S^T\yyy{\bm{U}^T} \\
      \bm{\yyy{S}} &= \bm{\xxx{V}}\S^2\Vt &&& \bm{\xxx{S}} &=  \U\S^2\yyy{\bm{U}^T} \\
      &\downarrow &&& &\downarrow \\
      \bm{\yyy{S}} &= \str{\bm{Q}}\eigL \str{\bm{Q}^T} &&& \bm{\xxx{S}} &= \str{\bm{Q}}\eigL \str{\bm{Q}^T} \\
      \xxx{R}(\bm{\yyy{S}}) &= \str{\bm{Q}}\eigL \str{\bm{Q}^T} &&& \yyy{C}(\bm{\xxx{S}}) &= \str{\bm{Q}}\eigL \str{\bm{Q}^T} \\
      \xxx{R}(\bm{A}) &= \str{\bm{Q}}\eigL \str{\bm{Q}^T} &&& \yyy{C}(\bm{A}) &= \str{\bm{Q}}\eigL \str{\bm{Q}^T} \\
      &\downarrow &&& &\downarrow \\ 
      \xxx{R}(\bm{A}) &= \bm{\xxx{V}}\S^2\Vt &&& \yyy{C}(\bm{A}) &=  \U\S^2\yyy{\bm{U}^T} 
    \end{align*}
    \item Thus, in the case of a real square matrix, \(\S^2 = \eigL\) when \tbm{A} is transposed with itself and \(\U\)\,/\,\(\Vt\) act as the orthogonal basis for the \yyy{column}/\xxx{row} spaces, respectively.
    \item Reviewing the singular value equation:
    \begin{align*}
      \bm{A} = &~\U \S \Vt \\ 
      \swarrow &\hspace{14pt} \searrow \\
      \bm{A}\rrr{\bm{V}} = \U \S \hspace{20pt}&\hspace{35pt} \bbb{\bm{U^T}}\bm{A} = \S \Vt \\
      \bm{A}\yyy{\bm{v}}= \yyy{\bm{u}} \s \hspace{24pt}&\hspace{37pt} \xxx{\bm{u}^T}\bm{A}=\s \xxx{\bm{v}^T} 
    \end{align*}
    \item 
  \end{itemize}

  \subsection{Relation to Matrix Subspaces}\label{SVD Relation to Subspaces}
  \begin{itemize}
    \item 
  \end{itemize}

  \subsection{Spectral Theory}\label{Spectral Theory}
  \begin{itemize}
    \item 
  \end{itemize}
  
\end{itemize}


\section{Applications of SVD}\label{Applications of SVD}
\begin{itemize}
  \item []
  
  \subsection{Low Rank Approximations}\label{Low Rank Approximations}
  \begin{itemize}
    \item 
  \end{itemize}

  \subsection{Percent Variance}\label{Percent Variance}
  \begin{itemize}
    \item 
  \end{itemize}
  
  \subsection{Pseudoinverse}\label{Pseudoinverse}
  \begin{itemize}
    \item 
  \end{itemize}

  \subsection{Condition Number}\label{Condition Number}
  \begin{itemize}
    \item 
  \end{itemize}
  
\end{itemize}



