\chapter{Vectors}\label{Vectors}
\section{Interpretations of Vectors}\label{Interpretations of Vectors}
\begin{itemize}
  \item \jjj{Algebraic vectors \jx{\left(\bm{v},~\vv{v}\right)}}: an ordered list of numbers.
    \begin{itemize}
      \item E.g., \(\bm{v}= \begin{bmatrix} 1 & 2 & 3 \end{bmatrix}\)
      \item Vectors can be written as rows (seen above) or columns (seen below), but  differ only at the level of notation and convention.
      \item The order of elements in a vector matters:
      \[%%%%%%%%%%%%%%%%%%%%%%%%%%%%%%%%%%%%%%%%%%%%%%%
      \begin{bmatrix}
      1 \\
      2 \\
      3 \
      \end{bmatrix}
      \neq
      \begin{bmatrix}
      2 \\
      1 \\
      3 \
      \end{bmatrix}
      \]%%%%%%%%%%%%%%%%%%%%%%%%%%%%%%%%%%%%%%%%%%%%%%%
    \end{itemize}
  \item \jjj{Dimensionality}: the number of elements in a vector.

  \item \jjj{Geometric vectors}: a line in geometric space that indicates the magnitude and direction from its start point (tail) to its end point (head).
    \begin{itemize}
      \item Geometric vectors can start at any point in space, but often represented as starting from the \emph{origin}---such vectors are in \emph{standard position}.
      \item Coordinates are not the same as vectors, but they do indicate where the head of a vector will land if it is in standard position. 
    \end{itemize}
  
  \subsection{Vector Addition and Subtraction}\label{Vector Addition and Subtraction}
  \begin{itemize}
      \item Algebraically, \emph{dimensionality} of vectors \emph{must be equal}. When they are, then addition or subtraction vectors is done on the corresponding elements of each vector, e.g.:
      \[%%%%%%%%%%%%%%%%%%%%%%%%%%%%%%%%%%%%%%%%%%%%%%%
      \begin{bmatrix}
      1 \\
      0 \\
      4 \\
      5 
      \end{bmatrix}
      +
      \begin{bmatrix}
      2 \\
      3 \\
      -6 \\
      11
      \end{bmatrix}
      =
      \begin{bmatrix}
      3 \\
      3 \\
      -2 \\
      16
      \end{bmatrix}
      \]%%%%%%%%%%%%%%%%%%%%%%%%%%%%%%%%%%%%%%%%%%%%%%%
      \item Geometrically, addition can be thought of translating the tail of one vector to the head of the other---resulting in a new vector. 
      \item Geometric interpretations of subtraction can be thought of in two ways:
        \begin{itemize}
          \item[1.] Multiplying one vector by -1, then applying vector addition method above.
          \item[2.] Placing both vectors in standard position, with the resulting vector between the two heads being the answer.
        \end{itemize}
  \end{itemize}
  
  \subsection{Vector-Scalar Multiplication}\label{Vector-Scalar Multiplication}
  \begin{itemize}
      \item \jjj{Scalar (\(\alpha,~\beta,~\lambda \))}: an element of a field (typically real numbers) used in scalar-multiplication of vectors. 
      \item Algebraically, scalar-multiplication is the multiplication of each element of a vector by a particular scalar.
      \[%%%%%%%%%%%%%%%%%%%%%%%%%%%%%%%%%%%%%%%%%%%%%%%
      \bbb{\lambda} \bm{v} \rightarrow \bbb{7} \begin{bmatrix}
      -1 \\
      0 \\
      1 \
      \end{bmatrix} = \begin{bmatrix}
      -7 \\
      0 \\
      7 \
      \end{bmatrix}
      \]%%%%%%%%%%%%%%%%%%%%%%%%%%%%%%%%%%%%%%%%%%%%%%%
      \item Geometrically, scalar-multiplication is the \emph{extension} (\(\lambda > 1\)) or \emph{compression} (\(\lambda\in(0,1)\)) of a vector.
        \begin{itemize}
          \item When \(\lambda \) < 0, then it can be thought of inverting its direction with respect to the origin.  
        \end{itemize}
  \end{itemize}  
\end{itemize}

\section{The Dot Product}\label{The Dot Product}
\begin{itemize}
    \item \jjj{Dot product (scalar product)}: an algebraic operation that takes two \emph{equal-length} sequences of numbers (usually coordinate vectors), and returns a \emph{single number}.
      \begin{itemize}
        \item The result of a dot product is a scalar, so often it is represented as a such (a lower case Greek letter).
        \item It can also be represented as multiplication between two vectors (\(\bm{a\cdot b}\)). 
        \item Most commonly it is represented as \(\bm{a}^T\bm{b}\) --- transpose will be explained in more detail when dealing with matrix products.
        \item Algebraically: \(\sum_{i = 1}^{n} \bm{a}_i \bm{b}_i  \) --- where \(\Sigma \) denotes summation and \(n\) is the dimension of the vector space, e.g.:
        \[%%%%%%%%%%%%%%%%%%%%%%%%%%%%%%%%%%%%%%%%%%%%%%%
        \rrr{\begin{bmatrix} 1 & 3 & 5 \end{bmatrix}}\bbb{\begin{bmatrix}
        4 \\
        -2 \\
        1 \
        \end{bmatrix}} = (\rrr{1}\cdot\bbb{4}) + (\rrr{3}\cdot \bbb{-2}) + (\rrr{-5} \cdot \bbb{-1}) = 3
        \]%%%%%%%%%%%%%%%%%%%%%%%%%%%%%%%%%%%%%%%%%%%%%%%
      \end{itemize}

  \subsection{Properties of the Dot Product}\label{Properties of the Dot Product}
  \begin{itemize}
      \item \ttt{\cmark~Distributive}: if \textbf{a}, \textbf{b}, and \textbf{c} and real vectors, then \(\bm{a}^T(\bm{b}+\bm{c})=\bm{a}^T\bm{b}+\bm{a}^T\bm{c}\)
      \item \fff{\xmark~Associative}: \(\bm{a}^T(\bm{b}^T\bm{c})\fff~\fff{\neq}~(\bm{a}^T\bm{b})\bm{c}\) --- in general the associative property does not hold, as the dot product would most likely produce different scalars.
        \begin{itemize}
          \item Additionally, \bm{a} could have a different dimensionality than \bm{b} and \bm{c}. I.e., even if \bm{b} and \bm{c} had the same dimensionality (\(\bm{a}^T(\bm{b}^T\bm{c})\) would be valid scalar-vector multiplication) then \(\bm{a}^T \bm{b}\) would be invalid.
        \end{itemize}
  \end{itemize}
  
\end{itemize}



