\chapter{Vectors}\label{Vectors}
\section{Interpretations of Vectors}\label{Interpretations of Vectors}
\begin{itemize}
  \item \jjj{Algebraic vectors \jx{\left(\bm{v},~\vv{v}\right)}}: an ordered list of numbers.
    \begin{itemize}
      \item E.g., \(\bm{v}= \begin{bmatrix} 1 & 2 & 3 \end{bmatrix}\)
      \item Vectors can be written as rows (seen above) or columns (seen below), but  differ only at the level of notation and convention.
      \item The order of elements in a vector matters:
      \[%%%%%%%%%%%%%%%%%%%%%%%%%%%%%%%%%%%%%%%%%%%%%%%
      \begin{bmatrix}
      1 \\
      2 \\
      3 \
      \end{bmatrix}
      \neq
      \begin{bmatrix}
      2 \\
      1 \\
      3 \
      \end{bmatrix}
      \]%%%%%%%%%%%%%%%%%%%%%%%%%%%%%%%%%%%%%%%%%%%%%%%
    \end{itemize}
  \item \jjj{Dimensionality}: the number of elements in a vector.

  \item \jjj{Geometric vectors}: a line in geometric space that indicates the magnitude and direction from its start point (tail) to its end point (head).
    \begin{itemize}
      \item Geometric vectors can start at any point in space, but often represented as starting from the \emph{origin}---such vectors are in \emph{standard position}.
    \end{itemize}

\end{itemize}


