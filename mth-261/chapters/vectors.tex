\chapter{Vectors}\label{Vectors}
\section{Interpretations of Vectors}\label{Interpretations of Vectors}
\begin{itemize}
  \item \jjj{Algebraic vectors \jx{\left(\bm{v},~\vv{v}\right)}}: an ordered list of numbers.
  \begin{itemize}
    \item E.g., \(\bm{v}= \begin{bmatrix} 1 & 2 & 3 \end{bmatrix}\)
    \item Vectors can be written as rows (seen above) or columns (seen below), but  differ only at the level of notation and convention.
    \item The order of elements in a vector matters:
    \[%%%%%%%%%%%%%%%%%%%%%%%%%%%%%%%%%%%%%%%%%%%%%%%
    \begin{bmatrix}
      1 \\
      2 \\
      3 \
    \end{bmatrix}
    \neq
    \begin{bmatrix}
      2 \\
      1 \\
      3 \
    \end{bmatrix}
    \]%%%%%%%%%%%%%%%%%%%%%%%%%%%%%%%%%%%%%%%%%%%%%%%
  \end{itemize}
  \item \jjj{Dimensionality}: the number of elements in a vector.
  
  \item \ddd{Euclidean (geometric, spatial) vectors}: a line in geometric space that indicates the \emph{magnitude} and \emph{direction} from a starting point (tail) to an end point (head).
  \begin{itemize}
    \item Geometric vectors can start at any point in space, but often represented as starting from the \emph{origin}---such vectors are in \emph{standard position}.
    \item Coordinates are not the same as vectors, but they do indicate where the head of a vector will land if it is in standard position. 
  \end{itemize}
  
  \subsection{Vector Addition and Subtraction}\label{Vector Addition and Subtraction}
  \begin{itemize}
    \item Algebraically, \emph{dimensionality} of vectors \emph{must be equal}. When they are, then addition or subtraction vectors is done on the corresponding elements of each vector, e.g.,
    \[%%%%%%%%%%%%%%%%%%%%%%%%%%%%%%%%%%%%%%%%%%%%%%%
    \begin{bmatrix}
      1 \\
      0 \\
      4 \\
      5 
    \end{bmatrix}
    +
    \begin{bmatrix}
      2 \\
      3 \\
      -6 \\
      11
    \end{bmatrix}
    =
    \begin{bmatrix}
      3 \\
      3 \\
      -2 \\
      16
    \end{bmatrix}
    \]%%%%%%%%%%%%%%%%%%%%%%%%%%%%%%%%%%%%%%%%%%%%%%%
    \item Geometrically, addition can be thought of translating the tail of one vector to the head of the other---resulting in a new vector. 
    \item Geometric interpretations of subtraction can be thought of in two ways:
    \begin{itemize}
      \item[1.] Multiplying one vector by -1, then applying vector addition method above.
      \item[2.] Placing both vectors in standard position, with the resulting vector between the two heads being the answer.
    \end{itemize}
  \end{itemize}
  
  \subsection{Vector-Scalar Multiplication}\label{Vector-Scalar Multiplication}
  \begin{itemize}
    \item \ddd{Scalar}: typically denoted with lower case Greek letters (e.g., \(\alpha,~\lambda \)) indicating an element of a field (typically real numbers) used in scalar-multiplication of vectors. 
    \item Algebraically, scalar-multiplication is the multiplication of each element of a vector by a particular scalar, e.g.,
    \[%%%%%%%%%%%%%%%%%%%%%%%%%%%%%%%%%%%%%%%%%%%%%%%
    \lambda \bm{v} \rightarrow 7 \begin{bmatrix}
      -1 \\
      0 \\
      1 \
    \end{bmatrix} = \begin{bmatrix}
      -7 \\
      0 \\
      7 \
    \end{bmatrix}
    \]%%%%%%%%%%%%%%%%%%%%%%%%%%%%%%%%%%%%%%%%%%%%%%%
    \item Geometrically, scalar-multiplication can be thought of as the \rrr{extension (\(\lambda > 1\))} or \xxx{compression (\(\lambda\in(0,1)\))} of a vector.
    \begin{itemize}
      \item When \bbb{\(\lambda \) < 0}, then it can be thought of inverting its direction with respect to the origin.  
    \end{itemize}
  \end{itemize}  
\end{itemize}
  

\section{The Dot Product}\label{The Dot Product}
\begin{itemize}
  \item \ddd{Dot (scalar, inner) product}: an algebraic operation that takes two \emph{equal-length} sequences of numbers (usually coordinate vectors), and returns a \emph{single number}.
  \begin{itemize}
    \item The result of a dot product is a \hyperref[Scalar]{\ulink{scalar}}, so often it is represented as such.
      \begin{itemize}
        \item It can also be represented as multiplication between two vectors (\(\bm{a\cdot b}\)). 
        \item However, it is commonly represented as \(\bm{a}^T\bm{b}\) --- \hyperref[tbd]{\dlink{transpose}} will be explained in more detail when dealing with \hyperref[tbd]{\dlink{matrix products}}.
      \end{itemize}
    \item Algebraically: \(\sum_{i = 1}^{n} \bm{a}_i \bm{b}_i  \) --- where \(\Sigma \) denotes summation and \(n\) is the dimension of the vector space, e.g.,
    \[%%%%%%%%%%%%%%%%%%%%%%%%%%%%%%%%%%%%%%%%%%%%%%%
    \rrr{\begin{bmatrix} 1 & 3 & 5 \end{bmatrix}}\bbb{\begin{bmatrix}
      4 \\
      -2 \\
      1 \
    \end{bmatrix}} = (\rrr{1}\cdot\bbb{4}) + (\rrr{3}\cdot \bbb{-2}) + (\rrr{-5} \cdot \bbb{-1}) = 3
    \]%%%%%%%%%%%%%%%%%%%%%%%%%%%%%%%%%%%%%%%%%%%%%%%
  \end{itemize}
  
  \subsection{Properties of the Dot Product}\label{Properties of the Dot Product}
  \begin{itemize}
    \item Note: the following properties hold as long as \(\bm{a}\), \(\bm{b}\), and \(\bm{c}\) are real vectors.
    \item \ttt{\cmark~Distributive}: \(\bm{a}^T(\bm{b}+\bm{c})=~\bm{a}^T\bm{b}+\bm{a}^T\bm{c}\) ---  vector multiplication distributes over vector addition. 
    \item \fff{\xmark~Associative}: \(\bm{a}^T(\bm{b}^T\bm{c})\neq(\bm{a}^T\bm{b})\bm{c}\) --- in general the associative property does not hold, as the dot product would most likely produce different scalars.
    \begin{itemize}
      \item Additionally, \bm{a} could have a different dimensionality than \bm{b} and \bm{c}. I.e., even if \bm{b} and \bm{c} had the same dimensionality (\(\bm{a}^T(\bm{b}^T\bm{c})\) would be valid vector-scalar multiplication) then \(\bm{a}^T \bm{b}\) would be invalid.
    \end{itemize}
    \item \ttt{\cmark~Commutative}: \(\bm{a}^T\bm{b} = \bm{b}^T\bm{a}\) --- the order of the vectors does not matter. 
  \end{itemize}

  \subsection{Vector Length}\label{Vector Length}
  \begin{itemize}
    \item \jjj{Vector norm (magnitude, length)}: denoted with double vertical bars \( \| \bm{v} \| \), indicating length of a vector in euclidean space. Not to be confused with absolute value \(|x|\) of a scalar's ``norm.'' However, sometimes the notation \(| \bm{v} |\) is used.
    \item Calculating \( \| \bm{v} \| \) is done using the Euclidean norm:
    \[%%%%%%%%%%%%%%%%%%%%%%%%%%%%%%%%%%%%%%%%%%%%%%%
    \| \bm{v} \| = \sqrt{v_1^2+v_2^2+v_3^2}
    \]%%%%%%%%%%%%%%%%%%%%%%%%%%%%%%%%%%%%%%%%%%%%%%%
    \begin{itemize}
      \item This is a consequence of the Pythagorean theorem, since the \hyperref[tbd]{\dlink{basis vectors}}  \\ \(\bm{e}_1\), \(\bm{e}_2\), and \(\bm{e}_3\) are \hyperref[Orthogonality]{\dlink{orthogonal}} \hyperref[tbd]{\dlink{unit vectors}}.
    \end{itemize}
    \item Thus, the \emph{norm} can easily be found by taking the square root of the dot product of the vector with itself:
    \[%%%%%%%%%%%%%%%%%%%%%%%%%%%%%%%%%%%%%%%%%%%%%%%
    \emph{\| \bm{v} \| = \sqrt{\bm{v}^T v}}
    \]%%%%%%%%%%%%%%%%%%%%%%%%%%%%%%%%%%%%%%%%%%%%%%%
  \end{itemize}

  \subsection{Geometric Interpretation of the Dot Product}\label{Geometric Interpretation of the Dot Product}
  \begin{itemize}
    \item The dot product of two \hyperref[Euclidean (geometric, spatial) vectors]{\ulink{Euclidean vectors}} \tbm{a} and \tbm{b} is defined by:
    \[%%%%%%%%%%%%%%%%%%%%%%%%%%%%%%%%%%%%%%%%%%%%%%%
    \emph{\lambda = \bm{a}^T \bm{b} = \| \bm{a} \| ~\| \bm{b} \| \cos(\theta)}
    \]%%%%%%%%%%%%%%%%%%%%%%%%%%%%%%%%%%%%%%%%%%%%%%%
    where \(\theta \) is the angle between \tbm{a} and \tbm{b}.
    \item \jjj{Features based on \tbm{\theta}}:
    \begin{itemize}
      \item When \(\rrr{\cos{\theta} > 0}\) \(\left(\theta < \ang{90}~\text{--- \rrr{acute}}\right)\) then \(\rrr{\lambda > 0~(+)}\)
      \item When \(\bbb{\cos{\theta} < 0}\) \(\left(\theta > \ang{90}~\text{--- \bbb{obtuse}}\right)\) then \(\bbb{\lambda < 0~(-)}\)
      \item When \(\minor{\cos{\theta} = 0}\)  (\(\theta = \ang{90}~\text{--- \minor{perpendicular}}\)) then \(\minor{\lambda = 0}\)
      \begin{itemize}
        \item This represents a special case where the vectors are said to be \emph{orthogonal}. 
        \item \ddd{Orthogonality}: the generalization of the notion of \emph{perpendicularity} to the linear algebra of \hyperref[tbd]{\dlink{bilinear forms}}.
      \end{itemize}
      \item When \(\emph{\cos{\theta} = 1}\) then the vectors are \emph{codirectional}:
      \[\bm{a}^T\bm{b} = \| \bm{a} \| ~\| \bm{b} \| \]\vspace{-25pt}
      \begin{itemize}
        \item Thus, the dot product with a vector \tbm{v} with itself is
        \[\bm{v}^T\bm{v} = \| \bm{v} \|^2\]\vspace{-25pt}
        \item Which gives us the \hyperref[Vector Length]{\ulink{norm}} as defined above, i.e., \( \| \bm{v} \| = \sqrt{\bm{v}^T v}\)
        \item If \(\cos{\theta} = -1\), then really vectors are still codirectional, but point in opposite directions with respect to the origin. 
      \end{itemize}
    \end{itemize}
  \end{itemize}
\end{itemize}

\newpage
\section{Other Vector Products}\label{Other Vector Products}
\begin{itemize}
  \item \hyperref[tbd]{\dlink{Matrices}} are properly defined later. However, vectors are technically single row or column matrices, so many of the following operations also work on vectors, thus use of matrices often appears in this section.
  
  \subsection{Hadamard Multiplication}\label{Vector Hadamard Multiplication}
  \begin{itemize}
    \item \jjj{Hadamard (element-wise) product}: a binary operation (only takes two operands) that matrices of the same dimensions and produces another matrix of the same dimension as the operands, e.g., vector Hadamard multiplication:
    \[%%%%%%%%%%%%%%%%%%%%%%%%%%%%%%%%%%%%%%%%%%%%%%%
    \begin{bmatrix}
      1 \\
      0 \\
      4 \\
      5 
    \end{bmatrix}
    +
    \begin{bmatrix}
      2 \\
      3 \\
      -6 \\
      11
    \end{bmatrix}
    =
    \begin{bmatrix}
      2 \\
      0 \\
      -24 \\
      55
    \end{bmatrix}
    \]%%%%%%%%%%%%%%%%%%%%%%%%%%%%%%%%%%%%%%%%%%%%%%%
  \end{itemize}

  \subsection{Outer Product}\label{Outer Product}
  \begin{itemize}
    \item Recall that the \hyperref[Dot (scalar, inner) product]{\ulink{dot (scalar, inner) product}} produces a \(1\times1 \) matrix, or rather a scalar, hence ``scalar'' product.
      \begin{itemize}
        \item Note: the typical notation used for dot products is \(\bm{v}^T \bm{w}\) --- part of reasoning for the ``inner'' product. 
      \end{itemize}
    \item \jjj{Outer product}: an \(N\times M\) matrix that results from the product of two vectors with dimensions \(n\) and \(m\).
    \[%%%%%%%%%%%%%%%%%%%%%%%%%%%%%%%%%%%%%%%%%%%%%%%
    \bm{v}\bm{w}^T=N\times M
    \]%%%%%%%%%%%%%%%%%%%%%%%%%%%%%%%%%%%%%%%%%%%%%%%
    \item The subtle change in notation matters in contrast to the dot product, as both represent distinct operations (assuming they are column vectors).
    \item The outer product allows for the multiplication of vectors with different dimensionality
    \item Can be thought of in two different ways:
    \begin{itemize}
      \item The \xxx{row} perspective:
      \[%%%%%%%%%%%%%%%%%%%%%%%%%%%%%%%%%%%%%%%%%%%%%%%
      \begin{bmatrix}
        1 \\
        0 \\
        4 \\
        2
      \end{bmatrix} \begin{bmatrix} \xxx{a} & \xxx{b} & \xxx{c} \end{bmatrix}
      =
      \begin{bmatrix}
        1\xxx{a} & 1\xxx{b} & 1\xxx{c} \\
        0\xxx{a} & 0\xxx{b} & 0\xxx{c} \\
        4\xxx{a} & 4\xxx{b} & 4\xxx{c} \\
        2\xxx{a} & 2\xxx{b} & 2\xxx{c}  
      \end{bmatrix}
      \]%%%%%%%%%%%%%%%%%%%%%%%%%%%%%%%%%%%%%%%%%%%%%%%
      \item The \yyy{column} perspective:
      \[%%%%%%%%%%%%%%%%%%%%%%%%%%%%%%%%%%%%%%%%%%%%%%%
      \begin{bmatrix}
        \yyy{1} \\
        \yyy{0} \\
        \yyy{4} \\
        \yyy{2}
      \end{bmatrix} \begin{bmatrix} a & b & c \end{bmatrix}
      =
      \begin{bmatrix}
        \yyy{1}a & \yyy{1}b & \yyy{1}c \\
        \yyy{0}a & \yyy{0}b & \yyy{0}c \\
        \yyy{4}a & \yyy{4}b & \yyy{4}c \\
        \yyy{2}a & \yyy{2}b & \yyy{2}c  
      \end{bmatrix}
      \]%%%%%%%%%%%%%%%%%%%%%%%%%%%%%%%%%%%%%%%%%%%%%%%
    \end{itemize}
  \end{itemize}
  
  \subsection{Cross Product}\label{Cross Product}
  \begin{itemize}
    \item \jjj{Cross (vector, directed area) product}: denoted by the symbol \jjj{\(\bm{\times} \)}, indicating a binary operation on two vectors in \emph{three-dimensional space \(\mathbb{R}^3 \)}.
      \begin{itemize}
        \item Given two linearly independent vectors \tbm{a} and \tbm{b}, then the cross product \(\bm{a}\times\bm{b}\) produces a new vector that is \emph{orthogonal} to both \tbm{a} and \tbm{b}, or \emph{normal} to the plane containing them.
        \item The \emph{direction} of the vector is given by the \emph{right-hand rule} (\tbm{a} = pointer, \tbm{b} = index, thumb = direction). 
        \item The \emph{magnitude} of the vector represents the \emph{area of the parallelogram} that the vectors span.
      \end{itemize}
    \item The cross product can be defined by the formula: 
    \[%%%%%%%%%%%%%%%%%%%%%%%%%%%%%%%%%%%%%%%%%%%%%%%
    \bm{a}\times \bm{b} = \| \bm{a} \|~\| \bm{b} \|\sin(\theta)\,\bm{v}
    \]%%%%%%%%%%%%%%%%%%%%%%%%%%%%%%%%%%%%%%%%%%%%%%%
    \begin{itemize}
      \item Notice this is similar to the \hyperref[Geometric Interpretation of the Dot Product]{\ulink{geometric interpretations of the dot product}} --- the contrast between the two leads to an intuitive interpretation:
      \begin{itemize}
        \item \(\cos(\theta)\) in the \emph{dot product} is used to measure how ``parallel'' the two vectors are, i.e., they are \emph{codirectional} when \(\theta = 1\), allowing for calculation of the \hyperref[Vector Length]{\ulink{norm}}. 
        \item \(\sin(\theta)\) in the \emph{cross product} is used to measure how ``perpendicular'' two vectors are, i.e., they are \emph{orthogonal} when \(\theta = 1\). There are multiple directions of the orthogonal vector, so calculation of the signed area returns vector \tbm{v} that describes both magnitude and direction as described above.
        \item The intuition described here will be more clear when the \hyperref[tbd]{\dlink{determinant}} is discussed in more detail.
      \end{itemize}
    \end{itemize}
    \item An algebraic example of vector e.g.,
    \[%%%%%%%%%%%%%%%%%%%%%%%%%%%%%%%%%%%%%%%%%%%%%%%
    \begin{bmatrix}
    1 \\
    2 \\
    3 
    \end{bmatrix}
    \times
    \begin{bmatrix}
    a \\
    b \\
    c 
    \end{bmatrix}
    =
    \begin{bmatrix}
    2c & - & 3b \\
    3a & - & 1c \\
    1b & - & 2a 
    \end{bmatrix}
    \]%%%%%%%%%%%%%%%%%%%%%%%%%%%%%%%%%%%%%%%%%%%%%%%
  \end{itemize}
  
  \subsection{Vectors with Complex Numbers}\label{Vectors with Complex Numbers}
  \begin{itemize}
    \item \ddd{Complex number}: a number that can be expressed in the form \(\xxx{a} + \yyy{bi}\), where \(a\) and \(b\) are real numbers, and \(i~(\: j\,)\) is a symbol called the imaginary unit that satisfies the equation \yyy{\(i = \sqrt{-1}\)}.
    \item This imaginary unit allows for the \yyy{imaginary set \((\mathbb{C})\)} to combined with the \\ \xxx{real number set \((\mathbb{R})\)} to create a 2-D plane that allows for more efficient calculations. 
    \item He
  \end{itemize}
  
\end{itemize}

  
\newpage

\section{To be defined}\label{tbd}
\begin{itemize}
  \item unit vectors
  \item basis vectors
  \item transpose
  \item matrix products
  \item determinant
\end{itemize}





