\chapter{Systems of Linear Equations}\label{Systems of Linear Equations}

\section{Linear Equations}\label{Linear Equations}
\begin{itemize}
  \item \jjj{Linear equation}: an equation that may be put in the form
  \[%%%%%%%%%%%%%%%%%%%%%%%%%%%%%%%%%%%%%%%%%%%%%%%
   \emph{a_{1}x_{1}+a_{2}x_{2}+\cdots +a_{n}x_{n}+b=0}
  \]%%%%%%%%%%%%%%%%%%%%%%%%%%%%%%%%%%%%%%%%%%%%%%%
  \begin{itemize}
    \item \jjj{Coefficients (parameters) \(a_1,a_2,\dots,a_n\)}: typically real numbers; sometimes they can be arbitrary expressions as long as they don't contain any new variables, though often \(\{x_{n+1},x_{n+2},x_{n+3},\dots\}\) are replaced with \(\{s,r,t,\dots\}\) for convenience. % chktex 21
    \item \jjj{Variables \(x_1,x_2,\dots,x_n\)}: the unknowns of the equation.
    \item \jjj{Constant term \(b\)}: typically a real number; technically also a coefficient, but not attached to any variables.
  \end{itemize}
  \item \jjj{System of linear equations}: a \emph{finite collection} of linear equations involving the \emph{same~set of variables}, e.g.,
  \[%%%%%%%%%%%%%%%%%%%%%%%%%%%%%%%%%%%%%%%%%%%%%%%
  \begin{cases}
    3x+2y-z &= 1 \\
    2x+2y+4z &= -2 \\
    -x+\frac{1}{2}y-z &= 0 \\
  \end{cases}\vspace{5pt}
  \]%%%%%%%%%%%%%%%%%%%%%%%%%%%%%%%%%%%%%%%%%%%%%%%
    \begin{itemize}
      \item Note: some variables can have a coefficient of 0, so some equations may appear to not contain the same set at first glance.
      \item Together, all individual equations are used to indicate one object, hence ``system.''
    \end{itemize}
  \subsection{Solutions}\label{Solutions}
  \begin{itemize}
    \item \jjj{Solution}: an assignment of values to variables such that all assignments yield a valid equation, i.e.,
    \[%%%%%%%%%%%%%%%%%%%%%%%%%%%%%%%%%%%%%%%%%%%%%%%
    a_1x_1+a_2x_2+\cdots+a_n x_n=b
    \]%%%%%%%%%%%%%%%%%%%%%%%%%%%%%%%%%%%%%%%%%%%%%%%
    \item \jjj{Liner system solution}: an assignment of values to the variables such that \emph{all~equations are simultaneously satisfied}, e.g., using the above system of equations:
    \begin{align*}
      x &= 1  & 3(1) + 2(-2) - (-2) &= 1 \\
      y &= -2 & 2(1) + 2(-2) + 4(-2)&= -2\\
      z &= -2 & -(1) + \tfrac{1}{2}(-2) - (-2) &= 0
    \end{align*}\vspace{-24pt}
    \item \jjj{Solution set}: the set of all possible solutions that satisfy all variables in a system of linear equations; there are three possibilities for such system: 
      \begin{enumerate}
        \item The system has \emph{infinitely many solutions} (the lines are the same).
        \item The system has a \emph{single unique solution} (the lines intersect at single point).
        \item The system has \emph{no solution} (the lines are parallel).
      \end{enumerate}
    \item \ddd{Inconsistent}: a system that has no solution.
    \item \ddd{Consistent}: a system that has at least one solution.
  \end{itemize}

  \subsection{Elementary Operations}\label{Elementary Operations}
  \begin{itemize}
    \item Solve system of equations is easily done if such systems are first converted to a form of matrix multiplication, i.e., taking this general form of a system of equations:
    \[%%%%%%%%%%%%%%%%%%%%%%%%%%%%%%%%%%%%%%%%%%%%%%%
    \begin{cases}
      a_{1,1}x_{1}+a_{1,2}x_{2}+\cdots +a_{1,n}x_{n}&=b_1 \\ 
      a_{2,1}x_{1}+a_{2,2}x_{2}+\cdots +a_{2,n}x_{n}&=b_2 \\ 
      &~\vdots \\ 
      a_{m,1}x_{1}+a_{m,2}x_{2}+\cdots +a_{m,n}x_{n}&=b_m \\ 
    \end{cases}
    \]%%%%%%%%%%%%%%%%%%%%%%%%%%%%%%%%%%%%%%%%%%%%%%%
    and then turning it into the \hyperref[Matrix Vector Multiplication]{\ulink{matrix vector multiplication}} form, specifically the \rrr{post-multiplication form} in order to create a \yyy{column vector of constants}:
    \[%%%%%%%%%%%%%%%%%%%%%%%%%%%%%%%%%%%%%%%%%%%%%%%
    \begin{bmatrix}
    a_{1,1} & a_{1,2} & \cdots & a_{1,n} \\
    a_{2,1} & a_{2,2} & \cdots & a_{2,n} \\
    \vdots & \vdots & \ddots & \vdots \\
    a_{m,1} & a_{m,2} & \cdots & a_{m,n}
    \end{bmatrix}
    \rrr{\begin{bmatrix} x_1 \\ x_2 \\ \vdots \\ x_n \end{bmatrix}}
    =
    \yyy{\begin{bmatrix} b_1 \\ b_2 \\ \vdots \\ b_m \end{bmatrix}}
    \]%%%%%%%%%%%%%%%%%%%%%%%%%%%%%%%%%%%%%%%%%%%%%%%
    \item \hyperref[Augmented and Complex Matrices]{\ulink{Augmented matrices}} are a useful intermediate that is used to represent the final product and perform elementary operations on.
      \begin{itemize}
        \item Creating an augmented matrix can easily be done by taking the above form, \rrr{dropping the variables} temporarily, then \yyy{concatenating the vector of constants} onto the \chap{matrix of coefficients}, i.e.,
      \[%%%%%%%%%%%%%%%%%%%%%%%%%%%%%%%%%%%%%%%%%%%%%%%
      \begin{bmatrix}[cccc|c]
        \chap{a_{1,1}} & \chap{a_{1,2}} & \chap{\cdots} & \chap{a_{1,n}} & \yyy{b_1}\\
        \chap{a_{2,1}} & \chap{a_{2,2}} & \chap{\cdots} & \chap{a_{2,n}} & \yyy{b_2}\\
        \chap{\vdots} & \chap{\vdots} & \chap{\ddots} & \chap{\vdots} & \yyy{\vdots} \\
        \chap{a_{m,1}} & \chap{a_{m,2}} & \chap{\cdots} & \chap{a_{m,n}} & \yyy{b_m}
        \end{bmatrix}
      \]%%%%%%%%%%%%%%%%%%%%%%%%%%%%%%%%%%%%%%%%%%%%%%%
      \end{itemize} 
    \item \jjj{Elementary operations}: a set of operations that can be performed on a matrix while \emph{maintaining equivalence} (when two systems have the same set of solutions); the operations are as follows:
      \begin{enumerate}
        \item Interchange two rows: \emph{\(R_i \updownarrow R_j\)}
        \item Multiply one row by a nonzero number: \emph{\(kR_i \rightarrow R_i,~k\neq 0\)}
        \item Add a multiple of one row to a different row: \emph{\(R_i + kR_j \rightarrow R_i,~i\neq j\)}
      \end{enumerate}
  \end{itemize}
\end{itemize}

\section{Gaussian Elimination}\label{Gaussian Elimination}
\begin{itemize}
  \item \jjj{Gaussian elimination (row reduction)}: an algorithm for solving systems of linear equations by using a sequence of elementary operations on a matrix until the \emph{row~echelon form} is obtained.
  \begin{itemize}
    \item \jjj{Gauss-Jordan elimination}: when Gaussian elimination is used until a matrix reaches the \emph{reduced row echelon form}; sometimes it is computationally beneficial to stop and before such form is reached.
    \item The \emph{reduced row echelon form is unique}, unlike the row echelon form, i.e., it is independent of the sequences of row operations used.
  \end{itemize}
  \subsection{Row Echelon form}\label{Row Echelon form}
  \begin{itemize}
    \item \jjj{Row echelon form}: a matrix that has converted into a pseudo \hyperref[Diagonal and Triagnular Matrices]{\ulink{upper triangular matrix}} using Gaussian elimination, e.g. (* = any number; can be zero),
    \[%%%%%%%%%%%%%%%%%%%%%%%%%%%%%%%%%%%%%%%%%%%%%%%
    \begin{bmatrix}
      1 & \rrr{*} & \rrr{*} & \rrr{*} & \rrr{*} & \rrr{*} \\
      \bbb{0} & \bbb{0} & 1 & \rrr{*} & \rrr{*} & \rrr{*} \\
      \bbb{0} & \bbb{0} & \bbb{0} & 1 & \rrr{*} & \rrr{*} \\
      \bbb{0} & \bbb{0} & \bbb{0} & \bbb{0} & \bbb{0} & 1 \\
      \bbb{0} & \bbb{0} & \bbb{0} & \bbb{0} & \bbb{0} & \bbb{0} \\
    \end{bmatrix}
    \]%%%%%%%%%%%%%%%%%%%%%%%%%%%%%%%%%%%%%%%%%%%%%%%
    \begin{itemize}
      \item More specifically, a matrix is in row echelon form if:
      \begin{itemize}
        \item All rows consisting of only zeroes are on the bottom.
        \item The leading coefficient (pivot) of a nonzero row is always strictly to the right of the leading coefficient (sometimes must be 1) of the row above it. 
      \end{itemize}
      \item The row echelon form allows for \hyperref[Matrix Rank]{\ulink{matrix rank}} and kernel (not discussed).
    \end{itemize}
    \item \ddd{Reduced row echelon (row canonical) form}: a matrix that satisfies all the requirements of the row echelon form while additionally satisfying the following:
    \begin{itemize}
      \item The leading coefficient of each nonzero row must be 1 (often called a leading 1). 
      \item Columns containing a leading 1 have zeros in all other entries (below \emph{and above}).
      \item E.g., using the row echelon form above:
      \[%%%%%%%%%%%%%%%%%%%%%%%%%%%%%%%%%%%%%%%%%%%%%%%
      \begin{bmatrix}
        1 & \rrr{*} & \emph{0} & \rrr{*} & \emph{0} & \rrr{*} \\
        \bbb{0} & \bbb{0} & 1 & \rrr{*} & \emph{0} & \rrr{*} \\
        \bbb{0} & \bbb{0} & \bbb{0} & 1 & \emph{0} & \rrr{*} \\
        \bbb{0} & \bbb{0} & \bbb{0} & \bbb{0} & \bbb{0} & 1 \\
      \end{bmatrix}
      \]%%%%%%%%%%%%%%%%%%%%%%%%%%%%%%%%%%%%%%%%%%%%%%%
    \end{itemize}
    \item \ddd{Parameters}: the nonleading variables (\(\rrr{*x_n}\): the left over variables associated with nonzero elements) are replaced with new variables \(\{s,r,t,\dots\}\) when converting back systems of equations, hence why \hyperref[Linear Equations]{\ulink{coefficients}} are sometimes referred to as parameters. % chktex 21
    \item Any matrix can be brought to the (reduced) row echelon form by a sequence of elementary row operations, this leads to the Gaussian algorithm.
  \end{itemize}
  
  \subsection{The Gaussian Algorithm}\label{The Gaussian Algorithm}
  \begin{itemize}
    \item 
  \end{itemize}
  
\end{itemize}
  
  \section{Homogeneous Equations}\label{Homogeneous Equations}
  \begin{itemize}
  \item 
\end{itemize}
  
  
