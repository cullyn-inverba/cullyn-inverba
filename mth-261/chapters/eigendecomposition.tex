\chapter{Eigendecomposition}\label{Eigendecomposition}
% chktex-file 1

\section{Eigendecomposition Fundamentals}\label{Eigendecomposition Fundamentals}
\begin{itemize}
  \item \dd{Eigendecomposition}: the factorization of a matrix into a canonical form, whereby the matrix is represented in terms of its \str{eigen\textbf{value}} and \chap{eigen\textbf{vector}}.
    \begin{itemize}
      \item Only defined for square matrices. 
      \item \hyperref[Singular Value Decomposition]{\dlink{Singular value decomposition}} works for any \(m\times n\) matrix.
    \end{itemize}
  \item For an \(m \times m\) matrix, there are \(m\) eigenvalues and \(m\) eigenvectors.
    \begin{itemize}
      \item Each eigenvalue has an associated eigenvector, with the possibility of there being both \hyperref[Eigenvectors of Distinct Eigenvalues]{\dlink{distinct}} and \hyperref[Eigenvectors of Repeated Eigenvalues]{\dlink{repeated}} eigenvalues.
    \end{itemize}
  \item \textbf{\str{Eigenvector \(\bm{v}\)}}: a nonzero vector that changes at most by a \hyperref[Vector Scalar Multiplication]{\ulink{scalar}} when a linear transformation is applied to it. 
  \item \textbf{\chap{Eigenvalue \(\lambda\)}}: the corresponding factor by which the eigenvector is scaled.
  \item Formally, if \(T\) is a linear transformation from vector space \(V\) over a field \(F\) into itself and \eigv~is a nonzero vector in \(V\), then \eigv~is an \str{eigenvector} of \(T\) if \(T(v)\) is a \chap{scalar multiple} \(\eigl \) of \(\eigv \), i.e.,
  \[%%%%%%%%%%%%%%%%%%%%%%%%%%%%%%%%%%%%%%%%%%%%%%%
  \chap{T}(\str{v}) = \eigl \eigv
  \]%%%%%%%%%%%%%%%%%%%%%%%%%%%%%%%%%%%%%%%%%%%%%%%
  \item If \(V\) is a finite-dimensional, then the above equation is equivalent to
  \[%%%%%%%%%%%%%%%%%%%%%%%%%%%%%%%%%%%%%%%%%%%%%%%
  \bm{\chap{A}\str{u}} = \eigl \str{\bm{u}}
  \]%%%%%%%%%%%%%%%%%%%%%%%%%%%%%%%%%%%%%%%%%%%%%%%
  where \chap{\tbm{A}} is the matrix representation of \(T\) and \str{\tbm{u}} is the coordinate vector (vector in terms of particular ordered basis) of \(\eigv \).
  \item Essentially, this is useful as a single \chap{eigenvalue} \(\eigl \) can represent an entire matrix \chap{\tbm{A}} given the associated \str{eigenvector} \(\eigv \). Thus, finding the eigenvectors allows for a set of basis vectors (principal axis) that can be used to represent a dataset, often in a more efficient way.
    \begin{itemize}
      \item E.g., it can be very useful as each data point can be represented as a vector, leading to the application of a linear transformation that will not change direction of the data, but instead efficiently scale the data along the eigenvectors.
      \item Alternatively, there are often various patterns within datasets that can be much more apparent when organized along new axes, whereby eigenvectors are the means of such reorganization.
    \end{itemize}

  \subsection{Finding Eigenvalues}\label{Eigenvalues}
  \begin{itemize}
    \item 
  \end{itemize}
  

  \subsection{Finding Eigenvectors}\label{Eigenvectors}
  \begin{itemize}
    \item 
  \end{itemize}
  
\end{itemize}


\section{Diagonalization}\label{Diagonalization}
\begin{itemize}
  \item []
  
  \subsection{Matrix Powers}\label{Matrix Powers}
  \begin{itemize}
    \item 
  \end{itemize}
  
\end{itemize}

\section{Properties of Eigendecomposition}\label{Properties of Eigendecomposition}
\begin{itemize}
  \item []
  
  \subsection{Eigenvectors of Distinct Eigenvalues}\label{Eigenvectors of Distinct Eigenvalues}
  \begin{itemize}
    \item 
  \end{itemize}
  
  \subsection{Eigenvectors of Repeated Eigenvalues}\label{Eigenvectors of Repeated Eigenvalues}
  \begin{itemize}
    \item 
  \end{itemize}

  \subsection{Eigendecomposition of Symmetric Matrices}\label{Eigendecomposition of Symmetric Matrices}
  \begin{itemize}
    \item 
  \end{itemize}
  
  \subsection{Eigenlayers}\label{Eigenlayers}
  \begin{itemize}
    \item 
  \end{itemize}
  
  \subsection{Generalized Eigendecomposition}\label{Generalized Eigendecomposition}
  \begin{itemize}
    \item 
  \end{itemize}
  
\end{itemize}



