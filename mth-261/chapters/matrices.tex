\chapter{Matrices}\label{Matrices}  
\section{Matrix Terminology}\label{Matrix Terminology}
\begin{itemize}
  \item \jjj{Matrix \(\bm{M}_{\xxx{r},\yyy{c}}\)}: a \emph{rectangular array} of elements arranged in \xxx{rows \(\leftrightarrow \)} and \yyy{columns \(\updownarrow \)}, e.g.,
  \[%%%%%%%%%%%%%%%%%%%%%%%%%%%%%%%%%%%%%%%%%%%%%%%
  M = \begin{bmatrix}
  1 & 0 & 3 \\
  5 & 4 & 2 \\
  7 & 6 & 9 
  \end{bmatrix}
  \qquad M_{\xxx{3},\yyy{2}} = 6
  \]%%%%%%%%%%%%%%%%%%%%%%%%%%%%%%%%%%%%%%%%%%%%%%%
  \item \ddd{Block (partitioned) matrix}: a matrix that is interpreted as having been broken into sections called blocks or submatrices, e.g.,
  \begin{align*}
    \bm{M} = 
    \begin{bmatrix}
      \bbb{\bm{D}} & \fff{\bm{N}} \\
      \ttt{\bm{Y}} & \bbb{\bm{D}} 
    \end{bmatrix} 
    =
    \begin{bmatrix}
    \bbb{4} & \bbb{2} & \fff{0} & \fff{0} \\
    \bbb{6} & \bbb{9} & \fff{0} & \fff{0} \\
    \ttt{1} & \ttt{1} & \bbb{4} & \bbb{2} \\
    \ttt{1} & \ttt{1} & \bbb{6} & \bbb{9}
    \end{bmatrix}
    \\
    \bm{D} = 
    \bbb{\begin{bmatrix}
      4 & 2 \\
      6 & 9 
    \end{bmatrix}}\quad
    \bm{N} =
    \fff{\begin{bmatrix}
      0 & 0 \\
      0 & 0 
    \end{bmatrix}}\quad
    \bm{Y} = 
    \ttt{\begin{bmatrix}
      1 & 1 \\
      1 & 1 
    \end{bmatrix}}
  \end{align*}
  \begin{itemize}
    \item Can be used for large matrices with high level structure, offering convenient notation, and sometimes providing computational benefits.
  \end{itemize}

  \item \ddd{Diagonal}: the elements of matrix starting from the \emph{top~left~\(\searrow \, \)~lower~right}.
    \begin{itemize}
      \item \jjj{Off-diagonal}: elements not along the diagonal (0s and 1s in example below)
      \item Works for both \hyperref[Square and Rectangular Matrices]{\dlink{square and rectangular matrices}}, e.g.,
      \[%%%%%%%%%%%%%%%%%%%%%%%%%%%%%%%%%%%%%%%%%%%%%%%
      \begin{bmatrix}
        \emph{4} & 1 & 0 & 1\\
        0 & \emph{2} & 0 & 1\\
        1 & 0 & \emph{6} & 0\\
        1 & 1 & 0 & \emph{9} 
      \end{bmatrix}
      \qquad
      \begin{bmatrix}
        \emph{4} & 1 & 0 & 1 & 1 & 0 & 0 \\
        0 & \emph{2} & 0 & 1 & 0 & 1 & 1 \\
        1 & 0 & \emph{6} & 0 & 1 & 0 & 1 \\
        1 & 1 & 0 & \emph{9} & 1 & 0 & 1
      \end{bmatrix}
      \]%%%%%%%%%%%%%%%%%%%%%%%%%%%%%%%%%%%%%%%%%%%%%%%
    \end{itemize}
  \item \ddd{Matrix size}: a matrix with \xxx{\(m\) rows} and \yyy{\(n\) columns} is called an \(\xxx{m} \times \yyy{n}\) matrix, or \(\xxx{m}\)-by-\(\yyy{n}\) matrix, while \(\xxx{m}\) and \(\yyy{n}\) are the dimensions.
    \begin{itemize}
      \item Order matters---the convention is rows then columns, i.e., \(\xxx{m} \times \yyy{n} \neq \yyy{n} \times \xxx{m} \).
      \item \xxx{\textbf{MR}}. \yyy{\textbf{N}}i\yyy{\textbf{C}}e guy: a useful mnemonic to remember typical conventions.
    \end{itemize}
  \item \ddd{Matrix dimensionality}: 
    \begin{itemize}
      \item \(\mathbb{R}^{mn}\): describes the total number of elements, here the multiplication of the dimensions is commutative (order doesn't matter). 
      \item \(\mathbb{R}^{m \times n}\): the specific matrix size using rows and columns as described above.
      \item \(C(M) \in \mathbb{R}^m\): a collection of column vectors, i.e., a matrix spanned by set column vectors with \(m\) elements.
      \item \(R(M) \in \mathbb{R}^n\): a collection of row vectors, the inverse of above.
    \end{itemize}

  \subsection{Square and Rectangular Matrices}\label{Square and Rectangular Matrices}
  \begin{itemize}
    \item \ddd{Square matrix}: a matrix with the same number of rows and columns. 
      \begin{itemize}
        \item An \(n\times n\) matrix is known as a square matrix of order \(n\).
        \item Any two square matrices of the same order can be added and multiplied.
      \end{itemize}
    \item \ddd{Rectangular matrix}: a matrix with an unequal number of rows and columns, i.e., \(m \neq n\). 
    \item Both square and rectangular matrices have a \hyperref[Diagonal]{\ulink{diagonal}}, as described above.
  \end{itemize}

  \subsection{Symmetric and Skew-Symmetric Matrices}\label{Symmetric and Skew-Symmetric Matrices}
  \begin{itemize}
    \item \ddd{Symmetric matrix}: a square matrix that can be mirrored across the diagonal, e.g.,
    \[%%%%%%%%%%%%%%%%%%%%%%%%%%%%%%%%%%%%%%%%%%%%%%%
    \begin{bmatrix}
    4 & \ttt{-6} & \ttt{-1} \\
    \ttt{-6} & 2 & \ttt{9} \\
    \ttt{-1} & \ttt{9} &  0
    \end{bmatrix}
    \]%%%%%%%%%%%%%%%%%%%%%%%%%%%%%%%%%%%%%%%%%%%%%%%
      \begin{itemize}
        \item Algebraically, it's a square matrix \(A\) that is equal to its \hyperref[tbd]{\dlink{transpose}}, i.e., \\ \(A = A^T\).
        \item Additionally, it does not matter what is on the diagonal, as any number is equal to itself. 
      \end{itemize}
    \item \ddd{Skew-symmetric matrix}: a square matrix that is still symmetric, but all elements mirrored across the diagonal and inverted, e.g.,
    \[%%%%%%%%%%%%%%%%%%%%%%%%%%%%%%%%%%%%%%%%%%%%%%%
    \begin{bmatrix}
    0 & \fff{+6} & \fff{+1} \\
    \ttt{-6} & 0 & \fff{-9} \\
    \ttt{-1} & \ttt{9} &  0
    \end{bmatrix}
    \]%%%%%%%%%%%%%%%%%%%%%%%%%%%%%%%%%%%%%%%%%%%%%%%
    \begin{itemize}
      \item Algebraically, it's a square matrix \(A\) that is equal to its negative transpose, i.e., \(A = -A^T\)
      \item However, the diagonal must be zero, as zero is the only number that can equal its inverse. 
    \end{itemize}
  \end{itemize}
  
  \subsection{Identity and Zero Matrices}\label{Identity and Zero Matrices}
  \begin{itemize}
    \item \jjj{Identity matrix \(\bm{I}_n\)}: a matrix with size \(n \times n\) with \emph{all elements along the diagonal = 1} and \bbb{all other elements = 0}, e.g, \(\bm{I}_3\) and \(\bm{I_n}\) (\(~\cdots~\)indicate continuation of pattern):
    \[%%%%%%%%%%%%%%%%%%%%%%%%%%%%%%%%%%%%%%%%%%%%%%%
    \begin{bmatrix}
    \emph{1} & \bbb{0} & \bbb{0} \\
    \bbb{0} & \emph{1} & \bbb{0} \\
    \bbb{0} & \bbb{0} & \emph{1} 
    \end{bmatrix}
    \qquad
    \begin{bmatrix}
    1 & 0 & \cdots & 0 \\
    0 & 1 & \cdots & 0 \\
    \vdots & \vdots & \ddots & \vdots \\
    0 & 0 & \cdots & 1
    \end{bmatrix}
    \]%%%%%%%%%%%%%%%%%%%%%%%%%%%%%%%%%%%%%%%%%%%%%%%
    \begin{itemize}
      \item Essentially, the identity matrix is the equivalent of the number 1 in linear algebra. 
    \end{itemize}
    \item \jjj{Zero matrix 0}: a matrix of \emph{all zeros}.
  \end{itemize}
  
  \subsection{Diagonal and Triangular Matrices}\label{Diagonal and Triagnular Matrices}
  \begin{itemize}
    \item \ddd{Diagonal matrix}: when all elements \emph{outside the main diagonal} are zero, i.e.,
    \[%%%%%%%%%%%%%%%%%%%%%%%%%%%%%%%%%%%%%%%%%%%%%%%
    \begin{bmatrix}
    e_{1,1} & 0 & \cdots & 0 \\
    0 & e_{2,2} & 0 & \vdots \\
    \vdots & 0 & \ddots & 0  \\
    0 & \cdots & 0 & e_{i,i}  \\
    \end{bmatrix}
    \]%%%%%%%%%%%%%%%%%%%%%%%%%%%%%%%%%%%%%%%%%%%%%%%
    \begin{itemize}
      \item Elements along the diagonal don't have to be the same, and they can be zero (meaning rectangular matrices still can be diagonal, technically).
      \item When all the elements along the main diagonal are the same, then it's a scaled version of the \hyperref[Identity and Zero Matrices]{\ulink{identity matrix}}, i.e., \(\lambda\bm{I}\).
    \end{itemize}
    \item \ddd{Triangular matrices}: when all elements above or below the diagonal are zero, but not on both sides.
      \begin{itemize}
        \item \rrr{Upper triangular matrix}: when all the elements \bbb{below} the diagonal are zero. 
        \item \bbb{Lower triangular matrix}: when all the elements \rrr{above} the diagonal are zero. 
        \[%%%%%%%%%%%%%%%%%%%%%%%%%%%%%%%%%%%%%%%%%%%%%%%
        \rrr{\text{Upper}=\begin{bmatrix}
          e_{1,1} & e_{1,2} & e_{1,3} \\
          \bbb{0} & e_{2,2} & e_{2,3} \\
          \bbb{0} &  \bbb{0} & e_{3,3}  \\
        \end{bmatrix}}
        \qquad
        \bbb{\text{Lower}=\begin{bmatrix}
          e_{1,1} & \rrr{0} & \rrr{0} \\
          e_{2,1} & e_{2,2} & \rrr{0} \\
          e_{3,1} &  e_{3,2}& e_{3,3}  \\
        \end{bmatrix}}
        \]%%%%%%%%%%%%%%%%%%%%%%%%%%%%%%%%%%%%%%%%%%%%%%%
        
      \end{itemize}
  \end{itemize}
  
  \subsection{Augmented and Complex Matrices}\label{Augmented and Complex Matrices}
  \begin{itemize}
    \item \jjj{Augmented (concatenated) matrix \(\bm{A}~\,\vline~\bm{B}\)}: a matrix obtained by appending the columns of two given matrices, e.g.,
    \[%%%%%%%%%%%%%%%%%%%%%%%%%%%%%%%%%%%%%%%%%%%%%%%
    \begin{bmatrix}
    4 & 2 & 0 \\
    3 & 7 & 6 \\
    1 & 6 & 9 
    \end{bmatrix}
    ~\vline~
    \begin{bmatrix} 4 \\ 2 \\ 0 \end{bmatrix}
    =
    \begin{bmatrix}[ccc|c]
      4 & 2 & 0 & 4 \\
      3 & 7 & 6 & 2 \\
      1 & 6 & 9 & 0 
    \end{bmatrix}
    \]%%%%%%%%%%%%%%%%%%%%%%%%%%%%%%%%%%%%%%%%%%%%%%%
    \begin{itemize}
      \item Used typically for the purpose of performing the same elementary row operations on each of the given matrices. 
      \item Matrices must have the same number of rows (or columns for vertical augmentation) for the concatenation to be applied.
    \end{itemize}
    \item \ddd{Complex matrix}: A matrix whose elements may contain complex numbers.
    \item The \hyperref[Conjugate Transpose]{\ulink{conjugate transpose}} discussed previously in vectors can is used here as well. 
    \item \hyperref[Transposition]{\dlink{Transposition}} will be discussed shortly, but for now, the complex conjugate still behaves the same (just imaginary numbers change sign), e.g.,
    \[%%%%%%%%%%%%%%%%%%%%%%%%%%%%%%%%%%%%%%%%%%%%%%%
    \begin{bmatrix}
    1 & -1\yyy{+5i} & 0 \\
    1 & -2 & -4 \\
   \yyy{ 6i} & -4 & 5\yyy{-2i}  
    \end{bmatrix}^H
    =
    \begin{bmatrix}
    1 & 1 & \yyy{-6i} \\
    -1\yyy{-5i} & -2 & -4 \\
    0 & -4 & 5\yyy{+2i} 
    \end{bmatrix}
    \]%%%%%%%%%%%%%%%%%%%%%%%%%%%%%%%%%%%%%%%%%%%%%%%
  \end{itemize}  
\end{itemize}

\section{Basic Matrix Operations}\label{Basic Matrix Operations}
\begin{itemize}
  \item []
  
  \subsection{Matrix Addition and Subtraction}\label{Matrix Addition and Subtraction}
  \begin{itemize}
    \item \ddd{Matrix addition (subtraction)}: the operation of adding (subtracting) two matrices of \emph{equal dimensions} \(m \times n\) by adding (subtracting) the corresponding elements together, e.g.,
    \[%%%%%%%%%%%%%%%%%%%%%%%%%%%%%%%%%%%%%%%%%%%%%%%
    \begin{bmatrix}
    1 & 2 & 5 \\
    0 & 6 & 8 \\
    9 & 6 & 4 
    \end{bmatrix} +
    \begin{bmatrix}
      0 & 3 & 5 \\
      1 & -6 & 9 \\
      -5 & -4 & 0 
    \end{bmatrix} 
    = 
    \begin{bmatrix}
    1 & 5 & 10 \\
    1 & 0 & 17 \\
    4 & 2 & 4 
    \end{bmatrix}
    \]%%%%%%%%%%%%%%%%%%%%%%%%%%%%%%%%%%%%%%%%%%%%%%%
    \item Note: there are other operations which could also be considered addition for matrices, such as the direct sum and the Kronecker sum (not discussed as of now).
    \item \ttt{\cmark~Commutative}: \(\bm{A}+\bm{B} = \bm{B} + \bm{A}\)
    \item \ttt{\cmark~Associative}: \(\bm{A} + (\bm{B}+\bm{C}) = (\bm{A}+\bm{B} + C)\)
  \end{itemize}

  \subsection{Matrix Scalar Multiplication}\label{Matrix Scalar Multiplication}
  \begin{itemize}
    \item \jjj{Matrix scalar multiplication}: the same as \hyperref[Vector Scalar Multiplication]{\ulink{vector scalar multiplication}} or simply, scalar multiplication, as vectors are \(m \times 1~(\text{or}~1 \times n)\) matrices.
    \item Scalar multiplication is true when both \bbb{left scalar} and \rrr{right scalar} are equal, i.e.,
    \[%%%%%%%%%%%%%%%%%%%%%%%%%%%%%%%%%%%%%%%%%%%%%%%
    \bbb{\lambda}(\bm{M})_{ij} = (\bbb{\lambda} \bm{M})_{ij} = ( \bm{M}\rrr{\lambda})_{ij}= (\bm{M})_{ij}\rrr{\lambda} % chktex 3
    \]%%%%%%%%%%%%%%%%%%%%%%%%%%%%%%%%%%%%%%%%%%%%%%%
    \item More explicitly: 
    \[%%%%%%%%%%%%%%%%%%%%%%%%%%%%%%%%%%%%%%%%%%%%%%%
    \begin{bmatrix}
    \bbb{\lambda} e & \bbb{\lambda} e & \cdots & \bbb{\lambda} e \\
    \bbb{\lambda} e & \bbb{\lambda} e & \cdots & \bbb{\lambda} e \\
    \vdots & \vdots & \ddots & \vdots \\
    \bbb{\lambda} e & \bbb{\lambda} e & \cdots & \bbb{\lambda} e
    \end{bmatrix}
    =
    \left(
      \bbb{\lambda}~\text{or}
    \begin{bmatrix}
    e & e & \cdots & e \\
    e & e & \cdots & e \\
    \vdots & \vdots & \ddots & \vdots \\
    e & e & \cdots & e
    \end{bmatrix}
      \text{or}~\rrr{\lambda}
    \right)
    =
    \begin{bmatrix}
    e\rrr{\lambda} & e\rrr{\lambda} & \cdots & e\rrr{\lambda} \\
    e\rrr{\lambda} & e\rrr{\lambda} & \cdots & e\rrr{\lambda} \\
    \vdots & \vdots & \ddots & \vdots \\
    e\rrr{\lambda} & e\rrr{\lambda} & \cdots & e\rrr{\lambda}
    \end{bmatrix}
    \]%%%%%%%%%%%%%%%%%%%%%%%%%%%%%%%%%%%%%%%%%%%%%%%
    \item The above is \ttt{true only} where the underlying ring (algebraic structure that generalize fields\dots I need to learn more about this\dots) \ttt{is commutative}. This fact is essential for later proofs.
  \end{itemize}
  
  \subsection{Transposition}\label{Transposition}
  \begin{itemize}
    \item 
  \end{itemize}
  
\end{itemize}



\newpage

\section{To be defined}\label{tbd}
\begin{itemize}
  \item Matrix multiplication
  \item matrix computations (determining weights)
  \item determinant
  \item change of basis
\end{itemize}