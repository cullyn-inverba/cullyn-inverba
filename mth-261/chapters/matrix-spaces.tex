\chapter{Matrix Spaces}\label{Matrix Spaces}
\section{Column Space}\label{Column Space}
\begin{itemize}
  \item \jjj{Column space (range, image) \(C(\bm{A})\)}: the vector subspace spanned (all possible weighted linear combinations) by the columns of a matrix \tbm{A}, i.e.,
  \[%%%%%%%%%%%%%%%%%%%%%%%%%%%%%%%%%%%%%%%%%%%%%%%
  C(\bm{A})=\{\lambda_1a_1 + \lambda_2a_2 + \cdots + \lambda_n a_n\}\quad \lambda\in \mathbb{R}  
  \]%%%%%%%%%%%%%%%%%%%%%%%%%%%%%%%%%%%%%%%%%%%%%%%
  \item Can also be expressed as the span of all the columns in a matrix, i.e.,
  \[%%%%%%%%%%%%%%%%%%%%%%%%%%%%%%%%%%%%%%%%%%%%%%%
  C(\bm{A})= \operatorname{span}(\{a_1,\dots,a_n\})
  \]%%%%%%%%%%%%%%%%%%%%%%%%%%%%%%%%%%%%%%%%%%%%%%%
    \begin{itemize}
      \item \jjj{Image}: the set of all output values of a function. 
      \item The column space is often called the image of a matrix since the span of all columns of a corresponding \hyperref[tbd]{\dlink{linear transformation}} is the image of that  transformation.
    \end{itemize}
  \item A common and important question in applied linear algebra is determining whether a vector is in a particular column space, and if not, then how close is it.
    \begin{itemize}
      \item Useful tools include \hyperref[Gaussian Elimination]{\dlink{reduced row echelon form}} and \hyperref[tbd]{\dlink{least squares algorithm}}.
    \end{itemize}

  \subsection{Row Space}\label{Row Space}
  \begin{itemize}
    \item \jjj{Row space \(R(\bm{A}), C(\bm{A}^T)\)}: the vector subspace spanned by all the rows of a matrix. 
    \item The row space is very similar to the column space, but does have some differences.
      \begin{itemize}
        \item When talking about the column space of a matrix, you need to \rrr{post}-multiply by a column vector, while the row space requires \bbb{pre}-multiplication of the matrix with a row vector, i.e.,
        \[%%%%%%%%%%%%%%%%%%%%%%%%%%%%%%%%%%%%%%%%%%%%%%%
        \bm{A}\rrr{\bm{x}}=\yyy{\bm{b}} \qquad \bbb{\bm{x}^T}\bm{A}=\xxx{\bm{b}}
        \]%%%%%%%%%%%%%%%%%%%%%%%%%%%%%%%%%%%%%%%%%%%%%%%
        \item The difference may be trivial at first, but row spaces and column spaces are often different when using real data; if you try to filter the data (contained in the matrix) with a vector, then the distinction here matters.
      \end{itemize}
    \item Using \hyperref[Row Echelon form]{\dlink{reduced row echelon form}} on a matrix will not change its row space, but it can (not always) change its column space.
    \item For a matrix that represents a \hyperref[Homogeneous Solution Sets]{\dlink{homogeneous system of linear equations}}, the row space consists of all linear equations that follow from those in the system. 
  \end{itemize}  
\end{itemize}

\section{Null Space}\label{Null Space}
\begin{itemize}
  \item []
  
  \subsection{Left Null Space}\label{Left Null Space}
  \begin{itemize}
    \item 
    
  \end{itemize}
\end{itemize}

\section{Dimensionality of Matrix Spaces}\label{Dimensionality of Matrix Spaces}
\begin{itemize}
  \item []
  
  \subsection{Examples of the Subspaces}\label{Examples of the Subspaces}
  \begin{itemize}
    \item 
  \end{itemize}
  
\end{itemize}
