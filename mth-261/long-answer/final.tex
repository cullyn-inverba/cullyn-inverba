\documentclass[basic]{inVerba-notes}
\usepackage{inVerba-math}
% chktex-file 21

\newcommand{\userName}{Cullyn Newman}
\newcommand{\class}{MTH:\@ 256}
\newcommand{\theTitle}{Final Exam}
\newcommand{\institution}{Portland State University}

\rfoot{\hyperref[toc]{\ulink{\thepage}}}

\begin{document}

\vspace*{90pt}

\begin{table}[h]
  \centering
  \scalebox{1.5}{
  \begin{tabular}{ccc}
    \toprule
    Page & Points  & Score \\
    \midrule
    \hyperlink{page.2}{\dlink{2}} & 3 & \\
    \hyperlink{page.3}{\dlink{3}} & 2 & \\
    \hyperlink{page.4}{\dlink{4}} & 4 & \\
    \hyperlink{page.5}{\dlink{5}} & 5 & \\
    \hyperlink{page.6}{\dlink{6}} & 6 & \\
    \hyperlink{page.7}{\dlink{7}} &  & \\
    \hyperlink{page.8}{\dlink{8}} &  & \\
    \hyperlink{page.9}{\dlink{9}} &  & \\
    \midrule
    Total: & 30 & \\
    \bottomrule
    \end{tabular}}
\end{table}\label{toc}

\newpage

\begin{enumerate}[align=left, leftmargin=0pt, labelindent=\parindent, listparindent=\parindent, labelwidth=0pt, itemindent=!]\color{minor}
  \item Use the fact to help respond to the prompts 1(a), 1(b), and 1(c) below.
  
  \textbf{Fact I:} The matrix equation below is consistent:
  \[%%%%%%%%%%%%%%%%%%%%%%%%%%%%%%%%%%%%%%%%%%%%%%%
  \begin{bmatrix}7 & -3\\2 & 1\\9 & -6\\-3 & 2 \end{bmatrix}
  \begin{bmatrix}-2\\-5\end{bmatrix}
  =
  \begin{bmatrix}1\\-9\\12\\-4\end{bmatrix}
  \]%%%%%%%%%%%%%%%%%%%%%%%%%%%%%%%%%%%%%%%%%%%%%%%

  \begin{enumerate}
    \item \pts{1} Write the matrix equation as an equivalent vector equation.
    \basec{\begin{align*}
      \bm{A\rrr{x}}&=\bm{\chap{b}} \\
      &\downarrow \\
      \begin{bmatrix}
        \yyy{a}_{\xxx{1}\yyy{1}} & \yyy{a}_{\xxx{1}\yyy{2}} & \cdots & \yyy{a}_{\xxx{1}\yyy{n}} \\
        \yyy{a}_{\xxx{2}\yyy{1}} & \yyy{a}_{\xxx{2}\yyy{2}} & \cdots & \yyy{a}_{\xxx{2}\yyy{n}} \\
        \vdots & \vdots & \ddots & \vdots \\
        \yyy{a}_{\xxx{m}\yyy{1}} & \yyy{a}_{\xxx{m}\yyy{2}} & \cdots & \yyy{a}_{\xxx{m}\yyy{n}}
        \end{bmatrix}
        \rrr{\begin{bmatrix} x_1 \\ x_2 \\ \vdots \\ x_n \end{bmatrix}}
        &=
        \chap{\begin{bmatrix} b_1 \\ b_2 \\ \vdots \\ b_m \end{bmatrix}} \\
        &\downarrow && \text{Given equation}\\
        \rrr{-2}\yyy{\bm{a_1}} - \rrr{5}\yyy{\bm{a_2}} &= \chap{\bm{b}} \\ \\
        \text{where}~ \yyy{\bm{a_1}},\yyy{\bm{a_2}} = \begin{bmatrix} 7 \\ 2 \\ 9 \\ -3 \end{bmatrix}, \begin{bmatrix} -3 \\ 1 \\ -6 \\ 2 \end{bmatrix}~&\text{and}~\chap{\bm{b}}= \begin{bmatrix} 1 \\ -9 \\ 12 \\ -4 \end{bmatrix}
    \end{align*}}
    
    \item \pts{2} Is \(\begin{bmatrix} 1 \\ -9 \\ 12\\-4 \end{bmatrix}\) in span\(\left\{ \begin{bmatrix} 7 \\ 2 \\ 9 \\ -3 \end{bmatrix}, \begin{bmatrix} -3 \\ 1 \\ -6 \\ 2 \end{bmatrix} \right\} \)? Justify your response. 
    \medskip
    \basec{\begin{itemize}
      \item \true{Yes}, \chap{\tbm{b}} is in the span of \tbm{A}.
      \item Span can be defined as set of all finite linear combinations of vectors of \tbm{A} over field \(K\), i.e.,
      \[%%%%%%%%%%%%%%%%%%%%%%%%%%%%%%%%%%%%%%%%%%%%%%%
      \operatorname{span}(\bm{A})=\left\lbrace\left. \sum_{i = 1}^{k} \rrr{\lambda_i}\,\yyy{\bm{a}_i} ~ \right| k \in \mathbb{N}, \yyy{\bm{a}_i} \in \bm{A}, \rrr{\lambda_i} \in K \right\rbrace
      \]%%%%%%%%%%%%%%%%%%%%%%%%%%%%%%%%%%%%%%%%%%%%%%%
      \item Essentially, this is equivalent to asking if there exists such vector \tbm{x} such that \(\bm{A\rrr{x}}=\chap{\bm{b}}\). Part (a) showed that there is such vector \((\rrr{\bm{x}}= \begin{bmatrix} -2 & 5 \end{bmatrix})\), thus \chap{\bm{b}} is in the span of \tbm{A}. 
      \item This can be confirmed by row reducing, just in case you don't trust random facts:
      \[%%%%%%%%%%%%%%%%%%%%%%%%%%%%%%%%%%%%%%%%%%%%%%%
      \operatorname{rref}\left(\begin{bmatrix}
      7 & -3 & 1 \\
      2 & 1 & -9 \\
      9 & -6 & 12 \\
      -3 & 2 & -4  
      \end{bmatrix}\right) =
      \begin{bmatrix}
      1 & 0 & \rrr{-2} \\
      0 & 1 & \rrr{-5} \\
      0 & 0 & 0 \\
      0 & 0 & 0 
      \end{bmatrix}, \quad \bm{x}= \begin{bmatrix} -2 \\ -5 \end{bmatrix}
      \]%%%%%%%%%%%%%%%%%%%%%%%%%%%%%%%%%%%%%%%%%%%%%%%
      
    \end{itemize}}
    
    \item \pts{2} Recall that a set of vectors \(\{ \bm{v_1}, \bm{v_2}, \, \ldots , \bm{v_k} \}\) is linearly independent if the only solution to the equation 
    \[c_1 \bm{v_1}+ c_2 \bm{v_2} + \cdots + c_k \bm{v_k} = \bm{0}
    \]
    is the trivial solution \(c_1=c_2=\cdots =c_k=0\).

    Is the set \(S=\left\{ \begin{bmatrix} 1 \\ -9 \\ 12\\-4 \end{bmatrix}, \begin{bmatrix} 7 \\ 2 \\ 9 \\ -3 \end{bmatrix}, \begin{bmatrix} -3 \\ 1 \\ -6 \\ 2 \end{bmatrix} \right\}\) linearly independent? 
    \basec{\begin{itemize}
      \item \false{No}, the set \(S\) is linearly dependent; the \yyy{column space} must span all of \(\R^\xxx{m}\) in order to be linearly dependent.
      \item A good way to test this is using the relationship between the \yyy{column space} and \xxx{cokernel}. The dot with the column space \(S\) and a vector from the \xxx{cokernel} must be orthogonal, i.e.,
      \[%%%%%%%%%%%%%%%%%%%%%%%%%%%%%%%%%%%%%%%%%%%%%%%
      \xxx{\bm{x}^T}\{\lambda_1 \yyy{a_1} + \cdots + \lambda_n \yyy{a_n}\} = 0
      \]%%%%%%%%%%%%%%%%%%%%%%%%%%%%%%%%%%%%%%%%%%%%%%%
      \item This implies that the only vector to make this equation true is the zero vector, if the set is linearly independent (trivial solution). Extrapolating using the rank-nullity theorem leads to the conclusion that the kernel must be empty and the set must but full rank in order to be linearly independent.
      \item First, the number of rows are more than number of columns, so without any calculation, one can tell nullity is out least one. However, using the rref form above clearly shows that dimension of the \yyy{column space} is \yyy{2}, implying the \xxx{cokernel's} dimension is \xxx{2} as well; a basis for the \xxx{cokernel} can be described as:
      \[%%%%%%%%%%%%%%%%%%%%%%%%%%%%%%%%%%%%%%%%%%%%%%%
      \xxx{\left\{\begin{bmatrix} -21/13 \\ 15/13 \\ 1 \\ 0 \end{bmatrix}\begin{bmatrix} 7/13 \\ -5/13 \\ 0 \\ 1 \end{bmatrix}\right\}}
      \]%%%%%%%%%%%%%%%%%%%%%%%%%%%%%%%%%%%%%%%%%%%%%%%
      thus, (\(\lambda \in \R\)):
      \begin{align*}
        \xxx{{x_1}^T}\{\lambda\yyy{a_1} + \lambda\yyy{a_2}+ \lambda\yyy{\bm{a_3}} \} = 0 \\
        \xxx{{x_2}^T}\{\lambda\yyy{a_1} + \lambda\yyy{a_2}+ \lambda\yyy{\bm{a_3}} \} = 0 
      \end{align*}
      I.e., not trivial solutions \to~\(S\) is \fff{linearly dependent}.
    \end{itemize}}
    
  \end{enumerate}

  \newpage

  \item Let \(\bm{C}=\begin{bmatrix}8 & 6\\5 & 4\end{bmatrix}\) and \(\bm{D}=\begin{bmatrix}2 & -3\\-2.5 & 4\end{bmatrix}\) for 2(a) and 2(b).
  
  \begin{enumerate}
    \item \pts{2} Show that \(\bm{D}=\bm{C}^{-1}\) by showing that \(\bm{DC}=\bm{I}\). You must show some details though you can use a calculator to check your work.
    
    \basec{\[%%%%%%%%%%%%%%%%%%%%%%%%%%%%%%%%%%%%%%%%%%%%%%%
    \hspace{-24pt}\bm{DC} =
    \begin{bmatrix}
    \bbb{8} & \chap{6} \\
    \bbb{5} & \chap{4} 
    \end{bmatrix} 
    \begin{bmatrix}
    \emph{2} & \rrr{-3} \\
    \emph{-2.5} & \rrr{4} 
    \end{bmatrix}
    = 
    \begin{pmatrix}
      \emph{2} \begin{bmatrix} \bbb{8} \\ \bbb{5} \end{bmatrix} + \emph{-2.5} \begin{bmatrix} \chap{6} \\ \chap{4} \end{bmatrix} &
      \rrr{-3} \begin{bmatrix} \bbb{8} \\ \bbb{5} \end{bmatrix} + \rrr{4} \begin{bmatrix} \chap{6} \\ \chap{4} \end{bmatrix} \end{pmatrix} 
      =
      \begin{bmatrix}
      1 & 0 \\
      0 & 1 
      \end{bmatrix}
    \]%%%%%%%%%%%%%%%%%%%%%%%%%%%%%%%%%%%%%%%%%%%%%%%%%%%%%%
    \[%%%%%%%%%%%%%%%%%%%%%%%%%%%%%%%%%%%%%%%%%%%%%%%
    \bm{C}^{-1}\bm{C} = \bm{I} = \bm{DC}, \quad \text{thus,}~\ttt{\bm{D}= \bm{C}^{-1}}
    \]}%%%%%%%%%%%%%%%%%%%%%%%%%%%%%%%%%%%%%%%%%%%%%%%
    
    \item \pts{2} Use the fact that \(\bm{D}=\bm{C}^{-1}\) to solve the system of linear equations (you can check your answer with another method, but full credit will only be awarded to the use of \(C^{-1}\)).
    \[
    \begin{cases}
    8x_1+6x_2=2 \\
    5x_1+4x_2=-1.
    \end{cases}
    \]

    \basec{\begin{align*}
      \bm{Cx}&=\bm{b} && \bm{b}= \begin{bmatrix} 2 \\ -1 \end{bmatrix} \\
      \bm{C}^{-1}\bm{Cx}&=\bm{C}^{-1}\bm{b} \\
      \bm{x} &= \bm{C}^{-1}\bm{b} \\
      \bm{x} &= \bm{D}\bm{b} && \ttt{\bm{D=C}^{-1}}\\ 
      \bm{x} &= \begin{bmatrix}
      2 & -3 \\
      -2.5 & 4  
      \end{bmatrix}\begin{bmatrix} 2 \\ -1 \end{bmatrix} \\
      \bm{x} &= \begin{bmatrix} 7 \\ -9 \end{bmatrix}
    \end{align*}}
    
  \end{enumerate}

  \newpage

  \item Below is a matrix \(\bm{A}\) and the reduced row echelon form of \(\bm{A}\). Use this \(\bm{A}\) for 3(a) and 3(b).
    \[
    A=
    \begin{bmatrix*}[r]
    \yyy{3 } & \yyy{-1}  & 7  & \yyy{3 }\\
    \yyy{-2}  &\yyy{ 2}  & -2  & \yyy{7 }\\
    \yyy{-5}  &\yyy{ 9}  & 3  & \yyy{3 }\\
    \yyy{1 } & \yyy{1 } & 5  & \yyy{10} \\
    \yyy{5 } & \yyy{-3}  & 9  & \yyy{-4}
    \end{bmatrix*} 
    \rightarrow
    \underbrace{
    \begin{bmatrix*}[r]
    \yyy{1} & \yyy{0} & 3 & \yyy{0} \\
    \yyy{0} & \yyy{1} & 2 & \yyy{0} \\
    \yyy{0} & \yyy{0} & 0 & \yyy{1} \\
    \yyy{0} & \yyy{0} & 0 & \yyy{0} \\
    \yyy{0} & \yyy{0} & 0 & \yyy{0}
    \end{bmatrix*}
    }_{\text{rref of }A}
    \]
  \begin{enumerate}
    \item \pts{2} Determine a basis for \(C(\bm{A})\). 
    
    \basec{\[%%%%%%%%%%%%%%%%%%%%%%%%%%%%%%%%%%%%%%%%%%%%%%%
    \yyy{C(\bm{A})}= \left\{
      \begin{bmatrix} 3 \\ -2 \\ -5 \\ 1 \\ 5 \end{bmatrix}
      \begin{bmatrix} -1 \\ 2 \\ 9 \\ 1 \\ -3 \end{bmatrix}
      \begin{bmatrix} 3 \\ 7 \\ 3 \\ 10 \\ -4 \end{bmatrix}
     \right\}
    \]}%%%%%%%%%%%%%%%%%%%%%%%%%%%%%%%%%%%%%%%%%%%%%%%
    \item \pts{2} Determine a basis for \(N(\bm{A})\).
    \basec{\[%%%%%%%%%%%%%%%%%%%%%%%%%%%%%%%%%%%%%%%%%%%%%%%
    \yyy{N(\bm{A})} = \begin{bmatrix} -3 \\ -2 \\ 1 \\ 0 \end{bmatrix}
    \]}%%%%%%%%%%%%%%%%%%%%%%%%%%%%%%%%%%%%%%%%%%%%%%%%%%%%%%
    \item \pts{1} Give the dimension of \(C(\bm{A})\) and the dimension of \(N(\bm{A})\)
    \basec{\[%%%%%%%%%%%%%%%%%%%%%%%%%%%%%%%%%%%%%%%%%%%%%%%
    \dim{\yyy{C(\bm{A})}} = 3, \quad \dim{\yyy{N(\bm{A})}} = 1
    \]%%%%%%%%%%%%%%%%%%%%%%%%%%%%%%%%%%%%%%%%%%%%%%%%%%%%%%
    Bonus: dimension of row space and left null space:
    \[%%%%%%%%%%%%%%%%%%%%%%%%%%%%%%%%%%%%%%%%%%%%%%%
    \dim{\xxx{C(\bm{A^T})}} = 3, \quad \dim{\xxx{N(\bm{A^T})}} = 2
    \]%%%%%%%%%%%%%%%%%%%%%%%%%%%%%%%%%%%%%%%%%%%%%%%
    Double-checking using rank-nullity theorem: \((\bm{A}~\text{is}~\xxx{5} \times \yyy{4})\):
    \[%%%%%%%%%%%%%%%%%%%%%%%%%%%%%%%%%%%%%%%%%%%%%%%
    \dim{\yyy{C(\bm{A})}} + \dim{\xxx{N(\bm{A^T})}} = \xxx{5} 
    \]%%%%%%%%%%%%%%%%%%%%%%%%%%%%%%%%%%%%%%%%%%%%%%%
    \[%%%%%%%%%%%%%%%%%%%%%%%%%%%%%%%%%%%%%%%%%%%%%%%
    \dim{\xxx{C(\bm{A^T})}} + \dim{\yyy{N(\bm{A})}} = \yyy{4}
    \]%%%%%%%%%%%%%%%%%%%%%%%%%%%%%%%%%%%%%%%%%%%%%%%
    
    }
  \end{enumerate}

    \newpage

  \item \pts{2} Is \(\begin{bmatrix}1\\1\\2\end{bmatrix}\) \((\eigv)\) an eigenvector of \(\begin{bmatrix}1 & -3 & 3\\3 & -5 & 3\\6 & -6 & 4\end{bmatrix}\) \((\basec{\bm{A}})\)?  If so, find the eigenvalue with which it associates.
  
  \basec{\[%%%%%%%%%%%%%%%%%%%%%%%%%%%%%%%%%%%%%%%%%%%%%%%
  \bm{A}\eigv = \begin{bmatrix} 4 \\ 4 \\ 8 \end{bmatrix}
  \]}%%%%%%%%%%%%%%%%%%%%%%%%%%%%%%%%%%%%%%%%%%%%%%%%%%%%%%
  \basec{\begin{itemize}
    \item \true{Yes}, \(\begin{bmatrix} 4 \\ 4 \\ 8 \end{bmatrix}\) is clearly a scalar multiple of \(\eigv\); the scalar is the associated \chap{eigenvalue}, i.e.,
    \[%%%%%%%%%%%%%%%%%%%%%%%%%%%%%%%%%%%%%%%%%%%%%%%
    \eigl\eigv=
    \chap{4}\str{\begin{bmatrix} 1 \\ 1 \\ 2 \end{bmatrix}} =
    \begin{bmatrix} 4 \\ 4 \\ 8 \end{bmatrix}
    \]%%%%%%%%%%%%%%%%%%%%%%%%%%%%%%%%%%%%%%%%%%%%%%%
    
  \end{itemize}}
  
  
  \item \pts{2} Find the eigenvalues of \(\begin{bmatrix}-5 & 2\\-7 & 4\end{bmatrix}\).
  
  \basec{\begin{align*}
    \det{\bm{A}-\eigl \bm{I}} = 0 \\
    \operatorname{det}\left(\begin{bmatrix} % chktex 35
    -5 - \eigl & 2 \\
    -7 & 4 - \eigl 
    \end{bmatrix}\right) = 0 \\
    (-5-\eigl)(4-\eigl)+14 = 0 \\
    \eigl = -3,2
  \end{align*}}

  \item \pts{2} Show that \(\lambda=-2\) is an eigenvalue of \(\begin{bmatrix}7 & 3\\3 & -1\end{bmatrix}\) by finding an eigenvector associated with it.
  \basec{\[%%%%%%%%%%%%%%%%%%%%%%%%%%%%%%%%%%%%%%%%%%%%%%%
  \operatorname{N}\left(\begin{bmatrix} % chktex 35
    7 \chap{-2} & 3 \\
    3 & -1 - \chap{-2} 
    \end{bmatrix}\right) = \str{\begin{bmatrix} -\tfrac{1}{3} \\ 1 \end{bmatrix}}
  \]}%%%%%%%%%%%%%%%%%%%%%%%%%%%%%%%%%%%%%%%%%%%%%%%%%%%%%%
  
  
  \newpage

  \item \pts{2} Let \(\bm{u}=\begin{bmatrix*}[r] -2 \\ 1 \\ 1 \end{bmatrix*}\),  \(\bm{v}=\begin{bmatrix} 3 \\ 4 \\ 1 \end{bmatrix}\), and \(\bm{w}=\begin{bmatrix} 1 \\ 2\\ 0 \end{bmatrix}\) for 7(a), 7(b), and 7(c).
  
    \begin{enumerate}
      \item \pts{2} Determine the distance between \(\bm{u}\) and \(\bm{v}\).
      \basec{\begin{align*}
        \bm{p}&= \bm{u}-\bm{v} = \begin{bmatrix} -5 \\ -3  \\ 0 \end{bmatrix}  \\
        \sqrt{\bm{p}^T \bm{p}} &= \emph{\sqrt{34}}
      \end{align*}}
      \item \pts{2} Which of the vectors \(\bm{u}\), \(\bm{v}\), and \(\bm{w}\) are orthogonal to one another?
      \basec{\begin{align*}
        \bm{u}^T \bm{v} &= -1\\
        \bm{u}^T \bm{w} &= 0 && \text{\true{\tbm{u} and \tbm{w} are orthogonal}}\\
        \bm{v}^T \bm{w} &= 11 
      \end{align*}}
      \item \pts{2} Consider the vectors \(\bm{u}\), \(\bm{v}\), and \(\bm{w}\) as points in \(\R^3\). Now think about the plane that contains those three points. Give a parametrization of all the points on this plane or give a scalar equation that defines the plane.
      
      \basec{\begin{align*}
        p_1 &= u - v = \begin{bmatrix} -5 & -3 & 0 \end{bmatrix}\\
        p_2 &= w - v = \begin{bmatrix} -2 & -2 & -1 \end{bmatrix} && \text{Generate two points on plane,}~p_1, p_2\\
        p_3 &= p_1 \times p_2 = \begin{bmatrix} 3 & -5 & 4 \end{bmatrix} && \text{Find vector normal to plane}\\
        b &= 3x-5y+4z && \text{Use normal vector to generate equation}\\ 
        b &= 3(-2)-5(-2)+4(-1)  && \text{Plug in a point on plane to find solution}\\ 
        b &= -7 && \downarrow
      \end{align*}
      \[%%%%%%%%%%%%%%%%%%%%%%%%%%%%%%%%%%%%%%%%%%%%%%%
      \text{Scalar equation of plane:}~\emph{3x-5y+4z+7=0}
      \]}%%%%%%%%%%%%%%%%%%%%%%%%%%%%%%%%%%%%%%%%%%%%%%%
      
    \end{enumerate}

    \newpage

    \item \pts{2} Consider three geometric objects in \(\R^3\), two planes and a line. One plane is given by the equation \(2x-z=0\) and the other is given by the equation \(3x+2z=0\). The line is parametrized by
    \[\bm{x}=\begin{bmatrix*}[r] x \\ y \\ z \end{bmatrix*} = \begin{bmatrix*}[r] 0 \\ 0 \\ 1 \end{bmatrix*} + t \begin{bmatrix*}[r] -1 \\ 5 \\ 0 \end{bmatrix*}
    \]
    for any real number \(t\). Do the three geometric objects share a common point of intersection? How do you know?
    
    \newpage

    \item \pts{2} Suppose \(f(x)\) is a degree two polynomial function in one variable. That is, let 
    \[ f(x)=ax^2+bx+c\]
    Determine coefficients \(a, b\), and \(c\) so that \(f(-1)=-2\), \(f(1)=4\), and \(f(3)=2\). \par (This is called an \textit{interpolating} polynomial since it might be used to estimate values between the data points given.) \par

    \textit{Hint:} The requirement that \(f(-1)=-2\) yields the equation 
    \[-2=a(-1)^2+b(-1)+c\] % chktex 3
    Use the other requirements to build a system of three equations in three unknowns. Then solve.
  \end{enumerate}
\end{document}