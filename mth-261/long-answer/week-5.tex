\documentclass[basic]{inVerba-notes}
\usepackage{inVerba-math}
% chktex-file 3
\newcommand{\userName}{Cullyn Newman}
\newcommand{\class}{MTH:\@ 261}
\newcommand{\theTitle}{More Multiplication Again}
\newcommand{\institution}{Portland State University}

 \begin{document}
   \begin{enumerate}[align=left, leftmargin=0pt, labelindent=\parindent, listparindent=\parindent, labelwidth=0pt, itemindent=!]\color{minor}
     \item Show that if \tbm{B} has a column of zeros, so too does \tbm{AB}.
     \basec{\begin{itemize}
       \item Visualizing matrix multiplication as building of the product matrix via scaling the columns of the left matrix by the columns of the right matrix allows for a good example of why the above is true, e.g.,
      \[%%%%%%%%%%%%%%%%%%%%%%%%%%%%%%%%%%%%%%%%%%%%%%%
       \begin{bmatrix} \bbb{a} & \emph{b} \\ \bbb{c} & \emph{d} \end{bmatrix}
       \begin{bmatrix} \ttt{\alpha} & \fff{0} \\ \ttt{\lambda} & \fff{0} \end{bmatrix}
       =
       \begin{pmatrix}
        \ttt{\alpha} \begin{bmatrix} \bbb{a} \\ \bbb{c} \end{bmatrix} + \ttt{\lambda} \begin{bmatrix} \emph{b} \\ \emph{d} \end{bmatrix} &
        \fff{0} \begin{bmatrix} \bbb{a} \\ \bbb{c} \end{bmatrix} + \fff{0} \begin{bmatrix} \emph{b} \\ \emph{d} \end{bmatrix}
      \end{pmatrix}
      =
      \begin{bmatrix}
        \ttt{\alpha}\bbb{a} + \ttt{\lambda}\emph{b} & \fff{0}+\fff{0} \\ \ttt{\alpha}\bbb{c}+  \ttt{\lambda}\emph{d} & \fff{0}+\fff{0} 
      \end{bmatrix}
    \]%%%%%%%%%%%%%%%%%%%%%%%%%%%%%%%%%%%%%%%%%%%%%%%%
    \item Under standard matrix multiplication, the product matrix dimensions are equal to \xxx{the~rows of the left matrix} \(\times \) \yyy{the columns of the right matrix}---thus, the product matrix must have a column of zeros if the right matrix contains a column of zeros.
    \end{itemize}}
    
    \item Create an example where \tbm{AB} has a column of zeros, but \tbm{B} does not.
    \basec{\begin{itemize}
      \item Again, the column perspective is very useful here---if the scaled columns of the left matrix summed together equal zero, then the entire column will be zero, e.g., 
      \vspace{-18pt}
    \end{itemize}}
    \basec{\[%%%%%%%%%%%%%%%%%%%%%%%%%%%%%%%%%%%%%%%%%%%%%%%
      \begin{bmatrix} \bbb{4} & \emph{2} \\ \bbb{6} & \emph{3} \end{bmatrix}
      \begin{bmatrix} \ttt{3} & \fff{1} \\ \ttt{1} & \fff{-2} \end{bmatrix}
      =
      \begin{pmatrix}
       \ttt{3} \begin{bmatrix} \bbb{4} \\ \bbb{6} \end{bmatrix} + \ttt{1} \begin{bmatrix} \emph{2} \\ \emph{3} \end{bmatrix} &
       \fff{1} \begin{bmatrix} \bbb{4} \\ \bbb{6} \end{bmatrix} + \fff{-2} \begin{bmatrix} \emph{2} \\ \emph{3} \end{bmatrix}
     \end{pmatrix}
     =
     \begin{bmatrix}
       \ttt{(}\bbb{12} + \emph{2}\ttt{)} & \fff{(}\bbb{4} - \emph{4}\fff{)} \\ \ttt{(}\bbb{18}+\emph{3}\ttt{)} & \fff{(}\bbb{6}-\emph{6}\fff{)} %chktex 9
     \end{bmatrix}
      \]%%%%%%%%%%%%%%%%%%%%%%%%%%%%%%%%%%%%%%%%%%%%%%%
      \[%%%%%%%%%%%%%%%%%%%%%%%%%%%%%%%%%%%%%%%%%%%%%%%
      =
      \begin{bmatrix}
        \ttt{14} & \fff{0} \\ \ttt{21} & \fff{0} 
      \end{bmatrix}
    \]}%%%%%%%%%%%%%%%%%%%%%%%%%%%%%%%%%%%%%%%%%%%%%%%
    \vspace{-18pt}
    \item For two numbers \(a\) and \(b\), note it's always true that
      \[%%%%%%%%%%%%%%%%%%%%%%%%%%%%%%%%%%%%%%%%%%%%%%%
      (a+b)^2 = a^2 + 2ab + b^2
      \]%%%%%%%%%%%%%%%%%%%%%%%%%%%%%%%%%%%%%%%%%%%%%%%
    Find two matrices \tbm{A} and \tbm{B} so that 
      \[%%%%%%%%%%%%%%%%%%%%%%%%%%%%%%%%%%%%%%%%%%%%%%%
      (\bm{A} + \bm{B})^2 \neq \bm{A}^2 + 2 \bm{AB} + \bm{B}^2
      \]%%%%%%%%%%%%%%%%%%%%%%%%%%%%%%%%%%%%%%%%%%%%%%%
      \vspace{-18pt}
    \basec{\begin{itemize}
      \item The above is true if and only if \(\bm{AB}=\bm{BA}\), i.e.,
      \[%%%%%%%%%%%%%%%%%%%%%%%%%%%%%%%%%%%%%%%%%%%%%%%
      (\bm{A} + \bm{B})^2 = \bm{A}(\bm{A}+\bm{B}) + \bm{B}(\bm{A}+\bm{B}) =  \bm{A}^2 + \bm{AB} + \bm{BA} + \bm{B}^2
      \]%%%%%%%%%%%%%%%%%%%%%%%%%%%%%%%%%%%%%%%%%%%%%%%
        \begin{itemize}
          \item This is true if \tbm{B} is the identity matrix, in which case \(\bm{AI}=\bm{IA}=\bm{A}\).
          \item Or if \tbm{A} is invertible and \tbm{B} is equal to the inverse, in which case \(\bm{AA}^{-1} = \bm{I} = \bm{A}^{-1}\bm{A}\)
        \end{itemize} 
      \item Thus, any two square matrices of equal size such that \(\bm{AB} \neq \bm{BA} \) will yield a counter example, e.g,
      \begin{align*}
      \bm{A} =\begin{bmatrix}
      4 & 2 & 0 \\
      0 & 0 & 6 \\
      0 & 0 & 9 
      \end{bmatrix} & ~\quad
     \begin{bmatrix}
      3 & 1 & 4 \\
      0 & 0 & 0 \\
      8 & 2 & 6 
      \end{bmatrix} = \bm{B}
      \\
      \bm{AB} =\begin{bmatrix}
        12 & 4  & 16 \\
        48 & 12 &  36 \\
        72 & 18 &  54 
      \end{bmatrix} &\neq
      \begin{bmatrix}
        12 & 6  & 42 \\
        0 & 0  & 0 \\
        32 & 16 &  66 
      \end{bmatrix} = \bm{BA} 
      \\
      \bm{2AB} = \begin{bmatrix}
        24  & 8 & 32 \\
        96  & 24 & 72\\
        144 & 36 & 108
      \end{bmatrix} &\neq
      \begin{bmatrix}
        24  & 10 & 58\\
        48  & 12 & 36 \\
        104 & 34 & 120
      \end{bmatrix} = \bm{AB} + \bm{BA}
      \end{align*}
    \end{itemize}}
    \item Let \tbm{A} and \tbm{B} denote invertible \(n\times n\) matrices. Show that if \(\bm{A}^{-1} = \bm{B}^{-1}\), then \(\bm{A}=\bm{B}\). 
    \basec{
    \[%%%%%%%%%%%%%%%%%%%%%%%%%%%%%%%%%%%%%%%%%%%%%%%
    \emph{\bm{A}^{-1}} = \emph{\bm{B}^{-1}}
    \]%%%%%%%%%%%%%%%%%%%%%%%%%%%%%%%%%%%%%%%%%%%%%%%
    \[%%%%%%%%%%%%%%%%%%%%%%%%%%%%%%%%%%%%%%%%%%%%%%%
    \downarrow
    \]%%%%%%%%%%%%%%%%%%%%%%%%%%%%%%%%%%%%%%%%%%%%%%%
    \[%%%%%%%%%%%%%%%%%%%%%%%%%%%%%%%%%%%%%%%%%%%%%%%
    \bm{A}^{-1}\bm{A} = \bm{AA}^{-1} = \bm{AB}^{-1} = \bbb{\bm{I}} = \bm{B}^{-1}\bm{A} = \bm{BB}^{-1} = \bm{B}^{-1}\bm{B}
    \]%%%%%%%%%%%%%%%%%%%%%%%%%%%%%%%%%%%%%%%%%%%%%%%%%%%%%%
    \[%%%%%%%%%%%%%%%%%%%%%%%%%%%%%%%%%%%%%%%%%%%%%%%
    \downarrow 
    \]%%%%%%%%%%%%%%%%%%%%%%%%%%%%%%%%%%%%%%%%%%%%%%%
    \[%%%%%%%%%%%%%%%%%%%%%%%%%%%%%%%%%%%%%%%%%%%%%%%
    \emph{\bm{A}} = \emph{\bm{A}}\bbb{\bm{I}} = \emph{\bm{A}}\bbb{(\bm{B}^{-1}\bm{B})} = \bbb{(\bm{A}\bm{B}^{-1})}\emph{\bm{B}} = \bm{\bbb{I}\emph{B}} = \emph{\bm{B}}
    \]}%%%%%%%%%%%%%%%%%%%%%%%%%%%%%%%%%%%%%%%%%%%%%%%
   \end{enumerate}
 \end{document}