\documentclass[basic]{inVerba-notes}
\usepackage{inVerba-math}

\newcommand{\userName}{Cullyn Newman}
\newcommand{\class}{MTH:\@ 256}
\newcommand{\theTitle}{Midterm}
\newcommand{\institution}{Portland State University}

\begin{document}

\vspace*{160pt}

\begin{table}[h]
  \centering
  \scalebox{1.4}{
  \begin{tabular}{ccc}
    \toprule
    Page & Points  & Score \\
    \midrule
    \hyperlink{page.2}{\dlink{2}} & 6 & \\
    \hyperlink{page.3}{\dlink{3}} & 8 & \\
    \hyperlink{page.4}{\dlink{4}} & 4 & \\
    \hyperlink{page.5}{\dlink{5}} & 4 & \\
    \hyperlink{page.6}{\dlink{6}} & 4 & \\
    \hyperlink{page.7}{\dlink{7}} & 4 & \\
    \midrule
    Total: & 30 & \\
    \bottomrule
    \end{tabular}}
\end{table}

\newpage

\begin{enumerate}[align=left, leftmargin=0pt, labelindent=\parindent, listparindent=\parindent, labelwidth=0pt, itemindent=!]\color{minor}

  \item Determine whether each statement below is true or false, then explain how you know. Note: that if the statement is false, then it might be easiest to provide a counterexample as justification.
  \begin{enumerate}
    \item \pts{2} If a linear system has \(n\) variables and \(m\) equations, then the augmented matrix has
    \(n\) columns.

    \basec{\begin{itemize}
      \item \true{True}: creating an augmented matrix can easily be done by dropping the variables temporarily, then \yyy{concatenating the vector of constants} onto the \chap{matrix~of coefficients}, i.e.,
      \[%%%%%%%%%%%%%%%%%%%%%%%%%%%%%%%%%%%%%%%%%%%%%%%
      \begin{bmatrix}[cccc|c]
        \chap{a_{1,1}} & \chap{a_{1,2}} & \chap{\cdots} & \chap{a_{1,n}} & \yyy{b_1}\\
        \chap{a_{2,1}} & \chap{a_{2,2}} & \chap{\cdots} & \chap{a_{2,n}} & \yyy{b_2}\\
        \chap{\vdots} & \chap{\vdots} & \chap{\ddots} & \chap{\vdots} & \yyy{\vdots} \\
        \chap{a_{m,1}} & \chap{a_{m,2}} & \chap{\cdots} & \chap{a_{m,n}} & \yyy{b_m}
        \end{bmatrix}
      \]%%%%%%%%%%%%%%%%%%%%%%%%%%%%%%%%%%%%%%%%%%%%%%%
      where the number of variables are equal to columns (\(n\)) of the matrix of coefficients and the numbers of equations are equal to the rows (\(m\)).
      \item Though, if you count the vector of constants as part of the augmented matrix then technically it's \fff{false}, since the totals columns is \(\chap{n}+\yyy{1}\).
    \end{itemize}}
    
    \item \pts{2} An inconsistent system can be made consistent by performing a sequence of elementary row operations.
    
    \basec{\begin{itemize}
      \item \false{False}: an inconsistent system is always inconsistent. For example, nothing can be down to the following matrix that would make it consistent:
      \[%%%%%%%%%%%%%%%%%%%%%%%%%%%%%%%%%%%%%%%%%%%%%%%
      \begin{bmatrix}[cccc|c]
        1 & -2 & 1 & -4 & 1 \\
        0 & 5 & 6 & 6 & 1 \\
        \rrr{0} & \rrr{0} & \rrr{0} & \rrr{0} & \rrr{6}
      \end{bmatrix}
      \]%%%%%%%%%%%%%%%%%%%%%%%%%%%%%%%%%%%%%%%%%%%%%%%
      One might try to say that multiplying the final row by 
      zero would make it consistent, but such multiplication would not be reversible, making it an invalid elementary row operation.
    \end{itemize}}
    
    \item \pts{2} A consistent system whose augmented matrix has 3 rows and 5 columns will have infinitely many solutions.
    
    \basec{\begin{itemize}
      \item \true{True}: a consistent system will have infinitely many solutions if the rank of the matrix is less than the number of columns.
      \item The max rank of matrix can be defined as a non-negative integer, including zero (\(\mathbb{N}_0\)), that is equal to the smaller of the two dimensions, either the \xxx{rows} or \yyy{columns}, i.e.,
      \[%%%%%%%%%%%%%%%%%%%%%%%%%%%%%%%%%%%%%%%%%%%%%%%
      \max(r) = r \in \mathbb{N}_0~\,|~\,0 \leq r \leq \min(\xxx{m},\yyy{n})
      \]%%%%%%%%%%%%%%%%%%%%%%%%%%%%%%%%%%%%%%%%%%%%%%%
      Thus, a \(3\times5\) matrix is rank deficient (\(r=3\)), meaning \(r<\yyy{n}\), which means that there are infinitely many solutions.
    \end{itemize}}
    
  \end{enumerate}

  \newpage

  \item Create an augmented matrix for the scenarios below or explain why it is impossible to do so. The associated system:
  
  \begin{enumerate}
    \item \pts{2} has infinite solutions, but the augmented matrix has no row of zeros.
  
    \basec{\[%%%%%%%%%%%%%%%%%%%%%%%%%%%%%%%%%%%%%%%%%%%%%%%
    \bm{A}=\begin{bmatrix}[cccc|c]
    \jjj{6} & 0 & 0 & 0 & 4 \\
    0 & \jjj{9} & 0 & 0 & 2 \\
    0 & 0 & 0 & \jjj{1} & 0 
    \end{bmatrix}
    \]}%%%%%%%%%%%%%%%%%%%%%%%%%%%%%%%%%%%%%%%%%%%%%%%%%%%%%%
    \basec{\[%%%%%%%%%%%%%%%%%%%%%%%%%%%%%%%%%%%%%%%%%%%%%%%
    \rank{\bm{A}} = 3,\quad r < n \to \text{infinite solutions}
    \]}%%%%%%%%%%%%%%%%%%%%%%%%%%%%%%%%%%%%%%%%%%%%%%%%%%%%%%
    \item \pts{2} has exactly one solution, but the augmented matrix has two rows of zeros.
    
    \basec{\[%%%%%%%%%%%%%%%%%%%%%%%%%%%%%%%%%%%%%%%%%%%%%%%
    \bm{B}=\begin{bmatrix}[cc|c]
    \jjj{6} & 0 & 4 \\
    0 & \jjj{9} & 2 \\
    0 & 0 & 0 \\
    0 & 0 & 0 
    \end{bmatrix}
    \]}%%%%%%%%%%%%%%%%%%%%%%%%%%%%%%%%%%%%%%%%%%%%%%%%%%%%%%
    \basec{\[%%%%%%%%%%%%%%%%%%%%%%%%%%%%%%%%%%%%%%%%%%%%%%%
    \rank{\bm{B}} = 2,\quad r = n \to \text{one solution}
    \]}%%%%%%%%%%%%%%%%%%%%%%%%%%%%%%%%%%%%%%%%%%%%%%%%%%%%%%
    
    \item \pts{2} is consistent, but has more equations than unknowns.
    
    \basec{\[%%%%%%%%%%%%%%%%%%%%%%%%%%%%%%%%%%%%%%%%%%%%%%%
    \bm{A}=\begin{bmatrix}[ccc|c]
    6 & \yyy{0} & \yyy{0} & 4 \\
    \xxx{0} & 9 & 0 & 2 \\
    \xxx{0} & 0 & 1 & 0 \\ 
    \xxx{0} & 0 & 0 & 0 \\ 
    \end{bmatrix}
    \]}%%%%%%%%%%%%%%%%%%%%%%%%%%%%%%%%%%%%%%%%%%%%%%%%%%%%%%
    \basec{\[%%%%%%%%%%%%%%%%%%%%%%%%%%%%%%%%%%%%%%%%%%%%%%%
    \xxx{m} > \yyy{n} 
    \]}%%%%%%%%%%%%%%%%%%%%%%%%%%%%%%%%%%%%%%%%%%%%%%%%%%%%%%
    
    \item \pts{2} is inconsistent, but the augmented matrix has a row of zeros.
    
    \basec{\[%%%%%%%%%%%%%%%%%%%%%%%%%%%%%%%%%%%%%%%%%%%%%%%
    \bm{A}=\begin{bmatrix}[cccc|c]
    6 & 0 & 0 & 0 & 3 \\
    0 & 9 & 0 & 0 & 1 \\
    \fff{0} & \fff{0} & \fff{0} & \fff{0} & \fff{4} \\
    \ttt{0} & \ttt{0} & \ttt{0} & \ttt{0} & \ttt{0}\\ 
    \end{bmatrix}
    \]}%%%%%%%%%%%%%%%%%%%%%%%%%%%%%%%%%%%%%%%%%%%%%%%%%%%%%%
  \end{enumerate}

  \newpage

  \item Consider the linear system whose augmented matrix is
  \[%%%%%%%%%%%%%%%%%%%%%%%%%%%%%%%%%%%%%%%%%%%%%%%
  \begin{bmatrix}[cc|c]
  1 & h & 4 \\
  3 & 6 & 8 \\
  \end{bmatrix}
  \]%%%%%%%%%%%%%%%%%%%%%%%%%%%%%%%%%%%%%%%%%%%%%%%
  \begin{enumerate}
    \item Determine a value for \(h\) so that the system is \ttt{consistent}.
    
    \basec{\[%%%%%%%%%%%%%%%%%%%%%%%%%%%%%%%%%%%%%%%%%%%%%%%
    \bbb{h = 1} \to 
    \begin{bmatrix}[cc|c]
      1 & \bbb{1} & 4 \\
      3 & 6 & 8 \\
      \end{bmatrix} \quad  R_2 - 3R_1 \to 
      \begin{bmatrix}[cc|c]
        1 & \bbb{1} & 4 \\
        0 & 3 & -4 \\
      \end{bmatrix} \quad \frac{1}{3}R_2 \to 
      \begin{bmatrix}[cc|c]
        1 & \bbb{1} & 4 \\
        \ttt{0} & \ttt{1} & \ttt{-\frac{4}{3}}~~~\\
      \end{bmatrix}
    \]}%%%%%%%%%%%%%%%%%%%%%%%%%%%%%%%%%%%%%%%%%%%%%%%%%%%%%%
   
    
    \item Determine a value for \(h\) so that the system is \fff{inconsistent}.
    
    \basec{\[%%%%%%%%%%%%%%%%%%%%%%%%%%%%%%%%%%%%%%%%%%%%%%%
    \emph{h = 2} \to 
    \begin{bmatrix}[cc|c]
      1 & \emph{2} & 4 \\
      3 & 6 & 8 \\
      \end{bmatrix} \quad 3R_1 - R_2 \to 
      \begin{bmatrix}[cc|c]
        1 & \emph{2} & 4 \\
        \fff{0 }& \fff{0 }& \fff{-4} \\
        \end{bmatrix}
    \]}%%%%%%%%%%%%%%%%%%%%%%%%%%%%%%%%%%%%%%%%%%%%%%%%%%%%%%

  \end{enumerate}

  \newpage

  \item \pts{2} Show that \(x=1\), \(y=2\), and \(z=3\) is not a solution to the following system.
  \[%%%%%%%%%%%%%%%%%%%%%%%%%%%%%%%%%%%%%%%%%%%%%%%
  \begin{cases}
    x+y+2z=9\\
    2x+4y-2z=1\\
    3x+6y-5z=0
  \end{cases}
  \]%%%%%%%%%%%%%%%%%%%%%%%%%%%%%%%%%%%%%%%%%%%%%%%
  \basec{\begin{align*}
    1 + 2 + 2(3) &= 9 \text{ \true{}} \\
    2(1) + 4(2) - 2(3) &= 4 \text{ \false{}}
  \end{align*}}
  \vspace{-24pt}
  \basec{\[%%%%%%%%%%%%%%%%%%%%%%%%%%%%%%%%%%%%%%%%%%%%%%%
  \fff{1\neq4},~\text{not a solution}
  \]}%%%%%%%%%%%%%%%%%%%%%%%%%%%%%%%%%%%%%%%%%%%%%%%
  
  \item \pts{2} Find a solution to the following system of linear equations: 
  \begin{align*}
    -450x_1 & \quad + & -22x_2 & \quad + & 1x_3 & \quad + & 1x_4 & \quad + & 0x_5 & \quad + & 333x_6 &= 0 \\
    3x_1 & \quad + & 2x_2 & \quad + & 1x_3 & \quad + & 0x_4 & \quad + & 900x_5 & \quad + & 0x_6 &= 0 \\
    -\pi x_1 & \quad + & 0x_2 & \quad + & 88x_3 & \quad + & 45x_4 & \quad + & 1x_5 & \quad + & 0x_6 &= 0 \\
    7x_1 & \quad + & 12x_2 & \quad + & 300x_3 & \quad + & 0x_4 & \quad + & 9x_5 & \quad + & 0x_6 &= 0 \\
    1x_1 & \quad + & 3x_2 & \quad + & 9x_3 & \quad + & 27x_4 & \quad + & 81x_5 & \quad + & 243x_6 &= 0 
  \end{align*}
  
  \basec{\begin{itemize}
    \item Ignoring the trivial solution, here is a nearly trivial solution:
    
    \basec{\[%%%%%%%%%%%%%%%%%%%%%%%%%%%%%%%%%%%%%%%%%%%%%%%
    \begin{bmatrix}[cccccc|c]
     1 & 1 & 1 & 1 & 1 & \frac{470}{333} & 0  \\
     0 & 0 & 0 & 0 & 0 & 0 & 0  \\
     0 & 0 & 0 & 0 & 0 & 0 & 0  \\
     0 & 0 & 0 & 0 & 0 & 0 & 0  \\
     0 & 0 & 0 & 0 & 0 & 0 & 0  
    \end{bmatrix}
    \]}%%%%%%%%%%%%%%%%%%%%%%%%%%%%%%%%%%%%%%%%%%%%%%%%%%%%%%
    \basec{\[%%%%%%%%%%%%%%%%%%%%%%%%%%%%%%%%%%%%%%%%%%%%%%%
    x_1 + x_2 + x_3 + x_4 + x_5 + \tfrac{470}{333}x_6 = 0
    \]}%%%%%%%%%%%%%%%%%%%%%%%%%%%%%%%%%%%%%%%%%%%%%%%%%%%%%%
    
    \minimal{\item Computer go brrrr\dots 
    \[%%%%%%%%%%%%%%%%%%%%%%%%%%%%%%%%%%%%%%%%%%%%%%%
    \begin{bmatrix}[cccccc|c]
      1 & 0 & 0 & 0 & 0 & -3.47 & 0 \\
      0 & 1 & 0 & 0 & 0 & 55.88 & 0  \\
      0 & 0 & 1 & 0 & 0 & -2.15 & 0  \\
      0 & 0 & 0 & 1 & 0 & 3.97 & 0  \\
      0 & 0 & 0 & 0 & 1 & -0.11 & 0  
     \end{bmatrix}
    \]%%%%%%%%%%%%%%%%%%%%%%%%%%%%%%%%%%%%%%%%%%%%%%%
    \[%%%%%%%%%%%%%%%%%%%%%%%%%%%%%%%%%%%%%%%%%%%%%%%
    3.47x_1 - 55.88x_2 + 2.15x_3 + -3.97x_4 + 0.11x_5 + \lambda x_6 = 0 
    \]}%%%%%%%%%%%%%%%%%%%%%%%%%%%%%%%%%%%%%%%%%%%%%%%%%%%%%%
    \vspace{-24pt}
    \begin{center}\color{minimal}
      Wait, is that right? \(\uparrow \)
    \end{center}
  \end{itemize}}
  
  
  \newpage 

  \item \pts{2} Suppose the matrix below is the augmented matrix of a system of linear equations. Write the general solution in parametric vector form (as a linear combination of vectors some scaled by parameters).
  \[%%%%%%%%%%%%%%%%%%%%%%%%%%%%%%%%%%%%%%%%%%%%%%%
  \begin{bmatrix}[ccccc|c]
  1 & 0 & \bbb{-4}  & 0 & \rrr{2} & -1  \\
  0 & 1 & \bbb{8 } & 0 & \rrr{-7} & 9  \\
  0 & 0 & \bbb{0 } & 1 & \rrr{0} & 3  
  \end{bmatrix}
  \]%%%%%%%%%%%%%%%%%%%%%%%%%%%%%%%%%%%%%%%%%%%%%%%
  \basec{\[%%%%%%%%%%%%%%%%%%%%%%%%%%%%%%%%%%%%%%%%%%%%%%%
  \begin{bmatrix} x_1 \\ x_2  \\ \bbb{x_3} \\ x_4 \\ \rrr{x_5} \end{bmatrix}
  = 
  \begin{bmatrix} -1 \\ 9 \\ \bbb{0} \\ 3 \\ \rrr{0} \end{bmatrix}
  + r
  \begin{bmatrix} 4 \\ -8 \\ \bbb{1} \\ 0 \\ \rrr{0} \end{bmatrix}
  + s
  \begin{bmatrix} -2 \\ 7 \\ \bbb{0} \\ 0 \\ \rrr{1} \end{bmatrix}
  \]}%%%%%%%%%%%%%%%%%%%%%%%%%%%%%%%%%%%%%%%%%%%%%%%%%%%%%%
  
  \bigskip
  
  \item \pts{2} Let
  \(%%%%%%%%%%%%%%%%%%%%%%%%%%%%%%%%%%%%%%%%%%%%%%%
  \bm{A} = \begin{bmatrix} 0 & -4 \\ 3 & 9 \end{bmatrix},
  \quad \bm{B} = \begin{bmatrix} -9 & 1  \\ -9 & 8 \end{bmatrix},~\text{and}
  \quad \bm{C} = \begin{bmatrix} 7 & 5 \\ 3 & -3 \end{bmatrix}.
  \)%%%%%%%%%%%%%%%%%%%%%%%%%%%%%%%%%%%%%%%%%%%%%%
  
  \medskip
  
  \hspace{58pt} Determine: \( -8\bm{C} + 2(5\bm{A}-3\bm{B}) + 4\bm{C} - 10\bm{A} + 4(\bm{C}+2\bm{B}) \)

  \medskip

  \basec{\begin{itemize}
    \item Matrix addition is:
      \begin{itemize}
        \item \true{Commutative}: \(\bm{A}+\bm{B} = \bm{B} + \bm{A}\)
        \item \true{Associative}: \(\bm{A} + (\bm{B}+\bm{C}) = (\bm{A}+\bm{B}+ \bm{C})\)
        \item \true{Distributive}: \(\bm{A}(\bm{B}+\bm{C}) = \bm{A}\bm{B}+ \bm{A}\bm{C}\)
      \end{itemize}
    \item Thus:
    \begin{align*}
      \emph{-8\bm{C} + 2(5\bm{A}-3\bm{B}) + 4\bm{C} - 10\bm{A} + 4(\bm{C}+2\bm{B})} &= \\
      -8\bm{C} + 10\bm{A}-6\bm{B} + 4\bm{C} - 10\bm{A} + 4\bm{C}+8\bm{B} &= \\
      10\bm{A}-6\bm{B} - 10\bm{A} +8\bm{B} &= \\
      -6\bm{B} + 8\bm{B} &= \\
      2\bm{B} &= \emph{\begin{bmatrix}
      -18 & 2 \\
      -18 & 16 
      \end{bmatrix}}
    \end{align*}
  \end{itemize}}
  
  \newpage

  \item \pts{2} Fact: the vector equation below is consistent
  \[%%%%%%%%%%%%%%%%%%%%%%%%%%%%%%%%%%%%%%%%%%%%%%%
  2 \begin{bmatrix} 1 \\ 4\end{bmatrix} + 
  3 \begin{bmatrix} 2 \\ -12 \end{bmatrix} -
  5 \begin{bmatrix} 7 \\ 0 \end{bmatrix} + 
  6 \begin{bmatrix} -3 \\ 5 \end{bmatrix} =
  \begin{bmatrix} -45 \\ 2 \end{bmatrix}
  \]%%%%%%%%%%%%%%%%%%%%%%%%%%%%%%%%%%%%%%%%%%%%%%%

  Use that fact to find a solution to the matrix equation \(\bm{Ax} = \bm{b }\) where 
  \[%%%%%%%%%%%%%%%%%%%%%%%%%%%%%%%%%%%%%%%%%%%%%%%
  \bm{A} = \begin{bmatrix}
  1 & 2 & 7 & -3 \\
  4 & -12 & 0 & 5 \\
  \end{bmatrix}
  \qquad \text{and} \qquad
  \bm{b} = \begin{bmatrix} -45 \\  2 \end{bmatrix}
  \]%%%%%%%%%%%%%%%%%%%%%%%%%%%%%%%%%%%%%%%%%%%%%%%
  
  \basec{\[%%%%%%%%%%%%%%%%%%%%%%%%%%%%%%%%%%%%%%%%%%%%%%%
  \operatorname{rref}\left(\begin{bmatrix}[cccc|c]
  1 & 2 & 7 & -3 & -45 \\
  4 & -12 & 0 & 5 & 2 
  \end{bmatrix}\right) \to 
  \begin{bmatrix}[cccc|c]
    1 & 0 & \dfrac{21}{5} & -\dfrac{13}{10} & -\dfrac{134}{5} \\
    &&&& \\
    0 & 1 & \dfrac{7}{5}  & -\dfrac{17}{20}  & -\dfrac{91}{10}
    \end{bmatrix} 
  \]}%%%%%%%%%%%%%%%%%%%%%%%%%%%%%%%%%%%%%%%%%%%%%%%
  \basec{\[%%%%%%%%%%%%%%%%%%%%%%%%%%%%%%%%%%%%%%%%%%%%%%%
  \begin{bmatrix} x_1 \\ x_2 \\ x_3 \\ x_4\end{bmatrix} = 
  \begin{bmatrix} 134/5 \\ -91/10\\ 0 \\ 0\end{bmatrix} + 
  r\begin{bmatrix} -21/5 \\ -7/5 \\ 1 \\ 0\end{bmatrix} +
  s\begin{bmatrix} 13/10 \\ 17/20 \\ 0 \\ 1\end{bmatrix}
  \]}%%%%%%%%%%%%%%%%%%%%%%%%%%%%%%%%%%%%%%%%%%%%%%%%%%%%%%
  
  \item \pts{2} Define a transformation 
  \(T: \mathbb{R}^3 \to \mathbb{R}^4 \qquad \bm{x} \to \bm{Ax}\) \\
  \medskip
  where \(\bm{A} = \begin{bmatrix}
  -5 & -4 & 1 \\
  3 & 2 & -1 \\
  -4 & 0 & 8 \\
  7 & 0 & 9
  \end{bmatrix}\). 
  It is a fact that 
  \(%%%%%%%%%%%%%%%%%%%%%%%%%%%%%%%%%%%%%%%%%%%%%%%
  T\left(\begin{bmatrix} 1 \\ -1 \\ 4 \end{bmatrix}\right) = \begin{bmatrix} 3 \\ -3 \\ 28 \\ 43 \end{bmatrix}
  \)%%%%%%%%%%%%%%%%%%%%%%%%%%%%%%%%%%%%%%%%%%%%%%%
  
  \medskip

  Use this fact to produce a solution to the system of linear equations below
  \begin{align*}
    \hspace{80pt} -5x_1 &\quad+ \hspace{-50pt}& -4x_2 &\quad+ \hspace{-50pt}& 1x_3 &= 3 \\
    \hspace{80pt} 3x_1 &\quad+ \hspace{-50pt}& 2x_2  &\quad+ \hspace{-50pt}& 1x_3 &= -3 \\
    \hspace{80pt} -4x_1 &\quad+ \hspace{-50pt}& 0x_2 &\quad+ \hspace{-50pt}& 3 8x_3 &= 28 \\
    \hspace{80pt} 7x_1 &\quad+ \hspace{-50pt}& 0x_2  &\quad+ \hspace{-50pt}& 9x_3 &= 43 
  \end{align*}
  \vspace{-12pt}
  \basec{\[%%%%%%%%%%%%%%%%%%%%%%%%%%%%%%%%%%%%%%%%%%%%%%%
  \text{If}~\bm{x}= \begin{bmatrix} 1 \\ -1 \\ 4 \end{bmatrix},~
  \text{then}~\bm{Ax} = \begin{bmatrix} 3 \\ -3 \\ 28 \\ 43 \end{bmatrix}
  \]}%%%%%%%%%%%%%%%%%%%%%%%%%%%%%%%%%%%%%%%%%%%%%%%%%%%%%%
  \basec{
  \[%%%%%%%%%%%%%%%%%%%%%%%%%%%%%%%%%%%%%%%%%%%%%%%
  \text{Thus: if}~\bm{B} = \begin{bmatrix}
    -5 & -4 & 1 \\
    3 & 2 & -1 \\
    -4 & 0 & \bbb{38} \\
    7 & 0 & 9
  \end{bmatrix},~\text{then}
  ~\bm{Bx} = \begin{bmatrix} 3 \\ -3 \\ \bbb{148} \\ 43 \end{bmatrix}~\text{is the solution}.
  \]%%%%%%%%%%%%%%%%%%%%%%%%%%%%%%%%%%%%%%%%%%%%%%%
  \begin{center}
    Double-checking by computing RREF of both augment matrices \(\downarrow \)
  \end{center}}
  
  \newpage

  \color{text}

  \[%%%%%%%%%%%%%%%%%%%%%%%%%%%%%%%%%%%%%%%%%%%%%%%
  \operatorname{rref}\left(\begin{bmatrix}[ccc|c]
    -5 & -4 & 1 & 3 \\
    3 & 2 & -1  & -3\\
    -4 & 0 & 8  & 28 \\
    7 & 0 & 9 & 43
    \end{bmatrix}\right) =
    \begin{bmatrix}[ccc|c]
      1 & 0 & 0 & 1  \\   
      0 & 1 & 0 & -1 \\
      0 & 0 & 1 & 4  \\
      0 & 0 & 0 & 0
    \end{bmatrix}
  \]%%%%%%%%%%%%%%%%%%%%%%%%%%%%%%%%%%%%%%%%%%%%%%%
  \[%%%%%%%%%%%%%%%%%%%%%%%%%%%%%%%%%%%%%%%%%%%%%%%
  r = n, \text{thus the matrix is full rank and only one unique solution exists}
  \]%%%%%%%%%%%%%%%%%%%%%%%%%%%%%%%%%%%%%%%%%%%%%%%
  
  \[%%%%%%%%%%%%%%%%%%%%%%%%%%%%%%%%%%%%%%%%%%%%%%%
  \operatorname{rref}\left(\begin{bmatrix}[ccc|c]
    -5 & -4 & 1 & 3 \\
    3 & 2 & -1  & -3\\
    -4 & 0 & 28  & 148 \\
    7 & 0 & 9 & 43
    \end{bmatrix}\right) =
    \begin{bmatrix}[ccc|c]
      1 & 0 & 0 & 1  \\   
      0 & 1 & 0 & -1 \\
      0 & 0 & 1 & 4  \\
      0 & 0 & 0 & 0
    \end{bmatrix}
  \]%%%%%%%%%%%%%%%%%%%%%%%%%%%%%%%%%%%%%%%%%%%%%%%
  \[%%%%%%%%%%%%%%%%%%%%%%%%%%%%%%%%%%%%%%%%%%%%%%%
  r = n, \text{thus the matrix is full rank and only one unique solution exists}
  \]%%%%%%%%%%%%%%%%%%%%%%%%%%%%%%%%%%%%%%%%%%%%%%%

  Does the rank matter here actually? I have not explored linear transformations much, so I'm exploring things here. Hmm, let's create a rank deficient matrix and apply the transformation.
  \[%%%%%%%%%%%%%%%%%%%%%%%%%%%%%%%%%%%%%%%%%%%%%%%
  \begin{bmatrix}[ccc|c]
    -5 & -4 & 1 \\
    3 & 2 & -1  \\
    3 & 2 & -1  
  \end{bmatrix} 
  \begin{bmatrix} 1 \\ -1 \\ 4 \end{bmatrix} =
  \begin{bmatrix} 3 \\ -3 \\ -3 \end{bmatrix}
  \]%%%%%%%%%%%%%%%%%%%%%%%%%%%%%%%%%%%%%%%%%%%%%%%
  This results in infinite solutions:
  \[%%%%%%%%%%%%%%%%%%%%%%%%%%%%%%%%%%%%%%%%%%%%%%%
  \operatorname{rref}\left(\begin{bmatrix}[ccc|c]
    -5 & -4 & 1 \\
    3 & 2 & -1  \\
    3 & 2 & -1  
  \end{bmatrix}\right)
  =
  \begin{bmatrix}[ccc|c]
    1 & 0 & -1 & 3 \\
    0 & 1 & 1 & -3 \\
    0 & 0 & 0 & 0 
  \end{bmatrix} 
  \to 
  \begin{bmatrix} x_1 \\ x_2 \\ x_3 \end{bmatrix} = 
  \begin{bmatrix} 3 \\ -3 \\ 0 \end{bmatrix} + r
  \begin{bmatrix} 1 \\ -1 \\ 1 \end{bmatrix}
  \]%%%%%%%%%%%%%%%%%%%%%%%%%%%%%%%%%%%%%%%%%%%%%%%
  So \bm{x} maps \(\mathbb{R}^2 \to \mathbb{R}^2\) in this case, hmm\dots maybe I'm getting at nothing here. 

\end{enumerate}


\end{document}