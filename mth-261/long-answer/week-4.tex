\documentclass[basic]{inVerba-notes}
\usepackage{inVerba-math}

\newcommand{\userName}{Cullyn Newman}
\newcommand{\class}{MTH:\@ 261}
\newcommand{\theTitle}{Multiplication Again}
\newcommand{\institution}{Portland State University}

\begin{document}
\begin{enumerate}[align=left, leftmargin=0pt, labelindent=\parindent, listparindent=\parindent, labelwidth=0pt, itemindent=!] \color{minor}
  \item For parts (a) and (b) below, let
  \[%%%%%%%%%%%%%%%%%%%%%%%%%%%%%%%%%%%%%%%%%%%%%%%
  \bm{A} =
  \begin{bmatrix}
    1 & 2 & 3 & 4 \\
    1 & 2 & 3 & 4 \\
    1 & 2 & 3 & 4 \\
    1 & 2 & 3 & 4 \\
    1 & 2 & 3 & 4
  \end{bmatrix}
  \]%%%%%%%%%%%%%%%%%%%%%%%%%%%%%%%%%%%%%%%%%%%%%%%
  \begin{enumerate}
    \item Common notation for the vector with a 1 in the third entry and zeros elsewhere (with the total number of entries given via context) is \tbm{e_3}. That is, in this case
    \[\bm{e}_3 = 
    \begin{bmatrix}
      0 \\
      0 \\
      1 \\
      0  
    \end{bmatrix}
    \]
    Calculate \tbm{Ae_3}
    \basec{\[%%%%%%%%%%%%%%%%%%%%%%%%%%%%%%%%%%%%%%%%%%%%%%%
    \bm{Ae_3} = \begin{bmatrix} 3 \\ 3 \\ 3 \\ 3 \end{bmatrix}
    \]}%%%%%%%%%%%%%%%%%%%%%%%%%%%%%%%%%%%%%%%%%%%%%%%
    
    \item Calculate \tbm{Ae_4}
    \basec{\[%%%%%%%%%%%%%%%%%%%%%%%%%%%%%%%%%%%%%%%%%%%%%%%
    \bm{Ae_4} = \begin{bmatrix} 4 \\ 4 \\ 4 \\ 4 \end{bmatrix}
    \]}%%%%%%%%%%%%%%%%%%%%%%%%%%%%%%%%%%%%%%%%%%%%%%%

    \item Write a sentence that explains what happens when you multiply an arbitrary matrix \tbm{B} by \tbm{e_j} where \tbm{B} and \tbm{e_j} have compatible sizes.
    \basec{\begin{itemize}
      \item Multiplication of a matrix \tbm{B} with a vector \tbm{e} where the number of columns in \tbm{B} is equal to the number of rows in \tbm{e} results in the weighted combination of columns in \tbm{B}.
      \item In the case of standard basis vectors, where \tbm{e_j} denotes the vector with a 1 in the \(j\)-th row and 0s elsewhere, then the result is simply the \(j\)-th column of \tbm{B}.
    \end{itemize}}
  \end{enumerate}

  \item In the parts below, let 
  \[ \bm{A} =
  \begin{bmatrix}
  1 & 2 & 3 & 4 \\
  1 & 2 & 3 & 4 \\
  1 & 2 & 3 & 4 \\
  1 & 2 & 3 & 4 
  \end{bmatrix} \qquad \text{and} \qquad
  \bm{B} =
  \begin{bmatrix*}[r]
  1 & 0 & 0 & 0 \\
  0 & 1 & 0 & 0 \\
  0 & 0 & -3 & 0 \\
  0 & 0 & 0 & 1 
  \end{bmatrix*}
  \]
  \newpage
  \begin{enumerate}
    \item Calculate \tbm{AB}. Write what you observed in a sentence.
    \basec{\[%%%%%%%%%%%%%%%%%%%%%%%%%%%%%%%%%%%%%%%%%%%%%%%
    \bm{A\rrr{B}} =
    \begin{bmatrix}
    1 & 2 & \yyy{-9} & 4 \\
    1 & 2 & \yyy{-9} & 4 \\
    1 & 2 & \yyy{-9} & 4 \\
    1 & 2 & \yyy{-9} & 4 
    \end{bmatrix}
    \]}%%%%%%%%%%%%%%%%%%%%%%%%%%%%%%%%%%%%%%%%%%%%%%%
    \basec{\begin{itemize}
      \item The row of the matrix \tbm{B} with the scaled basis vector (\(1 \to -3\)) results in the scaling of the column of the original matrix \tbm{A} by that same amount (\(3\to -9\)).
      \item In general, \rrr{post-multiplication} of a diagonal matrix (all elements outside the main diagonal are zero) results in the \textbf{scaling of the \yyy{columns}} of the original matrix.
    \end{itemize}}
    \item Calculate \tbm{BA}. Write what you observed in a sentence.
    \basec{\[%%%%%%%%%%%%%%%%%%%%%%%%%%%%%%%%%%%%%%%%%%%%%%%
    \bm{\bbb{B}A} =
    \begin{bmatrix}
    1 & 2 & 3 & 4 \\
    1 & 2 & 3 & 4 \\
    \xxx{-3 }& \xxx{-6 }& \xxx{-9 }& \xxx{-12} \\
    1 & 2 & 3 & 4 
    \end{bmatrix}
    \]}%%%%%%%%%%%%%%%%%%%%%%%%%%%%%%%%%%%%%%%%%%%%%%%
    \basec{\begin{itemize}
      \item Similar to (a), but the \bbb{pre-multiplication} of the diagonal matrix results in the \textbf{scaling of the \xxx{rows}} of the original matrix.
    \end{itemize}}
    \item Use what you learned in (b) to create a \(4\times4\) matrix \tbm{C} that would
    \begin{enumerate}
      \item zero out the first row
      \item multiply the second row by -1
      \item multiply the third row by 5
      \item leave the last row the same
    \end{enumerate}
    when you pre-multiply a size-compatible matrix \tbm{M} by \tbm{C}.
    \basec{\[%%%%%%%%%%%%%%%%%%%%%%%%%%%%%%%%%%%%%%%%%%%%%%%
    \bm{C} = \begin{bmatrix}
    \emph{0} & 0 & 0 & 0 \\
    0 & \emph{-1} & 0 & 0 \\
    0 & 0 & \emph{5} & 0 \\
    0 & 0 & 0 & \emph{1}
    \end{bmatrix} \quad 
    \bm{M} = 
    \begin{bmatrix}
    a & a & a & a \\
    b & b & b & b \\
    c & c & c & c \\
    d & d & d & d 
    \end{bmatrix}
    \]}%%%%%%%%%%%%%%%%%%%%%%%%%%%%%%%%%%%%%%%%%%%%%%%
    \basec{\[%%%%%%%%%%%%%%%%%%%%%%%%%%%%%%%%%%%%%%%%%%%%%%%
    \bm{CM} = 
    \begin{bmatrix}
    \emph{0}a & \emph{0}a & \emph{0}a & \emph{0}a \\
    \emph{-1}b & \emph{-1}b & \emph{-1}b & \emph{-1}b \\
    \emph{5}c & \emph{5}c & \emph{5}c & \emph{5}c \\
    \emph{1}d & \emph{1}d & \emph{1}d & \emph{1}d 
    \end{bmatrix}
    \]}%%%%%%%%%%%%%%%%%%%%%%%%%%%%%%%%%%%%%%%%%%%%%%%
    
  \end{enumerate}
\end{enumerate}
\end{document}