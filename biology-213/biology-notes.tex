\documentclass[12pt,a4paper]{article}
\usepackage{inverba}
\newcommand{\userName}{Cullyn Newman} 
\newcommand{\class}{BI 216} 
\newcommand{\institution}{Portland State University} 
\newcommand{\thetitle}{\hypertarget{home}{Biology Notes}}
\rfoot{\hyperlink{home}{\thepage}}

\begin{document}
%%%%%%%%%%%%%%%%%%%%%%%%%%%%%%%%%%%%%%%%%%%%%%%%%%%%%%%%%%%%%%%%%%%%%
\tableofcontents
\cleardoublepage
\fancyhead{}
\fancyhead[R]{\hyperlink{home}{\nouppercase\leftmark}}
%%%%%%%%%%%%%%%%%%%%%%%%%%%%%%%%%%%%%%%%%%%%%%%%%%%%%%%%%%%%%%%%%%%%%

%%%%%%%%%%%%%%%%%%%%%%%%%%%%% Chapter 43 %%%%%%%%%%%%%%%%%%%%%%%%%%%%
%\begingroup
\clearpage
\setcounter{section}{43}
\section{Animal Sensory Systems}
\subsection{How Do Sensory Organs Convey Information to the Brain?}
\begin{itemize}
    \item \textbf{Transduction}: conversion of an external stimulus to an internal signal in the form of action potentials along sensory neurons.
    \begin{itemize}
        \item Requires a sensory receptor:
        \begin{enumerate}
            \item \textit{Mechanoreceptors}: respond to changes in pressure.
            \item \textit{Photoreceptors}: respond to particular wavelengths of light.
            \item \textit{Chemoreceptors}: detect specific molecules.
            \item \textit{Thermoreceptors}: respond to changes in temperature.
            \item \textit{Nociceptors}: sense harmful stimuli such as tissue injury.
            \item \textit{Electroreceptors}: detect electric fields.
            \item \textit{Magnetoreceptors}: detect magnetic fields.
        \end{enumerate}
        \item Frequency of action potential firing rate can indicate the intensity of the stimulus.
    \end{itemize}
    \item \textbf{Transmission}: the process of sending the signal to the central nervous system.
\end{itemize}
\subsection{Mechanoreception: Sensing Pressure Changes}
\begin{itemize}
     \item \textbf{Statocyst}: an fluid filled organ that grabs use to help sense pressure created by gravity.
    \item Direct physical pressure causes ion channels to open and close, creating voltage gated action potentials.
    \item \textbf{The Mammalian Ear}:
    \begin{itemize}
        \item \textit{Outer ear}: collects incoming pressure waves and funnels them into tube known as the ear canal, which leads to the \textbf{tympanic membrane}, or eardrum.
        \item \textit{Middle ear}: contains three tiny bones that the eardrum passes vibrations to in order to amplify sound. One of the bones, called \textbf{stapes}, vibrates against a membrane called the \textbf{oval window}, which separates the middle ear from the inner ear.
        \item \textit{Inner ear}: the oval window oscillates in response to vibrations which generates waves in a chamber known as the \textbf{cochlea}. These waves are pressure inputs the hair cells respond to.
    \end{itemize}
    \item Hair cells, forming rows that sit in the middle chamber, are embedded in a tissue that sits atop the \textbf{basilar membrane}. In addition, the hair cells' stereocilia touch another smaller surface called the \textbf{tectorial membrane.}
    \item This sandwiching of hair cells produce very specific responses to various frequencies, allowing us to distinguish between them. 
    \item \textbf{Lateral line system}: a mechanoreceptor organ that most fish and larval amphibians use. Consists of embedded gel-like domed structures called cupulae that lay inside canals along the length of the body. 
\end{itemize}

\subsection{Photoreception: Sensing Light}
\begin{itemize}
    \item \textbf{The Insect Eye}:
        \begin{itemize}
            \item \textbf{Compound eyes}: eyes composed of hundreds of thousands of light-sensing columns called \textbf{ommatidia}.
            \item Each ommatidium has lens that focuses light into a smaller number of receptor cells--usually four.
        \end{itemize}
    \item \textbf{The Vertebrate Eye}:
        \begin{itemize}
            \item \textbf{Simple eye}: a structure with a single lens that focuses incoming light onto a layer of many receptor cells.
            \item \textbf{Structure of the Vertebrate Eye}:
                \begin{itemize}
                    \item \textit{white of the eye}: outermost layer of the eye that consists of tough rind of white tissue called the \textit{sclera}.
                    \item \textit{Cornea}: a transparent sheet of connective tissue on the front of the sclera.
                    \item \textit{Iris}: pigmented, round muscle just inside the cornea that can contract or expand to control the amount of light entering the eye.
                    \item \textit{Pupil}: hole in the center of the iris.
                    \item \textit{Lens}: works with the cornea to focus incoming light.
                    \item \textit{Retina}: a layer of photoreceptors and several layers of neurons.
                \end{itemize}
            \item The photoreceptors are held in place by the pigmented epithelium.
            \item photoreceptors synapse with an intermediate layer of connecting neurons called \textbf{bipolar cells}.
            \item Bipolar cells synapse with neurons called \textbf{ganglion cells}, which form the innermost layer of the retina.
            \item The axons of the ganglion cells project to the brian via optic nerve.
            \item Photoreceptors come in two distinct types:
                \begin{itemize}
                    \item \textbf{Rods}: sensitive to dim light but not to color.
                    \item \textbf{Cones}: sensitive to different wavelengths of light, but not dim light.
                \end{itemize}
            \item \textbf{Opsin}: a transparent membrane protein that associates with a molecule of pigment in the \textbf{retinal}.
            \item \textbf{Rhodopsin}: a two molecule complex similar ot opsin in rod cells. 
        \end{itemize}
\end{itemize}

\subsection{Chemoreception: Sensing Chemicals}
\begin{itemize}
    \item Chemoreception occurs when chemicals bind to chemoreceptors, initiating action potentials in sensory neurons.
    \item \textbf{Gustation}: the sense of taste.
    \item \textbf{Olfaction}: sense of smell.
    \item \textbf{Taste buds}: clustered structures containing about 100 spindle shaped taste cells that make synapses with sensory neurons.
    \item \textit{Salty}: due to sodium ions dissolved in food.
    \item \textit{Sourness}: due to presence of protons.
    \item \textit{Umami}: due to monosodium glutamate. 
    \item \textbf{Odorants}: airborne molecules.
    \item \textbf{Olfactory bulb}: part of the brain where olfactory signals are processed and interpreted.
    \item \textbf{Pheromone}: a chemical that is secreted into the environment that affects the behavior or physiology of animals.
\end{itemize}

\subsection{Other Sensory Systems}
\begin{itemize}
    \item Thermoreceptors are located in the central nervous system and also commonly found skin and other outer surfaces of animals. 
    \item \textit{nociceptor}: senses extreme temperatures as well as other painful stimuli produced by chemicals, excessive pressure, and tissue damage.
    \item \textbf{Ampullae of Lorenzini}: tiny pores scattered across a shark's head contain structures that are responsible for their electroreception. Sharks can detect electrical potentials as small as a nanovolt.
    \item \textit{Electrogenic fishes} have specialized organs near their tails that generate electric fields stronger than those of regular nerves or muscles. 
    \item \textit{Magnetoreception}: seen in many organisms, including fungi, invertebrates, and all other vertebrate classes.
\end{itemize}

%\endgroup
%%%%%%%%%%%%%%%%%%%%%%%%%%%%% Chapter 43 %%%%%%%%%%%%%%%%%%%%%%%%%%%%

%%%%%%%%%%%%%%%%%%%%%%%%%%%%% Chapter 44 %%%%%%%%%%%%%%%%%%%%%%%%%%%%
%\begingroup
\clearpage
\section{Animal Movement}
\subsection{How Do Muscles Contract?}
\begin{itemize}
    \item \textbf{Muscle fiber}: a muscle fiber is a long, thin muscle cell. 
    \item Within each muscle cell are many threadlike, contractile structures called \textbf{myofibrils}. 
    \item Myofibrils often look striped or striated due to alternating light and dark units called \textbf{sarcomeres}, which repeat along the length of a myofibril.
    \item The question of how muscles contract simplifies to how sarcomeres shorten.
    \item \textbf{Thin filaments} are composed of actin.
    \item \textbf{Thick filaments} are composed of myosin.
    \item \textbf{Sliding-filament model}: filaments slide past one another during contraction, with the sarcomere shorting with no change between the lengths of thick and thin filaments themselves. 
    \item ATP is required for myosin to release from actin once the two molecules have bound to each other.
    \item \textbf{Tropomyosin} and \textbf{Tropnin} work together to block the myosin binding sites on actin. When the sites are blocked, the myosin-actin interaction cannot occur, relaxing the muscle. 
\end{itemize}

\subsection{Muscle Tissues}
\begin{itemize}
    \item 
\end{itemize}

%\endgroup
%%%%%%%%%%%%%%%%%%%%%%%%%%%%% Chapter 44 %%%%%%%%%%%%%%%%%%%%%%%%%%%%
\end{document}