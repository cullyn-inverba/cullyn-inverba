\documentclass[12pt,letterpaper]{article}
\usepackage{inverba}
\newcommand{\userName}{Cullyn Newman} 
\newcommand{\class}{BI 213} 
\newcommand{\institution}{Portland State University} 
\newcommand{\thetitle}{\hypertarget{home}{How Animals Work}}
\rfoot{\hyperlink{home}{\thepage}}

\begin{document}

%%%%%%%%%%%%%%%%%%%%%%%%%%%%%%%%%%%%%%%%%%%%%%%%%%%%%%%%%%%%%%%%%%%%%%%%%%%%%%%%%%%%%%%%%
%                               %   %   %   %   %   %   %                               %
%                           %   %   %   %   %   %   %   %   %                           %
%                       %   %                               %   %                       %
%   %   %   %   %   %   %   %   O   U   T   L   I   N   E   %   %   %   %   %   %   %   %
%                       %   %                               %   %                       %
%                           %   %   %   %   %   %   %   %   %                           %
%                               %   %   %   %   %   %   %                               %
%%%%%%%%%%%%%%%%%%%%%%%%%%%%%%%%%%%%%%%%%%%%%%%%%%%%%%%%%%%%%%%%%%%%%%%%%%%%%%%%%%%%%%%%%
%\begingroup

\begin{chapbox}{Unit 7: How Animals Work}{ 
\begin{enumerate}[font=\bfseries, wide]
    \setcounter{enumi}{38}
\item \hyperlink{39}{\textbf{Animal Form and Function}}
    \begin{itemize}
        \item \hyperlink{39.1}{Form, Function, and Adaptation}
        \item \hyperlink{39.2}{Tissues, Organs, and Systems}
        \item \hyperlink{39.3}{How Does Body Size Affect Animal Physiology?}
        \item \hyperlink{39.4}{Homeostasis}
        \item \hyperlink{39.5}{Thermoregulation: A Closer Look}
    \end{itemize}
\item \hyperlink{40}{\textbf{Water and Electrolye Balance}}
    \begin{itemize}
        \item \hyperlink{40.1}{Osmoregulation and Exertion}
        \item \hyperlink{40.2}{Water and Electrolyte in Marine Fishes}
        \item \hyperlink{40.3}{Water and Electrolyte in Freshwater Fishes}
        \item \hyperlink{40.5}{Water and Electrolyte in Vertebrates}
    \end{itemize}
\item \hyperlink{41}{\textbf{Nutrition}}
    \begin{itemize}
        \item \hyperlink{41.1}{Nutritional Requirements}
        \item \hyperlink{41.3}{How Are Nutrients Digested and Absorbed?}
        \item \hyperlink{41.4}{Nutritional Homeostasis}
    \end{itemize}
\item \hyperlink{42}{\textbf{Gas Exchange}}
    \begin{itemize}
        \item 
    \end{itemize}
\item \hyperlink{43}{\textbf{Nervous System}}
    \begin{itemize}
        \item 
    \end{itemize}
\item \hyperlink{44}{\textbf{Sensory Systems}}
    \begin{itemize}
        \item 
    \end{itemize}
\item \hyperlink{45}{\textbf{Animal Movement}}
    \begin{itemize}
        \item 
    \end{itemize}
\item \hyperlink{46}{\textbf{Chemical Signals}}
    \begin{itemize}
        \item 
    \end{itemize}
\end{enumerate}
}\end{chapbox}

%\endgroup
%%%%%%%%%%%%%%%%%%%%%%%%%%%%%%%%%%%%%%%%%%%%%%%%%%%%%%%%%%%%%%%%%%%%%%%%%%%%%%%%%%%%%%%%%
%                               %   %   %   %   %   %   %                               %
%                           %   %   %   %   %   %   %   %   %                           %
%                       %   %                               %   %                       %
%   %   %   %   %   %   %   %       N   O   T   E   S       %   %   %   %   %   %   %   %
%                       %   %                               %   %                       %
%                           %   %   %   %   %   %   %   %   %                           %
%                               %   %   %   %   %   %   %                               %
%%%%%%%%%%%%%%%%%%%%%%%%%%%%%%%%%%%%%%%%%%%%%%%%%%%%%%%%%%%%%%%%%%%%%%%%%%%%%%%%%%%%%%%%%



%%%%%%%%%%%%%%%%%%%%%%%%%%%%%%%%%%%%%%%%%%%%%%%%%%%%%%%%%%%%%%%%%%%%%%%%%%%%%%%%%%%%%%%%%
%  vvvvvvvvvvvvvvvvvvvvvvvvvvvvvvvvvv   Chapter 39   vvvvvvvvvvvvvvvvvvvvvvvvvvvvvvvvv  %
%\begingroup
\clearpage

\renewcommand{\thetitle}{\hypertarget{39}{Form, Function, and Adaptation}}
\rfoot{\hyperlink{40}{39 --- \thepage}}
\hypertarget{39}{} 

\begin{chapbox}{\hyperlink{home}{Chapter 39: Form, Function, and Adaptation}}
    \begin{enumerate}
        \item \hyperlink{39.1}{Form, Function, and Adaptation}
        \item \hyperlink{39.2}{Tissues, Organs, and Systems}
        \item \hyperlink{39.3}{How Does Body Size Affect Animal Physiology?}
        \item \hyperlink{39.4}{Homeostasis}
        \item \hyperlink{39.5}{Thermoregulation: A Closer Look}
    \end{enumerate}
\end{chapbox}

\hypertarget{39.1}{}
\begin{secbox}{\hyperlink{39}{Form, Function, and Adaptation}}{
    \begin{itemize}
        \item\textbf{Anatomy}
        \item\textbf{Physiology}
        \item\textbf{Adaptations}
        \item\textbf{Trade-off}
        \item\textbf{Spermatophore}
        \item\textbf{Acclimatization}
    \end{itemize}    
}\end{secbox}

\hypertarget{39.2}{}
\begin{secbox}{\hyperlink{39}{Tissues, Organs, and Systems}}{
    \begin{itemize}
        \item \textbf{Tissue}
        \item \textbf{Connective tissue}
        \begin{itemize}
            \item \textbf{Loose connective tissue }
            \item \textbf{Dense connective tissue}
            \item \textbf{Supporting connective tissue}
            \item \textbf{Fluid connective tissue }
        \end{itemize}
        \item \textbf{Nervous tissue}
        \begin{itemize}
            \item \textbf{Neurons}
            \item \textbf{Dendrites}
            \item \textbf{Axon}
        \end{itemize}
        \item \textbf{Muscle tissue}
        \begin{itemize}
            \item \textbf{Skeletal muscle}
            \item \textbf{Cardiac muscle}
            \item \textbf{Smooth muscle }
        \end{itemize}
        \item \textbf{Epithelial tissues }
        \begin{itemize}
            \item \textbf{Organ}
            \item \textbf{Gland}
            \item \textbf{Apical }
            \item \textbf{Basolateral}
            \item \textbf{Basal lamina}
        \end{itemize}
        \item \textbf{Organ system }
    \end{itemize}
}\end{secbox}

\hypertarget{39.3}{}
\begin{secbox}{\hyperlink{39}{How Does Body Size Affect Animal Physiology?}}{
    \begin{itemize}
        \item \textbf{Surface area vs. volume theory}
        \item \textbf{Metabolic rate}
        \item \textbf{Basal metabolic rate (BMR)}
        \item \textbf{Gills}
        \item \textbf{Adaptations for increased surface area}
        \begin{itemize}
            \item \textbf{Flattening}
            \item \textbf{Folding}
            \item \textbf{Branching}
        \end{itemize}
    \end{itemize}
}\end{secbox}

\hypertarget{39.4}{}
\begin{secbox}{\hyperlink{39}{Homeostasis}}{
    \begin{itemize}
        \item \textbf{Homeostasis}
        \item \textbf{Regulate}
        \item \textbf{Conform}
        \item \textbf{Set point}
        \item \textbf{Homeostatic system:}
        \begin{itemize}
            \item \textbf{Sensor }
            \item \textbf{Integrator}
            \item \textbf{Effector}
        \end{itemize}
        \item \textbf{Hypothalamus}
        \item \textbf{Negative feedback }
    \end{itemize}
}\end{secbox}

\hypertarget{39.5}{}
\begin{secbox}{\hyperlink{39}{Thermoregulation: A Closer Look}}{
    \begin{itemize}
        \item \textbf{Thermoregulate}
        \item \textbf{Thermoregulatory strategies:}
        \begin{itemize}
            \item \textbf{Endotherm }
            \item \textbf{Ectotherm}
            \item \textbf{Homeotherms}
            \item \textbf{Poikilotherms}
            \item \textbf{Torpor}
            \item \textbf{Hibernation}
        \end{itemize}
        \item \textbf{Countercurrent exchanger}
    \end{itemize}

    \begin{probbox}{Chpater 39: Review}\end{probbox}
    \hyperlink{39.1}{Form, Function, and Adaptation}
    \begin{itemize}
            \item Animal structures and their functions represent adaptations, which are heritable traits that improve survival and reproduction in a certain environment.
            \item Adaptations involve trade-offs, or inescapable compromises between traits.
            \item Acclimatization is a reversible response to the environment that improves physiological function in that environment.
        \end{itemize}
    \hyperlink{39.2}{Tissues, Organs, and Systems}
    \begin{itemize}
            \item Animal cells with a common function are grouped together into
            four general types of tissue: connective tissue, nervous tissue, muscle tissue, and epithelial tissue.
            \item Organs are structures that are composed of two or more tissues that together perform specific tasks.
            \item Organ systems comprise organs that work together in an integrated fashion to perform one or more functions.
    \end{itemize}
    \hyperlink{39.3}{How Does Body Size Affect Animal Physiology?}
    \begin{itemize}
            \item Large animals have smaller surface area/volume ratios than small animals. As animals grow, their volume increases more rapidly than their surface area. 
            \item Large animals have low mass-specific metabolic rates, in keeping with their relatively small surface area for exchanging the oxygen and nutrients required to support metabolism and the wastes and heat produced by metabolism.
            \item The relatively high surface area of small animals means that they lose heat extremely rapidly.
    \end{itemize}
    \hyperlink{39.4}{Homeostasis}
    \begin{itemize}
            \item Homeostasis refers to relatively constant physical and chemical conditions inside the body. 
            \item Homeostasis in a fluctuating environment is usually achieved by regulation. 
            \item Animals have set points, or target values, for various body parameters. When a parameter is not at its set point, negative feedback occurs. Responses to negative feedback return the parameter to the set point and result in homeostasis. 
            \item Most animals have a set point for body temperature. If an individual overheats, it may pant, sweat, or seek a cool environment; if an individual is cold, it may shiver, bask in sunlight, or fluff its fur or feathers.
        \end{itemize}
    \hyperlink{39.5}{Thermoregulation: A Closer Look}
    \begin{itemize}
            \item Animals vary from endothermic to ectothermic and from homeothermic to poikilothermic. 
            \item Endotherms can be active in cold environments but must obtain a lot of energy to fuel their metabolism. Ectotherms do not require as much energy, but their activity depends on environmental temperature. 
            \item Countercurrent heat exchangers have vessels in close contact that carry warm and cool fluids in opposite directions.
    \end{itemize}        
 }\end{secbox}   
%\endgroup
%  ^^^^^^^^^^^^^^^^^^^^^^^^^^^^^^^^^^   Chapter 39   ^^^^^^^^^^^^^^^^^^^^^^^^^^^^^^^^^^ %   
%%%%%%%%%%%%%%%%%%%%%%%%%%%%%%%%%%%%%%%%%%%%%%%%%%%%%%%%%%%%%%%%%%%%%%%%%%%%%%%%%%%%%%%%%

%%%%%%%%%%%%%%%%%%%%%%%%%%%%%%%%%%%%%%%%%%%%%%%%%%%%%%%%%%%%%%%%%%%%%%%%%%%%%%%%%%%%%%%%%
%  vvvvvvvvvvvvvvvvvvvvvvvvvvvvvvvvvv   Chapter 40   vvvvvvvvvvvvvvvvvvvvvvvvvvvvvvvvvv %
%\begingroup

\clearpage

\renewcommand{\thetitle}{\hypertarget{40}{Water and Electrolye Balance}}
\rfoot{\hyperlink{40}{40 --- \thepage}}
\hypertarget{40}{}
\setcounter{section}{40}

\begin{chapbox}{\hyperlink{home}{Chapter 40: Water and Electrolyte}}
    \begin{enumerate}
        \item \hyperlink{40.1}{Osmoregulation and Exertion}
        \item \hyperlink{40.2}{Water and Electrolyte in Marine Fishes}
        \item \hyperlink{40.3}{Water and Electrolyte in Freshwater Fishes}
        \item [5.] \hyperlink{40.5}{Water and Electrolyte in Vertebrates}
    \end{enumerate}
\end{chapbox}

\hypertarget{40.1}{}
\begin{secbox}{\hyperlink{40}{Osmoregulation and Exertion}}{
    \begin{itemize}
        \item \textbf{Electrolyte}
        \item \textbf{Diffusion}
        \item \textbf{Osmosis}
        \item \textbf{Osmolarity}
        \item \textbf{Osmoregulation }
        \item \textbf{Osmoconformers}
        \item \textbf{Isosmotic}
        \item \textbf{osmoregulators}
        \item \textbf{Hyperosmotic }
        \item \textbf{Hyposmotic}
        \item \textbf{Aquaporins}
        \item \textbf{Ammonia}
        \item \textbf{Urea}
        \item \textbf{Uric acid}
    \end{itemize}
}\end{secbox}

\hypertarget{40.2}{}
\begin{secbox}{\hyperlink{40}{Water and Electrolyte in Marine Fish}}{
    \begin{itemize}
        \item \textbf{Rectal gland}
        \item \textbf{Ouabain }
        \item \textbf{Interstitial fluid}
    \end{itemize}
}\end{secbox}

\hypertarget{40.3}{}
\begin{secbox}{\hyperlink{40}{Water and Electrolyte in Freshwater Fish}}{
    \begin{itemize}
        \item \textbf{Osmoregulatory cells may be in different locations}
        \item \textbf{Different forms of Na\(^+\)/K\(^+\)-ATPase may be activated}
        \item \textbf{The orientation of key transport proteins “flips.”}
    \end{itemize}
}\end{secbox}

\hypertarget{40.5}{}
\begin{secbox}{\hyperlink{40}{Water and Electrolyte in Vertebrates}}{
    \begin{itemize}
        \item \textbf{Kidney}
        \item \textbf{Ureter}
        \item \textbf{Bladder}
        \item \textbf{Urethra}
        \item \textbf{Nephron}
        \item \textbf{Cortex}
        \item \textbf{Medulla}
        \item \textbf{The Mammalian Kidney:}
        \item \textbf{Filtration}
        \item \textbf{Proximal tubule}
        \item \textbf{Microvilli}
        \item \textbf{Ion and Water Movement:}
        \item \textbf{Loop of Henle}
        \item \textbf{A Comprehensive View of the Loop of Henle:}
        \item \textbf{Vasa recta}
        \item \textbf{Distal tubule}
        \item \textbf{Collecting duct}
        \item \textbf{Hormones Involved:}
        \item \textbf{Cloaca}
    \end{itemize}
    \begin{probbox}{Chaper 40: Review}\end{probbox}
        \hyperlink{40.1}{Osmoregulation and Exertion}
        \begin{itemize}
            \item Solutes move across membranes via passive transport, facilitated diffusion, or active transport. Water moves across membranes by osmosis. 
            \item In most animals, epithelial cells that selectively transport water and electrolytes are responsible for homeostasis. 
            \item The mechanisms involved in regulating water and electrolyte balance vary widely among animal groups because different habitats present different types of osmotic stress.
            \item The type of nitrogenous waste excreted by an animal is affected by its phylogeny and its habitat type. Most fishes excrete ammonia; mammals and most adult amphibians excrete urea; and insects and reptiles excrete uric acid.
        \end{itemize}
        \hyperlink{40.2}{Water and Electrolyte in Marine Fishes}
        \begin{itemize}
            \item Seawater is strongly hyperosmotic to the tissues of marine bony fishes, so they tend to lose water by osmosis and gain electrolytes by diffusion. 
            \item Marine bony fishes are osmoregulators, whereas cartilaginous fishes including sharks are osmoconformers. 
            \item Epithelial cells in the shark rectal gland and in the gills of marine bony fishes excrete excess salt using Na\(^+\)/K\(^+\)­ATPase and Na\(^+\)/Cl\(^+\)/K\(^+\) cotransporters located in the basolateral membrane. 
            \item Similar salt­excreting cells also exist in the salt glands of marine birds and other reptiles and in the kidneys of mammals.
        \end{itemize}
        \hyperlink{40.3}{Water and Electrolyte in Freshwater Fishes}
        \begin{itemize}
            \item Freshwater is strongly hyposmotic to the blood of freshwater fishes, so they tend to gain water by osmosis and lose electrolytes by diffusion. 
            \item Epithelial cells in the gills of freshwater fishes import ions using Na\(^+\)/K\(^+\)­ATPase located in the basolateral membrane and Na\(^+\)/Cl\(^-\)/K\(^+\) cotransporters located in the apical membrane.
        \end{itemize}
        [5.] \hyperlink{40.5}{Water and Electrolyte in Vertebrates} 
        \begin{itemize}
            \item Nephrons in the vertebrate kidney form a filtrate in the renal corpuscle and then reabsorb valuable nutrients, electrolytes, and water in the proximal tubule. 
            \item A solution containing urea and electrolytes flows through the loop of Henle of mammalian kidneys, where changes in the permeability of epithelial cells to water and salt—along with active transport of salt—create a steep osmotic gradient. 
            \item Antidiuretic hormone increases the water permeability of the collecting duct, causing water to be reabsorbed along the osmotic gradient and hyperosmotic urine to be produced. 
            \item The nephrons of fishes, amphibians, and non­avian reptiles do not have loops of Henle and therefore cannot produce urine that is hyperosmotic to the body fluids. However, some of these vertebrates can produce hyperosmotic urine by reabsorbing water from the cloaca or bladder.
        \end{itemize}    
}\end{secbox}

%\endgroup
%  ^^^^^^^^^^^^^^^^^^^^^^^^^^^^^^^^^^   Chapter 40   ^^^^^^^^^^^^^^^^^^^^^^^^^^^^^^^^^^ %  
%%%%%%%%%%%%%%%%%%%%%%%%%%%%%%%%%%%%%%%%%%%%%%%%%%%%%%%%%%%%%%%%%%%%%%%%%%%%%%%%%%%%%%%%%

%%%%%%%%%%%%%%%%%%%%%%%%%%%%%%%%%%%%%%%%%%%%%%%%%%%%%%%%%%%%%%%%%%%%%%%%%%%%%%%%%%%%%%%%%
%  vvvvvvvvvvvvvvvvvvvvvvvvvvvvvvvvvv   Chapter 41   vvvvvvvvvvvvvvvvvvvvvvvvvvvvvvvvvv %
%\begingroup

\clearpage

\renewcommand{\thetitle}{\hypertarget{41}{Nutrition}}
\rfoot{\hyperlink{41}{41 --- \thepage}}
\hypertarget{41}{}
\setcounter{section}{41}

\begin{chapbox}{\hyperlink{home}{Chapter 41: Nutrition}}
    \begin{enumerate}
        \item \hyperlink{41.1}{Osmoregulation and Exertion}
        \item [3.]\hyperlink{41.3}{Water and Electrolyte in Marine Fishes}
        \item \hyperlink{41.4}{Water and Electrolyte in Freshwater Fishes}
    \end{enumerate}
\end{chapbox}

\hypertarget{41.1}{}
\begin{secbox}{\hyperlink{41}{Nutritional Requirements}}{

}\end{secbox}

\hypertarget{41.3}{}
\begin{secbox}{\hyperlink{41}{How are Nutrients Digested and Absorbed}}{

}\end{secbox}

\hypertarget{41.4}{}
\begin{secbox}{\hyperlink{41}{Nutritional Homeostasis}}{

}\end{secbox}

%\endgroup
%  ^^^^^^^^^^^^^^^^^^^^^^^^^^^^^^^^^^   Chapter 41   ^^^^^^^^^^^^^^^^^^^^^^^^^^^^^^^^^^ %  
%%%%%%%%%%%%%%%%%%%%%%%%%%%%%%%%%%%%%%%%%%%%%%%%%%%%%%%%%%%%%%%%%%%%%%%%%%%%%%%%%%%%%%%%%
\end{document}