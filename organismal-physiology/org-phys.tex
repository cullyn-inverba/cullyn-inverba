\documentclass[12pt,a4paper]{article}
\usepackage{inverba}

\newcommand{\userName}{Cullyn Newman} 
\newcommand{\class}{BI 320} 
\newcommand{\institution}{Portland State University} 
\newcommand{\theTitle}{\color{g-Leaf} Organismal Physiology}

\begin{document}
%%%%%%%%%%%%%%%%%%%%%%%%%%%%%%%%%%%%%%%%%%%%%%%%%%%%%%%%%%%%%%%%%%%%%
\tableofcontents
\cleardoublepage
\fancyhead{}
\fancyhead[R]{\hyperlink{home}{\nouppercase\leftmark}}
%%%%%%%%%%%%%%%%%%%%%%%%%%%%%%%%%%%%%%%%%%%%%%%%%%%%%%%%%%%%%%%%%%%%%

\clearpage
\fancyhead[L]{Midterm I}
%\begingroup
%%%%%%%%%%%%%%%%%%%%%%%%%%%%% Chapter 1 %%%%%%%%%%%%%%%%%%%%%%%%%%%%%
%\begingroup
\clearpage
\section{Animals and Environments}\phantomsection
\subsection{Introduction}
\begin{itemize}
    \item What is physiology?
        \begin{itemize}
            \item Form and function of organisms; the study of how organisms work.
        \end{itemize}
    \item Central questions of physiology: {\color{o-Sun}Mechanism} and {\color{o-Sun}Origin}.
    \item Mechanism:
        \begin{itemize}
            \item refers to the {\color{o-Sun}components} of living organisms and understanding {\color{o-Sun}how} components interact to enable the organism to function.
        \end{itemize}
    \item Origin:
        \begin{itemize}
            \item asks why a mechanism exists, or {\color{o-Sun}what} is the mechanistic {\color{o-Sun}adaptive significance} of the mechanism.
        \end{itemize}
    \item August Krogh, \begin{quote}
        "For such a large number of problems there will be some animal of choice or a few such animals on which it can be most conveniently studied."
    \end{quote}
\end{itemize}

\subsection{Homeostasis}
\begin{itemize}
    \item 
\end{itemize}

\subsection{Physiology and Time}
\begin{itemize}
    \item 
\end{itemize}
%\endgroup
%%%%%%%%%%%%%%%%%%%%%%%%%%%%% Chapter 1 %%%%%%%%%%%%%%%%%%%%%%%%%%%%%

%%%%%%%%%%%%%%%%%%%%%%%%%%%%% Chapter 2 %%%%%%%%%%%%%%%%%%%%%%%%%%%%%
%\begingroup
\clearpage
\section{Molecules and Cells in Animal Physiology}\phantomsection
\subsection{Cell Membranes}
\begin{itemize}
    \item 
\end{itemize}

\subsection{Enzyme Fundamentals}
\begin{itemize}
    \item 
\end{itemize}
%\endgroup
%%%%%%%%%%%%%%%%%%%%%%%%%%%%% Chapter 2 %%%%%%%%%%%%%%%%%%%%%%%%%%%%%

%%%%%%%%%%%%%%%%%%%%%%%%%%%%% Chapter 3 %%%%%%%%%%%%%%%%%%%%%%%%%%%%%
%\begingroup
\clearpage
\section{Genomics and Proteomics}\phantomsection
\subsection{}
\begin{itemize}
    \item 
\end{itemize}
%\endgroup
%%%%%%%%%%%%%%%%%%%%%%%%%%%%% Chapter 3 %%%%%%%%%%%%%%%%%%%%%%%%%%%%%

%%%%%%%%%%%%%%%%%%%%%%%%%%%%% Chapter 4 %%%%%%%%%%%%%%%%%%%%%%%%%%%%%
%\begingroup
\clearpage
\section{Physiological Development}\phantomsection
\subsection{}
\begin{itemize}
    \item 
\end{itemize}
%\endgroup
%%%%%%%%%%%%%%%%%%%%%%%%%%%%% Chapter 4 %%%%%%%%%%%%%%%%%%%%%%%%%%%%%

%%%%%%%%%%%%%%%%%%%%%%%%%%%%% Chapter 5 %%%%%%%%%%%%%%%%%%%%%%%%%%%%%
%\begingroup
\clearpage
\section{Transport of Solutes and Water}\phantomsection
\subsection{}
\begin{itemize}
    \item 
\end{itemize}
%\endgroup
%%%%%%%%%%%%%%%%%%%%%%%%%%%%% Chapter 5 %%%%%%%%%%%%%%%%%%%%%%%%%%%%%

%%%%%%%%%%%%%%%%%%%%%%%%%%%%% Chapter 27 %%%%%%%%%%%%%%%%%%%%%%%%%%%%%
%\begingroup
\clearpage
\setcounter{section}{26}
\section{Water and Salt Physiology: Mechanisms}\phantomsection
\subsection{}
\begin{itemize}
    \item 
\end{itemize}
%\endgroup
%%%%%%%%%%%%%%%%%%%%%%%%%%%%% Chapter 27 %%%%%%%%%%%%%%%%%%%%%%%%%%%%%

%%%%%%%%%%%%%%%%%%%%%%%%%%%%% Chapter 7 %%%%%%%%%%%%%%%%%%%%%%%%%%%%%
%\begingroup
\clearpage
\setcounter{section}{6}
\section{Nutrition, Feeding, and Digestion}\phantomsection
\subsection{}
\begin{itemize}
    \item 
\end{itemize}
%\endgroup
%%%%%%%%%%%%%%%%%%%%%%%%%%%%% Chapter 7 %%%%%%%%%%%%%%%%%%%%%%%%%%%%%
%\endgroup

\clearpage
\fancyhead[L]{Midterm II}
%\begingroup

%\endgroup
\end{document}