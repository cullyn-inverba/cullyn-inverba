\documentclass[12pt,a4paper]{article}
\usepackage{inverba}

\newcommand{\userName}{Cullyn Newman} 
\newcommand{\class}{BI 320} 
\newcommand{\institution}{Portland State University} 
\newcommand{\theTitle}{\color{g-Leaf} Organismal Physiology}

\begin{document}
%%%%%%%%%%%%%%%%%%%%%%%%%%%%%%%%%%%%%%%%%%%%%%%%%%%%%%%%%%%%%%%%%%%%%
\tableofcontents
\cleardoublepage
\fancyhead{}
\fancyhead[R]{\hyperlink{home}{\nouppercase\leftmark}}
%%%%%%%%%%%%%%%%%%%%%%%%%%%%%%%%%%%%%%%%%%%%%%%%%%%%%%%%%%%%%%%%%%%%%

\clearpage
\fancyhead[L]{Week 1}
%\begingroup
%%%%%%%%%%%%%%%%%%%%%%%%%%%%% Chapter 1 %%%%%%%%%%%%%%%%%%%%%%%%%%%%%
%\begingroup
\clearpage
\section{Animals and Environments}\phantomsection
\subsection{Introduction}
\begin{itemize}
    \item What is physiology?
        \begin{itemize}
            \item Form and function of organisms; the study of how organisms work.
        \end{itemize}
    \item Central questions of physiology: {\color{o-Sun}mechanism} and {\color{o-Sun}origin}.
        \begin{itemize}
            \item Mechanism:
                \begin{itemize}
                    \item Refers to the {\color{o-Sun}components} of living organisms and understanding {\color{o-Sun}how} components interact to enable the organism to function.
                \end{itemize}
            \item Origin:
                \begin{itemize}
                    \item Asks why a mechanism exists, or {\color{o-Sun}what} is the mechanistic {\color{o-Sun}adaptive significance} of the mechanism.
                \end{itemize}
            \item Mechanism and adaptive significance are distinct concepts; knowing about one doesn't necessarily mean you know anything about the other.
        \end{itemize}
    \item Krogh's principle: \begin{quote}\color{G-Moon}
        "For such a large number of problems there will be some animal of choice or a few such animals on which it can be most conveniently studied."\end{quote}
        \begin{itemize}
            \item This idea is central to disciplines that rely on the \textit{comparative method}. The key take away: there is unity in diversity; many organisms are very much alike at the most fundamental levels. 
        \end{itemize}
    \item Physiology subdisciplines: 
        \begin{itemize}
            \item Mechanistic: emphasizes the mechanisms by which organisms perform their life functions.
            \item Evolutionary: emphasizes evolutionary origins and the adaptive significance of traits.
            \item Comparative: emphasizes the way in which diverse phylogenetic groups resemble and differ from each other.
            \item Environmental: emphasizes the ways in which physiology and ecology interact.
            \item Integrative: emphasizes the importance of all levels of organization, from genes to proteins and tissues to organs in order to better understand whole physiological systems.
        \end{itemize}
\end{itemize}

\subsection{Homeostasis}
\begin{itemize}
    \item Important ideas to remember:
        \begin{itemize}
            \item Organisms are structurally dynamic; form stays relatively static while individual cells recycle frequently.
            \item Most cells are exposed to the {\color{o-Sun}internal} environment, not external.
            \item Internal cells may vary or kept constant with the environment.
        \end{itemize}
    \item Temperature regulation:
        \begin{itemize}
            \item \textbf{Conformity}: organism's internal temperature {\color{o-Sun}correlates} with external temperature in a particular range of temperatures. 
            \item \textbf{Regulation}: internal environment is held mostly {\color{o-Sun}contant} using celluar mechanisms.
        \end{itemize}
    \item \textbf{Homeostasis}: the coordinated physiological processes that maintain a relatively constant state in the organism.
        \begin{itemize}
            \item {\color{pos}Positive feedback}: less common in homeostasis due difficulty in regulation; leads to runaway effect easily.
            \item {\color{neg}Negative feedback}: more common in homeostasis due to self correcting nature.
            \item \textbf{Effector}: executes the change in action that produces an effect, e.g. signals to increase temperature.
            \item \textbf{Sensor}: sense changes in environment and sends information to the effector.
        \end{itemize}
\end{itemize}

\subsection{Physiology and Time}
\begin{itemize}
    \item Timeframes of physiological change:
        \begin{itemize}
            \item \textbf{Acute}: short-term, reversible, and quick to adapt to changes in environment. Usually minutes to hours.
            \item \textbf{Chronic}: long-term after prolonged exposure to new environments. Changes are usually reversible, but often slower. 
                \begin{itemize}
                    \item Chronic can be termed acclimation, or phenotypic plasticity/flexibility.
                     \item Repetitive acute responses usually lead to chronic responses.
                \end{itemize}
            \item \textbf{Evolutionary}: changes due to alteration in gene frequencies in {\color{o-Sun}populations} exposed to new environments.
        \end{itemize}
    \item Acclimation is {\color{false}not the same} as adaption.
        \begin{itemize}
            \item \textit{Adaption} is an evolutionary trait presnet at high frequency in a population due to survival/reproductive advantages. 
            \item Not all traits are adaptations.
            \item The amount of natural variation in a trait must be considered across populations, species etc.
        \end{itemize}
\end{itemize}
%\endgroup
%%%%%%%%%%%%%%%%%%%%%%%%%%%%% Chapter 1 %%%%%%%%%%%%%%%%%%%%%%%%%%%%%

%%%%%%%%%%%%%%%%%%%%%%%%%%%%% Chapter 2 %%%%%%%%%%%%%%%%%%%%%%%%%%%%%
%\begingroup
\clearpage
\section{Molecules and Cells in Animal Physiology}\phantomsection
\subsection{Cell Membrane Review}
\begin{itemize}
    \item Major cell memberane structures:
        \begin{itemize}
            \item \textbf{Glycoproteins}: carbohydrate chain attached to a protein.
            \item \textbf{Glycolipids}: similar to glycoproteins, but attached to lipid molecues.
            \item \textit{Glycocalyx}: combination of glycoproteins and glycolipids on the surface of cell.
            \item \textbf{Integral proteins}: embedded in phospholipid bilayer.
            \item \textbf{Peripheral proteins}: associated with one side of the bilayer.
        \end{itemize}
    \item \textbf{Unsaturated phospholipid}: whey hydrocarbon tails contain double bonds (less hydrogen).
        \begin{itemize}
            \item Increase membrane fluidity due to extra space created.
        \end{itemize}
    \item The fluidity of the cell membrane allows proteins to from complexes and dynamically change shape.
\end{itemize}

\subsection{Enzyme Fundamentals}
\begin{itemize}
    \item \textbf{Enzymes}: a protein catalyst that plays two primary roles: {\color{o-Sun}accelerating} and {\color{o-Sun}regulating} chemical reactions. 
    \item \textit{Substrates}: the initial reactants of the reaction that an enzyme catalyzes.
    \item \textbf{Enzyme-substrate-complex (E-S)}: a combination of enzyme (E) with a molecule of substrate (S) that starts a reaction.
        \begin{itemize}
            \item Usually stabalized by {\color{o-Sun}non-covalent} bonds.
            \item The substrate is converted to a product by first becomeing an \textit{enzyme-product complex (E-P)}, then dissociates to yield free product and free enzyme.
            \item {\color{o-Sun}\(E+S\rightleftharpoons \text{E-S} \rightleftharpoons \text{E-P} \rightleftharpoons E + P\)}
        \end{itemize}
    \item \textbf{Saturation kinetics}:
        \begin{itemize}
            \item \textbf{V\(_{\text{max}}\)}: the maximum velocity of a reaction and is determined by:
                \begin{itemize}
                    \item the {\color{o-Sun}number} of active enzyme molecues present relative to substrate.
                    \item the catalytic {\color{o-Sun}effectiveness} of each enzyme molecule.
                    \item These properties usually undergo heavy selection pressure.
                \end{itemize}
            \item \textit{Saturated}: all enzymes are occupied by a substrate molecule nearly all the time and now unable to increase reaction velocity.
            \item \textbf{Hyperbolic}: asymptotically approaches V\(_{\text{max}}\)
                \begin{itemize}
                    \item Tends to happen when enzymes have just one substrate binding site.
                    \item Or when substrate sites behave independently
                \end{itemize}
            \item \textbf{Sigmodal}: approaches V\(_{\text{max}}\) with a sigmodal trajectory.
                \begin{itemize}
                    \item When multiple sites influence each other.
                \end{itemize}
            \item \textbf{Turnover number (k\(_{\text{cat}}\))}: the {\color{o-Sun}total effectiveness}, expressed as the number of substrate molecules coverted to product per second by each enzyme molecule when saturated.
                \begin{itemize}
                    \item Depends partly on the \textit{activation energy} of the enzyme-catalyzed reaction.
                    \item \textbf{Activation energy}: the energy required for the substrate to enter the \textit{transition state}. 
                    \item \textbf{Transition state}: the intermediate chemical state between substrate and product.
                    \item Enzymes {\color{o-Sun}lower the activation energy} required to enter transition state.
                \end{itemize}
        \end{itemize}
    \item \textbf{Enzyme-substrate affinity}:
        \begin{itemize}
            \item The proclivity of the enzyme to form a complex with the substrate when they meet.
                \begin{itemize}
                    \item {\color{pos}Likely} complex formation results in {\color{pos}high-affinity}.
                    \item {\color{neg}Unlikely} complex formation results in {\color{neg}low-affinity}.
                \end{itemize}
            \item Affinity affects the shape of the reaction velocity.
                \begin{itemize}
                    \item {\color{pos}Higher} affinity produces a {\color{pos}steeper} velocity, and {\color{neg}lower} produces a more {\color{neg}linear} result.
                \end{itemize}
            \item \textbf{Half-saturation constant, \(\mathbf{K_m}\)}: the substrate concentration required to attain one-half maximum reaction velocity.
                \begin{itemize}
                    \item \(K_m\) and enzyme-substrate affinity are{\color{o-Sun}inversely related}.
                    \item i.e. {\color{neg}low-affinity} enzyme has a {\color{pos}greater \(K_{m}\)}.
                \end{itemize}
        \end{itemize}
\end{itemize}
%\endgroup
%%%%%%%%%%%%%%%%%%%%%%%%%%%%% Chapter 2 %%%%%%%%%%%%%%%%%%%%%%%%%%%%%
%\endgroup

\clearpage
\fancyhead[L]{Week 2}
%\begingroup

%\endgroup
%%%%%%%%%%%%%%%%%%%%%%%%%%%%% Chapter 3 %%%%%%%%%%%%%%%%%%%%%%%%%%%%%
%\begingroup
\clearpage
\section{Genomics and Proteomics}\phantomsection
\subsection{}
\begin{itemize}
    \item 
\end{itemize}
%\endgroup
%%%%%%%%%%%%%%%%%%%%%%%%%%%%% Chapter 3 %%%%%%%%%%%%%%%%%%%%%%%%%%%%%

%%%%%%%%%%%%%%%%%%%%%%%%%%%%% Chapter 4 %%%%%%%%%%%%%%%%%%%%%%%%%%%%%
%\begingroup
\clearpage
\section{Physiological Development}\phantomsection
\subsection{}
\begin{itemize}
    \item 
\end{itemize}
%\endgroup
%%%%%%%%%%%%%%%%%%%%%%%%%%%%% Chapter 4 %%%%%%%%%%%%%%%%%%%%%%%%%%%%%

%%%%%%%%%%%%%%%%%%%%%%%%%%%%% Chapter 5 %%%%%%%%%%%%%%%%%%%%%%%%%%%%%
%\begingroup
\clearpage
\section{Transport of Solutes and Water}\phantomsection
\subsection{}
\begin{itemize}
    \item 
\end{itemize}
%\endgroup
%%%%%%%%%%%%%%%%%%%%%%%%%%%%% Chapter 5 %%%%%%%%%%%%%%%%%%%%%%%%%%%%%

%%%%%%%%%%%%%%%%%%%%%%%%%%%%% Chapter 27 %%%%%%%%%%%%%%%%%%%%%%%%%%%%
%\begingroup
\clearpage
\setcounter{section}{26}
\section{Water and Salt Physiology: Mechanisms}\phantomsection
\subsection{}
\begin{itemize}
    \item 
\end{itemize}
%\endgroup
%%%%%%%%%%%%%%%%%%%%%%%%%%%%% Chapter 27 %%%%%%%%%%%%%%%%%%%%%%%%%%%%

%%%%%%%%%%%%%%%%%%%%%%%%%%%%% Chapter 7 %%%%%%%%%%%%%%%%%%%%%%%%%%%%%
%\begingroup
\clearpage
\setcounter{section}{6}
\section{Nutrition, Feeding, and Digestion}\phantomsection
\subsection{}
\begin{itemize}
    \item 
\end{itemize}
%\endgroup
%%%%%%%%%%%%%%%%%%%%%%%%%%%%% Chapter 7 %%%%%%%%%%%%%%%%%%%%%%%%%%%%%


\clearpage
\fancyhead[L]{Midterm II}
%\begingroup

%\endgroup
\end{document}