\documentclass[12pt,a4paper]{article}
\usepackage{inverba}
\newcommand{\userName}{Cullyn Newman} 
\newcommand{\class}{} 
\newcommand{\institution}{Portland State University} 
\newcommand{\thetitle}{\hypertarget{home}{Principles of Neural Science}}
\rfoot{\hyperlink{home}{\thepage}}

\begin{document}
%%%%%%%%%%%%%%%%%%%%%%%%%%%%%%%%%%%%%%%%%%%%%%%%%%%%%%%%%%%%%%%%%%%%%
\tableofcontents
\cleardoublepage
\fancyhead{}
\fancyhead[R]{\hyperlink{home}{\nouppercase\leftmark}}
%%%%%%%%%%%%%%%%%%%%%%%%%%%%%%%%%%%%%%%%%%%%%%%%%%%%%%%%%%%%%%%%%%%%%

%%%%%%%%%%%%%%%%%%%%%%%%%%%%% Chapter 1 %%%%%%%%%%%%%%%%%%%%%%%%%%%%%
%\begingroup
\clearpage
\section{The Brain and Behavior}

\subsection{The Brain Has Distinct Functional Regions}

    \textbf{The Central Nervous System Has Seven Main Parts}
    \begin{itemize}
        \item \textbf{Spinal cord}: most caudal part of the central nervous system. It is subdivided into cervical, thoracic, lumbar, and sacral regions.
        \item \textbf{Brain stem}: consists of the medulla oblongata, pons, and midrain. Relays input from the spinal cord and back, and controls input to and from the head.
        \item \textbf{Medulla oblongata}: rostral to spinal cord and includes several centers responsible for vital autonomic functions. 
        \item \textbf{Pons}: rostral to medulla and conveys information about movement.
        \item \textbf{Cerebellum}: lies behind pons, modulates force and range of movement, and involved in learning motor skills.
        \item \textbf{Diencephalon}: lies rostral to midrain and contains two structures, thalamus (processes information reaching cerebral cortex) and hypthalamus (regulates autonomic, endocrine, and visceral functions).
        \item \textbf{Cerebrum}: comprises two cerebral hemispheres, each consisting of wrinkled outer layer (the cerebral cortex), and three deep lying structures (basal ganglia, the hippocampus, and the amygdaloid nuclei).
        \item \textbf{Cerebral cortex}: divided into four distinct lobes--- frontal, parietal, occipital, and temporal. The frontal lobe is largely concerned with short-term memory and planning, as well as movement; the parietal lobe with somatic sensation, forming a body image, and relating it to extrapersonal space; the occipital lobe with vision; and the temporal lobe with hearing---combined with deeper structures---with learing, memory, and emotion.
    \end{itemize}
%\endgroup
%%%%%%%%%%%%%%%%%%%%%%%%%%%%% Chapter 1 %%%%%%%%%%%%%%%%%%%%%%%%%%%%%

%%%%%%%%%%%%%%%%%%%%%%%%%%%%% Chapter 2 %%%%%%%%%%%%%%%%%%%%%%%%%%%%%
%\begingroup
\clearpage
\section{Chapter}
\subsection{}
\begin{itemize}
    \item
\end{itemize}
%\endgroup
%%%%%%%%%%%%%%%%%%%%%%%%%%%%% Chapter 2 %%%%%%%%%%%%%%%%%%%%%%%%%%%%%
\end{document}