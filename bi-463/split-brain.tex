\documentclass[basic]{inVerba-notes}

\newcommand{\userName}{Cullyn Newman}
\newcommand{\class}{BI:\@ 463}
\newcommand{\theTitle}{\color{link}{Split-Brain Presentation Script}}
\newcommand{\institution}{Portland State University}

\begin{document}
    
\textbf{Current State of Research}

Hi, so now we will begin to transition the current state of research in the field of split brains. There is a lot of exciting stuff, so we can't really cover it all unfortunately. What we tried to do was focus on how they did what they did, what the results were, and what those results might mean.

To do this, we will first explain some current experimental techniques used understand the asymmetries between hemispheres of the brain. Next, we will then dive deeper into some past and current paradigms in regard to some more specific research. And finally, we will attempt tackle some of the more open questions and interesting implications that this research generates.

That being said, I will now pass it off to Derrick to discuss some the techniques used.

\textbf{Current Paradigm}

So hopefully we are now aware of what the corpus callosum is, the history behind the experiments, and how we go about investigating the hemispheric asymmetries in the brain. With this all in mind I'd then like to give an outline of where I'll try to take us next.

First I'll attempt to summarize and contrast the classic and current view of in regard to the role of the corpus callosal in various disconnection syndromes. Then I'll jump into a more specific realm of research that uses facial recognition as a means investigate some uncertainties around hemispheric asymmetries and also how emotional processing may affect social cognition.

\textbf{Sources}
 
I do want to quickly point out the main source I used for this discussion of current research. As I said, there were so many directions we could go in, but the most up to date review paper I found came out of a University in Italy. Of course, split brain patients were our focus, but in particular this paper focused on a review of facial perception and the roles of emotional and social cognition. 

There are other extremely fascinating topics that could have been discussed are definitely out there. For example, I found some research on something called hemineglect, which I would have absolutely loved to cover, and to be honest it was probably more interesting, but unfortunately I found it super late and did not have time to integrate it into this presentation. Regardless, I do think I have some interesting stuff to show you, so let's jump right in. 

\textbf{Classical View}

The first general conclusion around hemispheric lateralization emerged from the observations centered around white matter lesions that gave rise to cognitive, behavioral, and psychological dysfunctions as we previously discussed. These dysfunctions are broadly defined as disconnection syndromes. The callosal syndrome relates to specific dysfunctions of split-brain patients. 

The classical view is built on the study of various callosal syndromes in order to analyze these functional differences. Historical left vs.\ right hemisphere superiority of certain tasks gave rise to various pop culture interpretations that stemmed from the hypothesis that the regions were mostly separate. In the most basic sense, it was thought that results of these left and right brain computations were just sent over to other side in cases that the other sided needed it. 

\textbf{Integrative View}

Recently however, it has become clear that the corpus callosum may aid in \textbf{creating} these functional asymmetries in some cases, rather that just facilitating transfer. The first observation that suggests this involves selectively active cells within the corpus callosum. 

This suggests that symptoms of callosal disconnection are not simply due to the loss of information transfer, but also due to loss of distributed balance mediated by the callosal fibers. Also, an important note, these balancing meditations worked together with the other cortical and subcortical nerve connections between hemispheres, but I decided to cut the details of those connections from this presentation. The key fact was that the corpus callosum doesn't work alone. This might be one of the reasons why it is possible to cut the corpus callosum while still appearing relatively normal. 

There's also a notion of an equilibrating role that supports this mediated view, due to evidence that the

(read second point)

This observation suggests that some callosal syndromes might be due to the result of the poor responses from an improperly informed hemisphere. Basically, information may still be transferred by other subcortical pathways, but not processed correctly due to improper integration of bilateral transfer. 

Also, curiosity is aroused from the observation that transfer time was found to be different depending on the direction. Molecularly speaking, it was thought to be due to that fact that the right hemisphere has a greater number of fast-conducting, myelinated fibers. But, why is this the case? It's not completely certain, but it is significant, because this asymmetry helps support the prevalent, more integrative interpretation, by demonstrating another factor responsible for dysfunctional asymmetries that arise in connection syndromes. 

To sum it all up and put it a little more succinctly: the prevalent interpretation is one that views the interhemispheric communication taking place both by white matter and by bilateral subcortical projections that themselves influence informational processing. This makes the old view of the corpus callosum simply being used for information transfer a bit weaker.

\textbf{Hemispheric Asymmetry For Faces}

Now, to dive a little deeper into this, the authors focused on human faces as specific stimuli, since the nature of stimuli present can often change outcomes.

It's well established that there is a strong asymmetry in facial processing, with a strong superiority of the right hemisphere, particularly in the fusiform face area (FFA). There is a significant amount of evidence backing this up in the paper, but I will not elaborate on it here. Instead, what I will focus on, and what the main point the paper makes, is that there is interhemispheric (or between hemispheres) cooperation that also plays a role in facial processing. One study illuminates such cooperation in many areas. There were also reports of covariation in activity between corresponding areas in the two hemispheres, e.g., left and right fusiform gyrus. They state that in some cases interhemispheric connectivity was stronger (more active) between opposite facial processing areas, rather than typical one-sided hemispheric superiority often observed. The more complex the processing however, the more specialized to the right activity tended to be. 

These findings, as well as others that the paper goes over, support the integrative view previously discussed. More importantly however, it suggests a hypothesis that pushes back on the previously, mostly unquestioned, cerebral asymmetry and superiority of the right hemisphere in regard to facial processing; implicating the importance of understanding interhemispheric connections and their crucial role in facial processing.

\textbf{Faces in the Disconnected Brain}

Now his is where collosotomy patients and the data behind callosal syndromes could shed more light on the issue, especially regarding the complexity of facial stimuli. This focus is a great demonstration of the current state of research regarding split brains. 

Much of the current research around disconnected brains have produced much data supporting the superiority of right hemisphere in regard to facial processing. Again, mostly in specific types of tasks, such as when the faces share same gander as the patient, in self recognition, or in faces that show other characteristics that make them appear visually similar to one's self. More research on collosotomy patients will help develop this area of research into more conclusive directions. 

Overall (read slide). However, don't get me wrong, there is still plenty of evidence that supports the fact that each hemisphere is more specialized in and various tasks. Here the current research is mostly trying to figure out just how integrative the role of the corpus callosum is in regard to a certain type of task, i.e., facial perception.

\textbf{Social Perception and Cognition}

There research didn't end with facial processing however. Another specific debate in current research around disconnected brain has to do with emotional processing. They key observation consist of the fact that a major part of facial perception is emotional decoding. Often expressions are detected automatically and our perception of others is changed, sometimes dramatically, by this decoding. 

There are two main theories when it comes to the location of where this processing takes place: one, the valence hypothesis, which states that there is an asymmetry in emotional processing depending on emotional valence (perceived positive and negative emotions), and two, the right hemisphere hypothesis, which states that all processing mostly occurs in the right hemisphere.

\textbf{Continued\dots}

What's strange though, is that it appears that both theories are well-supported, even though they appear to be competing with each other. Some prior evidence suggest it depends on the amount of emotional stimuli present, where a greater degree of complexity in the stimuli could be involved in hemispheric specialization. There authors also questioned the if the degree of intensity could play a role, but there was conflicting data in regard to that observation. 

Limited research and the fact that some patients already had problems, e.g., the seizures, brings into question just how functional their brains were compared to healthy patients in the first place. Factors like these make the research in this field even more difficult. There are plenty of open questions and various implications that the data from this field produces. That being said, that brings us to Dania, who is going to discuss how some of these data are being used in fascinating fields of research.o

\end{document}