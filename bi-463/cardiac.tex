\documentclass[basic,plain]{inVerba-notes}

\newcommand{\userName}{Cullyn Newman}
\newcommand{\class}{BI:\@ 463}
\newcommand{\theTitle}{Cardiac Physiology Assignment}
\newcommand{\institution}{Portland State University}
% chktex-file 37

\begin{document}

\begin{enumerate}\color{minimal}
    \item Explain what long QT syndrome is?
    
    {\color{black}
        Long QT syndrome (LQTS) is a heart condition that affects the repolarization phase after a heartbeat occurs, increasing the risk of irregular heart rhythms (arrhythmias). Not all people show signs or symptoms, but if they do occur, then symptoms such as fainting, seizures, sudden death, or others stemming from the heart arrhythmias could be present. 
        
        It is called a long QT syndrome because the QT interval (the time from start of Q wave to end of T wave) is prolonged due to changes in the ion channels that influence the electrical signals used to coordinate the heart cells. There are several mechanisms in particular that are responsible for the change in heart rhythm, each with different degrees of severity. Many variants of LQTS are inherited genetically, generally resulting in malfunctioning ion channels, but some can be drug-induced. }

    Points: [\hspace{16pt} / 5 ]

    Feedback: 

    \vspace*{50pt}

    \item What role do HERG channels play in cardiac physiology under baseline conditions? 
    {\color{black}

    Human Ether-a-go-go-Related Gene (HERG) is a gene that codes for a \(\alpha \)-subunit of a voltage gated potassium ion channel that is involved in the rapidly activating delayed rectifier current (\(I_{Kr}\)). Under baseline conditions, the HERG channel mediates the \(I_{Kr}\) current for the cardiac action potential that help keeps the coordination of regular heart beats.

    The HERG channels have unusual gating properties. They activate with depolarization like most potassium channels, but inactivate much more rapidly at positive voltages. This allows them to have brief periods of potassium conductivity. However, during repolarization, the inactivated channels reopen quickly and then slowly return to a rested state, longer open states and producing large-amplitude tail currents.

    Changes to the expression of HERG channels can result in errors in the timing of coordinated heartbeats to changes in the amplitude, timing, or deactivation of the current effecting the cell polarization leading to the irregular arrhythmias and various LQTSs.}

    Points: [\hspace{16pt} / 5 ]

    Feedback: 

    \vspace*{50pt}
    
    \item What happens to HERG channels with the N470D mutation? How does this contribute to long QT syndrome? 
    {\color{black}

    The N47OD mutation was one of the human LQT2 (a reduction in \(I_{Kr}\)) mutations that caused apparent dominant negative suppression of the wild-type HERG function leading to abnormal current.

    N470D in particular is associated with HERG channels that activate at more negative voltages than wild-type channels. Usually activation begins at above -50 mV according to the paper provided. Lowering of the activation voltages appears to contribute to the decrease the amplitude of \(I_{Kr}\) current, which could lead to the abnormal timings.}

    Points: [\hspace{16pt} / 5 ]

    Feedback: 

    \vspace*{70pt}

    \item Explain the role that temperature plays in Long QT syndrome in patients with the N470D mutation in HERG channels. 
    {\color{black}

    The LQT2 channel is highly dependent on temperature, with N47OD mutation shown to may play a role in influencing the defective protein trafficking in the mutant HERG proteins. This defective trafficking was shown to be a major mechanism for decreased \(I_{Kr}\) in LQT2. 

    The paper cites biochemical studies that confirmed the that the trafficking was temperature dependent due to the mutant protein retaining intracellularly at physiologic temperature. The paper doesn't exactly explain why, but from I presume the change in activation energy provided by the change in temperature is what causes the change protein trafficking. The mutant form may be more sensitive (or less robust) to changes in the activation energy of the system compared to the wild-type. This high sensitivity could influence the expression/function of HERG protein overall, resulting in ion channel abnormalities. 
    
    }

    Points: [\hspace{16pt} / 5 ]

    Feedback: 

    \newpage

    \item IN YOUR OWN WORDS, explain the controversy regarding HERG channels potentially acting as dominant negatives when inherited as an autosomal dominant mutation. (N470D).
    
    {\color{black}
    The dominant negative effect implies that the result of just a single mutant subunit would convey the mutant phenotype since four HERG proteins molecules are needed to coassemble in order to form the functional ion channel. 

    The issue is that there are two proposed mechanism of HERG channel assembly. No co-assembly results in all four subunits having the mutant form, leading to approximately half of the HERG channels showing mutant phenotype.  Alternatively, a random co-assembly with a dominant negative effect in would result in much more than half showing channel dysfunction (only 1/16).

    The paper doesn't seem to conclude which possibility is correct. From what I can tell they are implying there is a dominant negative effect, thus many channels are affected, but before this paper the former was assumed (?) I really can't tell for certain. The in the conclusion they state that ``It now is clear that LQT2 represents a large number of mutations in the HERG channel\dots'' which is what formed my assumption that this was not previously clear. 
    }

    Points: [\hspace{16pt} / 5 ]

    Feedback: 

    \vspace*{50pt}
\end{enumerate}

\end{document}