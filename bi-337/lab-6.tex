\documentclass[basic,plain]{inVerba-notes}

\newcommand{\userName}{Cullyn Newman}
\newcommand{\class}{BI:\@ 337}
\newcommand{\theTitle}{Week 6: Signaling Mechanisms}
\newcommand{\institution}{Portland State University}

\begin{document}

\chonk{Relevant Concepts} --- \textbf{15-PGDH inhibition promotes liver regeneration}
\begin{itemize}
  \item The authors investigated the effects of 15-PGDH on a variety of different tissue types and how it impacted tissue regeneration and repair in each tissue type.
  \item Previous findings demonstrated that SW033291 inhibits15-PGDH, which has been shown to have therapeutic applicability due to increases of PGE2 levels and enhanced hematopoietic capacity, all of which leads to increased tissue regeneration.
  \item Liver regeneration was of particular interest for patients undergoing surgical resection of primary liver tumors or colon cancers metastatic to the liver. The limiting factor to get these procedures often is dependent on the postoperative liver being able to regenerate quickly enough.
    \begin{itemize}
      \item If SW033291 is shown to have similar effects in humans as mice, then this could lead to more patients being able to undergo the procedures they need.
    \end{itemize}
\end{itemize}

\chonk{Techniques} --- \textbf{BrdU incorporation following partial hepatectomy}
\begin{itemize}
  \item BrdU is often used to identify proliferating cells by using a synthetic nucleoside analogue with a chemical structure similar to thymidine, giving the ability to detect cells that recently performed DNA replication or repair with specific antibodies.
    \begin{itemize}
      \item The authors used BrdU in order to test for cell proliferation after a partial hepatectomy to test for function of 15-PGDH and effect of SW033291.
      \item Given past results, the cells that are treated with a 15-PGDH knockout have more cells that are BrdU positive compared to the wild-type.
    \end{itemize}
  \item Hepatectomy: surgical resection (removal of all or part) of the liver.
\end{itemize}
\medskip

\chonk{Illustration of Techniques}

\begin{center}
  \Image{0.72\columnwidth}{images/week-6-biorend.png}
\end{center}

\end{document}