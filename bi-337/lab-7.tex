\documentclass[basic,plain]{inVerba-notes}

\newcommand{\userName}{Cullyn Newman}
\newcommand{\class}{BI:\@ 337}
\newcommand{\theTitle}{Week 7: Cytoskeleton}
\newcommand{\institution}{Portland State University}

\begin{document}

\chonk{Relevant Concepts} --- \textbf{Focal Adhesion \& Integrins}
\begin{itemize}
  \item \textbf{Focal Adhesions (FA)}: large macromolecular assemblies, which are large chemical structures made up of complex mixtures of polypeptides, polynucleotide, polysaccharide, and/or other polymeric macromolecules, that allow transfer mechanical force and regulatory signals between the extracellular matrix and a cell. 
    \begin{itemize}
      \item Technically, they are sub-cellular structures that mediate the regulatory effects in response to  extracellular adhesion by serving as mechanical linkages. 
      \item They also direct numerous signaling proteins at sites of integrin binding and clustering.
    \end{itemize}
  \item \textbf{Integrins}: transmembrane receptors that facilitate and modulate cell to cell and cell to extracellular matrix adhesion via ligand binding to activate signal various transduction pathways, allowing for a wide range of influence over the cell behavior.
\end{itemize}

\medskip

\chonk{Techniques} --- \textbf{TIRF \& Speckle Microscopy}

\begin{itemize}
  \item \textbf{Total internal reflection fluorescence (TIRF)}: a type microscopy that shows visualizations of certain molecules under specific conditions such as vesicle transport, signaling events, and single-molecule detection. 
    \begin{itemize}
      \item TIRF is done by shining a laser at a specific angle that will display the entire internal reflection. By using TIRF the authors were able to detect single fluorescent molecules because in regular microscope practices it can be harder to achieve small illuminations. 
    \end{itemize}

  \item \textbf{Speckle microscopy}: a technique that allows for the visualization of microtubules as a ``speckled'' form when the fluorescence tubulin is inserted into the living cells.
    \begin{itemize}
      \item The advantage of speckle microscopy is that it avoids intensified light usage that would usually be needed for activation or bleaching. 
    \end{itemize}
 
  \item The authors combined both TIRF microscopy and speckle microscopy in order to measure the interactions within the actin filaments and FAs. The use of TIRF creates an image contrast that allows visualization of where the actin filaments combine with the FAs, while the speckle microscopy incorporates flouriohone clusters in order to mark and track the assemblies after formation.
  
\end{itemize}

\chonk{Illustration of Techniques}\\
Honestly, I don't understand this paper well enough to be able to generate a useful illustration. I (sort of) think I understand the results, why they needed TIRF to visualize the movement of the integrins in question, and why speckle microscopy was needed to track their movement. However, I don't understand how they came to the conclusions they did, and I'm worried I'm going to spend time cementing blatantly false and misguided information by creating a graphic. I'm okay with taking reduced points this week as a result, since it does demonstrate my inability to understand this paper.

\end{document}