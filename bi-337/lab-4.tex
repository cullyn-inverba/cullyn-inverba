\documentclass[basic,plain]{inVerba-notes}

\newcommand{\userName}{Cullyn Newman}
\newcommand{\class}{BI:\@ 337}
\newcommand{\theTitle}{Trafficking Inside the Cell}
\newcommand{\institution}{Portland State University}

\begin{document}

\chonk{Relevant Concepts} --- \textbf{Lethal Gene Combinations Block Transport}
\begin{itemize}
  \item The authors were not certain of the limitations of the SEC genes; they thought that the SEC genes could be involved in other cellular processes besides intracellular transport. 
  \item To test this, the authors investigated the properties of double-mutant combinations.
    \begin{itemize}
      \item \textbf{Double-mutant}: two genes can often have genetic interactions with each other, producing an alternative phenotype than they would if isolated.
      \item Genes can have a variety of double mutants, depending on their interaction, i.e., they can be~\cite{perez2009understanding}:
        \begin{enumerate}
          \item Additive, where the interaction between genes can play a supportive role, amplifying general function; 
          \item Epistatic, where gene mutation is dependent on the presence of absence of mutation of one or more genes (modifier genes); 
          \item Suppressive, where one suppresses the other, though not entirely like epistasis, more like the opposite of additive.
          \item Synergistic, where a complete new, or lack of, either original, function occurs.
        \end{enumerate}
    \end{itemize}
    \item The main interest of the authors was to test whether the observed lethality of the double-mutants combination (sec16--1, sec23--1 and sec23--1, sec17--1) resulted from a block in protein transport, or due to a specific double-mutant interaction type.
\end{itemize}

\medskip

\chonk{Techniques} --- \textbf{Pulse-chase analysis}
\begin{itemize}
  \item A pulse-chase experiment is pretty simple and is mainly used for determining activity of certain cells over a prolonged period of time.
  \item The pulse portion, or pulse labeling, is when a group of cells are exposed to a radioactive compound, mainly done to identify the stage at which the messenger RNA is being produced in a cell~\cite{miglani2010developmental}.
  \item In figure 4, carboxypeptidase Y was labeled (pulse) with radioactive [\(^35\)S]methionine. Time is allotted (5 min) for the compound to start to undergo its reaction or metabolic pathway (converted to p1, p2 and m forms), then the chase is added.
  \item The chase is the same unlabeled compound that is introduced in excess later, allowing for the continuation of the process to occur, but without any labeling. This allows for the tracking of the previously labeled compound throughout the remainder of the experiment.
  \item The result was that the double mutants had pronounced secretory defects, showing no conversion of carboxypeptidase Y, indicating the pair of defects combine synergistically to block ER to Golgi transport.
\end{itemize}

\newpage
\chonk{Illustration of Techniques}
\bigskip

\begin{center}
  \Image{\columnwidth}{images/week-4-biorend.png}
\end{center}

\nocite{perez2009understanding}
\bibliographystyle{apacite}
\bibliography{citations.bib}

\end{document}