\documentclass[basic,plain]{inVerba-notes}
\usepackage{inVerba-chem}

\newcommand{\userName}{Cullyn Newman}
\newcommand{\class}{BI:\@ 337}
\newcommand{\theTitle}{Lab 1: Proteins in Membrane}
\newcommand{\institution}{Portland State University}

\begin{document}

\begin{center}
  {\Large\textbf{\rrr{Description of Techniques} and \bbb{Explanation of Concepts}}}
\end{center}

\hrule
\medskip

{\large\textbf{Monolipidated Substrates for S-Palmitoylation Rapidly and Randomly Partition Over All Membranes}}
\begin{itemize}
  \item ``Palmitoylated peripheral membrane proteins typically contain an \bbb{irreversibly attached prenyl or myristoyl moiety} in proximity to the palmitoylation site.''
  \begin{itemize}
    \item \bbb{Prenylation (lipidation)}: the addition of hydrophobic molecules to a protein or chemical compound.
      \begin{itemize}
        \item Prenyl groups (are often assumed to) facilitate attachment to cell membranes, similar to lipid anchors like the GPI anchor.
      \end{itemize}
    \item \bbb{Myristoylation}: a lipidation modification where a myristoyl group (a common saturated fatty acid derived from myristic acid) is covalently attached by an amide bond to the alpha-amino group of an N-terminal glycine residue.
    \begin{itemize}
      \item Myristoylation allows for weak protein–protein and protein–lipid interactions and plays an essential role in membrane targeting, protein–protein interactions, and functions widely in a variety of signal transduction pathways. 
    \end{itemize}
    \item I.e., these \bbb{irreversible modifications} that help facilitate protein-protein and protein-lipid interactions may help \rrr{identify spatial organization} of palmitoylated peripheral membrane proteins, as \bbb{palmitoylation is a reversible lipid modification and thus hard to localize}.
  \end{itemize}
\end{itemize}

\medskip 

\begin{center}
  \chemname{\Image{\textwidth}{images/week-1-a.png}}{\textbf{(A) Steady-State Localization}}
\end{center}
  
\medskip 
  
  \begin{itemize}
    \item ``We first \rrr{questioned whether singly lipidated proteins} exhibit a \bbb{specific membrane distribution to facilitate their subsequent S-palmitoylation}. Mutant solely farnesylated HRasC181,184S and solely myristoylated MyrSer that cannot get palmitoylated were \rrr{equipped with mCherry and mCitrine, respectively}, to observe their \bbb{steady-state localization}. In both cases, the fluorescence distribution did not show a preference for any membrane compartment, but merely reflected membrane densities (Figure A).''
    \begin{itemize}
      \item \bbb{Steady-state localization}: steady state refers to the maintenance of constant internal concentrations of molecules and ions in the cells and organs of living systems, basically homeostasis at a cellular level. 
      \begin{itemize}
        \item I'm uncertain on what the ``localization'' exactly is specifying, but I'm assuming that is means where proteins (cellular organelles, lipids, and possibly others components) actually end up when steady-state is maintained.
      \end{itemize}
      \item The use of \rrr{mCherry and mCitrine (stains)} allowed for \rrr{visualization of the localization via fluorescence} of HRasC181,184S and MySer (singly lipidated proteins); these proteins were \bbb{mutated in order to control for any preemptive palmitoylation}, allowing researchers to \rrr{test for any prior distribution} that effects later palmitoylation---``results reflected membrane densities (Figure A).''
    \end{itemize}
  \end{itemize}
  
  \medskip
  \begin{center}
    \chemname{\Image{\textwidth}{images/week-1-b.png}}{\textbf{(B) Fluorescence Loss of Photoactivated MySer-paGFP}}

    \chemname{\Image{\textwidth}{images/week-1-c.png}}{\textbf{(C) Fluorescence Loss of Photoactivated paGFP-HRasC181S,C184S}}
  \end{center}
  \medskip

\begin{itemize}
  \item ``\bbb{Photoactivatable GFP-fused} versions of both proteins rapidly redistributed over all membranes, reaching steady-state within the first seconds after \rrr{photo-activation} (Figures B/C)''. 
    \begin{itemize}
      \item \rrr{Photo-activation localization}: fluorescence microscopy imaging methods that allow obtaining images with a resolution beyond the diffraction limit targeted biophysical imaging method was largely prompted by the discovery of new species and the engineering of mutants of fluorescent proteins displaying a controllable \bbb{photochromism} (a reversible change of color upon exposure to light, using the photo-activatible GFP).
    \end{itemize}
  \item Essentially, the \rrr{use of photoactivation} allowed for the researchers to \rrr{determine the rate of steady-state localization}. 
\end{itemize}

\medskip
\begin{center}
  \Image{0.9\textwidth}{images/week-1-d.png}
  \chemname{\Image{0.9\textwidth}{images/week-1-e.png}}{\textbf{(D) Total Internal Reflection Fluorescence}}
\end{center}  
\medskip 

\begin{itemize}
  \item  ``In order to confirm that the monolipidated
  proteins also had access to the PM, \rrr{TIRF microscopy was
  performed} on the wild-type and monolipidated mutant proteins.
  As expected, the fully lipidated wild-type proteins clearly
  showed an \bbb{enrichment at the PM} (Figure D).''
    \begin{itemize}
      \item \rrr{Total internal reflection fluorescence (TIRF)}: a fluorescence microscope technique that allows for a thin region of a specimen, usually less than 200 nanometers to be observed. 
        \begin{itemize}
          \item The fluorescence signals from the lipidated proteins were \rrr{normalized to soluble mCherry} to show the \bbb{disparity in contrast reflecting their PM localization}.
          \item Free mCitrine/ Free mCitrine/mCherry images are shown as controls, showing disparity arising due to differences in optical parameters of the TIRF field.
          \item Scale bars represent 10 mM. Color bar indicates normalized range of pixel ratios from minimum (blue) to maximum (red).
        \end{itemize}
      \item \bbb{Enrichment at the plasma membrane (PM)}: depalmitoylation was shown to occur at least at the plasma membrane (El-Hus-seini et al., 2002; Rocks et al., 2005)---this is why it was expected. 
    \end{itemize}
    \item ``However, both \bbb{monolipidated mutants also exhibited clear PM localization}, establishing that \bbb{they have access to this membrane}. These \bbb{experiments are inconsistent with} the presence of receptors for monolipidated proteins on specific membrane compartments \bbb{(Choy et al., 1999)}. Instead, \rrr{proteins with only one attached lipid} rapidly and randomly sample all membranes until they are \bbb{trapped because of an increase in their affinity for membranes} by the acquisition of additional lipid anchors at the site of palmitoylation. This \bbb{kinetic trapping (Shahinian and Silvius, 1995)} is an essential aspect of the spatial organization of palmitoylated peripheral membrane proteins that can be \rrr{exploited to detect the subcellular site of palmitoylation}.''
      \begin{itemize}
        \item \bbb{Kinetic traps}: folding kinetics may trap a protein in a high-energy conformation, i.e., a high-energy intermediate conformation blocks access to the lowest-energy conformation. 
          \begin{itemize}
            \item The authors go on to use this phenomenon, stating that ``kinetic tapping is apparent from a local probe accumulation caused by a decrease in effective diffusion.''
          \end{itemize}
        \item Essentially, what I'm gathering is that \rrr{exploitation of the higher energy state} may lead to different more observable functions, such as decreased effective diffusion, which is the technique used to identify the subcellular cites that are currently (previously?) not well known.
      \end{itemize}
\end{itemize}




\end{document}