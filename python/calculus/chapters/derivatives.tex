\chapter{Derivatives}

\section{Derivative Fundamentals}
\src{\link{https://en.wikipedia.org/wiki/Derivative}{Derivative} | \thomas{3.2}}
\begin{itemize}
  \item \ddd{Derivative}: the measure of \emph{sensitivity to change} of the function \yyy{value} with respect to some change in its in \xxx{argument}.
    \begin{itemize}
      \item Often described as the \emph{instantaneous rate of change} of a single variable function, since it is the slope a tangent line at a particular point, when it exists. 
        \begin{itemize}
          \item \ddd{Tangent line}: the line through a pair of points on a curve \prn{secant line}, except the points are \emph{infinitely close}, thus, it's the rate of change at that ``instant''.
        \end{itemize}
    \end{itemize} 

  \subsection{Definition, Notation}
  \begin{itemize}
    \item Formally, a \emph{derivative} of the function \(f(x)\) with respect to the variable \(x\) is the function \(\fp\) whose value at \(x\) is \prn{provided the limit exists}
    \[%%%%%%%%%%%%%%%%%%%%%%%%%%%%%%%%%%%%%%%%%%%%%%%
      \fp(x) = \lim_{h \to 0} \frac{f(x+h)-f(x)}{h}
    \]%%%%%%%%%%%%%%%%%%%%%%%%%%%%%%%%%%%%%%%%%%%%%%%
    \begin{itemize}
      \item Let \(z = x + h\), then \(h = z-x \land h \to 0 \ifandif z\to x\); this leads to an equivalent definition of the derivative \prn{sometimes more convenient}:
      \[%%%%%%%%%%%%%%%%%%%%%%%%%%%%%%%%%%%%%%%%%%%%%%%
        \fp(x) = \lim_{z \to x} \frac{f(z)-f(x)}{z-x}
      \]%%%%%%%%%%%%%%%%%%%%%%%%%%%%%%%%%%%%%%%%%%%%%%%3
    \end{itemize}
    
    \item \ddd{Notation}: there are many ways to denote the derivative, I prefer Leibniz's, but other notations can be useful in various contexts; some common notations \prn{for \(y = f(x)\)}:
    \[%%%%%%%%%%%%%%%%%%%%%%%%%%%%%%%%%%%%%%%%%%%%%%%
      \fp(x) = y' = \dot{y} = \dydx = \ddx f(x) = D(f)(x) = D_x f(x)
    \]%%%%%%%%%%%%%%%%%%%%%%%%%%%%%%%%%%%%%%%%%%%%%%%
    
    \item \ddd{Differentiation}: the process of finding a derivative; if \(\fp\) exists at a particular point, then \(f\) is said to be differentiable at that point. 
      \begin{itemize}
        \item If \(\fp\) exists at every point on an interval, then \(f\) is differentiable on that interval.
        \item \(\fp\) is differentiable on a closed interval \([a,b]\) if both \hyperref[ss: One-Sided Limit]{\ulink{one-sided limits}} of the function \prn{\(\mathsmaller{h\to\{\rrr{0^+:a},~\bbb{0^-:b}\}}\)} exist at the end points  and is differentiable on the interior. 
        \item Not all continuous functions have a derivative, but \ttt{functions with a derivative are continuous}; functions with any of following \fff{do not have derivatives}:
          \begin{itemize}
            \item \fff{corners} \prn{one-sided derivatives differ at a point}, 
            \item \fff{cusps} \prn{slope approaches alternating \(\pm \infty\) on both sides of a point},
            \item \fff{discontinuities}, or \fff{vertical tangent lines}. 
          \end{itemize}
      \end{itemize}
  \end{itemize}

\end{itemize}

\section{Differentiation Rules}
  \begin{itemize}
    \item 

  \subsection{Linear, Product, Chain, Inverse}
  \begin{itemize}
    \item 
  \end{itemize}
  
  \subsection{Powers, Polynomials, Quotients, Reciprocals}
  \begin{itemize}
    \item 
  \end{itemize}

  \subsection{Exponential, Logarithmic}
  \begin{itemize}
    \item 
  \end{itemize}

  \subsection{Trigonometric, Hyperbolic}
  \begin{itemize}
    \item 
  \end{itemize}

\end{itemize}


\section{Differentials and Related Concepts}
\begin{itemize}
  \item []
  
  \subsection{Differentials}
  \begin{itemize}
    \item 
  \end{itemize}

  \subsection{Linearization}
  \begin{itemize}
    \item 
  \end{itemize}

  \subsection{Implicit Differentiation}
  \begin{itemize}
    \item 
  \end{itemize}

  \subsection{Related Rates}
  \begin{itemize}
    \item 
  \end{itemize}
  
\end{itemize}

