\chapter{Derivatives}

\section{Derivative Fundamentals}
\src{\link{https://en.wikipedia.org/wiki/Derivative}{Derivative} | \thomas{3.2}}
\begin{itemize}
  \item \ddd{Derivative}: the measure of \emph{sensitivity to change} of the function \yyy{value} with respect to some change in its in \xxx{argument}.
    \begin{itemize}
      \item Often described as the \emph{instantaneous rate of change} of a single variable function, since it is the slope a tangent line at a particular point, when it exists. 
        \begin{itemize}
          \item \ddd{Tangent line}: the line through a pair of points on a curve \prn{secant line}, except the points are \emph{infinitely close}, thus, it's the rate of change at that ``instant''.
        \end{itemize}
    \end{itemize} 

  \subsection{Definition, Notation}
  \begin{itemize}
    \item Formally, a derivative of the function \(f(x)\) with respect to the variable \(x\) is the function \(f'\) whose value at \(x\) is \prn{provided the limit exists}
    \[%%%%%%%%%%%%%%%%%%%%%%%%%%%%%%%%%%%%%%%%%%%%%%%
      f'(x) = \lim_{h \to 0} \frac{f(x+h)-f(x)}{h}
    \]%%%%%%%%%%%%%%%%%%%%%%%%%%%%%%%%%%%%%%%%%%%%%%%
    \begin{itemize}
      \item Let \(z = x + h\), then \(h = z-x \land h \to 0 \ifandif z\to x\); this leads to an equivalent definition of the derivative \prn{sometimes more convenient}:
      \[%%%%%%%%%%%%%%%%%%%%%%%%%%%%%%%%%%%%%%%%%%%%%%%
        f'(x) = \lim_{z \to x} \frac{f(z)-f(x)}{z-x}
      \]%%%%%%%%%%%%%%%%%%%%%%%%%%%%%%%%%%%%%%%%%%%%%%%3
    \end{itemize}
    
    \item \ddd{Notation}: there are many ways to denote the derivative; notations can be useful in various contexts, some common notations \prn{for \(y = f(x)\)}:
    \[%%%%%%%%%%%%%%%%%%%%%%%%%%%%%%%%%%%%%%%%%%%%%%%
      f'(x) = y' = \dot{y} = \dydx = \ddx{} f(x) = D(f)(x) = D_x f(x)
    \]%%%%%%%%%%%%%%%%%%%%%%%%%%%%%%%%%%%%%%%%%%%%%%%
    
    \item \ddd{Differentiation}: the process of finding a derivative; if \(f'\) exists at a particular point, then \(f\) is said to be differentiable at that point. 
      \begin{itemize}
        \item If \(f'\) exists at every point on an interval, then \(f\) is differentiable on that interval.
        \item \(f'\) is differentiable on a closed interval \([a,b]\) if both \hyperref[ss: One-Sided Limit]{\ulink{one-sided limits}} of the function \prn{\(\mathsmaller{h\to\{\rrr{0^+:a},~\bbb{0^-:b}\}}\)} exist at the end points  and is differentiable on the interior. 
        \item Not all continuous functions have a derivative, but \ttt{functions with a derivative are continuous}; functions with any of following \fff{do not have derivatives}:
          \begin{itemize}
            \item \fff{corners} \prn{one-sided derivatives differ at a point}, 
            \item \fff{cusps} \prn{slope approaches alternating \(\pm \infty\) on both sides of a point},
            \item \fff{discontinuities}, or \fff{vertical tangent lines}. 
          \end{itemize}
      \end{itemize}
  \end{itemize}

\end{itemize}

\section{Differentiation Rules}
\src{\link{https://en.wikipedia.org/wiki/Differentiation_rules}{Differentiation rules} | \thomas{3.3, 3.5, 3.7}}
  \begin{itemize}
    \item Derivatives can be found by computing its limit, but there are several methods that use of combinations of simpler functions to make computation easier.

  \subsection{Linear, Product, Chain, Inverse}
  \begin{itemize}
    \item \ddd{Linear}: differentiation of linear functions consists of the constant and sum \prn{\& subtraction} rules, given the following
    \[%%%%%%%%%%%%%%%%%%%%%%%%%%%%%%%%%%%%%%%%%%%%%%%
    \forall (f \land g) \land \forall(\ppp{a} \land \ppp{b} \in \R) \then
    \ddx{(\ppp{a}f+\ppp{b}g)} = \ppp{a}\ddx{f} + \ppp{b}\ddx{g}
    \]%%%%%%%%%%%%%%%%%%%%%%%%%%%%%%%%%%%%%%%%%%%%%%%
    \begin{multicols}{3}
      \ddD{Constant} \\ \(\ddx{}(c) = 0\) \\
      \ddD{Constant factor}\\ \((\ppp{a}f)' = \ppp{a}f'\) \\
      \ddD{Sum (Difference)}\\ \((f \mathsmaller{+/-}\, g)' = f' \mathsmaller{+/-}\, g'\) \\
    \end{multicols}

    \item \ddd{Product rule}: used for the product of two functions; \hyperref[ss: General Leibniz Rule]{\dlink{can be generalized}}
    \[%%%%%%%%%%%%%%%%%%%%%%%%%%%%%%%%%%%%%%%%%%%%%%%
    \ddx{(fg)} = \ddx{f}g + f\ddx{g}
    \]%%%%%%%%%%%%%%%%%%%%%%%%%%%%%%%%%%%%%%%%%%%%%%%
    
    \item \ddd{Chain rule}: used for the composition of two functions \(f(g(x))\); if \(\yyy{z}\) depends on \(\y\), which is dependent on \(\x\), then \yyy{z} depends on \(\x\) as well, i.e., 
    \[%%%%%%%%%%%%%%%%%%%%%%%%%%%%%%%%%%%%%%%%%%%%%%%
    \dd{\yyy{z}}{\x} = \dd{\yyy{z}}{\y} \cdot \dd{\y}{x}
    \]%%%%%%%%%%%%%%%%%%%%%%%%%%%%%%%%%%%%%%%%%%%%%%%
    The following is used to indicate which points the derivatives have to evaluated at:
    \[%%%%%%%%%%%%%%%%%%%%%%%%%%%%%%%%%%%%%%%%%%%%%%%
    \left.\dd{\yyy{z}}{\x}\right|_\x = \left.\dd{\yyy{z}}{\y}\right|_{\y(\x)} \cdot \left.\dd{\y}{\x}\right|_\x
    \]%%%%%%%%%%%%%%%%%%%%%%%%%%%%%%%%%%%%%%%%%%%%%%%
    \begin{itemize}
      \item \ddD{``Outside-Inside Rule''}: take the derivative of the ``outside'' function, leave ``inside'' alone, and multiply it be the derivative of the ``inside.''
      \item This method must be recursively ``chained'' when there are further compositions in the inside function, hence the name. 
    \end{itemize}
  
    \item \ddd{Inverse function rule}: can be applied if the function \(f\) has an inverse function \(g\), i.e., a function that ``undoes'' the effect of \(f\). 
    \[%%%%%%%%%%%%%%%%%%%%%%%%%%%%%%%%%%%%%%%%%%%%%%%
    \ttt{\left\{\fg{g(f(\x)) = \x \land f(g(\y)) = \y}\right\}} \then \dd{\x}{\y} = \left(\dd{\y}{\x}\right)^{-1}
    \]%%%%%%%%%%%%%%%%%%%%%%%%%%%%%%%%%%%%%%%%%%%%%%%
      \begin{itemize}
        \item Application of the chain rule on \(f ^{-1} (\y) = \x\) in terms of \(\x\) clearly shows the result if the derivatives exist and are reciprocal, 
        \[%%%%%%%%%%%%%%%%%%%%%%%%%%%%%%%%%%%%%%%%%%%%%%%
        \dd{\x}{\y} \cdot \dd{\y}{\x} = \dd{\x}{\x} = 1
        \]%%%%%%%%%%%%%%%%%%%%%%%%%%%%%%%%%%%%%%%%%%%%%%%
        
      \end{itemize}
  \end{itemize}
  
  \subsection{Powers, Polynomials, Quotients, Reciprocals}
  \begin{itemize}
    \item 
  \end{itemize}

  \subsection{Exponential, Logarithmic}
  \begin{itemize}
    \item 
  \end{itemize}

  \subsection{Trigonometric, Hyperbolic}
  \begin{itemize}
    \item 
  \end{itemize}

\end{itemize}


\section{Differentials and Related Concepts}
\begin{itemize}
  \item []
  
  \subsection{Differentials}
  \begin{itemize}
    \item 
  \end{itemize}

  \subsection{Linearization}
  \begin{itemize}
    \item 
  \end{itemize}

  \subsection{Implicit Differentiation}
  \begin{itemize}
    \item 
  \end{itemize}

  \subsection{Related Rates}
  \begin{itemize}
    \item 
  \end{itemize}
  
\end{itemize}

