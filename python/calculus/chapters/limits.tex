\chapter{Limits and Continuity}

\section{Limits}

\src{\link{https://en.wikipedia.org/wiki/Limit_(mathematics)}{Limit (Wikipedia)} | \link{https://book4you.org/s/0134438981}{Thomas' Calculus (2.2--2.6)}}

\smallskip

\begin{itemize}
  \item \ddd{Limit \(\lim_{x \to c}\)}: the value of a function \prn{or sequence} approaches as the input \prn{or index} approaches some value \prn{informal definition}  
  \begin{itemize}
    \item Limits are used to define \hyperref[s: Continuity]{\dlink{continuity}}, \hyperref[c: Derivatives]{\dlink{derivatives}}, and \hyperref[c: Integrals]{\dlink{integrals}}
  \end{itemize}

  \subsection{Limits of a Functions and Sequences}
  \src{\link{https://en.wikipedia.org/wiki/Limit_of_a_function}{Limit of a function (Wikipedia)} | \link{https://en.wikipedia.org/wiki/Limit_of_a_sequence}{Limit of a sequence (Wikipedia)}}
  \begin{itemize}
    \item \ddd{Limit of a function}: a fundament concept in calculus and analysis concerning the behavior \(\ppp{L}\) of a function near a particular input \(\yyy{p}\), i.e., 
    \[%%%%%%%%%%%%%%%%%%%%%%%%%%%%%%%%%%%%%%%%%%%%%%%
    \lim_{\x \to\yyy{p}}\f(\x) = \ppp{L}
    \]%%%%%%%%%%%%%%%%%%%%%%%%%%%%%%%%%%%%%%%%%%%%%%%
    \begin{itemize}
      \item Reads as ``\(\f\) of \(\x\) tends to \ppp{L} as \(\x\) tends to \(\yyy{p}\)''
    \end{itemize}
    \item \ddd{\(\epsilon,\delta\) Limit of function}: a formalized definition, wherein \(f(x)\) is defined on an open interval, except possibly at c itself, leading to above definition, \ssec{if and only if}:
      \begin{itemize}
        \item For every real \(\bbb{\epsilon > 0}\), there exists a corresponding real \(\rrr{\delta >0}\) such that for all real \(\x, 0 < | \x - \yyy{p} | < \delta\) implies that \(| \f(x) - \ppp{L} | < \bbb{\epsilon}\), i.e.,
        \[%%%%%%%%%%%%%%%%%%%%%%%%%%%%%%%%%%%%%%%%%%%%%%%
        \f : \R \to \R,~ \yyy{p}, \ppp{L} \in \R \Rightarrow \lim_{\x \to\yyy{p}}\f(\x) = \ppp{L} 
        \]%%%%%%%%%%%%%%%%%%%%%%%%%%%%%%%%%%%%%%%%%%%%%%%
        \vspace{-24pt}
        \[%%%%%%%%%%%%%%%%%%%%%%%%%%%%%%%%%%%%%%%%%%%%%%%
        \ssec{\Updownarrow}
        \]%%%%%%%%%%%%%%%%%%%%%%%%%%%%%%%%%%%%%%%%%%%%%%%
        \[%%%%%%%%%%%%%%%%%%%%%%%%%%%%%%%%%%%%%%%%%%%%%%%
        \forall\bbb{\depsilon >0}\left(
          \exists\rrr{\delta >0} : \forall \x, 0 < | \x - \yyy{p} | < \rrr{\delta} 
          \Rightarrow 
          | \f(\x) - \ppp{L} | < \bbb{\depsilon}
        \right)
        \]%%%%%%%%%%%%%%%%%%%%%%%%%%%%%%%%%%%%%%%%%%%%%%%
        
      \end{itemize}
    \item Functions do not have a limit when the function:
      \begin{itemize}
        \item has a unit step, i.e., it ``jumps'' at a point;
        \item is not bounded, i.e., it tends towards infinity;
        \item or does not stay close to any single number, i.e., it oscillates too much. 
      \end{itemize}
    \item \ddd{Limit of a sequence}: the value that the terms of a sequence (\(x_\ppp{n}\)) ``tends to'' \prn{and not to any other} as \(\ppp{n}\) approaches infinity \prn{or some point}, i.e.,
    \[%%%%%%%%%%%%%%%%%%%%%%%%%%%%%%%%%%%%%%%%%%%%%%%
    \lim_{\ppp{n} \to \yyy{\infty}} x_\ppp{n} = x
    \]%%%%%%%%%%%%%%%%%%%%%%%%%%%%%%%%%%%%%%%%%%%%%%%
      \item \ddd{\(\epsilon \) Limit of sequence}: for every measure of closeness \(\bbb{\epsilon} \), the sequence's term eventually converge to the limit, i.e.,
      \[%%%%%%%%%%%%%%%%%%%%%%%%%%%%%%%%%%%%%%%%%%%%%%%
      \forall\bbb{\depsilon > 0} \left(
        \exists \ppp{N} \in \N \left(
          \forall\ppp{n} \in \N \left(
            \ppp{n} \geq \ppp{N} \Rightarrow |x_\ppp{n}-x| < \bbb{\depsilon}
          \right)
        \right)
      \right)
      \]%%%%%%%%%%%%%%%%%%%%%%%%%%%%%%%%%%%%%%%%%%%%%%%
    \begin{itemize}  
      \item \ddd{Convergent}: when a limit of a sequence \ttt{exists}.
      \item \ddd{Divergent}: a sequence that \fff{does not} converge. 
    \end{itemize}
  \end{itemize}

  \subsection{Limit Laws and Theorems}
  \begin{itemize}
    \item 
  \end{itemize}
  
\end{itemize}

\section{Continuity}
\begin{itemize}
  \item Sources:
  
  \subsection{Continuity at a Point}
  \begin{itemize}
    \item 
  \end{itemize}

  \subsection{Continuous Functions}
  \begin{itemize}
    \item 
  \end{itemize}
  
  \subsection{Intermediate Value Theorem}
  \begin{itemize}
    \item 
  \end{itemize}
  
\end{itemize}

\section{Limits Involving Infinity}
\begin{itemize}
  \item Sources:

  \subsection{Limits at Infinity}
  \begin{itemize}
    \item 
  \end{itemize}

  \subsection{Infinite Limits}
  \begin{itemize}
    \item 
  \end{itemize}
  
  
\end{itemize}






