\chapter{Limits and Continuity}

\section{Limits}

\src{\link{https://en.wikipedia.org/wiki/Limit_(mathematics)}{Limit} | \link{https://book4you.org/s/0134438981}{Thomas' Calculus (2.2--2.4)}}

\smallskip

\begin{itemize}
  \item \ddd{Limit \(\lim_{x \to c}\)}: the value of a function \prn{or sequence} approaches as the input \prn{or index} approaches some value \prn{informal definition}  
  \begin{itemize}
    \item Limits are used to define \hyperref[s: Continuity]{\dlink{continuity}}, \hyperref[c: Derivatives]{\dlink{derivatives}}, and \hyperref[c: Integrals]{\dlink{integrals}}
  \end{itemize}

  \subsection{Limits of a Functions and Sequences}
  \src{\link{https://en.wikipedia.org/wiki/Limit_of_a_function}{Limit of a function} | \link{https://en.wikipedia.org/wiki/Limit_of_a_sequence}{Limit of a sequence}}
  \begin{itemize}
    \item \ddd{Limit of a function}: a fundament concept in calculus and analysis concerning the behavior \(\ppp{L}\) of a function near a particular input \(\yyy{c}\), i.e., 
    \[%%%%%%%%%%%%%%%%%%%%%%%%%%%%%%%%%%%%%%%%%%%%%%%
    \lim_{\x \to\yyy{c}}\f(\x) = \ppp{L}
    \]%%%%%%%%%%%%%%%%%%%%%%%%%%%%%%%%%%%%%%%%%%%%%%%
    \begin{itemize}
      \item Reads as ``\(\f\) of \(\x\) tends to \ppp{L} as \(\x\) tends to \(\yyy{c}\)''
    \end{itemize}
    \item \ddd{\(\epsilon,\delta\) Limit of function}: a formalized definition, wherein \(f(x)\) is defined on an open interval, except possibly at c itself, leading to above definition, \ssec{if and only if}:
      \begin{itemize}
        \item For every real measure of \bbb{closeness} \(\bbb{\epsilon > 0}\), there exists a real \rrr{corresponding} \(\rrr{\delta >0}\), such that for all existing further approaches there exist a smaller \(\bbb{\epsilon}\), i.e.,
        \[%%%%%%%%%%%%%%%%%%%%%%%%%%%%%%%%%%%%%%%%%%%%%%%
        \f : \R \to \R,~ \yyy{c}, \ppp{L} \in \R \Rightarrow \lim_{\x \to\yyy{c}}\f(\x) = \ppp{L} 
        \]%%%%%%%%%%%%%%%%%%%%%%%%%%%%%%%%%%%%%%%%%%%%%%%
        \vspace{-24pt}
        \[%%%%%%%%%%%%%%%%%%%%%%%%%%%%%%%%%%%%%%%%%%%%%%%
        \ssec{\Updownarrow}
        \]%%%%%%%%%%%%%%%%%%%%%%%%%%%%%%%%%%%%%%%%%%%%%%%
        \[%%%%%%%%%%%%%%%%%%%%%%%%%%%%%%%%%%%%%%%%%%%%%%%
        \forall\bbb{\depsilon >0}\left(
          \exists\rrr{\delta >0} : \forall \x, 0 < | \x - \yyy{c} | < \rrr{\delta} 
          \Rightarrow 
          | \f(\x) - \ppp{L} | < \bbb{\depsilon}
        \right)
        \]%%%%%%%%%%%%%%%%%%%%%%%%%%%%%%%%%%%%%%%%%%%%%%%
        
      \end{itemize}
    \item Functions do not have a limit when the function:
      \begin{itemize}
        \item has a unit step, i.e., it ``jumps'' at a point;
        \item is not bounded, i.e., it tends towards infinity;
        \item or does not stay close to any single number, i.e., it oscillates too much. 
      \end{itemize}
    \item \ddd{Limit of a sequence}: the value that the terms of a sequence (\(x_\ppp{n}\)) ``tends to'' \prn{and not to any other} as \(\ppp{n}\) approaches infinity \prn{or some point}, i.e.,
    \[%%%%%%%%%%%%%%%%%%%%%%%%%%%%%%%%%%%%%%%%%%%%%%%
    \lim_{\ppp{n} \to \yyy{\infty}} x_\ppp{n} = x
    \]%%%%%%%%%%%%%%%%%%%%%%%%%%%%%%%%%%%%%%%%%%%%%%%
      \item \ddd{\(\epsilon \) Limit of sequence}: for every measure of closeness \(\bbb{\epsilon} \), the sequence's term eventually converge to the limit, i.e.,
      \[%%%%%%%%%%%%%%%%%%%%%%%%%%%%%%%%%%%%%%%%%%%%%%%
      \forall\bbb{\depsilon > 0} \left(
        \exists \ppp{N} \in \N \left(
          \forall\ppp{n} \in \N \left(
            \ppp{n} \geq \ppp{N} \Rightarrow |x_\ppp{n}-x| < \bbb{\depsilon}
          \right)
        \right)
      \right)
      \]%%%%%%%%%%%%%%%%%%%%%%%%%%%%%%%%%%%%%%%%%%%%%%%
    \begin{itemize}  
      \item \ddd{Convergent}: when a limit of a sequence \ttt{exists}.
      \item \ddd{Divergent}: a sequence that \fff{does not} converge. 
    \end{itemize}
  \end{itemize}

  \subsection{Properties of Limits}
  \src{\link{https://en.wikipedia.org/wiki/List_of_limits}{List of limits} | \link{https://en.wikipedia.org/wiki/Squeeze_theorem}{Squeeze theorem}}
  \begin{itemize}
    \item \ddd{Operations on a single known limit}: if \(\lim_{\x \to \yyy{c}}f(\x)=\ppp{L}\) then:
      \begin{itemize}
        \item \(\lim_{\x \to \yyy{c}}[f(\x)\pm \alpha] = \ppp{L} \pm \alpha\)  
        \item \(\lim_{\x \to \yyy{c}}\alpha f(\x)= \alpha \ppp{L}\)
        \item \(\lim_{\x \to \yyy{c}}f(\x)^{-1}=\ppp{L}^{-1}, \ppp{L}\neq 0\)
        \item \(\lim_{\x \to \yyy{c}}f(\x)^n = \ppp{L}^n, n \in \N\)
        \item \(\lim_{\x \to \yyy{c}}f(\x)^{n ^{-1}} = \ppp{L}^{n ^{-1}}, n \in \N,~\text{if}~n \in \N_e \then \ppp{L} > 0\) 
      \end{itemize}
    \item \ddd{Operations on two known limits}: if \(\lim_{\x \to \yyy{c}}f(\x)= \ppp{L_1}\) and \(\lim_{\x \to \yyy{c}}g(\x)=\ppp{L_2}\) 
      \begin{itemize}
        \item \(\lim_{\x \to \yyy{c}}[f(\x)\pm g(\x)]=\ppp{L_1}\pm\ppp{L_2}\)
        \item \(\lim_{\x \to \yyy{c}}[f(\x)g(\x)] = \ppp{L_1 L_2}\)
        \item \(\lim_{\x \to \yyy{c}}f(\x)g(\x)^{-1} = \ppp{L_1}\ppp{L_2}^{-1}\)
      \end{itemize}
    \smallskip
    \item \ddd{Squeeze theorem}: used to confirm the limit of a function via comparison with two other functions whose limits are easily known or computed.
      \begin{itemize}
        \item Let \(\bBb{I}\) be an interval having the point \(\yyy{a}\) as a limit point. 
        \item Let \(g,f\), and \(h\), be functions defined on \(\bBb{I}\), except possibly at \(\yyy{a}\) itself. 
        \item Suppose that \(\forall \x \in \bBb{I}~\land \neq \yyy{a} \then \bbb{g(\x)}\leq f(\x) \leq \rrr{h(\x)}\)
        \item And suppose that \(\lim_{\x \to \yyy{a}}\bbb{g(\x)}=\lim_{\x \to \yyy{a}}\rrr{h(\x)} = \ppp{L}\)
        \item Then, \(\lim_{\x \to \yyy{a}}f(\x) = \ppp{L}\)
        \item Essentially, the hard to compute limit of the ``middle function'' is found by finding two other easy functions that ``squeeze'' the middle function at that point.
      \end{itemize}
  \end{itemize}

  \subsection{One-Sided Limit}
  \src{\link{https://en.wikipedia.org/wiki/One-sided_limit}{One-Sided Limit}}
  \begin{itemize}
    \item 
  \end{itemize}
  
\end{itemize}

\section{Continuity}
\begin{itemize}
  \item Sources:
  
  \subsection{Continuity at a Point}
  \begin{itemize}
    \item 
  \end{itemize}

  \subsection{Continuous Functions}
  \begin{itemize}
    \item 
  \end{itemize}
  
  \subsection{Intermediate Value Theorem}
  \begin{itemize}
    \item 
  \end{itemize}
  
\end{itemize}

\section{Limits Involving Infinity}
\begin{itemize}
  \item Sources:

  \subsection{Limits at Infinity}
  \begin{itemize}
    \item 
  \end{itemize}

  \subsection{Infinite Limits}
  \begin{itemize}
    \item 
  \end{itemize}
  
  
\end{itemize}






