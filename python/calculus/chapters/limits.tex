\chapter{Limits and Continuity}

\section{Limits}

\src{\link{https://en.wikipedia.org/wiki/Limit_(mathematics)}{Limit (Wikipedia)} | \link{https://book4you.org/s/0134438981}{Thomas' Calculus (2.2--2.6)}}

\smallskip

\begin{itemize}
  \item \ddd{Limit \(\lim_{x \to c}\)}: the value of a function \prn{or sequence} approaches as the input \prn{or index} approaches some value \prn{informal definition}  
  \begin{itemize}
    \item Limits are used to define \hyperref[s: Continuity]{\dlink{continuity}}, \hyperref[c: Derivatives]{\dlink{derivatives}}, and \hyperref[c: Integrals]{\dlink{integrals}}
  \end{itemize}

  \subsection{Limits of a Functions and Sequences}
  \src{\link{https://en.wikipedia.org/wiki/Limit_of_a_function}{Limit of a function (Wikipedia)} | \link{https://en.wikipedia.org/wiki/Limit_of_a_sequence}{Limit of a sequence (Wikipedia)}}
  \begin{itemize}
    \item \ddd{Limit of a function}: a fundament concept in calculus and analysis concerning the behavior \(\emph{L}\) of a function near a particular input \(\yyy{p}\), i.e., 
    \[%%%%%%%%%%%%%%%%%%%%%%%%%%%%%%%%%%%%%%%%%%%%%%%
    \lim_{\x \to\yyy{p}}\f(\x) = \emph{L}
    \]%%%%%%%%%%%%%%%%%%%%%%%%%%%%%%%%%%%%%%%%%%%%%%%
    \begin{itemize}
      \item Reads as ``\(\f\) of \(\x\) tends to \emph{L} as \(\x\) tends to \(\yyy{p}\)''
    \end{itemize}
    \item Functions do not have a limit when the function:
      \begin{itemize}
        \item has a unit step, i.e., it ``jumps'' at a point;
        \item is not bounded, i.e., it approaches tends towards infinity;
        \item or does not stay close to any single number, i.e., it oscillates too much. 
      \end{itemize}
    \item \ddd{Limit of a sequence}: the value that the terms of a sequence ``tends to'' as \(n\) approaches infinity \prn{or some point}, i.e.,
    \[%%%%%%%%%%%%%%%%%%%%%%%%%%%%%%%%%%%%%%%%%%%%%%%
    \lim_{\xxx{n} \to \yyy{\infty}} a_\xxx{n} x_n = c
    \]%%%%%%%%%%%%%%%%%%%%%%%%%%%%%%%%%%%%%%%%%%%%%%%
    \begin{itemize}
      \item \ddd{Convergent}: when a limit of a sequence \ttt{exists}.
      \item \ddd{Divergent}: a sequence that \fff{does not} converge. 
      \item 
    \end{itemize}
  \end{itemize}

  \subsection{Limit Laws and Theorems}
  \begin{itemize}
    \item 
  \end{itemize}
  
\end{itemize}

\section{Continuity}
\begin{itemize}
  \item Sources:
  
  \subsection{Continuity at a Point}
  \begin{itemize}
    \item 
  \end{itemize}

  \subsection{Continuous Functions}
  \begin{itemize}
    \item 
  \end{itemize}
  
  \subsection{Intermediate Value Theorem}
  \begin{itemize}
    \item 
  \end{itemize}
  
\end{itemize}

\section{Limits Involving Infinity}
\begin{itemize}
  \item Sources:

  \subsection{Limits at Infinity}
  \begin{itemize}
    \item 
  \end{itemize}

  \subsection{Infinite Limits}
  \begin{itemize}
    \item 
  \end{itemize}
  
  
\end{itemize}






