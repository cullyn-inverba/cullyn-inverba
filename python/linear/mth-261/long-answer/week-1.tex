\documentclass[basic]{inVerba-notes}
\usepackage{inVerba-math}

\newcommand{\userName}{Cullyn Newman}
\newcommand{\class}{MTH:\@ 261}
\newcommand{\theTitle}{Solution Sets and Gaussian Elimination}
\newcommand{\institution}{Portland State University}

\begin{document}

\begin{enumerate}
  \minor{\item Consider a system of linear equations with augmented matrix \tbm{A} and coefficient matrix \tbm{C}. In each case explain why the statement is true or give an example showing that it is false.}
  \begin{enumerate}
    \minor{\item If there is more than one solution, \tbm{A} has a row of zeros.}
    \begin{itemize}
    \item If a row with all zeros occurs, then that row added no new information and was only a multiple of another row, i.e., the system is reduced rank. 
    \item Matrix rank corresponds to the maximal number of linearly independent columns. 
    \item Linear independence is when no vector in the matrix can be expressed as a linear combination, i.e., 
    \[%%%%%%%%%%%%%%%%%%%%%%%%%%%%%%%%%%%%%%%%%%%%%%%
    a_1\bm{v}_1 +  a_2\bm{v}_2 +  a_3\bm{v}_3 +  a_n\bm{v}_n   
    \]%%%%%%%%%%%%%%%%%%%%%%%%%%%%%%%%%%%%%%%%%%%%%%%
    \item For example, in a 3-D vector space \(\mathbb{R}^3\), then any vector in the space is can be made by a linear combination of the following three vectors \(\bm{e}_1,~\bm{e}_2,~\bm{e}_3\) (unit vectors) multiplied by some scalar \(\lambda \):
    \[%%%%%%%%%%%%%%%%%%%%%%%%%%%%%%%%%%%%%%%%%%%%%%%
    \lambda\begin{bmatrix} 1 \\ 0 \\ 0 \end{bmatrix} +
    \lambda\begin{bmatrix} 0 \\ 1 \\ 0 \end{bmatrix} + 
    \lambda\begin{bmatrix} 0 \\ 0 \\ 1 \end{bmatrix}
    \]%%%%%%%%%%%%%%%%%%%%%%%%%%%%%%%%%%%%%%%%%%%%%%%
    \item If the third column vector was all zeros (making the matrix have a row of all zeros) then it would just be describing a plane in \(\mathbb{R}^2\), with the last row (or rather, third vector) contributing no information to the system,
    \[%%%%%%%%%%%%%%%%%%%%%%%%%%%%%%%%%%%%%%%%%%%%%%%
    \lambda\begin{bmatrix} 1 \\ 0 \\ \minimal{0} \end{bmatrix} +
    \lambda\begin{bmatrix} 0 \\ 1 \\ \minimal{0} \end{bmatrix} + 
    \minimal{\lambda\begin{bmatrix} 0 \\ 0 \\ 0 \end{bmatrix}}
    \]%%%%%%%%%%%%%%%%%%%%%%%%%%%%%%%%%%%%%%%%%%%%%%%
    \item That being said, the matrix above could be an augmented matrix with the constant matrix containing values for only the first two vectors, indicating a single solution. 
    \[%%%%%%%%%%%%%%%%%%%%%%%%%%%%%%%%%%%%%%%%%%%%%%%
    \begin{bmatrix}[ccc|c]
    1 & 0 & \minimal{0} & 6 \\
    0 & 1 & \minimal{0} & 9 \\
    \minimal{0} & \minimal{0} & \minimal{0} & \minimal{0}
    \end{bmatrix}
    \]%%%%%%%%%%%%%%%%%%%%%%%%%%%%%%%%%%%%%%%%%%%%%%%
    \item And now I've realized I actually answered why the statement in (b) is false and not directly addressing the validity of (a).
    \item (a) is false, the row of zeros puts no restriction on the \rrr{parameter}, but the row doesn't have to be there; the rank just has to be less than the number of columns. 
    \[%%%%%%%%%%%%%%%%%%%%%%%%%%%%%%%%%%%%%%%%%%%%%%%
    \begin{bmatrix}[ccc|c]
      1 & 0 & \rrr{2} & 4 \\
      0 & 1 & \minimal{0} & 2 \\
      \minimal{0} & \minimal{0} & \minimal{0} & \minimal{0}
    \end{bmatrix}
    \Arrow 
    \begin{bmatrix}[ccc|c]
      1 & 0 & \rrr{2} & 4 \\
      0 & 1 & \minimal{0} & 2 \\
    \end{bmatrix}~\text{row zeros doesn't have to exist}
    \]%%%%%%%%%%%%%%%%%%%%%%%%%%%%%%%%%%%%%%%%%%%%%%%
    \end{itemize}
    \minor{\item If \tbm{A} has a row of zeros, there is more than one solution.}
    \begin{itemize}
      \item Not necessarily, see above answer.
    \end{itemize}
    \minor{\item If there is no solution, the reduced row-echelon form of \tbm{C} has a row of zeros.}
      \begin{itemize}
        \item True, the rref form needs to have a row of zeros in the coefficient matrix and a non-zero in the same row in the constant matrix, i.e., \(\begin{bmatrix}[cccc|c] 0 & 0 & 0 &\cdots & a \neq 0 \end{bmatrix}\), in order to have no solution. 
      \end{itemize}
    \minor{\item If the row-echelon form of \tbm{C} has a row of zeros, there is no solution.}
    \begin{itemize}
      \item Not necessarily, a row of zeros is more likely to indicate infinite solutions than no solution or a single solution, albeit, all are still possible. 
      \item For example, the following matrix is in ref and contains a row of all zeros.
      \[%%%%%%%%%%%%%%%%%%%%%%%%%%%%%%%%%%%%%%%%%%%%%%%
      \begin{bmatrix}[ccc|c]
        1 & 6 & 9 & 4 \\
        0 & 1 & 0 & 2 \\
        0 & 0 & 0 & 0
      \end{bmatrix}
      \]%%%%%%%%%%%%%%%%%%%%%%%%%%%%%%%%%%%%%%%%%%%%%%%
    \end{itemize}
\end{enumerate}

\minor{\item 

\[\text{Verify that} \begin{cases}
x_1 & = 2s + 12t +13 \\
x_2 & = s \\
x_3 & = -s - 3t - 3 \\
x_4 & = t
\end{cases} \quad \text{is a solution of }
\begin{cases}
2x_1 + 5x_2 + 9x_3 + 3x_4 &= -1 \\
\hphantom{2}x_1 + 2x_2 + 4x_3  &= 1.
\end{cases}
\]}
\begin{align*}
  \begin{bmatrix}[cccc|c]
    2 & 5 & 9 & 3 & -1 \\
    1 & 2 & 4 & 0 & 1 \\
  \end{bmatrix}
  ~R_1\updownarrow R_2~
  \begin{bmatrix}[cccc|c]
    1 & 2 & 4 & 0 & 1 \\
    2 & 5 & 9 & 3 & -1 \\
  \end{bmatrix}\\ 
  ~-2R_1 + R_2~
  \begin{bmatrix}[cccc|c]
    1 & 2 & 4 & 0 & 1 \\
    0 & 1 & 1 & 3 & -3 \\
  \end{bmatrix}\\
  -2R_2 + R_1~
  \begin{bmatrix}[cccc|c]
    1 & 0 & 2 & -6 & 7 \\
    0 & 1 & 1 & 3 & -3 \\
  \end{bmatrix}\\
\end{align*}
Two parameters are left: \(s,~t\). 
\begin{align*}
  x_1 &= 2s - 6t + 7 \\
  x_2 &= s \\
  x_3 &= -s - 3t - 3 \\
  x_4 &= t \\
\end{align*}
I am confused.

Let's try something different:
\begin{align*}
  2(2s + 12t + 13) + 5(s) + 9(-s -3t -3) + 3(t) = -1 \\
  4s + 24t + 26 + 5s - 9s - 27t - 27 + 3t = -1 \\
  24t - 27t+ 3t + 26 -27  = - 1 \\ 
  26 - 27 = -1 \\
  \ttt{-1 = -1} \\
  2s + 12t + 13 + 2(s) + 4(-s -3t - 3) = 1 \\
  2s + 12t + 13 + 2s - 4s -12t - 12 = 1 \\
  12t - 12t + 13 - 12 = 1 \\
  \ttt{1 = 1}
\end{align*}
Hmmm\dots why doesn't rref work.
\begin{align*}
  2(2s - 6t + 7) + 5(s) + 9 (-s -3t -3) + 3(t) = - 1 \\
  4s - 12t + 14 + 5s - 9s - 27t - 27 + 3t = - 1 \\
  -12t + 14 - 27t -27 + 3t = -1 \\
  \text{yeah, no}  
\end{align*}
Did I do parameters wrong?
\[%%%%%%%%%%%%%%%%%%%%%%%%%%%%%%%%%%%%%%%%%%%%%%%
\begin{bmatrix}[cccc|c]
  1 & 0 & 2 & -6 & 7 \\
  0 & 1 & 1 & 3 & -3 \\
\end{bmatrix}
\]%%%%%%%%%%%%%%%%%%%%%%%%%%%%%%%%%%%%%%%%%%%%%%%
\begin{align*}
  x_1 &= 2s~\rrr{+}~6t + 7 \\
  x_2 &= -s -3t - 3 \\
  x_3 &= s \\
  x_4 &= t
\end{align*}
Idk, let's try it.
\begin{align*}
  2(2s + 6t + 7) + 5(-s -3t -3) + 9(s) + 3(t) &= -1 \\
  4s + 12t + 14 -10s -15t - 15 + 9s + 3t &= -1 \\
  3s - 1 &= -1 \\ 
  s &= 0
\end{align*}
Well, why doesn't that work? I guess rref removes information\dots or maybe I'm doing something wrong with parameters.

Update after a couple of days later: the rank is 2, with 4 columns, i.e., \(r < n\), thus there are infinite solutions. You just gave us one of many possible solutions, hence why \(s=0\) I guess? \(s\), and I bet \(t\) if I worked it out, would both be zero, or unrestricted.
\end{enumerate}
\end{document}