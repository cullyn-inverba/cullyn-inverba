\documentclass[basic]{inVerba-notes}
\usepackage{inVerba-math}
% chktex-file 3
\newcommand{\userName}{Cullyn Newman}
\newcommand{\class}{MTH:\@ 261 | Sauls}
\newcommand{\theTitle}{Abstraction Gratification}
\newcommand{\institution}{Portland State University}

\newcommand{\R}{\mathbb{R}}

\begin{document}

\begin{enumerate}[align=left, leftmargin=0pt, labelindent=\parindent, listparindent=\parindent, labelwidth=0pt, itemindent=!]\color{minor}
  \item Consider the set of all points in the region \(U\) shown in \(\R^2\) below. Assume the set includes the boundary line. Give a specific reason why the set \(U\) is \textit{not} a subspace of \(\mathbb{R}^2\).
   
  \medskip
  \begin{center}
    \begin{tikzpicture}
      
      \draw[-,color=text] (-2.9 ,0.) -- (2.9,0.);
      \draw[-,color=text] (0.,-.9) -- (0.,0.9);
      \fill[color=bbb,fill=bbb,fill opacity=0.1] (-3.,1.) -- (3.,-1.) -- (3.,1.) -- cycle;
      \draw [line width=1.5pt,color=bbb] (-3.,1)-- (3.,-1.);
      \draw (1.2,0.7) node[anchor=north west] {\small{\({\color{ssec}U}\)}};
      
    \end{tikzpicture}
  \end{center}
  \medskip 

  \basec{\begin{itemize}
    \item If the set of vectors that describe \(\R^2\) is defined as \(V\), then \ssec{\(U\)} is just a subset of \(V\). All subspaces are subsets, but not all subsets are subspaces.
      \begin{itemize}
        \item Subsets don't need to include the origin, don't need to be closed under addition and scalar multiplication, or could have boundaries.
      \end{itemize}
    \item That being said, a \chap{vector subspace \(L\)} must be closed under addition and \emph{scalar \(\lambda, \alpha \)} multiplication while also containing the zero vector, i.e., 
    \[%%%%%%%%%%%%%%%%%%%%%%%%%%%%%%%%%%%%%%%%%%%%%%%
    \forall~\bm{v}, \bm{w} \in \chap{L};\quad \forall~\emph{\lambda}, \emph{\alpha}\in \R; \quad \emph{\lambda} \bm{v} + \emph{\alpha}\bm{w} \in \chap{L}
    \]%%%%%%%%%%%%%%%%%%%%%%%%%%%%%%%%%%%%%%%%%%%%%%%
    \begin{itemize}
      \item This shows how the region \ssec{\(U\)} is not a subspace, because if you take two vectors in \ssec{\(U\)}, then some linear combinations would yield points that were not in \ssec{\(U\)}.
    \end{itemize}
    \item Interestingly, the boundary line of \ssec{\(U\)} is a linear subspace of \(V\).
  \end{itemize}}

  \vspace*{50pt}

  \item Let \(W\) be the set of all points on either the \(x\)- or \(y\)- axis. That is, \(W\) is all the points of the form \(\begin{bmatrix} a \\ 0 \end{bmatrix}\) or \(\begin{bmatrix} 0 \\ b \end{bmatrix}\) for any real numbers \(a\) and \(b\). Show that \(W\) is not a subspace of \(\R^2\)
  
  \bigskip

  \basec{\begin{itemize}
    \item \(W\) is not a subspace of \(\R^2\) because \(W\) is \(\R^2\). This is because any vector in \(\R^2\) can be made via a linear combination of \(\begin{bmatrix} a \\ 0 \end{bmatrix}\) and \(\begin{bmatrix} 0 \\ b \end{bmatrix}\).
    \item \(W\) would have to be limited to only the \(x\)- or \(y\)- axis, but not both, in order to be a subspace of \(\R^2\). 
    \item If \(W\) contains both axes, but not any other points on the plane, then one is putting boundaries the space and limiting the ability to take linear combination of vectors in the subspace, making it only subset.
  \end{itemize}}
  
  \newpage

  \item Consider \(A\) and its reduced row echelon form below.
  \[
  A=\begin{bmatrix*}[r] -3 & 9 & -2 & -7 \\
  2 & -6 & 4 & 8 \\
  3 & -9 & -2 & 2\end{bmatrix*}
  \rightarrow
  \underbrace{\begin{bmatrix*}[r]1 & -3 & 0 & 3/2\\
  0 & 0 & 1 & 5/4 \\
  0 & 0 & 0 & 0\end{bmatrix*}}_{\text{rref of }A}
  \]
  Let \(\bm{b}_1\), \(\bm{b}_2\), \(\bm{b}_3\), and \(\bm{b}_4\) be the columns of rref\((A)\). Note that 
  \[-3\bm{b}_1 = \bm{b}_2 \quad \text{and} \quad \frac{3}{2}\bm{b}_1 + \frac{5}{4}\bm{b}_3 = \bm{b}_4
  \]
  Now let \(\bm{a}_1\), \(\bm{a}_2\), \(\bm{a}_3\), and \(\bm{a}_4\) be the columns of \(A\).
  
  \begin{enumerate}
    
    \item Show that \(-3\bm{a}_1 =\bm{a}_2 \).
    
    \basec{\[%%%%%%%%%%%%%%%%%%%%%%%%%%%%%%%%%%%%%%%%%%%%%%%
    -3 \bm{a_1} = -3 \begin{bmatrix} -3 \\ 2 \\ 3 \end{bmatrix} = 
    \begin{bmatrix} 9 \\ -6 \\ -9 \end{bmatrix} =
    \bm{a_2}
    \]}%%%%%%%%%%%%%%%%%%%%%%%%%%%%%%%%%%%%%%%%%%%%%%%%%%%%%%
    \item Find a linear combination of \(\bm{a}_4\) in terms of \(\bm{a}_1\) and \(\bm{a}_3\).
    
    \basec{\[%%%%%%%%%%%%%%%%%%%%%%%%%%%%%%%%%%%%%%%%%%%%%%%
    \frac{3}{2}\bm{a_1}+\frac{5}{4}\bm{a_3} =
    \frac{3}{2}\begin{bmatrix} -3 \\ 2 \\ 3 \end{bmatrix}+\frac{5}{4}\begin{bmatrix} -2 \\ 4 \\ -2 \end{bmatrix} =
    \begin{bmatrix} -4.5 \\ 3 \\ 4.5 \end{bmatrix}+
    \begin{bmatrix} -2.5 \\ 5 \\ -2.5 \end{bmatrix} =
    \begin{bmatrix} -7 \\ 8 \\ 2 \end{bmatrix} = 
    \bm{a_4}
    \]}%%%%%%%%%%%%%%%%%%%%%%%%%%%%%%%%%%%%%%%%%%%%%%%%%%%%%%
    
    \item Show that the set \(A\) \( \{ \bm{a}_1, \, \bm{a}_2, \, \bm{a}_3, \, \bm{a}_4 \}  \) is linearly dependent.
    
    \basec{\begin{itemize}
      \item Algebraically, a sequence of vectors \(\bm{v}_1,\bm{v}_2,\ldots ,\bm{v}_k\) from a vector space are linearly dependent if there exist scalars (not all zero) \(\lambda_1,\lambda_2\ldots s,\lambda_k\) such that they form the zero vector \(\bm{o}\), i.e.,
      \[%%%%%%%%%%%%%%%%%%%%%%%%%%%%%%%%%%%%%%%%%%%%%%%
      \lambda_1\bm{v}_1,\lambda_2\bm{v}_2,\ldots,\lambda_k\bm{v}_k = \bm{o} \qquad \lambda \in \mathbb{R}
      \]%%%%%%%%%%%%%%%%%%%%%%%%%%%%%%%%%%%%%%%%%%%%%%%
      \item We already showed that at least one vector in the set can be expressed as a linear combination of another, meaning A is linearly dependent, but it can still be done explicitly:
      \[%%%%%%%%%%%%%%%%%%%%%%%%%%%%%%%%%%%%%%%%%%%%%%%
      3\bm{a_1} + \bm{a_2} + 0\bm{a_3} + 0\bm{a_4} = \bm{o}
      \]%%%%%%%%%%%%%%%%%%%%%%%%%%%%%%%%%%%%%%%%%%%%%%%
      or
      \[%%%%%%%%%%%%%%%%%%%%%%%%%%%%%%%%%%%%%%%%%%%%%%%
      -\frac{3}{2}\bm{a_1} + 0\bm{a_2} -\frac{5}{4}\bm{a_3} + \bm{a_4} = \bm{o}
      \]%%%%%%%%%%%%%%%%%%%%%%%%%%%%%%%%%%%%%%%%%%%%%%%
      
    \end{itemize}}
    
  \end{enumerate}
\end{enumerate}

\end{document}