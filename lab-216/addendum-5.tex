\documentclass[12pt,a4paper]{article}
\usepackage{inverba}
\newcommand{\userName}{Cullyn Newman} 
\newcommand{\class}{BI 216} 
\newcommand{\institution}{Portland State University} 
\newcommand{\thetitle}{\hypertarget{home}{Lab 5 Addendum: The Nervous System}}
\rfoot{\hyperlink{home}{\thepage}}

\begin{document}
\section*{Part A: Brain Workouts}
\begin{enumerate}[font=\bfseries, wide]
    {\color{under}\item Wnat is brain plasticity? \textbf{(0.5 pts)}}

    >>
    {\color{under}\item What is the term that refers to the formation of new neurons? \textbf{(0.5 pts)}}

    >> 
    {\color{under}\item What term refers to the genetically programmed death of neurons in the brain? Is this genetically programmed neuronal death only found in the brains of people with Alzheimer’s disease? Please explain. \textbf{(2 pts)}}

    >> 
    {\color{under}\item Explain what dendritic spines, sprouting, branching, and arborization are and how these concepts relate to synapses. Then, draw a picture that you could share with Anthony and Darrell that demonstrates these concepts and that willhelp them understand. Be sure to label the neurons’ other relevant structures,as well.  Insert your drawing below.  \textbf{(2 pts)}}

    >> 
    {\color{under}\item Anthony is correct that there are chemicals in the brain that help neurons survive, help promote neural growth, and are involved in neurogenesis. Theseare called neurotrophins. Describe how the neurotrophins nerve growth factor (NGF) and brain-derived neurotrophic factor (BDNF) promote neuron survival,growth, and/or neurogenesis? \textbf{(1 pt)}}

    >> 
    {\color{under}\item Darrell’s father used the phrase, “use it or lose it.” What neuronal activities arerelated to this idea? \textbf{(1 pt)}}

    >> 
    {\color{under}\item Darrell’s dad insists that there are scientific research findings supporting his claim that playing brain games helps one’s brain and keeps memory from declining, whereas Anthony’s dad insists that there are scientific research results supporting his claim that physical exercise helps one’s brain and slowsmemory decline. Based on what you’ve learned about synapses and about the chemicals that promote neural survival and growth, is one of the dads correct? Is neither correct? Are both correct? Give evidence to support your answer.\textbf{(2 pts)}}

    >> 
    {\color{under}\item Which of these two “brain workouts” do you believe would be the most beneficial for you as you experience adulthood and aging? Why? Which ofthese brain workouts do you believe would be the most beneficial for you in terms of learning material in your current college classes? Why? \textbf{(2 pt)}}

    >> 
\end{enumerate}

\section*{Part B: Neurophysiology}
\begin{enumerate}[font=\bfseries, wide, resume]
    {\color{under}\item Where on the leech are the initial dissection cuts made, and why are they made there? \textbf{(1 pt)}}

    >> 
    {\color{under}\item How many different types of neurons are probed and identified in the simulation?\textbf{(0.5 pts)}}

    >> 
    {\color{under}\item What part of the neuron is the fluorescent dye injected? \textbf{(0.5 pts)}}

    >> 
    {\color{under}\item The simulation provided two tools by which to identify a neuron. Please explain what those two factors are, and which factor was easier for you to use when identifying the neuron? \textbf{(1 pt)}}

    >> 
\end{enumerate}
    
\section*{Part C: Brain Control}
\begin{enumerate}[font=\bfseries, wide, resume]
    {\color{under}\item It is difficult to measure feelings and emotions experimentally. How does Dr. Tyeovercome this problem in her research? \textbf{(1 pt)}}

    >> 
    {\color{under}\item The amygdala is a region of the brain and is thought to be important for emotion. What did Dr. Tye and colleagues discover about thes part of the brain? \textbf{(1 pt)}}

    >> 
    {\color{under}\item Why was it impotant for mice to explore the 'open arms' of the maze compared to the 'closed arms'? What is the relevance of this study? \textbf{(1 pt)}}

    >> 
    {\color{under}\item Give one example of how you see the potential of optogenetic treatment being effective for depression/anxiety. \textbf{(2 pt)}}

    >> 
\end{enumerate}

\end{document}